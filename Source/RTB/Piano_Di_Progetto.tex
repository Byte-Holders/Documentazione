\documentclass[a4paper, 11pt]{article}

% ====== PACCHETTI NECESSARI ======
\usepackage[utf8]{inputenc}
\usepackage[T1]{fontenc}
\usepackage[italian]{babel}
\usepackage{geometry}
\usepackage{graphicx}
\usepackage[table]{xcolor}
\usepackage{tabularx}
\usepackage{array}
\usepackage{amssymb}
\usepackage{fancyhdr}
\setlength{\headheight}{14pt}
\usepackage{titlesec}
\usepackage{helvet}
\renewcommand{\familydefault}{\sfdefault}
\usepackage{lipsum}
\usepackage{hyperref}
\usepackage{booktabs}
\usepackage{enumitem}
\usepackage[utf8]{inputenc} % Specifica la codifica del file (necessaria per le accentate)
\usepackage[T1]{fontenc}    % Migliora l'output dei font per le lingue europee
\usepackage{graphicx} % Necessario per \rotatebox
\usepackage{multirow}
\usepackage{float}

% Comando rapido per ruotare il testo di 45 gradi
\newcommand{\rot}[1]{%
  \multicolumn{1}{c}{%
    \makebox[0pt][l]{%
      \rotatebox[origin=bl]{45}{#1}%
    }%
  }%
}


% ====== IMPOSTAZIONI GLOBALI DI STILE ======

% 1. DEFINIZIONE COLORI BLU-VIOLA
\definecolor{AccentColor}{RGB}{80, 90, 180} % Blu-viola principale
\definecolor{AccentLight}{RGB}{80, 90, 180} % Versione più chiara
\definecolor{AccentDark}{RGB}{50, 60, 140} % Versione più scura
\definecolor{LightGray}{RGB}{245, 245, 250}
\definecolor{MediumGray}{RGB}{200, 200, 210}

% 2. IMPOSTAZIONE MARGINI
\geometry{a4paper, left=2.5cm, right=2.5cm, top=3.5cm, bottom=3.5cm}

% 3. STILE DEI TITOLI DI SEZIONE
\titleformat{\section}
  {\normalfont\sffamily\Large\bfseries\color{AccentColor}}
  {\thesection}
  {1em}
  {}
\titleformat{\subsection}
  {\normalfont\sffamily\large\bfseries\color{AccentDark}}
  {\thesubsection}
  {1em}
  {}
\titleformat{\subsubsection}
  {\normalfont\sffamily\large\bfseries\color{AccentDark}}
  {\thesubsubsection}
  {1em}
  {}

% --- CONFIGURAZIONE LIVELLI AGGIUNTIVI (4 e 5) ---
\setcounter{secnumdepth}{5} 
\setcounter{tocdepth}{5}

\titleformat{\paragraph}
  {\normalfont\sffamily\normalsize\bfseries\color{AccentDark}}
  {\theparagraph}
  {1em}
  {}
\titlespacing*{\paragraph}
  {0pt}
  {3.25ex plus 1ex minus .2ex}
  {1.5ex plus .2ex}

% Alias comando
\newcommand{\subsubsubsection}[1]{\paragraph{#1}}

\titleformat{\subparagraph}
  {\normalfont\sffamily\normalsize\bfseries\color{AccentDark}} % Stile
  {\thesubparagraph}
  {1em}
  {}
\titlespacing*{\subparagraph}
  {0pt}
  {3.25ex plus 1ex minus .2ex}
  {1.5ex plus .2ex}

% Alias comando
\newcommand{\subsubsubsubsection}[1]{\subparagraph{#1}}



% 4. IMPOSTAZIONE HEADER E FOOTER
\pagestyle{fancy}
\fancyhf{} 
\fancyhead[L]{\sffamily\bfseries\color{AccentColor}\@BYTE HOLDERS}
\fancyhead[R]{\sffamily\color{AccentColor}\thepage}
\renewcommand{\headrulewidth}{0.8pt}
\renewcommand{\headrule}{\color{AccentColor}\hrule width\headwidth height\headrulewidth \vskip-\headrulewidth}

% 5. IMPOSTAZIONE LINK
\hypersetup{
    colorlinks=true,
    linkcolor=AccentColor,
    urlcolor=AccentLight,
    citecolor=AccentDark,
}

% 6. PERSONALIZZAZIONE ELENCHI
\setlist[itemize]{itemsep=2pt, topsep=4pt}
\setlist[enumerate]{itemsep=2pt, topsep=4pt}

% ====== COMANDI PERSONALIZZATI ======
\makeatletter
\newcommand{\NomeGruppo}[1]{\def\@NomeGruppo{#1}}
\newcommand{\TitoloVerbale}[1]{\def\@TitoloVerbale{#1}}
\newcommand{\Sommario}[1]{\def\@Sommario{#1}}
\newcommand{\Autore}[1]{\def\@Autore{#1}}
\newcommand{\Verificatore}[1]{\def\@Verificatore{#1}}
\makeatother

% ====== STILE TABELLE MIGLIORATO ======
\newcolumntype{Y}{>{\raggedright\arraybackslash}X} % Colonna giustificata a sinistra
\setlength{\arrayrulewidth}{0.4pt} % Linee più sottili
\setlength{\tabcolsep}{10pt} % Spaziatura interna celle
\renewcommand{\arraystretch}{1.4} % Altezza righe

% ====== INIZIO DEL DOCUMENTO ======
\begin{document}

% ====== INFORMAZIONI PER LA PAGINA DI TITOLO ======
\NomeGruppo{BYTE HOLDERS}
\TitoloVerbale{Verbale Riunione Interna}
\Sommario{Questo verbale documenta la riunione interna avvenuta il 16/10/2025 per la discussione dei capitolati e la definizione degli strumenti di lavoro del team.}
\Autore{}
\Verificatore{}

\pagestyle{empty}

% ====== PAGINA DI TITOLO ======
\begin{titlepage}
    \centering
    
    \includegraphics[width=0.55\textwidth]{../Assets/ByteHolders1.png}\par\vspace{1.5cm}
    
    {\LARGE \sffamily \color{AccentColor}\bfseries Piano di Progetto}\par
    
    \vfill
    
    \noindent\color{AccentColor}\rule{\textwidth}{1pt}\par
    \vspace{0.5cm}
    
    \begin{tabularx}{0.9\textwidth}{@{}>{\bfseries\sffamily}l X@{}}
    Autori & \sffamily Giacomo Nalotto, Damiano Berti\\
    \arrayrulecolor{MediumGray}\hline \\[-1.5ex]
    Verificatori & \sffamily XXX\\
    \arrayrulecolor{MediumGray}\hline \\[-1.5ex] 
    Approvazione & \sffamily YYY\\ 
    \arrayrulecolor{MediumGray}\hline 
\end{tabularx}
    
    \vfill
\end{titlepage}

\newpage

% ====== TABELLA DI VERSIONAMENTO ======
{\normalfont\sffamily\huge\bfseries\color{AccentColor} Registro delle versioni}
\vspace{1cm}

\begin{center}
    \rowcolors{2}{LightGray}{white}
    \begin{tabular}{>{\centering\arraybackslash}m{1.6cm} >{\centering\arraybackslash}m{2cm} >{\raggedright\arraybackslash}m{1.7cm} >{\raggedright\arraybackslash}m{1.7cm} >{\raggedright\arraybackslash}m{5cm}}
        \rowcolor{AccentColor}
          \textcolor{white}{\textbf{Versione}} & 
          \textcolor{white}{\textbf{Data}} & 
          \multicolumn{1}{c}{\textcolor{white}{\textbf{Autore}}} & 
          \multicolumn{1}{c}{\textcolor{white}{\textbf{verificatore}}} &
          {\textcolor{white}{\textbf{Descrizione delle modifiche}}} \\

        0.5.0 & 30/12/2025 & Damiano Berti &  & aggiunte e modifiche varie alle sezioni di pianificazione e alla sezione dei rischi\\         
        \hline
        0.4.0 & 29/12/2025 & Giulia Romanato & Damiano Berti & Redazione sprint 2 \\ 
        \hline
        0.3.0 & 21/12/2025 & Giulia Romanato & Damiano Berti & Redazione sprint 1 \\ 
        \hline
        0.2.1 & 21/12/2025 & Giulia Romanato & Damiano Berti & creazione tabelle tracciamento sprint \\ 
        \hline
        0.2.0 & 20/12/2025 & Giulia Romanato & Damiano Berti & Inserita introduzione generale, sezione "analisi e gestione dei rischi" e prima lista di rischi\\ 
        \hline
        0.1.0 & 01/12/2025 & Giulia Romanato & Damiano Berti & Inizio stesura \\ 
        
    \end{tabular}
\end{center}


% ====== INDICE ======
\pagestyle{fancy}
\newpage
\tableofcontents
\newpage
\listoftables
\newpage
\listoffigures
\newpage

\section{Introduzione}

\subsection{informazioni generali}
Il presente documento costituisce il \textbf{Piano di Progetto} ufficiale per lo sviluppo del software Code Guardian. Esso definisce la pianificazione temporale, l'allocazione delle risorse e la metodologia di gestione adottata per garantire il rispetto degli standard qualitativi e dei vincoli contrattuali. \\[0.5cm]
Lo scopo primario è fornire una guida strutturata per il monitoraggio dell'avanzamento lavori e la mitigazione dei rischi nell'intero ciclo di vita del progetto, il presente piano sarà quindi soggetto ad aggiornamenti progressivi.

\subsection{Glossario}
Data la natura tecnica del progetto è naturale l'utilizzo di termini settoriali e talvolta anche definiti dal gruppo, per questo è stato appositamente creato un glossario presente su un documento separato, ogni parola che rimanda al glossario sarà contrassegnata come segue:
\begin{center}
  termine\textsuperscript{G}
\end{center}

\subsection{Riferimenti}
\subsubsection{Riferimenti normativi}
\begin{itemize}
  \item \href{https://www.math.unipd.it/~tullio/IS-1/2025/Progetto/C2p.pdf}{Capitolato d'appalto C2: Code Guardian - Var Group}
  \item norme di progetto
\end{itemize}
\subsubsection{Riferimenti informativi}
\begin{itemize}
  \item \href{https://www.math.unipd.it/~tullio/IS-1/2025/Dispense/T02.pdf}{T2: i processi di ciclo di vita del software}
  \item \href{https://www.math.unipd.it/~tullio/IS-1/2025/Dispense/T04.pdf}{T4: Gestione di progetto}
  \item glossario
\end{itemize}

\newpage

\section{Analisi e gestione dei rischi}
\subsection{Introduzione}
L'analisi e la gestione dei rischi rappresentano il processo strategico volto a identificare, valutare e mitigare i potenziali eventi avversi che potrebbero compromettere gli obiettivi del gruppo. \\[0.4cm]
Identificare ed analizzare i rischi in anticipo è essenziale per guidare lo sviluppo del progetto con maggiore controllo, così da poter anticipare le criticità anziché subirle e poter meglio pianificare. Inoltre permette stimare più accuratamente il tempo necessario per le attività così da poter meglio calibrare il backlog degli sprint. \\[0.4cm]
Il processo di gestione dei rischi si articola in quattro fasi essenziali:
\begin{itemize}
  \item \textbf{Identificazione}: consiste nel rintracciare e scrivere un elenco di tutti i possibili problemi o imprevisti che potrebbero ostacolare il progetto
  \item \textbf{Analisi}: serve a capire quanto ogni rischio sia , valutando quante probabilità ha di accadere e quanti danni potrebbe causare.
  \item \textbf{Pianificazione}: Rappresenta l’insieme delle azioni di mitigazione pianificate per ridurre la probabilità che un rischio si manifesti o per limitarne i possibili effetti dannosi.
  \item \textbf{Constrollo}È l’attività di monitoraggio continuo che verifica nel tempo l'efficacia delle misure di prevenzione adottate. Il controllo serve a garantire che i rischi restino entro livelli accettabili, a rilevare l'insorgenza di nuove minacce e ad aggiornare dinamicamente le startegie di mitigazione in base all'evoluzione del progetto.
\end{itemize}
\vspace{0.4cm}
Il gruppo Byte Holders indivudua la seguente lista di possibili rischi, suddividendoli per tipo con annesso codice identificativo:
\begin{itemize}
  \item \textbf{T} rischi teconologici
  \item \textbf{I} rischi individuali
  \item \textbf{O} rischi organizzativi
\end{itemize}
\vspace{0.4cm}
Ogni rischio è quindi contrassegnato con un codice composto da un prefisso indicante il tipo ed una cifra univoca.

\newpage

\subsection{Rischi tecnologici}

\subsubsection{T1 - Inesperienza con le nuove tecnologie}
\begin{table}[H]
    \centering
    \rowcolors{2}{LightGray}{white}
    \begin{tabular}{|m{5cm}|m{8cm}|}
        \hline
        \multicolumn{2}{|>{\centering\arraybackslash}m{14cm}|}{\cellcolor{AccentColor}\textcolor{white}{\textbf{T1 - Inesperienza con le nuove tecnologie}}} \\ 
        \hline
        Descrizione & il gruppo ha scarsa o nulla esperienza con le tecnologie richieste dal capitolato: architetture ad agenti con LLM, Node.js, Python, React, AWS, OWASP e GitHub Actions \\ 
        \hline
        Mitigazione & dedicare tempo allo studio delle tecnologie, sfruttare le sessioni di mentoring tecnico offerte dalla proponente e le risorse informative fornite nel capitolato, realizzare più test di Proof of Concept per familiarizzare con i componenti critici prima dell'implementazione \\ 
        \hline
        Probabilità di occorrenza & alta \\
        \hline
        Pericolosità & elevata \\
        \hline
    \end{tabular}
    \caption{Informazioni sul rischio T1}
\end{table}

\subsubsection{T2 - Mancanza della documentazione necessaria}
\begin{table}[H]
    \centering
    \rowcolors{2}{LightGray}{white}
    \begin{tabular}{|m{5cm}|m{8cm}|}
        \hline
        \multicolumn{2}{|>{\centering\arraybackslash}m{14cm}|}{\cellcolor{AccentColor}\textcolor{white}{\textbf{T2 - Mancanza della documentazione necessaria}}} \\ 
        \hline
        Descrizione & le tecnologie da utilizzare potrebbero avere documentazione insufficiente, obsoleta o poco chiara, in particolare per framework di agenti LLM, strumenti OWASP e SDK AWS \\ 
        \hline
        Mitigazione & identificare precocemente le aree con documentazione carente, utilizzare risorse alternative come tutorial, video dimostrativi e forum. Valutare eventualmente tecnologie alternative meglio documentate previa approvazione del committente \\ 
        \hline
        Probabilità di occorrenza & media \\
        \hline
        Pericolosità & media \\
        \hline
    \end{tabular}
    \caption{Informazioni sul rischio T2}
\end{table}

\subsubsection{T3 - Guasti tecnici hardware o software}
\begin{table}[H]
    \centering
    \rowcolors{2}{LightGray}{white}
    \begin{tabular}{|m{5cm}|m{8cm}|}
        \hline
        \multicolumn{2}{|>{\centering\arraybackslash}m{14cm}|}{\cellcolor{AccentColor}\textcolor{white}{\textbf{T3 - Guasti tecnici hardware o software}}} \\ 
        \hline
        Descrizione & possibili malfunzionamenti dell'hardware dei membri del gruppo o dei servizi esterni utilizzati (GitHub, AWS, GitHub Actions) che potrebbero interrompere le attività di sviluppo e testing \\ 
        \hline
        Mitigazione & utilizzare sistemi di versionamento del codice per garantire backup continuo, implementare deployment su più ambienti, utilizzare Docker per ambienti di sviluppo locali in caso di indisponibilità dei servizi cloud, mantenere copie locali dei repository \\ 
        \hline
        Probabilità di occorrenza & bassa \\
        \hline
        Pericolosità & media \\
        \hline
    \end{tabular}
    \caption{Informazioni sul rischio T3}
\end{table}

\subsection{Rischi individuali}

\subsubsection{I1 - Impegno personale imprevisto}
\begin{table}[H]
    \centering
    \rowcolors{2}{LightGray}{white}
    \begin{tabular}{|m{5cm}|m{8cm}|}
        \hline
        \multicolumn{2}{|>{\centering\arraybackslash}m{14cm}|}{\cellcolor{AccentColor}\textcolor{white}{\textbf{I1 - Impegno personale imprevisto}}} \\ 
        \hline
        Descrizione & uno o più membri del gruppo potrebbero trovarsi impossibilitati a dedicare tempo al progetto a causa di emergenze personali, problemi familiari o di salute \\ 
        \hline
        Mitigazione & comunicare tempestivamente l'indisponibilità, gli altri membri si dividono il carico di lavoro, rimandando attività meno urgenti se necessario; una volta risolta l'emergenza, il membro recupera le attività rimanenti \\ 
        \hline
        Probabilità di occorrenza & media \\
        \hline
        Pericolosità & bassa \\
        \hline
    \end{tabular}
    \caption{Informazioni sul rischio I1}
\end{table}

\subsubsection{I2 - Mancanza di tempo dovuto ad altre attività universitarie}
\begin{table}[H]
    \centering
    \rowcolors{2}{LightGray}{white}
    \begin{tabular}{|m{5cm}|m{8cm}|}
        \hline
        \multicolumn{2}{|>{\centering\arraybackslash}m{14cm}|}{\cellcolor{AccentColor}\textcolor{white}{\textbf{I2 - Mancanza di tempo dovuto ad altre attività universitarie}}} \\ 
        \hline
        Descrizione & i membri del gruppo devono gestire altri corsi, esami e progetti universitari che limitano la disponibilità di tempo da dedicare al progetto, specialmente durante le sessioni d'esame \\ 
        \hline
        Mitigazione & pianificare il carico di lavoro considerando il calendario accademico, identificare in anticipo i periodi critici e ridurre il carico in quegli sprint, comunicare tempestivamente sovrapposizioni con altri impegni e bilanciare il lavoro per permettere recuperi successivi \\ 
        \hline
        Probabilità di occorrenza & alta \\
        \hline
        Pericolosità & media \\
        \hline
    \end{tabular}
    \caption{Informazioni sul rischio I2}
\end{table}

\subsection{Rischi organizzativi}

\subsubsection{O1 - Sottostima del tempo necessario per una task}
\begin{table}[H]
    \centering
    \rowcolors{2}{LightGray}{white}
    \begin{tabular}{|m{5cm}|m{8cm}|}
        \hline
        \multicolumn{2}{|>{\centering\arraybackslash}m{14cm}|}{\cellcolor{AccentColor}\textcolor{white}{\textbf{O1 - Sottostima del tempo necessario per una task}}} \\ 
        \hline
        Descrizione & durante la pianificazione degli sprint, il gruppo potrebbe sottostimare il tempo necessario per completare attività complesse \\ 
        \hline
        Mitigazione & utilizzare buffer di tempo nelle stime, specialmente per tecnologie nuove, basarsi sulle retrospettive per migliorare le stime, monitorare costantemente l'avanzamento e segnalare tempestivamente ritardi, riorganizzare le priorità durante gli sprint; eventualmente consultare la proponente per validare stime su attività complesse. \\ 
        \hline
        Probabilità di occorrenza & alta \\
        \hline
        Pericolosità & elevata \\
        \hline
    \end{tabular}
    \caption{Informazioni sul rischio O1}
\end{table}

\subsubsection{O2 - Sovrastima del tempo necessario per una task}
\begin{table}[H]
  \begin{center}
    \rowcolors{2}{LightGray}{white}
    \begin{tabular}{|m{5cm}|m{8cm}|}
        \hline
        \multicolumn{2}{|>{\centering\arraybackslash}m{14cm}|}{\cellcolor{AccentColor}\textcolor{white}{\textbf{O2 - Sovrastima del tempo necessario per una task}}} \\ 
        \hline
        Descrizione & alcune attività potrebbero richiedere meno tempo del previsto, lasciando membri del gruppo con tempo disponibile non pianificato \\ 
        \hline
        Mitigazione & mantenere un backlog di attività secondarie utilizzabili in caso di anticipo, i membri in anticipo supportano i colleghi in difficoltà o iniziano task dello sprint successivo; utilizzare il tempo extra per migliorare qualità del codice, test e documentazione. \\ 
        \hline
        Probabilità di occorrenza & bassa \\
        \hline
        Pericolosità & bassa \\
        \hline
    \end{tabular}
  \end{center}
  \caption{Informazioni sul rischio O2}
\end{table}


\begin{table}[H]
  \subsubsection{O3 - Problemi comunicativi interni al gruppo}
  \begin{center}
    \rowcolors{2}{LightGray}{white}
    \begin{tabular}{|m{5cm}|m{8cm}|}
        \hline
        \multicolumn{2}{|>{\centering\arraybackslash}m{14cm}|}{\cellcolor{AccentColor}\textcolor{white}{\textbf{O3 - Problemi comunicativi interni al gruppo}}} \\ 
        \hline
        Descrizione & possibili incomprensioni o disallineamento tra i membri del gruppo riguardo obiettivi, priorità o scelte tecniche, la complessità del progetto amplifica questo rischio \\ 
        \hline
        Mitigazione & stabilire canali di comunicazione regolari, documentare decisioni importanti e renderle accessibili, utilizzare GitHub Projects per trasparenza sullo stato delle attività, definire chiaramente ruoli e responsabilità per ogni sprint, effettuare code review per mantenere allineamento tecnico. \\ 
        \hline
        Probabilità di occorrenza & media \\
        \hline
        Pericolosità & media \\
        \hline
    \end{tabular}
  \end{center}
  \caption{Informazioni sul rischio O3}
\end{table}

\subsubsection{O4 - Problemi comunicativi con la proponente}

\begin{table}[H]
  \begin{center}
    \rowcolors{2}{LightGray}{white}
    \begin{tabular}{|m{5cm}|m{8cm}|}
        \hline
        \multicolumn{2}{|>{\centering\arraybackslash}m{14cm}|}{\cellcolor{AccentColor}\textcolor{white}{\textbf{O4 - Problemi comunicativi con la proponente}}} \\ 
        \hline
        Descrizione & possibili incomprensioni sui requisiti, aspettative non allineate o ritardi nelle risposte da parte della proponente potrebbero causare sviluppo di funzionalità non conformi o decisioni bloccate \\ 
        \hline
        Mitigazione & sfruttare al massimo le sessioni di design thinking e gli incontri periodici per chiarire i requisiti, documentare ogni incontro con verbali dettagliati., preparare domande specifiche per le sessioni di Q\&A, utilizzare diagrammi UML e documenti di requisiti per validare la comprensione, richiedere chiarimenti tempestivamente senza attendere gli incontri programmati mediante comunicazione asincrona \\ 
        \hline
        Probabilità di occorrenza & media \\
        \hline
        Pericolosità & elevata \\
        \hline
    \end{tabular}
  \end{center}
  \caption{Informazioni sul rischio O4}
\end{table}



\newpage


\section{Modello di sviluppo}

\subsection{Modello adottato}
Il gruppo \textbf{ByteHolders} ha scelto di adottare la metodologia \textbf{Agile}, ritenendola l'approccio più efficace per la gestione di un progetto software complesso. L'organizzazione del lavoro si baserà su iterazioni di due settimane, denominate \textbf{sprint}.
Il processo seguirà una logica di miglioramento continuo (simile al modello a spirale) grazie alle retrospettive al termine di ogni ciclo, il gruppo punterà a maturare e a rendere più efficienti i processi di sprint in sprint. Ogni aggiornamento sui dettagli degli verrà puntualmente riportato nella sezione \hyperref[pianificazione_breve_termine]{Pianificazione nel breve termine}.

\subsection{Gestione dei ruoli}
All'inizio di ciascun sprint, il gruppo procederà alla pianificazione delle attività e alla contestuale rotazione dei ruoli. Questa strategia ha il duplice scopo di permettere a ogni membro di acquisire competenze trasversali in tutte le aree del progetto e di individuare l'assetto organizzativo più efficiente.

\subsection{Comunicazione e gestione interna}
Per il coordinamento interno e le comunicazioni rapide, il gruppo si avvarrà di strumenti di messaggistica asincrona quali \textbf{WhatsApp} e \textbf{Discord}.
La gestione del \textbf{Backlog}, l'assegnazione e il tracciamento delle task avverranno tramite lo strumento \textbf{GitHub Projects}, garantendo così trasparenza e controllo sull'avanzamento dei lavori.

\subsection{Comunicazione con la proponente}
Il confronto costante con la proponente, \textbf{Var Group}, è ritenuto essenziale per la buona riuscita del progetto.
Gli accordi prevedono incontri di allineamento a cadenza bisettimanale, uno per ogni sprint. A questi appuntamenti fissi si affiancherà l'uso di canali asincroni e, qualora fosse necessario, verranno concordate ulteriori riunioni straordinarie per garantire il pieno allineamento sugli obiettivi.

\newpage

\section{Pianificazione nel lungo termine}
Come riportato nella \href{https://byte-holders.github.io/Documentazione/Candidatura/Dichiarazione_Impegni.pdf}{dichiarazione degli impegni} presentata alla candidatura il gruppo prevede di terminare il progetto entro il giorno 20/03/2026 con un budget corrispondente a \textbf{13.290 €}, suddiviso per ruolo secondo quanto riportato nella seguente tabella:
\begin{table}[H]
  \centering
  \begin{tabular}{|l|c|c|c|c|}
    \hline
    \cellcolor{AccentColor}\textbf{\textcolor{white}{Ruolo}} & 
    \cellcolor{AccentColor}\textbf{\textcolor{white}{Ore}} & 
    \cellcolor{AccentColor}\textbf{\textcolor{white}{Percentuale}} & 
    \cellcolor{AccentColor}\textbf{\textcolor{white}{Costo(€/h)}} & 
    \cellcolor{AccentColor}\textbf{\textcolor{white}{Costo Totale}} \\
    \hline
    
    Responsabile & 66 & 10,48\% & 30 €/h & 1980 € \\ \hline
    Amministratore & 50 & 7,94\% & 20 €/h & 1000 € \\ \hline
    Analista & 100 & 15,87\% & 25 €/h & 2500 € \\ \hline
    Progettista & 160 & 25,40\% & 25 €/h & 4000 € \\ \hline
    Programmatore & 127 & 20,16\% & 15 €/h & 1905 € \\ \hline
    Verificatore & 127 & 20,16\% & 15 €/h & 1905 € \\ \hline
    \textbf{Totale} & \textbf{630} & \textbf{100\%} & - & \textbf{13.290 €} \\ \hline
  \end{tabular}
  \caption{Tabella prospetto costi}
\end{table}

\subsection{Requirements and Technology Baseline (RTB)}
\vspace{0.4cm}
\begin{tabular}{ll}
Inizio: & \textbf{04/11/2025} \\
Fine: & \textbf{30/01/2026} \\
\end{tabular}
\vspace{0.2cm}
\subsubsection{Attività da svolgere}
\subsubsubsection{Redazione analisi dei requisiti}
Documento esterno che riporta i casi d'uso ed i requisti del programma rilevati, necessari per allineare le aspettative del clienete e degli stakeholders e per la futura progettazione e implementazione del software.
\subsubsubsection{Redazione Piano di Progetto}
Documento esterno che formalizza la pianificazione delle attività del gruppo. Il suo obiettivo primario è definire le strategie per il rispetto dei vincoli di tempi e budget concordati.
Esso include l'analisi dei rischi di progetto, la scelta del modello di sviluppo e la definizione del preventivo economico, fungendo da riferimento per il monitoraggio dell'avanzamento lavori.
\subsubsubsection{Redazione piano di qualifica}
Documento esterno che riporta come la qualità del progetto verrà gestita, monitorata e garantita.
\subsubsubsection{Redazione norme di progetto}
Documento interno che riporta il riferimento normativo interno del gruppo. Il documento descrive le procedure, gli strumenti e le convenzioni adottate per lo svolgimento del progetto. Il suo obiettivo è disciplinare il way of working.
\subsubsubsection{Redazione del glossario}
Documento interno che riporta una descrizione per termini tecnici e non, utile per una iù semplice comprensione dei docuementi
\subsubsubsection{Realizzazione del Proof of Concept}
Ovvero un prototipo semplificato ma funzionante, realizzato allo scopo di dimostrare la fattibilità tecnica della soluzione ideata relativamente ai requisiti.

\subsection{Product Baseline 
(PB)}
Questa sezione verrà redatta all'inizio della PB per garantire una pianificazione più accurata.

\newpage

\section{Pianificazione nel breve termine} \label{pianificazione_breve_termine}

\subsection{Introduzione}
In questa sezione vengono esposti i dettagli relativi alla pianificazione e alla rendicontazione di ogni sprint. Il monitoraggio dell'avanzamento avverrà attraverso l'analisi di preventivi e consuntivi, seguita da un momento di retrospettiva.\\
Nello specifico, per ogni iterazione verranno riportati:
\begin{itemize}
  \item Informazioni generali e attività da svolgere
  \item Rischi attesi
  \item Preventivo
  \item Consuntivo
  \item Aggiornamento delle risorse rimanenti
  \item Rischi incontrati
  \item Retrospettiva
\end{itemize}

\subsection{Requirements and Technology Baseline (RTB)}

\textbf{NOTA: } sebbene la data d'inizio della RTB sia il 04/11/2025 l'effettiva organizzazione in sprint è iniziata il 25/11/2025 a causa di una serie di problemi come:
\begin{itemize}
  \item difficoltà iniziale a causa dell'inesperianza.
  \item imprevisti personali.
  \item ritardi di comunicazione con l'azienda proponente.
\end{itemize}
Il gruppo ha in ogni caso sfruttato questo periodo con attività di formazione

\subsubsection{Sprint1}
\begin{tabular}{ll}
Inizio: & \textbf{25/11/2025} \\
Fine prevista: & \textbf{9/12/2025} \\
Fine reale: & \textbf{9/12/2025} \\
Giorni di ritardo: & \textbf{0} \\
\end{tabular}


\subsubsubsection{informazioni generali e attività da svolgere}
In questo primo sprint ufficiale, iniziato subito dopo la prima riunione con l'azienda proponente sono stati definiti degli obiettivi per la maggior parte organizzativi e di studio. \\
Gli obiettivi sono:
\begin{itemize}
  \item riorganizzare la struttura della repo GitHub
  \item sistemare e aggiornare il sito
  \item studio su come affrontare l'analisi dei requisiti
  \item individuazione degli attori e dei casi d'uso
  \item individuazione delle milestones
  \item individuazione della struttura dei documenti
  \item continuazione pelestra e studio
\end{itemize}

\subsubsubsection{rischi attesi}
I rischi che il gruppo si aspetta sono:
\begin{itemize}
  \item I1 - Impegno personale imprevisto
  \item O1 - Sottostima del tempo necessario per una task
  \item O2 - Sovrastima del tempo necessario per una task
  \item 03 - Problemi comunicativi interni al gruppo
  \item 04 - Problemi comunicativi con la proponente
\end{itemize}

\subsubsubsection{Preventivo}
\begin{table}[H]
    \centering
    \renewcommand{\arraystretch}{1.5}
    
    \begin{tabular}{|m{3.5cm}|*{6}{>{\centering\arraybackslash}m{1.2cm}|}}
        
        \multicolumn{1}{c}{} & 
        \multicolumn{1}{c}{\rule{0pt}{2cm}\makebox[0pt][l]{\rotatebox[origin=bl]{45}{\textcolor{AccentColor}{\textbf{Responsabile}}}}} & 
        
        \rot{\textcolor{AccentColor}{\textbf{Amministratore}}} & 
        \rot{\textcolor{AccentColor}{\textbf{Analista}}} & 
        \rot{\textcolor{AccentColor}{\textbf{Progettista}}} & 
        \rot{\textcolor{AccentColor}{\textbf{Programmatore}}} & 
        \rot{\textcolor{AccentColor}{\textbf{Verificatore}}} \\ \hline 
        
        Giacomo Nalotto     & - & 2 & - & - & - & - \\ \hline
        Damiano Berti       & - & 3 & 2 & - & - & 1 \\ \hline
        Alessandro Morabito & - & - & - & - & - & - \\ \hline
        Alessandro Frison   & - & 1 & - & - & - & - \\ \hline
        Giulia Romanato     & - & - & 2 & - & - & 1 \\ \hline
        Nicolò Lattanzio    & 2 & 2 & - & - & - & - \\ \hline
        Lorenzo Grolla      & - & 2 & - & - & - & 1 \\ \hline
    \end{tabular}
    \caption{Preventivo sprint 1}
\end{table}

\begin{figure}[h]
    \centering
    \includegraphics[width=0.8\textwidth]{../Assets/pdp/sprint1/preventivo.png}
    \caption{Grafico preventivo sprint 1}
\end{figure}

\subsubsubsection{Consuntivo}
\begin{table}[H]
    \centering
    \renewcommand{\arraystretch}{1.5}
    
    \begin{tabular}{|m{3.5cm}|*{6}{>{\centering\arraybackslash}m{1.2cm}|}}
        
        \multicolumn{1}{c}{} & 
        \multicolumn{1}{c}{\rule{0pt}{2.5cm}\makebox[0pt][l]{\rotatebox[origin=bl]{45}{\textcolor{AccentColor}{\textbf{Responsabile}}}}} & 
        
        \rot{\textcolor{AccentColor}{\textbf{Amministratore}}} & 
        \rot{\textcolor{AccentColor}{\textbf{Analista}}} & 
        \rot{\textcolor{AccentColor}{\textbf{Progettista}}} & 
        \rot{\textcolor{AccentColor}{\textbf{Programmatore}}} & 
        \rot{\textcolor{AccentColor}{\textbf{Verificatore}}} \\ \hline 
        
        Giacomo Nalotto     & - & 2 & - & - & - & - \\ \hline
        Damiano Berti       & - & 3 & 2 & - & - & 1 \\ \hline
        Alessandro Morabito & - & - & - & - & - & - \\ \hline
        Alessandro Frison   & - & 2 \textcolor{red}{(+1)} & - & - & - & - \\ \hline
        Giulia Romanato     & - & - & 2 & - & - & 1 \\ \hline
        Nicolò Lattanzio    & 3 \textcolor{red}{(+1)} & 2 & - & - & - & - \\ \hline
        Lorenzo Grolla      & - & 2 & - & - & - & 1 \\ \hline
    \end{tabular}
    \caption{Consuntivo sprint 1}
\end{table}

\begin{figure}[h]
    \centering
    \includegraphics[width=0.8\textwidth]{../Assets/pdp/sprint1/consuntivo.png}
    \caption{Grafico consuntivo sprint 1}
\end{figure}

\subsubsubsection{Aggiornamento delle risorse rimanenti}
\begin{table}[H]
    \centering
    \renewcommand{\arraystretch}{1.5}
    
    \begin{tabular}{|c|c|c|c|c|c|}
        \hline
        \textcolor{AccentColor}{\textbf{Ruolo}} &
        \textcolor{AccentColor}{\textbf{Costo}} &
        \textcolor{AccentColor}{\textbf{Ore}} &
        \textcolor{AccentColor}{\textbf{Costo}} &
        \textcolor{AccentColor}{\textbf{Ore rimanenti}} &
        \textcolor{AccentColor}{\textbf{Budget Rimanente}} \\ \hline
        
        Responsabile & 30€/h & 3 & 90€ &  63 \textcolor{red}{(-3)} &  1.890€ \textcolor{red}{(-90€)}\\ \hline
        
        Amministratore & 20€/h & 11 & 220€ & 39 \textcolor{red}{(-11)} &  780€ \textcolor{red}{(-220€)} \\ \hline
        
        Analista & 25€/h & 4 & 100€ &  96 \textcolor{red}{(-4)} &  2.400€ \textcolor{red}{(-100€)}\\ \hline
        
        Progettista & 25€/h & 0 & 0€ & 160 &  4.000€ \\ \hline
        
        Programmatore & 15€/h & 0 & 0€ &  127 &  1.905€ \\ \hline
        
        Verificatore & 15€/h & 3 & 45€ &  124 \textcolor{red}{(-3)}&  1.860€ \textcolor{red}{(-45€)} \\ \hline
        
        \textbf{Totale} & - & 21 & \textbf{455€} & \textbf{609} \textcolor{red}{(-21)} & \textbf{12.835€} \textcolor{red}{(-455€)}\\ \hline
    \end{tabular}
    \caption{Aggiornamento risorse rimanenti sprint 1}
\end{table}
\subsubsubsection{Rischi incontrati}
\begin{itemize}
  \item O1 - Sottostima del tempo necessario per una task
  \item O2 - Sovrastima del tempo necessario per una task
  \item 03 - Problemi comunicativi interni al gruppo
\end{itemize}
\subsubsubsection{Retrospettiva}
Questo primo sprint è stato fondamentale per capire le problematiche dovute alla gestione delle tempistiche. Sono emersi casi di sovrastima, come per l'aggiornamento del sito, ma anche criticità dovute alla sottostima dell'impegno richiesto, in particolare nell'individuazione dei casi d'uso. In quest'ultima fase, il valore prodotto è stato limitato da problemi comunicativi interni e discordanze d'opinione che hanno rallentato i processi decisionali. Ulteriori ostacoli sono derivati da una gestione del lavoro eccessivamente sincrona (svoltasi prevalentemente in riunione) e da una carente automazione dei processi, che ha pesato sull'efficienza complessiva.









\newpage









\subsubsection{Sprint2}
\begin{tabular}{ll}
Inizio: & \textbf{10/12/2025} \\
Fine prevista: & \textbf{27/12/2025} \\
Fine reale: & \textbf{29/12/2025} \\
Giorni di ritardo: & \textbf{2} \\
\end{tabular}


\subsubsubsection{informazioni generali e attività da svolgere}
Nel secondo sprint, sono stati definiti degli obiettivi per identificare e provare le tecnologie da usare
 e scrivere i casi d'uso identificati nello sprint precedente. \\
Gli obiettivi sono:
\begin{itemize}
  \item completati i casi d'uso identificati nel precedente sprint
  \item integrazione di nuovi casi d'uso
  \item confronto con il prof. Cardin riguardo i casi d'uso identificati
  \item individuazione delle tecnologie da utilizzare
  \item completamento degli incontri di formazione sulle tecnologie da parte di VARGroup
  \item identificazione della struttura del poc (proof of concept)
\end{itemize}

\subsubsubsection{rischi attesi}
I rischi che il gruppo si aspetta sono:
\begin{itemize}
  \item I1 - Impegno personale imprevisto
  \item O1 - Sottostima del tempo necessario per una task
  \item O2 - Sovrastima del tempo necessario per una task
  \item 03 - Problemi comunicativi interni al gruppo
  \item 04 - Problemi comunicativi con la proponente
  \item T1- Inesperienza con le nuove tecnologie
\end{itemize}

\subsubsubsection{Preventivo}
\begin{table}[H]
    \centering
    \renewcommand{\arraystretch}{1.5}
    
    \begin{tabular}{|m{3.5cm}|*{6}{>{\centering\arraybackslash}m{1.2cm}|}}
        
        \multicolumn{1}{c}{} & 
        \multicolumn{1}{c}{\rule{0pt}{2cm}\makebox[0pt][l]{\rotatebox[origin=bl]{45}{\textcolor{AccentColor}{\textbf{Responsabile}}}}} & 
        
        \rot{\textcolor{AccentColor}{\textbf{Amministratore}}} & 
        \rot{\textcolor{AccentColor}{\textbf{Analista}}} & 
        \rot{\textcolor{AccentColor}{\textbf{Progettista}}} & 
        \rot{\textcolor{AccentColor}{\textbf{Programmatore}}} & 
        \rot{\textcolor{AccentColor}{\textbf{Verificatore}}} \\ \hline 
        
        Giacomo Nalotto     & - & - & 4 & - & - & - \\ \hline
        Damiano Berti       & - & - & 8 & - & - & 1 \\ \hline
        Alessandro Morabito & - & 0,5 & 8 & - & - & 0,5\\ \hline
        Alessandro Frison   & - & - & 4 & - & - & 1 \\ \hline
        Giulia Romanato     & 3 & 3 & 10 & - & - & - \\ \hline
        Nicolò Lattanzio    & - & 4 & 6 & - & - & - \\ \hline
        Lorenzo Grolla      & - & 6 & 3 & - & - & 0,5 \\ \hline
    \end{tabular}
    \caption{Preventivo sprint 2}
\end{table}

\begin{figure}[h]
    \centering
    \includegraphics[width=0.8\textwidth]{../Assets/pdp/sprint2/preventivo.png}
    \caption{Grafico preventivo sprint 2}
\end{figure}

\subsubsubsection{Consuntivo}
\begin{table}[H]
    \centering
    \renewcommand{\arraystretch}{1.5}
    
    \begin{tabular}{|m{3.5cm}|*{6}{>{\centering\arraybackslash}m{1.2cm}|}}
        
        \multicolumn{1}{c}{} & 
        \multicolumn{1}{c}{\rule{0pt}{2.5cm}\makebox[0pt][l]{\rotatebox[origin=bl]{45}{\textcolor{AccentColor}{\textbf{Responsabile}}}}} & 
        
        \rot{\textcolor{AccentColor}{\textbf{Amministratore}}} & 
        \rot{\textcolor{AccentColor}{\textbf{Analista}}} & 
        \rot{\textcolor{AccentColor}{\textbf{Progettista}}} & 
        \rot{\textcolor{AccentColor}{\textbf{Programmatore}}} & 
        \rot{\textcolor{AccentColor}{\textbf{Verificatore}}} \\ \hline 
        
        Giacomo Nalotto     & - & - & 5\textcolor{red}{(+1)}& - & - & - \\ \hline
        Damiano Berti       & - & - & 10\textcolor{red}{(+2)} & - & - & 1 \\ \hline
        Alessandro Morabito & - & 0,5 & 10\textcolor{red}{(+2)} & - & - & 0,5 \\ \hline
        Alessandro Frison   & - & - & 5\textcolor{red}{(+1)} & - & - & 1,5\textcolor{red}{(+1)} \\ \hline
        Giulia Romanato     & 4\textcolor{red}{(+1)} & 2 & 7 & - & - & - \\ \hline
        Nicolò Lattanzio    & - & 3 & - & - & - & - \\ \hline
        Lorenzo Grolla      & - & 5 & 3 & - & - & 0,5 \\ \hline
    \end{tabular}
    \caption{Consuntivo sprint 2}
\end{table}

\begin{figure}[h]
    \centering
    \includegraphics[width=0.8\textwidth]{../Assets/pdp/sprint2/consuntivo.png}
    \caption{Grafico consuntivo sprint 2}
\end{figure}

\subsubsubsection{Aggiornamento delle risorse rimanenti}
\begin{table}[H]
    \centering
    \renewcommand{\arraystretch}{1.5}
    
    \begin{tabular}{|c|c|c|c|c|c|}
        \hline
        \textcolor{AccentColor}{\textbf{Ruolo}} &
        \textcolor{AccentColor}{\textbf{Costo}} &
        \textcolor{AccentColor}{\textbf{Ore}} &
        \textcolor{AccentColor}{\textbf{Costo}} &
        \textcolor{AccentColor}{\textbf{Ore rimanenti}} &
        \textcolor{AccentColor}{\textbf{Budget Rimanente}} \\ \hline
        
        Responsabile & 30€/h & 4 & 120€ &  59 \textcolor{red}{(-4)} &  1.770€ \textcolor{red}{(-120€)}\\ \hline
        
        Amministratore & 20€/h & 10,5 & 210€ & 28,5 \textcolor{red}{(-10,5)} &  570€ \textcolor{red}{(-210€)} \\ \hline
        
        Analista & 25€/h & 40 & 1000€ &  56 \textcolor{red}{(-40)} &  1.400€ \textcolor{red}{(-1000€)}\\ \hline
        
        Progettista & 25€/h & 0 & 0€ & 160 &  4.000€ \\ \hline
        
        Programmatore & 15€/h & 0 & 0€ &  127 &  1.905€ \\ \hline
        
        Verificatore & 15€/h & 3,5 & 52,5€ &  120,5 \textcolor{red}{(-3,5)}&  1.807€ \textcolor{red}{(-52,5€)} \\ \hline
        
        \textbf{Totale} & - & 58 & \textbf{1382€} & \textbf{551} \textcolor{red}{(-58)} & \textbf{11.452€} \textcolor{red}{(-1382€)}\\ \hline
    \end{tabular}
    \caption{Aggiornamento risorse rimanenti sprint 1}
\end{table}
\subsubsubsection{Rischi incontrati}
\begin{itemize}
  \item I1 - Impegno personale imprevisto: malattia e impegni familiari
  \item O1 - Sottostima del tempo necessario per una task
  \item O2 - Sovrastima del tempo necessario per una task
  \item 03 - Problemi comunicativi interni al gruppo: divergenze di opinione sui casi d'uso
  \item 04 - Problemi comunicativi con la proponente: domande riguardo i casi d'uso ancora in attesa di risposta
   e mancanza dell'account di AWS che doveva essere fornito da VARGroup
  \item T1- Inesperienza con le nuove tecnologie: difficoltà nell'apprendimento dovuta alla grande varietà di tecnologie suggerite
\end{itemize}
\subsubsubsection{Retrospettiva}
La non chiara definizione della struttura e del grado di approfondimento di ogni singola sezione (Attore, precondizioni, postcondizioni, ect) di un caso d'uso ha portato a dover modificare i casi d'uso precedentemente scritti in modo da renderli coerenti. Inoltre, la difficoltà nell'individuare in modo chiaro gli attori che agiscono nel sistema ha comportato una sottostima dei tempi prevista per la stesura dei casi d'uso. \\
La mancanza dell'account AWS, che doveva essere fornito dall'azienda, ha impedito di studiare in modo pratico il funzionamento degli agenti che poi dovranno essere usati nell'applicazione. \\
Inoltre la grande varietà di servizi AWS presentati dall'azienda, ha reso complicato capire come effettivamente vadano usati e come interagiscono tra loro.\\
Ciò ha portato ad un rallentamento dello studio delle relative tecnologie e di conseguenza dello sviluppo del poc. In attesa, è stata sperimentata la creazione di agenti con mezzi alternativi quali Google AI studio.


\end{document}