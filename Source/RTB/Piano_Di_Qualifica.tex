\documentclass[a4paper, 11pt]{article}

% ====== PACCHETTI NECESSARI ======
\usepackage[utf8]{inputenc}
\usepackage[T1]{fontenc}
\usepackage[italian]{babel}
\usepackage{geometry}
\usepackage{graphicx}
\usepackage[table]{xcolor}
\usepackage{tabularx}
\usepackage{array}
\usepackage{amssymb}
\usepackage{fancyhdr}
\setlength{\headheight}{14pt}
\usepackage{titlesec}
\usepackage{helvet}
\renewcommand{\familydefault}{\sfdefault}
\usepackage{lipsum}
\usepackage{hyperref}
\usepackage{booktabs}
\usepackage{enumitem}
\usepackage{amsmath}
\usepackage[utf8]{inputenc} % Specifica la codifica del file (necessaria per le accentate)
\usepackage[T1]{fontenc}    % Migliora l'output dei font per le lingue europee

% ====== IMPOSTAZIONI GLOBALI DI STILE ======

% 1. DEFINIZIONE COLORI BLU-VIOLA
\definecolor{AccentColor}{RGB}{80, 90, 180} % Blu-viola principale
\definecolor{AccentLight}{RGB}{80, 90, 180} % Versione più chiara
\definecolor{AccentDark}{RGB}{50, 60, 140} % Versione più scura
\definecolor{LightGray}{RGB}{245, 245, 250}
\definecolor{MediumGray}{RGB}{200, 200, 210}

% 2. IMPOSTAZIONE MARGINI
\geometry{a4paper, left=2.5cm, right=2.5cm, top=3.5cm, bottom=3.5cm}

% 3. STILE DEI TITOLI DI SEZIONE
\titleformat{\section}
  {\normalfont\sffamily\Large\bfseries\color{AccentColor}}
  {\thesection}
  {1em}
  {}
\titleformat{\subsection}
  {\normalfont\sffamily\large\bfseries\color{AccentDark}}
  {\thesubsection}
  {1em}
  {}

% 4. IMPOSTAZIONE HEADER E FOOTER
\pagestyle{fancy}
\fancyhf{} 
\fancyhead[L]{\sffamily\bfseries\color{AccentColor}\@BYTE HOLDERS}
\fancyhead[R]{\sffamily\color{AccentColor}\thepage}
\renewcommand{\headrulewidth}{0.8pt}
\renewcommand{\headrule}{\color{AccentColor}\hrule width\headwidth height\headrulewidth \vskip-\headrulewidth}

% 5. IMPOSTAZIONE LINK
\hypersetup{
    colorlinks=true,
    linkcolor=AccentColor,
    urlcolor=AccentLight,
    citecolor=AccentDark,
}

% 6. PERSONALIZZAZIONE ELENCHI
\setlist[itemize]{itemsep=2pt, topsep=4pt}
\setlist[enumerate]{itemsep=2pt, topsep=4pt}

% ====== COMANDI PERSONALIZZATI ======
\makeatletter
\newcommand{\NomeGruppo}[1]{\def\@NomeGruppo{#1}}
\newcommand{\Sommario}[1]{\def\@Sommario{#1}}
\newcommand{\Autore}[1]{\def\@Autore{#1}}
\newcommand{\Verificatore}[1]{\def\@Verificatore{#1}}
\makeatother

% ====== STILE TABELLE MIGLIORATO ======
\newcolumntype{Y}{>{\raggedright\arraybackslash}X} % Colonna giustificata a sinistra
\setlength{\arrayrulewidth}{0.4pt} % Linee più sottili
\setlength{\tabcolsep}{10pt} % Spaziatura interna celle
\renewcommand{\arraystretch}{1.4} % Altezza righe

% ====== INIZIO DEL DOCUMENTO ======
\begin{document}

% ====== INFORMAZIONI PER LA PAGINA DI TITOLO ======
\NomeGruppo{BYTE HOLDERS}
\Sommario{Questo documento descrive il piano di qualifica del progetto, delineando le strategie e le metodologie che verranno adottate per garantire la qualità del prodotto finale.}
\Autore{}
\Verificatore{}

\pagestyle{empty}

% ====== PAGINA DI TITOLO ======
\begin{titlepage}
    \centering
    
    \includegraphics[width=0.55\textwidth]{../Assets/ByteHolders1.png}\par\vspace{1.5cm}
    
    {\LARGE \sffamily \color{AccentColor}\bfseries Piano di Qualifica}\par
    
    \vfill
    
    \noindent\color{AccentColor}\rule{\textwidth}{1pt}\par
    \vspace{0.5cm}
    
    \begin{tabularx}{0.9\textwidth}{@{}>{\bfseries\sffamily}l X@{}}
    Autori & \sffamily Alessandro Frison \\
    \arrayrulecolor{MediumGray}\hline \\[-1.5ex]
    Verificatori & \sffamily Lorenzo Grolla\\
    \arrayrulecolor{MediumGray}\hline \\[-1.5ex] 
    Approvazione & \sffamily YYY\\ 
    \arrayrulecolor{MediumGray}\hline 
\end{tabularx}
    
    \vfill
\end{titlepage}

\newpage

% ====== TABELLA DI VERSIONAMENTO ======
{\normalfont\sffamily\huge\bfseries\color{AccentColor} Registro delle versioni}
\vspace{1cm}

\begin{center}
    \rowcolors{2}{LightGray}{white}
    \begin{tabular}{>{\centering\arraybackslash}m{2cm} >{\centering\arraybackslash}m{2cm} >{\raggedright\arraybackslash}m{2.5cm} >{\raggedright\arraybackslash}m{2.5cm} >{\raggedright\arraybackslash}m{6.5cm}}
        \rowcolor{AccentColor}
          \textcolor{white}{\textbf{Versione}} & 
          \textcolor{white}{\textbf{Data}} & 
          \multicolumn{1}{c}{\textcolor{white}{\textbf{Autore}}} &
          \multicolumn{1}{c}{\textcolor{white}{\textbf{Verificatore}}} & 
          \multicolumn{1}{c}{\textcolor{white}{\textbf{Descrizione delle modifiche}}} \\
        0.0.3 & 04/01/2025 & Alessandro Frison & Lorenzo Grolla & Aggiunta metriche di prodotto \\
        0.0.2 & 21/12/2025 & Lorenzo Grolla & Alessandro Frison & Aggiunta metriche processi \\ 
        0.0.1 & 01/12/2025 & Alessandro Frison & Lorenzo Grolla & Inizio stesura \\
    \end{tabular}
\end{center}


% ====== INDICE ======
\pagestyle{fancy}
\newpage
\tableofcontents    % Elenco dei contenuti
\newpage
\listoftables       % Elenco delle tabelle
\newpage
\listoffigures      % Elenco delle immagini
\newpage

% ====== INTRODUZIONE ======
\section{Introduzione}

\subsection{Scopo del documento}
Il Piano di Qualifica\textsuperscript{G} costituisce il riferimento principale per la gestione e il monitoraggio continuo della qualità del progetto software e dei processi coinvolti nel suo ciclo di vita. Esso definisce le strategie, gli standard e le metriche necessarie per assicurare che il software prodotto soddisfi pienamente i requisiti concordati e le aspettative dei committenti.

L'obiettivo del documento è stabilire un approccio sistematico che si sviluppa attraverso tre dimensioni interconnesse:

\begin{itemize}
    \item \textbf{Piano della Qualità}: definisce gli obiettivi qualitativi da perseguire, stabilisce gli standard di riferimento e delinea le politiche e le strategie necessarie per raggiungere l'eccellenza nel prodotto finale.
    \item \textbf{Controllo di Qualità}: implementa meccanismi di misurazione oggettivi per verificare la conformità ai requisiti. Attraverso l'uso di metriche predefinite, il gruppo monitora costantemente le prestazioni e lo stato di avanzamento, assicurando che le attività svolte siano allineate con quanto pianificato.
    \item \textbf{Miglioramento Continuo}: si basa sull'analisi periodica dei risultati ottenuti per identificare opportunità di ottimizzazione. Questo processo prevede l'adattamento costante dei processi e degli obiettivi per correggere eventuali deviazioni e migliorare l'efficienza complessiva.
\end{itemize}

Attraverso questo strumento strategico, il gruppo si assicura che il progetto rispetti integralmente i requisiti definiti, consegua gli obiettivi prefissati e mantenga elevati standard qualitativi. L'approccio metodologico adottato non configura la qualità come un elemento statico, bensì come un processo dinamico di apprendimento e perfezionamento continuo.

\subsection{Glossario}
Al fine di prevenire ambiguità e garantire una comunicazione uniforme e precisa tra i membri del gruppo e i committenti, è stato redatto un Glossario apposito.
Per facilitare la lettura del presente documento, i termini tecnici o dotati di un significato specifico all'interno del dominio di progetto sono contrassegnati da una lettera «G» posta in apice (es. Parola\textsuperscript{G}). La definizione estesa di tali termini è reperibile nel documento citato tra i riferimenti informativi.

\subsection{Maturità e miglioramenti}
La gestione della qualità adottata dal gruppo Byte Holders non è intesa come una verifica statica, bensì come un processo dinamico ed evolutivo.
La maturità del progetto viene monitorata attraverso l'analisi periodica delle metriche di processo e di prodotto; i dati raccolti permettono di individuare eventuali criticità organizzative o tecniche e di applicare tempestivamente contromisure mirate. Questo approccio iterativo garantisce un costante affinamento del \textit{Way of Working}, elevando progressivamente gli standard qualitativi con l'avanzare degli sprint.
\subsection{Riferimenti}

\subsubsection{Riferimenti normativi}
\begin{itemize}
    \item \textbf{Norme di Progetto ver. 1.0.0}\textsuperscript{G}: \\ \url{https://github.com/Byte-Holders/Documentazione/blob/main/RTB/Norme_Di_Progetto.pdf}.
    \item \textbf{Capitolato d'appalto C2 - Code Guardian}: \\ \url{https://www.math.unipd.it/~tullio/IS-1/2025/Progetto/C2.pdf}.
\end{itemize}

\subsubsection{Riferimenti informativi}
\begin{itemize}
    \item \textbf{Glossario}: \\ \url{https://github.com/Byte-Holders/Documentazione/blob/main/RTB/Glossario.pdf}.
    \item \textbf{Standard ISO/IEC 9126}: \\ \url{https://it.wikipedia.org/wiki/ISO/IEC_9126}.
    \item \textbf{Standard ISO/IEC 12207:1995}: \\ \url{https://www.math.unipd.it/~tullio/IS-1/2009/Approfondimenti/ISO_12207-1995.pdf}.
\end{itemize}


% ====== QUALITÀ DI PROCESSO ======
\newpage
\section{Qualità di processo}
La qualità di processo costituisce un requisito fondamentale per il successo di un progetto software. Essa garantisce che i processi adottati siano efficaci, efficienti e conformi agli standard di qualità stabiliti.  
Per assicurare tale qualità, il progetto si avvale dei seguenti strumenti e metodologie:

\begin{itemize}
    \item \textbf{Modelli di riferimento:} vengono utilizzati il \textit{Capability Maturity Model Integration (CMMI)} e la norma \textit{ISO/IEC 12207}, che forniscono linee guida per la definizione, la gestione e il miglioramento dei processi software;
    
    \item \textbf{Metriche di processo:} consentono di valutare le prestazioni e l’efficienza dei processi adottati. Per ciascuna metrica sono definite soglie quantitative che rappresentano i livelli minimi accettabili di qualità;
    
    \item \textbf{Revisioni periodiche:} comprendono sessioni di verifica e controllo mirate ad analizzare i risultati ottenuti, confrontandoli con gli obiettivi predefiniti, al fine di individuare eventuali deviazioni e applicare azioni correttive.
\end{itemize}

\subsection{Processi Primari}

\subsubsection{Fornitura}
In questa sezione vengono descritte le metriche utilizzate per monitorare l'efficienza economica e temporale del progetto.

\begin{table}[h!]
    \centering
    \rowcolors{2}{LightGray}{white}
    \begin{tabular}{c p{7cm} c c}
        \rowcolor{AccentColor}
        \textcolor{white}{\textbf{ID}} & \textcolor{white}{\textbf{Nome}} & \textcolor{white}{\textbf{Accettabile}} & \textcolor{white}{\textbf{Ottimo}} \\
        
        MPC-01 & \textbf{Earned Value (EV)} & $\geq 0$ & $\leq EAC$ \\
        MPC-02 & \textbf{Planned Value (PV)} & $\geq 0$ & $\leq BAC$ \\
        MPC-03 & \textbf{Actual Cost (AC)} & $\geq 0$ & $\leq EAC$ \\
        MPC-04 & \textbf{Cost Performance Index (CPI)} & $\geq 0.90$ & $\geq 1.00$ \\
        MPC-05 & \textbf{Schedule Performance Index (SPI)} & $\geq 0.90$ & $\geq 1.00$ \\
        MPC-06 & \textbf{Estimate At Completion (EAC)} & $\geq 0$ & $\leq BAC$ \\
        MPC-07 & \textbf{Estimate To Complete (ETC)} & $\geq 0$ & $\leq BAC$ \\

    \end{tabular}
    \caption{Metriche di Processo - Fornitura}
\end{table}

\subsubsection{Sviluppo}
Queste metriche mirano a monitorare la stabilità dei requisiti durante la fase di analisi e progettazione.

\begin{table}[h!]
    \centering
    \rowcolors{2}{LightGray}{white}
    \begin{tabular}{c p{7cm} c c}
        \rowcolor{AccentColor}
        \textcolor{white}{\textbf{ID}} & \textcolor{white}{\textbf{Nome}} & \textcolor{white}{\textbf{Accettabile}} & \textcolor{white}{\textbf{Ottimo}} \\
        
        MPC-08 & \textbf{Requirements Stability Index (RSI)} & $\geq 70\%$ & $100\%$ \\
    \end{tabular}
    \caption{Metriche di Processo - Sviluppo}
\end{table}

\subsection{Processi di Supporto}
\subsubsection{Documentazione}
Metriche volte a garantire la leggibilità e la correttezza formale della documentazione prodotta.

\begin{table}[h!]
    \centering
    \rowcolors{2}{LightGray}{white}
    \begin{tabular}{c p{7cm} c c}
        \rowcolor{AccentColor}
        \textcolor{white}{\textbf{ID}} & \textcolor{white}{\textbf{Nome}} & \textcolor{white}{\textbf{Accettabile}} & \textcolor{white}{\textbf{Ottimo}} \\
        
        MPC-09 & \textbf{Indice di Gulpease} & $\geq 40$ & $\geq 60$ \\
        MPC-10 & \textbf{Correttezza ortografica (Errori)} & 0 & 0 \\
    \end{tabular}
    \caption{Metriche di Processo - Documentazione}
\end{table}

\subsubsection{Verifica}

\begin{table}[h!]
    \centering
    \rowcolors{2}{LightGray}{white}
    \begin{tabular}{c p{7cm} c c}
        \rowcolor{AccentColor}
        \textcolor{white}{\textbf{ID}} & \textcolor{white}{\textbf{Nome}} & \textcolor{white}{\textbf{Accettabile}} & \textcolor{white}{\textbf{Ottimo}} \\
        
        MPC-11 & \textbf{Code Coverage} & $\geq 70\%$ & $\geq 80\%$ \\
        MPC-12 & \textbf{Test Success Rate} & $100\%$ & $100\%$ \\
    \end{tabular}
    \caption{Metriche di Processo - Verifica}
\end{table}

\subsubsection{Gestione della qualità}
\begin{table}[h!]
    \centering
    \rowcolors{2}{LightGray}{white}
    \begin{tabular}{c p{7cm} c c}
        \rowcolor{AccentColor}
        \textcolor{white}{\textbf{ID}} & \textcolor{white}{\textbf{Nome}} & \textcolor{white}{\textbf{Accettabile}} & \textcolor{white}{\textbf{Ottimo}} \\
        
        MPC-13 & \textbf{Quality metrics satisfied} & $\geq 80\%$ & $\geq 100\%$ \\
    \end{tabular}
    \caption{Metriche di Processo - Gestione della qualità}
\end{table}

\subsection{Processi Organizzativi}
Attività per gestire l'infrastruttura e le risorse umane
\begin{table}[h!]
    \centering
    \rowcolors{2}{LightGray}{white}
    \begin{tabular}{c p{7cm} c c}
        \rowcolor{AccentColor}
        \textcolor{white}{\textbf{ID}} & \textcolor{white}{\textbf{Nome}} & \textcolor{white}{\textbf{Accettabile}} & \textcolor{white}{\textbf{Ottimo}} \\
        
        MPC-14 & \textbf{Time Efficency} & $\geq 60\%$ & $\geq 100\%$ \\
        MPC-15 & \textbf{Sprint Goal Achievement} & $\geq 80\%$ & $\geq 100\%$ \\


    \end{tabular}
    \caption{Processi Organizzativi - Gestione dei processi}
\end{table}

\newpage 

% ====== QUALITÀ DI PRODOTTO ======
\section{Qualità di prodotto}
L'obiettivo cardine dello sviluppo software risiede nella qualità di prodotto, intesa come l'attitudine del sistema a conformarsi ai requisiti e alle aspettative di utenti e stakeholder. Tale proprietà non è un attributo isolato, ma deriva direttamente dal rigore e dalla qualità dei processi implementati lungo l'intero ciclo di vita del progetto. \\
Un software si definisce di elevata qualità quando soddisfa i seguenti pilastri metodologici:
\begin{itemize}
    \item \textbf{Conformità Funzionale}: il prodotto aderisce rigorosamente ai requisiti funzionali e non funzionali specificati nel documento di Analisi dei Requisiti v1.1.0.
    \item \textbf{Affidabilità}: il sistema è in grado di operare in modo costante e resiliente, minimizzando i guasti e garantendo l'integrità dei dati nel tempo.
    \item \textbf{Usabilità}: l'interfaccia e l'esperienza d'uso sono progettate per essere intuitive, permettendo all'utente di raggiungere i propri obiettivi con il minimo sforzo cognitivo.
    \item \textbf{Efficienza}: le risorse computazionali sono ottimizzate per garantire tempi di risposta rapidi e un consumo energetico o di memoria proporzionato al carico di lavoro.
    \item \textbf{Manutenibilità}: l'architettura è modulare e ben documentata, facilitando interventi correttivi o evolutivi senza introdurre instabilità nel sistema.
\end{itemize}

\subsection{Funzionalità}
\begin{table}[h!]
    \centering
    \rowcolors{2}{LightGray}{white}
    \begin{tabular}{c p{7cm} c c}
        \rowcolor{AccentColor}
        \textcolor{white}{\textbf{ID}} & \textcolor{white}{\textbf{Nome}} & \textcolor{white}{\textbf{Accettabile}} & \textcolor{white}{\textbf{Ottimo}} \\
        MPD-01 & \textbf{Requisiti Obbligatori Soddisfatti} & $\geq 100\%$ & $100\%$ \\
        MPD-02 & \textbf{Requisiti Desiderabili Soddisfatti} & $\geq 0\%$ & $100\%$ \\
        MPD-03 & \textbf{Requisiti Opzionali Soddisfatti} & $\geq 0\%$ & $100\%$ \\
        MPD-04 & \textbf{Function Point} & da determinare & da determinare \\
    \end{tabular}
    \caption{Metriche di Prodotto - Funzionalità}
\end{table}

\subsection{Affidabilità}
\begin{table}[h!]
    \centering
    \rowcolors{2}{LightGray}{white}
    \begin{tabular}{c p{7cm} c c}
        \rowcolor{AccentColor}
        \textcolor{white}{\textbf{ID}} & \textcolor{white}{\textbf{Nome}} & \textcolor{white}{\textbf{Accettabile}} & \textcolor{white}{\textbf{Ottimo}} \\
        MPD-05 & \textbf{Code Coverage} & $\geq 80\%$ & $100\%$ \\
        MPD-06 & \textbf{Statement Coverage} & $\geq 80\%$ & $100\%$ \\
        MPD-07 & \textbf{Branch Coverage} & $\geq 70\%$ & $100\%$ \\
        MPD-08 & \textbf{Condition Coverage } & $\geq 60\%$ & $100\%$ \\
    \end{tabular}
    \caption{Metriche di Prodotto - Affidabilità}
\end{table}

\newpage

\subsection{Efficienza}
\begin{table}[h!]
    \centering
    \rowcolors{2}{LightGray}{white}
    \begin{tabular}{c p{7cm} c c}
        \rowcolor{AccentColor}
        \textcolor{white}{\textbf{ID}} & \textcolor{white}{\textbf{Nome}} & \textcolor{white}{\textbf{Accettabile}} & \textcolor{white}{\textbf{Ottimo}} \\
        MPD-09 & \textbf{Response Time} & $\leq 10s$ & $\leq 4s$ \\
    \end{tabular}
    \caption{Metriche di Prodotto - Efficienza}
\end{table}

\subsection{Usabilità}
\begin{table}[h!]
    \centering
    \rowcolors{2}{LightGray}{white}
    \begin{tabular}{c p{7cm} c c}
        \rowcolor{AccentColor}
        \textcolor{white}{\textbf{ID}} & \textcolor{white}{\textbf{Nome}} & \textcolor{white}{\textbf{Accettabile}} & \textcolor{white}{\textbf{Ottimo}} \\
        MPD-10 & \textbf{Facilità di utilizzo} & $\geq 7$ click & $\geq 5$ click\\
        MPD-11 & \textbf{Tempo medio di apprendimento} & $\geq 5$ minuti & $\geq 2$ minuti\\
    \end{tabular}
    \caption{Metriche di Prodotto - Usabilità}
\end{table}

\subsection{Manutenibilità}
\begin{table}[h!]
    \centering
    \rowcolors{2}{LightGray}{white}
    \begin{tabular}{c p{7cm} c c}
        \rowcolor{AccentColor}
        \textcolor{white}{\textbf{ID}} & \textcolor{white}{\textbf{Nome}} & \textcolor{white}{\textbf{Accettabile}} & \textcolor{white}{\textbf{Ottimo}} \\
        MPD-12 & \textbf{Accoppiamento Moduli} & $\leq 4$ & $\leq 2$ \\
        MPD-13 & \textbf{Linee per Metodo} & $\leq 30$ & $\leq 15$ \\
        MPD-14 & \textbf{Parametri per Metodo} & $\leq 4$ & $\leq 2$ \\
        MPD-15 & \textbf{Attributi per Classe} & $\leq 7$ & $\leq 5$ \\
        MPD-16 & \textbf{Structure Fan-In} & - & massimizzato \\
        MPD-17 & \textbf{Structure Fan-Out} & - & minimizzato \\
    \end{tabular}
    \caption{Metriche di Prodotto - Manutenibilità}
\end{table}

\newpage

% ====== METODI DI TESTING ======
\section{Metodi di testing}

% ====== CRUSCOTTO DI VALUTAZIONE ======
\section{Cruscotto di Valutazione}

% ====== INIZIATIVE DI AUTOMIGLIORAMENTO ======
\section{Iniziative di automiglioramento}

\end{document}
