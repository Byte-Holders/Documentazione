\documentclass[a4paper, 11pt]{article}

% ====== PACCHETTI NECESSARI ======
\usepackage[utf8]{inputenc}
\usepackage[T1]{fontenc}
\usepackage[italian]{babel}
\usepackage{geometry}
\usepackage{graphicx}
\usepackage[table]{xcolor}
\usepackage{tabularx}
\usepackage{array}
\usepackage{amssymb}
\usepackage{fancyhdr}
\setlength{\headheight}{14pt}
\usepackage{titlesec}
\usepackage{helvet}
\renewcommand{\familydefault}{\sfdefault}
\usepackage{lipsum}
\usepackage{hyperref}
\usepackage{booktabs}
\usepackage{enumitem}
\usepackage[utf8]{inputenc} % Specifica la codifica del file (necessaria per le accentate)
\usepackage[T1]{fontenc}    % Migliora l'output dei font per le lingue europee

% ====== IMPOSTAZIONI GLOBALI DI STILE ======

% 1. DEFINIZIONE COLORI BLU-VIOLA
\definecolor{AccentColor}{RGB}{80, 90, 180} % Blu-viola principale
\definecolor{AccentLight}{RGB}{80, 90, 180} % Versione più chiara
\definecolor{AccentDark}{RGB}{50, 60, 140} % Versione più scura
\definecolor{LightGray}{RGB}{245, 245, 250}
\definecolor{MediumGray}{RGB}{200, 200, 210}

% 2. IMPOSTAZIONE MARGINI
\geometry{a4paper, left=2.5cm, right=2.5cm, top=3.5cm, bottom=3.5cm}

% 3. STILE DEI TITOLI DI SEZIONE
\titleformat{\section}
  {\normalfont\sffamily\Large\bfseries\color{AccentColor}}
  {\thesection}
  {1em}
  {}
\titleformat{\subsection}
  {\normalfont\sffamily\large\bfseries\color{AccentDark}}
  {\thesubsection}
  {1em}
  {}

% 4. IMPOSTAZIONE HEADER E FOOTER
\pagestyle{fancy}
\fancyhf{} 
\fancyhead[L]{\sffamily\bfseries\color{AccentColor}\@BYTE HOLDERS}
\fancyhead[R]{\sffamily\color{AccentColor}\thepage}
\renewcommand{\headrulewidth}{0.8pt}
\renewcommand{\headrule}{\color{AccentColor}\hrule width\headwidth height\headrulewidth \vskip-\headrulewidth}

% 5. IMPOSTAZIONE LINK
\hypersetup{
    colorlinks=true,
    linkcolor=AccentColor,
    urlcolor=AccentLight,
    citecolor=AccentDark,
}

% 6. PERSONALIZZAZIONE ELENCHI
\setlist[itemize]{itemsep=2pt, topsep=4pt}
\setlist[enumerate]{itemsep=2pt, topsep=4pt}

% ====== COMANDI PERSONALIZZATI ======
\makeatletter
\newcommand{\NomeGruppo}[1]{\def\@NomeGruppo{#1}}
\newcommand{\Sommario}[1]{\def\@Sommario{#1}}
\newcommand{\Autore}[1]{\def\@Autore{#1}}
\newcommand{\Verificatore}[1]{\def\@Verificatore{#1}}
\makeatother

% ====== STILE TABELLE MIGLIORATO ======
\newcolumntype{Y}{>{\raggedright\arraybackslash}X} % Colonna giustificata a sinistra
\setlength{\arrayrulewidth}{0.4pt} % Linee più sottili
\setlength{\tabcolsep}{10pt} % Spaziatura interna celle
\renewcommand{\arraystretch}{1.4} % Altezza righe

% ====== INIZIO DEL DOCUMENTO ======
\begin{document}

% ====== INFORMAZIONI PER LA PAGINA DI TITOLO ======
\NomeGruppo{BYTE HOLDERS}
\Sommario{Questo documento descrive il piano di qualifica del progetto, delineando le strategie e le metodologie che verranno adottate per garantire la qualità del prodotto finale.}
\Autore{}
\Verificatore{}

\pagestyle{empty}

% ====== PAGINA DI TITOLO ======
\begin{titlepage}
    \centering
    
    \includegraphics[width=0.55\textwidth]{../Assets/ByteHolders1.png}\par\vspace{1.5cm}
    
    {\LARGE \sffamily \color{AccentColor}\bfseries Piano di Qualifica}\par
    
    \vfill
    
    \noindent\color{AccentColor}\rule{\textwidth}{1pt}\par
    \vspace{0.5cm}
    
    \begin{tabularx}{0.9\textwidth}{@{}>{\bfseries\sffamily}l X@{}}
    Autori & \sffamily Alessandro Frison, Lorenzo Grolla\\
    \arrayrulecolor{MediumGray}\hline \\[-1.5ex]
    Verificatori & \sffamily XXX\\
    \arrayrulecolor{MediumGray}\hline \\[-1.5ex] 
    Approvazione & \sffamily YYY\\ 
    \arrayrulecolor{MediumGray}\hline 
\end{tabularx}
    
    \vfill
\end{titlepage}

\newpage

% ====== TABELLA DI VERSIONAMENTO ======
{\normalfont\sffamily\huge\bfseries\color{AccentColor} Registro delle versioni}
\vspace{1cm}

\begin{center}
    \rowcolors{2}{LightGray}{white}
    \begin{tabular}{>{\centering\arraybackslash}m{2cm} >{\centering\arraybackslash}m{2cm} >{\raggedright\arraybackslash}m{2.5cm} >{\raggedright\arraybackslash}m{6.5cm}}
        \rowcolor{AccentColor}
          \textcolor{white}{\textbf{Versione}} & 
          \textcolor{white}{\textbf{Data}} & 
          \multicolumn{1}{c}{\textcolor{white}{\textbf{Autore}}} & 
          \multicolumn{1}{c}{\textcolor{white}{\textbf{Descrizione delle modifiche}}} \\
        0.0.1 & 01/12/2025 & Alessandro Frison & Inizio stesura \\ 
    \end{tabular}
\end{center}


% ====== INDICE ======
\pagestyle{fancy}
\newpage
\tableofcontents
\newpage

\section{Introduzione}

\subsection{Scopo del documento}
\subsection{Glossario}
\subsection{Maturià e miglioramenti}
\subsection{Riferimenti}
\subsubsection{Riferimenti normativi}
\subsubsection{Riferimenti informativi}

\section{Piano di qualità}

\section{Strategie di test}

\section{Cruscotto di qualità}

\section{Valutazioni per il miglioramento}

\end{document}