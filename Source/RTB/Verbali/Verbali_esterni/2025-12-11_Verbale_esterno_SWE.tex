\documentclass[a4paper, 11pt]{article}

% ====== PACCHETTI NECESSARI ======
\usepackage[utf8]{inputenc}
\usepackage[T1]{fontenc}
\usepackage[italian]{babel}
\usepackage{geometry}
\usepackage{graphicx}
\usepackage[table]{xcolor}
\usepackage{tabularx}
\usepackage{array}
\usepackage{amssymb}
\usepackage{fancyhdr}
\setlength{\headheight}{14pt}
\usepackage{titlesec}
\usepackage{helvet}
\renewcommand{\familydefault}{\sfdefault}
\usepackage{lipsum}
\usepackage{hyperref}
\usepackage{booktabs}
\usepackage{enumitem}
\usepackage[utf8]{inputenc} % Specifica la codifica del file (necessaria per le accentate)
\usepackage[T1]{fontenc}    % Migliora l'output dei font per le lingue europee

% ====== IMPOSTAZIONI GLOBALI DI STILE ======

% 1. DEFINIZIONE COLORI BLU-VIOLA
\definecolor{AccentColor}{RGB}{80, 90, 180} % Blu-viola principale
\definecolor{AccentLight}{RGB}{80, 90, 180} % Versione più chiara
\definecolor{AccentDark}{RGB}{50, 60, 140} % Versione più scura
\definecolor{LightGray}{RGB}{245, 245, 250}
\definecolor{MediumGray}{RGB}{200, 200, 210}

% 2. IMPOSTAZIONE MARGINI
\geometry{a4paper, left=2.5cm, right=2.5cm, top=3.5cm, bottom=3.5cm}

% 3. STILE DEI TITOLI DI SEZIONE
\titleformat{\section}
  {\normalfont\sffamily\Large\bfseries\color{AccentColor}}
  {\thesection}
  {1em}
  {}
\titleformat{\subsection}
  {\normalfont\sffamily\large\bfseries\color{AccentDark}}
  {\thesubsection}
  {1em}
  {}
\titleformat{\subsubsection}
    {\normalfont\sffamily\large\bfseries\color{AccentDark}}
    {\thesubsubsection}
    {1em}
    {}


% 4. IMPOSTAZIONE HEADER E FOOTER
\pagestyle{fancy}
\fancyhf{}
\fancyhead[L]{\sffamily\bfseries\color{AccentColor}\@BYTE HOLDERS}
\fancyhead[R]{\sffamily\color{AccentColor}\thepage}
\renewcommand{\headrulewidth}{0.8pt}
\renewcommand{\headrule}{\color{AccentColor}\hrule width\headwidth height\headrulewidth \vskip-\headrulewidth}

% 5. IMPOSTAZIONE LINK
\hypersetup{
    colorlinks=true,
    linkcolor=AccentColor,
    urlcolor=AccentLight,
    citecolor=AccentDark,
}

% 6. PERSONALIZZAZIONE ELENCHI
\setlist[itemize]{itemsep=2pt, topsep=4pt}
\setlist[enumerate]{itemsep=2pt, topsep=4pt}

% ====== COMANDI PERSONALIZZATI ======
\makeatletter
\newcommand{\NomeGruppo}[1]{\def\@NomeGruppo{#1}}
\newcommand{\TitoloVerbale}[1]{\def\@TitoloVerbale{#1}}
\newcommand{\Sommario}[1]{\def\@Sommario{#1}}
\newcommand{\Autore}[1]{\def\@Autore{#1}}
\newcommand{\Verificatore}[1]{\def\@Verificatore{#1}}
\makeatother

% ====== STILE TABELLE MIGLIORATO ======
\newcolumntype{Y}{>{\raggedright\arraybackslash}X} % Colonna giustificata a sinistra
\setlength{\arrayrulewidth}{0.4pt} % Linee più sottili
\setlength{\tabcolsep}{10pt} % Spaziatura interna celle
\renewcommand{\arraystretch}{1.4} % Altezza righe

% ====== INIZIO DEL DOCUMENTO ======
\begin{document}

% ====== INFORMAZIONI PER LA PAGINA DI TITOLO ======
\NomeGruppo{BYTE HOLDERS}
\TitoloVerbale{Verbale riunione esterna}
\Sommario{Questo Verbale documenta la riunione esterna avvenuta l'11 dicembre 2025}
\Autore{}
\Verificatore{}

\pagestyle{empty}

% ====== PAGINA DI TITOLO ======
\begin{titlepage}
    \centering
    
    \includegraphics[width=0.55\textwidth]{../../../Assets/ByteHolders1.png}\par\vspace{1.5cm}
    
    {\LARGE \sffamily \color{AccentColor}\bfseries Verbale esterno}\par
    \vspace{0.5cm}
    {\large \color{AccentColor}\sffamily 11 dicembre 2025}\par
    
    \vfill
    
    \noindent\color{AccentColor}\rule{\textwidth}{1pt}\par
    \vspace{0.5cm}
    
    \begin{tabularx}{0.9\textwidth}{@{}>{\bfseries\sffamily}l X@{}}
    Autore & \sffamily Damiano Berti\\
    \arrayrulecolor{MediumGray}\hline \\[-1.5ex]
    Verificatore & \sffamily Alessandro Frison\\
    \arrayrulecolor{MediumGray}\hline \\[-1.5ex] 
    Approvazione & \sffamily Giulia Romanato\\ 
    \arrayrulecolor{MediumGray}\hline 
\end{tabularx}
    
    \vfill
\end{titlepage}

% ====== INDICE ======
\pagestyle{fancy}
\newpage
\tableofcontents
\newpage

% ====== TABELLA DI VERSIONAMENTO ======
\section{Registro delle versioni}
\begin{center}
    \rowcolors{2}{LightGray}{white}
    \begin{tabular}{>{\centering\arraybackslash}m{2cm} >{\centering\arraybackslash}m{2cm} >{\raggedright\arraybackslash}m{2.5cm} >{\raggedright\arraybackslash}m{6.5cm}}
        \rowcolor{AccentColor}
          \textcolor{white}{\textbf{Versione}} & 
          \textcolor{white}{\textbf{Data}} & 
          \multicolumn{1}{c}{\textcolor{white}{\textbf{Autore}}} & 
          \multicolumn{1}{c}{\textcolor{white}{\textbf{Descrizione delle modifiche}}} \\
        1.0.0 & 15/12/2025 & Giulia Romanato & Approvazione del verbale \\
        0.1.0 & 14/12/2025 & Alessandro Frison & Verifica del verbale \\
        0.0.1 & 12/12/2025 & Damiano Berti & Prima stesura del verbale \\
    \end{tabular}
\end{center}

%\vspace{1cm}

% ====== SEZIONE INFORMAZIONI INTRODUTTIVE ======
\section{Informazioni introduttive}

\subsection{Durata e luogo}
\begin{itemize}
    \item \textbf{Inizio:} 17:00
    \item \textbf{Fine:} 17:55
    \item \textbf{Luogo:} Meeting Microsoft Teams
\end{itemize}

% ====== TABELLA PRESENZE ======
\subsection{Partecipanti}
\begin{center}
    \rowcolors{2}{LightGray}{white}
    \begin{tabular}{>{\raggedright\arraybackslash}p{6cm} c c}
        \rowcolor{AccentColor}
        \textcolor{white}{\textbf{Nome e Cognome}} & 
        \textcolor{white}{\textbf{Presente}} & 
        \textcolor{white}{\textbf{Assente}} \\
        
        Damiano Berti     & \textcolor{AccentColor}{$\blacksquare$}    & $\square$        \\
        Alessandro Frison     & \textcolor{AccentColor}{$\blacksquare$}    & $\square$        \\
        Lorenzo Grolla     & \textcolor{AccentColor}{$\blacksquare$}    & $\square$        \\
        Nicolò Lattanzio    &   $\square$   & \textcolor{AccentColor}{$\blacksquare$}      \\
        Alessandro Morabito   & \textcolor{AccentColor}{$\blacksquare$}    & $\square$        \\
        Giacomo Nalotto   & \textcolor{AccentColor}{$\blacksquare$}    & $\square$        \\
        Giulia Romanato   & \textcolor{AccentColor}{$\blacksquare$}    & $\square$        \\
    \end{tabular}
\end{center}

% ====== SEZIONE PRINCIPALE DEL VERBALE ======
\section{Contenuto della riunione}

\subsection{Ordine del giorno}
\begin{enumerate}
    \item Discussione con l'azienda proponente in merito alle funzionalità del programma Code Guardian
\end{enumerate}

\newpage

\subsection{Riassunto della discussione}
La riunione si è da subito focalizzata sull'esposizione all'azienda proponente delle funzionalità e del flusso di utilizzo del programma che il gruppo ha definito dopo una prima fase di analisi dei requisiti. I punti di discussione principale sono riportati nella sezione: \hyperref[punti-discussione]{Altri punti di discussione}, e tra questi si è distinto un \hyperref[problema]{punto problematico} che ha reso necessario ripensare una parte del programma.

\subsection{Gestione degli accessi e delle repo} \label{problema}
L'utilizzo delle \textbf{organizzazioni} di GitHub per gestire gruppi di repo, ruoli e permessi è troppo limitante, inoltre i ruoli delle organizzazioni interni al programma potrebbero andare in conflitto con quelli delle organizzazioni di GitHub.
\\[0.5cm]
La soluzione trovata è quella di utilizzare dei \textbf{workspace} creati e gestiti all'interno del programma. Questo comporta diversi vantaggi tra cui:
\begin{itemize}
    \item non essere più limitati dai ruoli presenti su GitHub;
    \item semplificazione della gestione dei ruoli e dei permessi;
    \item la possibilità per chi crea il workspace di diventarne l'owner.
\end{itemize}
\subsection{Altri punti di discussione} \label{punti-discussione}

\subsubsection{Scansioni parziali e storico}
Discussione in merito alla gestione delle scansioni, con particolare attenzione alle scansioni parziali, ovvero scansioni dedicate solo a una o più sezioni (es: sicurezza, test...) e su come gestirne lo storico.

\subsubsection{Pull request} 
È stata presentata all'azienda l'idea di includere nel programma una sezione dedicata alla visione e/o gestione delle pull request.
A seguito di discussione abbiamo concluso che questa funzionalità potrebbe portare alla complicazione dell'alberatura del programma.

\subsubsection{Gestione branch} 
Si è discusso di come il programma debba gestire i vari branch di una repo. La conclusione è che il programma debba permettere di selezionare il branch su cui eseguire la scansione per ogni repo.

\subsubsection{Ruoli di un workspace}
Si è discusso sulle differenze dei vari ruoli, e su come una personalizzazione complessa per ogni ruolo possa complicare il progetto.

\subsubsection{Qualità del codice}
Abbiamo esposto come si presenterà la sezione: "qualità del codice" relativa ad un report di una scansione a cui è seguita una discussione per aggiungere dati significativi andando a toglierne altri meno utili.

\subsubsection{Visione aggregata} 
Dopo un'attenta analisi su come gestire la visione aggregata, abbiamo concluso che: 
\begin{itemize}
    \item non deve essere una visuale molto specifica;
    \item sia piuttosto facile da leggere;
    \item che comunichi al volo l'andamento della repo mediante grafici.
\end{itemize}

\subsubsection{Valutazione documentazione} \label{metriche-documentazione}
Si è discusso su come effettuare la valutazione della documentazione. Sono stati individuati 2 importanti punti di valutazione: 
\begin{itemize}
    \item valutazione tramite LLM della coerenza tra la documentazione e il codice sorgente;
    \item analisi di quante funzioni pubbliche hanno la documentazione nel codice.
\end{itemize}

\subsubsection{Tag delle repo} \label{tags}
Durante una precedente analisi abbiamo pensato di utilizzare un sistema di tag delle repo per raggrupparle secondo una specifica semantica.
Dopo averne parlato con Var Group abbiamo individuato 2 tipologie di tag:
\begin{enumerate}
    \item tag gestito dall'utente per categorizzare le raccolte arbitrariamente;
    \item tag preso in automatico da GitHub per taggare le repo in base al linguaggio.
\end{enumerate}

\subsubsection{Login} 
Si è infine discusso in merito al modo ottimale di gestire il login nel programma. 
In particolare abbiamo vagliato la possibilità di eseguire il login tramite GitHub, senza l'appoggio di strumenti esterni o con un'approccio misto.



% ====== SEZIONE DECISIONI E AZIONI ======
\section{Decisioni e azioni}
\begin{center}
    \rowcolors{2}{LightGray}{white}
    \begin{tabular}{>{\centering\arraybackslash}m{3cm} >{\raggedright\arraybackslash}p{7cm} >{\centering\arraybackslash}p{2.5cm}}
        \rowcolor{AccentColor}
        \textcolor{white}{\textbf{Codice}} & 
        \multicolumn{1}{c}{\textcolor{white}{\textbf{Descrizione}}} & 
        \textcolor{white}{\textbf{Assegnatario}} \\
        DEC-RTB-016 & Abbandonata l'idea di utilizzare l'importazione automatica e le organizzazioni di GitHub e sistituita con dei workspace &  \\
        DEC-RTB-017 & Il programma non prevedrà scansioni parziali &  \\
        DEC-RTB-018 & Rimandata la discussione su come trattare le pull request &  \\
        DEC-RTB-019 & La visione aggregata non prevedrà informazioni troppo specifiche e sarà tenuta semplice  &  \\
        DEC-RTB-020 & Verranno usati \hyperref[metriche-documentazione]{questi 2 punti} per la valutazione della documentazione &  \\
        DEC-RTB-021 & Verrano utilizzati \hyperref[tags]{2 tipi di tag} per le repo &  \\
        DEC-RTB-022 & Il programma dovrà permettere di selezionare il branch su cui eseguire la scansione per ogni repo &  \\

        
        \midrule
    \end{tabular}

    \vspace{1.5cm}
    \rule{6cm}{0.4pt}\\
    \textit{Firma del proponente}

\end{center}

\end{document}