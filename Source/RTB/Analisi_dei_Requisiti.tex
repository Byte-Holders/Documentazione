\documentclass[a4paper, 11pt]{article}

% ====== PACCHETTI NECESSARI ======
\usepackage[utf8]{inputenc}
\usepackage[T1]{fontenc}
\usepackage[italian]{babel}
\usepackage{geometry}
\usepackage{graphicx}
\usepackage[table]{xcolor}
\usepackage{tabularx}
\usepackage{array}
\usepackage{amssymb}
\usepackage{fancyhdr}
\setlength{\headheight}{14pt}
\usepackage{titlesec}
\usepackage{helvet}
\renewcommand{\familydefault}{\sfdefault}
\usepackage{lipsum}
\usepackage{hyperref}
\usepackage{float}
\usepackage{booktabs}
\usepackage{enumitem}
\usepackage{titlecaps}
\usepackage[utf8]{inputenc} % Specifica la codifica del file (necessaria per le accentate)
\usepackage[T1]{fontenc}    % Migliora l'output dei font per le lingue europee


\setcounter{secnumdepth}{5} % Numera fino ai sottoparagrafi (livello 5)
\setcounter{tocdepth}{5}    % Mostra nell'indice fino ai sottoparagrafi (livello 5)
% ====== IMPOSTAZIONI GLOBALI DI STILE ======

\makeatletter

% ====== PARAGRAPH come subsubsection ======
\renewcommand\paragraph{\@startsection{paragraph}{4}
  {0pt}
  {2.5ex plus 1ex minus .2ex}
  {1.5ex plus .2ex}
  {\normalfont\normalsize\bfseries\color{AccentDark}}}

% ====== SUBPARAGRAPH come paragraph ======
\renewcommand\subparagraph{\@startsection{subparagraph}{5}
  {0pt}
  {2ex plus 0.8ex minus .2ex}
  {1ex plus .2ex}
  {\normalfont\normalsize\bfseries\color{AccentColor}}}

\makeatother


% 1. DEFINIZIONE COLORI BLU-VIOLA
\definecolor{AccentColor}{RGB}{80, 90, 180} % Blu-viola principale
\definecolor{AccentLight}{RGB}{80, 90, 180} % Versione più chiara
\definecolor{AccentDark}{RGB}{50, 60, 140} % Versione più scura
\definecolor{LightGray}{RGB}{245, 245, 250}
\definecolor{MediumGray}{RGB}{200, 200, 210}

% 2. IMPOSTAZIONE MARGINI
\geometry{a4paper, left=2.5cm, right=2.5cm, top=3.5cm, bottom=3.5cm}

% 3. STILE DEI TITOLI DI SEZIONE
\titleformat{\section}
  {\normalfont\sffamily\Large\bfseries\color{AccentColor}}
  {\thesection}
  {1em}
  {}
\titleformat{\subsection}
  {\normalfont\sffamily\large\bfseries\color{AccentDark}}
  {\thesubsection}
  {1em}
  {}

% 4. IMPOSTAZIONE HEADER E FOOTER
\pagestyle{fancy}
\fancyhf{} 
\fancyhead[L]{\sffamily\bfseries\color{AccentColor}@BYTE HOLDERS}
\fancyhead[R]{\sffamily\color{AccentColor}\thepage}
\renewcommand{\headrulewidth}{0.8pt}
\renewcommand{\headrule}{\color{AccentColor}\hrule width\headwidth height\headrulewidth \vskip-\headrulewidth}

% 5. IMPOSTAZIONE LINK
\hypersetup{
	colorlinks=true,
	linkcolor=AccentColor,
	urlcolor=AccentLight,
	citecolor=AccentDark,
}

% 6. PERSONALIZZAZIONE ELENCHI
\setlist[itemize]{itemsep=2pt, topsep=4pt}
\setlist[enumerate]{itemsep=2pt, topsep=4pt}

% ====== COMANDI PERSONALIZZATI ======
\newcommand{\refterm}[1]{\textit{#1}\textsuperscript{G}}
\newcommand{\term}[1]{\subsubsection{\refterm{#1}}}

% ====== STILE TABELLE MIGLIORATO ======
\newcolumntype{Y}{>{\raggedright\arraybackslash}X} % Colonna giustificata a sinistra
\setlength{\arrayrulewidth}{0.4pt} % Linee più sottili
\setlength{\tabcolsep}{10pt} % Spaziatura interna celle
\renewcommand{\arraystretch}{1.4} % Altezza righe

% ====== INIZIO DEL DOCUMENTO ======
\begin{document}
	

\pagestyle{empty}

% ====== PAGINA DI TITOLO ======
\begin{titlepage}
	\begin{center}
		\includegraphics[width=0.55\textwidth]{../Assets/ByteHolders1.png}\vspace{1.5cm}

		{\LARGE \sffamily \color{AccentColor}\bfseries Analisi dei Requisiti}
	\end{center}
	
	\vfill
	\rule{\textwidth}{1pt}\par
	\textit{Versione: 0.1.0}
\end{titlepage}

\newpage

% ====== TABELLA DI VERSIONAMENTO ======
{\normalfont\sffamily\huge\bfseries\color{AccentColor} Registro delle versioni}
\vspace{1cm}

\begin{center}
	\rowcolors{2}{LightGray}{white}
	\begin{tabular}{>{\centering\arraybackslash}m{1.5cm} >{\centering\arraybackslash}m{2cm} >{\raggedright\arraybackslash}m{2.5cm} >{\raggedright\arraybackslash}m{6.5cm}}
		\rowcolor{AccentColor}
		\textcolor{white}{\textbf{Versione}} & 
		\textcolor{white}{\textbf{Data}} & 
		\multicolumn{1}{c}{\textcolor{white}{\textbf{Autore}}} &
		\multicolumn{1}{c}{\textcolor{white}{\textbf{Descrizione delle modifiche}}} \\
		0.5.0 & 04/02/2025 & Lorenzo Grolla & Stesura UC Visione Aggregata\\ 
		0.4.0 & 25/01/2025 & Lorenzo Grolla & Stesura UC Dettagli Repository e Selezione Branch\\ 
        0.3.0 & 18/01/2025 & Giacomo Nalotto & Stesura UC Gestione Workspace\\
		0.2.0 & 14/01/2025 & Lorenzo Grolla & Stesura UC Autenticazione\\ 
		0.1.0 & 19/12/2025 & Alessandro Morabito & Inizio stesura\\ 
	\end{tabular}
\end{center}


% ====== INDICE ======
\pagestyle{fancy}
\newpage
\tableofcontents
\newpage

% ====== INTRODUZIONE ======
\section{Introduzione}
\subsection{Scopo del documento}
Con il presente documento il gruppo Byte Holders stabilisce i requisiti funzionali e non funzionali del software CodeGuardian.

Questo documento è rivolto:
\begin{itemize}
	\item all'Azienda Var Group, destinatari anche del software sviluppato
	\item al gruppo Byte Holders, che farà riferimento a questo documento nel corso del progetto
	\item ai professori Tullio Vardanega e Riccardo Cardin
\end{itemize}

All'interno del documento si proponge una visione generale del software proposto nella Sezione \ref{sec:descrizione generale}, per poi passare in rassegna i casi d'uso individuati nella Sezione \ref{sec:casi d'uso}.

Per la redazione del documento si è fatto riferimento allo standard IEEE 830-1998.

\subsection{Scopo del prodotto}
Il prodotto CodeGuardian permetterà di effettuare analisi della qualità di repository GitHub, con una particolare attenzione in merito ai permessi di visualizzazione delle informazioni e lancio delle stesse analisi.

CodeGuardian si propone come soluzione per team di sviluppo eterogenei che vogliono poter monitorare lo stato di repository GitHub e ottenere informazioni aggregate su insemi di progetti analizzati.

\subsection{Glossario}
Per evitare ambiguità, nel corso del documento si farà riferimento a termini indicati nel \href{https://byte-holders.github.io/Documentazione/RTB/Glossario.pdf}{\refterm{Glossario}} utilizzando la lettera \textit{G} ad apice della formula corrispondente, che viene indicata in corsivo (ad es. \refterm{formula in glossario}). La corrispondenza di termini è a meno di coniugazioni e declinazioni.

% Dal momento che il glossario è un documento interno, possiamo mettere qua dentro tutti i termini che ci servono
\subsection{Definizioni, acronimi e abbreviazioni}
\term{Caso d'uso}
Un Caso d'uso è un insieme di scenari che hanno in comune uno scopo finale per un utente.

\term{Scenario}

\term{Attore}


\subsection{Riferimenti}
\subsubsection{Riferimenti normativi}
\begin{itemize}
	\item 
		Norme Di Progetto\\
		\url{https://byte-holders.github.io/Documentazione/RTB/Norme_Di_Progetto.pdf}
	\item
		830-1998 - IEEE Recommended Practice for Software Requirements Specifications\\
		\url{https://ieeexplore.ieee.org/document/720574}
	\item
		Capitolato\\
		\url{https://www.math.unipd.it/~tullio/IS-1/2025/Progetto/C2.pdf}
\end{itemize}
\subsubsection{Riferimenti informativi}
\begin{itemize}
	\item
		\refterm{Glossario}\\
		\url{https://byte-holders.github.io/Documentazione/RTB/Glossario.pdf}
	\item
		Specifica UML 2.5.1\\
		\url{https://www.omg.org/spec/UML/2.5.1/PDF}
\end{itemize}

\section{Descrizione generale} \label{sec:descrizione generale}
\subsection{Prospettiva del prodotto}
Il gruppo Byte Holders propone il software CodeGuardian, un sistema ad agenti che permette di analizzare la qualità del codice, il livello di sicurezza e di manutenzione per una repository GitHub. L'esito dell'analisi sarà disponibile sotto forma di report agli utenti, ai quali sono proposte eventuali soluzioni alle problematiche individuate.

Il gruppo Byte Holders ha offerto particolare attenzione alla rolistica all'interno dell'applicazione, che si è tradotta nella distinzione di tipologie di utenti in base ai loro permessi.

A tutti gli utenti sarà comune la presenza di una dashboard che comprenderà vari workspace, nonché la capacità di effettuare ricerche avanzate al loro interno.

Il prodotto si propone quindi come soluzione per diverse tipologie di utenti, come quelle individuate in prima sessione di \textit{Design Thinking}, che condividono gli stessi progetti.

\subsection{Funzioni del prodotto}

\subsection{Caratteristiche dell'utente}

\section{Casi d'uso} \label{sec:casi d'uso}
\subsection{Lista degli Attori}
Nella creazione dei casi d'uso sono stati individuati i seguenti attori:
\begin{itemize}
	\item 
		\textbf{Utente non Autenticato}\\
		Un utente non riconosciuto dal sistema
	\item 
		\textbf{Utente Autenticato}\\
		Utente generico riconosciuto dal sistema
	\item 
		\textbf{Utente Permesso OWASP}  (eredita da \textit{Utente autenticato})\\
		Utente Autenticato con il permesso per la visione completa delle informazioni su OWASP
	\item
		\textbf{Utente Permesso Utenti/Ruoli} (eredita da \textit{Utente autenticato})\\
	 	Utente Autenticato con il permesso per la gestione degli utenti e dei ruoli
	\item 
		\textbf{Utente Permesso Test} (eredita da \textit{Utente autenticato})\\
		Utente Autenticato con il permesso per la visione completa delle informazioni sui test
	\item
		\textbf{Utente Permesso Documentazione} (eredita da \textit{Utente autenticato})\\
		Utente Autenticato con il permesso per la visione completa delle informazioni sulla documentazione
	\item
		\textbf{Utente Permesso Scansione} (eredita da \textit{Utente autenticato})\\
		Utente Autenticato con il permesso per il lancio di una scansione
	\item
		\textbf{Utente Permesso Qualità Codice} (eredita da \textit{Utente autenticato})\\
		Utente Autenticato con il permesso per la visione completa delle informazioni sulla qualità del codice
	\item
		\textbf{Utente Permesso Informazioni Tecniche} (eredita da \textit{Utente autenticato})\\
		Utente Autenticato con il permesso per la visione completa delle informazioni tecniche di una repository
\end{itemize}
\subsection{Struttura generale di un caso d'uso}
Si è deciso di descrivere ciascun \refterm{caso d'uso} seguendo la seguente struttura (\underline{sottolineati} i campi sempre popolati):

\begin{tabular}{|p{.25\linewidth} p{.6\linewidth}|}
	\hline
	\textbf{\underline{Codice}} 
	&
	Codice identificativo utilizzato per far riferimento al \refterm{caso d'uso} corrente\\
	
	\textbf{\underline{Titolo}} 
	&
	Titolo del \refterm{caso d'uso} corrente\\
	
	\textbf{\underline{Attori principali}} 
	&
	Attori che agiscono sul sistema dando inizio allo scenario\\
	
	\textbf{Attori secondari} 
	&
	Attori di supporto che agiscono in risposta a stimoli del sistema\\
	
	\textbf{\underline{Precondizioni}}
	&
	Condizioni necessarie per l'esecuzione del \refterm{caso d'uso} corrente\\
	
	\textbf{\underline{Postcondizioni}} 
	&
	Condizioni in cui viene lasciato il sistema al termine dello \refterm{scenario principale}\\
	
	\textbf{\underline{Scenario principale}}
	&
	Descrizione degli eventi che avvengono all'interno dello \refterm{scenario principale}\\
	
	\textbf{Inclusioni} 
	&
	Lista dei riferimenti a \refterm{casi d'uso} terzi \refterm{inclusi} dal \refterm{caso d'uso} corrente e al quale si fa riferimento nella sezione \textit{Scenario principale}\\
	
	\textbf{Scenari alternativi} 
	&
	Descrizione delle situazione che portano a \refterm{scenari alternativi}\\
	
	\textbf{Eredita da} 
	&
	Codice del \refterm{caso d'uso} terzo da cui eredita il \refterm{caso d'uso} corrente (non ammettiamo ereditarietà multipla)\\
	
	\hline
\end{tabular}

\newpage
\subsection{Lista dei \refterm{casi d'uso}}

% UC1: REGISTRAZIONE UTENTE
\subsubsection{UC1 - Registrazione}\label{UC1}
\begin{figure}[h]
	\centering
	\includegraphics[width=0.8\textwidth]{../Assets/AdR/UC1ext.png}
	\caption{UC1 - Registrazione}
	\label{fig:UC1ext}
\end{figure}

\begin{itemize}
	\item \textbf{Attori principali:} Utente non Autenticato
	\item \textbf{Precondizioni:} 
	\begin{itemize}
		\item Il sistema è attivo e funzionante
		\item L'utente non possiede ancora un account attivo nel sistema
	\end{itemize}
	\item \textbf{Postcondizioni:} Viene creato un nuovo profilo utente nel sistema CodeGuardian con stato "da confermare".
	\item \textbf{Scenario principale:}
	\begin{enumerate}
		\item L'utente accede alla pagina di registrazione
		\item L'utente inserisce l'username (\hyperref[UC1.1]{UC1.1})
		\item L'utente inserisce l'email (\hyperref[UC1.2]{UC1.2})
		\item L'utente inserisce la password (\hyperref[UC1.3]{UC1.3})
		\item L'utente conferma la registrazione
		\item Il sistema valida i dati e crea l'utente su Amazon Cognito
		\item Il sistema invia un codice OTP all'email fornita
	\end{enumerate}
	\item \textbf{Scenario alternativo:}
	\begin{enumerate}
		\item L'utente ha inserito un errore nei dati inseriti come username già esistente, email già registrata oppure password non valida (\hyperref{UC1.4}(UC1.4))
	\end{enumerate}
	\item \textbf{Inclusioni:} \hyperref[UC1.1]{UC1.1}, \hyperref[UC1.2]{UC1.2}, \hyperref[UC1.3]{UC1.3},

	\item \textbf{Estensioni:} \hyperref[UC1.4]{UC1.4}
	
\end{itemize}
\newpage

Il caso d'Uso UC1 include ulteriori casi d'uso come rappresentato nella seguente immagine:
\begin{figure}[H]
	\centering
	\includegraphics[width=0.6\textwidth]{../Assets/AdR/UC1.png}
	\caption{Inclusioni di UC1: UC1.1,UC1.2,UC1.3}
	\label{fig:UC1}
\end{figure}

% --- Sottocasi di UC1 ---
\paragraph{UC1.1 - Inserimento username}\label{UC1.1}
\begin{itemize}
	\item \textbf{Attori principali:} Utente non Autenticato
	\item \textbf{Precondizioni:} 
	\begin{itemize}
		\item Il sistema è attivo e funzionante
		\item L'utente si trova nella pagina di registrazione
	\end{itemize}
	\item \textbf{Postcondizioni:} L'username è inserito nel sistema.
	\item \textbf{Scenario principale:} L'utente digita l'username scelto nell'apposito campo.
\end{itemize}

\paragraph{UC1.2 - Inserimento email}\label{UC1.2}
\begin{itemize}
	\item \textbf{Attori principali:} Utente non Autenticato
	\item \textbf{Precondizioni:} 
	\begin{itemize}
		\item Il sistema è attivo e funzionante
		\item L'utente si trova nella pagina di registrazione
	\end{itemize}
	\item \textbf{Postcondizioni:} L'email è inserita nel sistema.
	\item \textbf{Scenario principale:} L'utente digita il proprio indirizzo email nell'apposito campo.
\end{itemize}

\paragraph{UC1.3 - Inserimento password}\label{UC1.3}
\begin{itemize}
	\item \textbf{Attori principali:} Utente non Autenticato
	\item \textbf{Precondizioni:} 
	\begin{itemize}
		\item Il sistema è attivo e funzionante
		\item L'utente si trova nella pagina di registrazione
	\end{itemize}
	\item \textbf{Postcondizioni:} La password è inserita nel sistema.
	\item \textbf{Scenario principale:} L'utente digita la password desiderata nell'apposito campo.
\end{itemize}

\paragraph{UC1.4 - Registrazione fallita}\label{UC1.4}
\begin{itemize}
	\item \textbf{Attori principali:} Utente non Autenticato
	\item \textbf{Precondizioni:} 
	\begin{itemize}
		\item Il sistema è attivo e funzionante
		\item L'utente ha tentato la conferma della registrazione con dati non validi
	\end{itemize}
	\item \textbf{Postcondizioni:} La registrazione non viene completata; l'utente rimane nella pagina di registrazione.
	\item \textbf{Scenario principale:}
	\begin{enumerate}
		\item Il sistema rileva un errore nei dati inseriti (username già esistente, email già registrata oppure password non conforme ai requisiti di sicurezza)
		\item Il sistema mostra un messaggio di errore specifico all'utente
		\item Il sistema permette all'utente di correggere i dati mantenendo quelli validi
	\end{enumerate}
\end{itemize}


% UC2: CONFERMA REGISTRAZIONE 
\subsubsection{UC2 - Conferma registrazione}\label{UC2}
\begin{figure}[H]
	\centering
	\includegraphics[width=0.7\textwidth]{../Assets/AdR/UC2.png}
	\caption{UC2 - Conferma Registrazione}
	\label{fig:UC2}
\end{figure}

\begin{itemize}
	\item \textbf{Attori principali:} Utente non Autenticato
	\item \textbf{Precondizioni:} 
		\begin{itemize}
			\item Il sistema è attivo e funzionante
			\item L'utente ha completato la prima fase di registrazione (\hyperref[UC1]{UC1})
			\item L'utente ha ricevuto il codice OTP via email
		\end{itemize}
	\item \textbf{Postcondizioni:} L'account viene attivato e l'utente può effettuare il login.
	\item \textbf{Scenario principale:}
	\begin{enumerate}
		\item L'utente accede alla pagina di conferma registrazione
		\item Il sistema richiede il codice di verifica
		\item L'utente inserisce il codice OTP (\hyperref[UC2.1]{UC2.1})
		\item L'utente conferma l'invio
		\item Il sistema verifica il codice e attiva l'account
	\end{enumerate}
	\item \textbf{Scenario alternativo:}
	\begin{enumerate}
		\item L'utente ha inserito un codice OTP non valido o scaduto(\hyperref[UC2.2]{UC2.2})
	\end{enumerate}
	\item \textbf{Inclusioni:} \hyperref[UC2.1]{UC2.1}
	\item \textbf{Estensioni:} \hyperref[UC2.2]{UC2.2}
\end{itemize}

% --- Sottocasi di UC2 ---
\paragraph{UC2.1 - Inserimento codice OTP}\label{UC2.1}
\begin{itemize}
	\item \textbf{Attori principali:} Utente non Autenticato
	\item \textbf{Precondizioni:} 
	\begin{itemize}
		\item Il sistema è attivo e funzionante
		\item L'utente si trova nella pagina di conferma registrazione
	\end{itemize}
	\item \textbf{Postcondizioni:} Il codice OTP è inserito nel sistema.
	\item \textbf{Scenario principale:} L'utente inserisce il codice numerico ricevuto via email nell'apposito campo.
\end{itemize}

\paragraph{UC2.2 - Verifica fallita}\label{UC2.2}
\begin{itemize}
	\item \textbf{Attori principali:} Utente non Autenticato
	\item \textbf{Precondizioni:} 
	\begin{itemize}
		\item Il sistema è attivo e funzionante
		\item L'utente ha inviato un codice OTP errato o scaduto
	\end{itemize}
	\item \textbf{Postcondizioni:} L'account rimane nello stato "da confermare".
	\item \textbf{Scenario principale:}
	\begin{enumerate}
		\item Il sistema rileva che il codice non è valido o è scaduto
		\item Il sistema mostra un messaggio di errore "Codice non valido o scaduto"
		\item Il sistema offre l'opzione per richiedere un nuovo codice OTP
	\end{enumerate}
\end{itemize}


% UC3: LOGIN 
\subsubsection{UC3 - Login} \label{UC3}
\begin{figure}[H]
	\centering
	\includegraphics[width=0.7\textwidth]{../Assets/AdR/UC3ext.png}
	\caption{UC3 - Login}
	\label{fig:UC3ext}
\end{figure}
\begin{itemize}
	\item \textbf{Attori principali:} Utente non Autenticato
	\item \textbf{Precondizioni:} 
		\begin{itemize}
			\item Il sistema è attivo e funzionante
			\item L'utente possiede un account attivo nel sistema
		\end{itemize}
	\item \textbf{Postcondizioni:} L'utente è autenticato e accede alla home del sistema
	\item \textbf{Scenario principale:}
	\begin{enumerate}
		\item L'utente accede alla pagina di login
		\item L'utente inserisce l'username (\hyperref[UC3.1]{UC3.1})
		\item L'utente inserisce la password (\hyperref[UC3.2]{UC3.2})
		\item L'utente conferma l'accesso
		\item Il sistema valida le credenziali tramite Amazon Cognito
		\item Il sistema reindirizza l'utente alla home
	\end{enumerate}
	\item \textbf{Scenari Alternativi}
	\begin{enumerate}
		\item L'utente ha inserito credenziali errate o inesistenti(\hyperref[UC3.3]{UC3.3})
		\item L'utente ha dimenticato la password e sta cercando di ripristinarla (\hyperref[UC4]{UC4})
	\end{enumerate}
	\item \textbf{Inclusioni:} \hyperref[UC3.1]{UC3.1}, \hyperref[UC3.2]{UC3.2}
	\item \textbf{Estensioni:} \hyperref[UC3.3]{UC3.3}, \hyperref[UC4]{UC4}
	
\end{itemize}

% --- Sottocasi di UC3 ---
Il caso d'Uso UC3 include ulteriori casi d'uso come rappresentato nella seguente immagine:
\begin{figure}[H]
	\centering
	\includegraphics[width=0.6\textwidth]{../Assets/AdR/UC3.png}
	\caption{Inclusioni di UC3: UC3.1,UC3.2}
	\label{fig:UC3}
\end{figure}


\paragraph{UC3.1 - Inserimento username o email}\label{UC3.1}
\begin{itemize}
	\item \textbf{Attori principali:} Utente non Autenticato
	\item \textbf{Precondizioni:} 
	\begin{itemize}
		\item Il sistema è attivo e funzionante
		\item L'utente si trova nella pagina di login
	\end{itemize}
	\item \textbf{Postcondizioni:} L'username è inserito nel sistema.
	\item \textbf{Scenario principale:} L'utente inserisce il proprio username o email nell'apposito campo.
\end{itemize}

\paragraph{UC3.2 - Inserimento password}\label{UC3.2}
\begin{itemize}
	\item \textbf{Attori principali:} Utente non Autenticato
	\item \textbf{Precondizioni:} 
	\begin{itemize}
		\item Il sistema è attivo e funzionante
		\item L'utente si trova nella pagina di login
	\end{itemize}
	\item \textbf{Postcondizioni:} La password è inserita nel sistema.
	\item \textbf{Scenario principale:} L'utente inserisce la propria password nell'apposito campo.
\end{itemize}

\paragraph{UC3.3 - Login fallito}\label{UC3.3}
\begin{itemize}
	\item \textbf{Attori principali:} Utente non Autenticato
	\item \textbf{Precondizioni:} 
	\begin{itemize}
		\item Il sistema è attivo e funzionante
		\item L'utente ha inviato credenziali non valide
	\end{itemize}
	\item \textbf{Postcondizioni:} L'utente rimane non autenticato.
	\item \textbf{Scenario principale:}
	\begin{enumerate}
		\item Il sistema verifica che le credenziali non corrispondono a nessun account attivo
		\item Il sistema mostra il messaggio "Username o password errati"
		\item Il sistema permette di riprovare l'inserimento delle credenziali
	\end{enumerate}
\end{itemize}

% UC4: RECUPERO PASSWORD 
\subsubsection{UC4 - Recupero password}\label{UC4}
\begin{figure}[H]
	\centering
	\includegraphics[width=0.7\textwidth]{../Assets/AdR/UC4ext.png}
	\caption{UC4}
	\label{fig:UC4ext}
\end{figure}

\begin{itemize}
	\item \textbf{Attori principali:} Utente non Autenticato
	\item \textbf{Precondizioni:} 
	\begin{itemize}
		\item Il sistema è attivo e funzionante
		\item L'utente possiede un account registrato
		\item L'utente ha dimenticato la password
	\end{itemize}
	\item \textbf{Postcondizioni:} La password viene reimpostata con successo.
	\item \textbf{Scenario principale:}
	\begin{enumerate}
		\item L'utente accede alla funzionalità di recupero password dalla pagina di login
		\item L'utente inserisce l'email associata al proprio account (\hyperref[UC4.1]{UC4.1})
		\item L'utente conferma la richiesta
		\item Il sistema valida l'email e genera un codice OTP
		\item Il sistema invia il codice OTP via email
		\item L'utente inserisce il codice OTP ricevuto (\hyperref[UC4.2]{UC4.2})
		\item L'utente inserisce la nuova password (\hyperref[UC4.3]{UC4.3})
		\item L'utente conferma il cambio password
		\item Il sistema valida il codice OTP, verifica che la nuova password rispetti i requisiti di sicurezza e aggiorna la password
		\item Il sistema conferma l'aggiornamento e reindirizza alla pagina di login
	\end{enumerate}
	\item \textbf{Scenario alternativo:}
	\begin{enumerate}
		\item L'utente ha inserito credenziali errate oppure il codice OTP è scaduto o errato(\hyperref[UC4.4]{UC4.4})
	\end{enumerate}
	\item \textbf{Inclusioni:} \hyperref[UC4.1]{UC4.1}, \hyperref[UC4.2]{UC4.2}, \hyperref[UC4.3]{UC4.3}
	\item \textbf{Estensioni:} \hyperref[UC4.4]{UC4.4}
\end{itemize}

% --- Sottocasi di UC4 ---
Il caso d'Uso UC4 include ulteriori casi d'uso come rappresentato nella seguente immagine:
\begin{figure}[h]
	\centering
	\includegraphics[width=0.6\textwidth]{../Assets/AdR/UC4.png}
	\caption{Inclusioni di UC4: UC4.1,UC4.2,UC4.3}
	\label{fig:UC4}
\end{figure}

\paragraph{UC4.1 - Inserimento email per recupero}\label{UC4.1}
\begin{itemize}
	\item \textbf{Attori principali:} Utente non Autenticato
	\item \textbf{Precondizioni:} 
	\begin{itemize}
		\item Il sistema è attivo e funzionante
		\item L'utente si trova nella pagina di recupero password
	\end{itemize}
	\item \textbf{Postcondizioni:} Il sistema invia il codice OTP all'email fornita.
	\item \textbf{Scenario principale:} L'utente inserisce l'email associata all'account nell'apposito campo e conferma la richiesta.
\end{itemize}

\paragraph{UC4.2 - Inserimento codice OTP}\label{UC4.2}
\begin{itemize}
	\item \textbf{Attori principali:} Utente non Autenticato
	\item \textbf{Precondizioni:} 
	\begin{itemize}
		\item Il sistema è attivo e funzionante
		\item L'utente ha ricevuto il codice OTP via email
		\item L'utente si trova nella pagina di reset password
	\end{itemize}
	\item \textbf{Postcondizioni:} Il codice OTP è inserito nel sistema.
	\item \textbf{Scenario principale:} L'utente inserisce il codice OTP ricevuto via email nell'apposito campo.
\end{itemize}

\paragraph{UC4.3 - Inserimento nuova password}\label{UC4.3}
\begin{itemize}
	\item \textbf{Attori principali:} Utente non Autenticato
	\item \textbf{Precondizioni:} 
	\begin{itemize}
		\item Il sistema è attivo e funzionante
		\item L'utente ha inserito il codice OTP
		\item L'utente si trova nella pagina di reset password
	\end{itemize}
	\item \textbf{Postcondizioni:} La nuova password è inserita nel sistema.
	\item \textbf{Scenario principale:} L'utente inserisce la nuova password desiderata nell'apposito campo.
\end{itemize}

\paragraph{UC4.4 - Ripristino fallito}\label{UC4.4}
\begin{itemize}
	\item \textbf{Attori principali:} Utente non Autenticato
	\item \textbf{Precondizioni:} 
	\begin{itemize}
		\item Il sistema è attivo e funzionante
		\item L'utente ha inserito un codice OTP non valido o scaduto, oppure una password non conforme ai requisiti
	\end{itemize}
	\item \textbf{Postcondizioni:} La password rimane invariata.
	\item \textbf{Scenario principale:}
	\begin{enumerate}
		\item Il sistema rileva che il codice OTP è errato o scaduto, oppure che la password non rispetta i requisiti di sicurezza
		\item Il sistema mostra un messaggio di errore specifico
		\item Il sistema impedisce il cambio password e permette di richiedere un nuovo codice o correggere la password
	\end{enumerate}
\end{itemize}

\subsubsection{UC5 - Invito utente in workspace}
\begin{itemize}
	\item \textbf{Attori principali:} 
	\item \textbf{Precondizioni:} 
	\begin{itemize}
		\item Il sistema è attivo e funzionante
		\item L'utente è riconosciuto dal sistema come Project Manager
        \item L'utente fa parte del workspace all'interno del quale invita un altro utente
	\end{itemize}
    \item \textbf{Postcondizioni:} L'invito viene inviato ed un utente esterno al workspace lo riceve
	\item \textbf{Scenario principale:}
	\begin{enumerate}
		\item Un utente che fa parte di un workspace vuole invitare un altro utente che non ne fa parte
		\item Inserisce l'username dell'utente che vuole invitare (\hyperref[UC5.2]{UC5.1})
		\item Seleziona al nuovo utente il ruolo scegliendo dalla lista dei ruoli del workspace (\hyperref[UC5.2]{UC5.2})
	\end{enumerate}
    \item \textbf{Inclusioni:} \hyperref[UC5.1]{UC5.1}, \hyperref[UC5.2]{UC5.2}
    \item \textbf{Estensioni:} \hyperref[UC5.3]{UC5.3}
	
\end{itemize}

% TODO: Immagine mancante
% \begin{figure}[h]
% 	\centering
% 	\includegraphics[width=1\textwidth]{Figura 8.png}
% 	\caption{Figura 8: UC5 e relative inclusioni}
% 	\label{fig:UC1}
% \end{figure}

\subsubsection{UC5.1 - Inserimento username nuovo utente}\label{UC5.1}
\begin{itemize}
	\item \textbf{Attori principali:} 
	\item \textbf{Precondizioni:} 
	\begin{itemize}
		\item Il sistema è attivo e funzionante
		\item L'utente è autenticato nel sistema
        \item L'utente fa parte del workspace all'interno del quale invita un altro utente
        \item L'utente ha avviato la procedura di invito di un nuovo utente
	\end{itemize}
    \item \textbf{Postcondizioni:} Il sistema riceve l'username dell'utente da invitare
	\item \textbf{Scenario principale:} Il sistema presenta un campo di input per l'inserimento dell'username, l'utente digita l'username dell'utente che desidera invitare
	
\end{itemize}

\subsubsection{UC5.2 - Selezione ruolo da assegnare al nuovo utente}\label{UC5.2}
\begin{itemize}
	\item \textbf{Attori principali:} 
	\item \textbf{Precondizioni:} 
	\begin{itemize}
		\item Il sistema è attivo e funzionante
		\item L'utente è autenticato nel sistema
        \item L'utente fa parte del workspace all'interno del quale invita un altro utente
        \item L'utente ha avviato la procedura di invito di un nuovo utente
	\end{itemize}
    	\item \textbf{Postcondizioni:} Il sistema riceve la selezione del ruolo da assegnare all'utente invitato
	\item \textbf{Scenario principale:} 
    \begin{enumerate}
    \item Il sistema recupera la lista dei ruoli disponibili nel workspace
    \item Il sistema presenta una lista a tendina con i ruoli disponibili
    \item L'utente seleziona il ruolo desiderato dalla lista
    \item Il sistema memorizza il ruolo selezionato
    \end{enumerate}
    \item \textbf{Inclusioni:} \hyperref[UC5.2.1]{UC5.2.1}
\end{itemize}

\subsubsection{UC5.2.1 - Visualizzazione lista ruoli nel workspace}\label{UC5.2.1}
\begin{itemize}
	\item \textbf{Attori principali:} 
	\item \textbf{Precondizioni:} 
	\begin{itemize}
		\item Il sistema è attivo e funzionante
		\item L'utente è autenticato nel sistema
        \item L'utente sta selezionando un ruolo da assegnare durante l'invito
        \item Esiste almeno un ruolo definito nel workspace
	\end{itemize}
    \item \textbf{Postcondizioni:} Viene visualizzata la lista completa dei ruoli disponibili nel workspace
	\item \textbf{Scenario principale:} 
    \begin{enumerate}
    \item Il sistema recupera la lista dei ruoli disponibili nel workspace
    \item Il sistema presenta una lista a tendina con i ruoli disponibili
    \item L'utente seleziona il ruolo desiderato dalla lista
    \item Il sistema memorizza il ruolo selezionato
    \end{enumerate}
\end{itemize}

\subsubsection{UC5.3 - Invito non riuscito}\label{UC5.3}
\begin{itemize}
	\item \textbf{Attori principali:} 
	\item \textbf{Precondizioni:} 
	\begin{itemize}
		\item Il sistema è attivo e funzionante
		\item L'utente è autenticato nel sistema
        \item L'utente ha tentato l'invio di un invito
        \item L'utente ha avviato la procedura di invito di un nuovo utente
	\end{itemize}
    \item \textbf{Postcondizioni:} 
    \begin{itemize}
    \item Il sistema annulla il tentativo di invito e nessun invito viene creato o inviato
    \item Viene mostrato a schermo un messaggio di errore specifico
    \end{itemize}
	\item \textbf{Scenario principale:} 
    \begin{enumerate}
    \item Il sistema tenta di validare l'username inserito dall'utente
    \item Il sistema rileva che l'username è già presente nel workspace o non esiste
    \end{enumerate}
	
\end{itemize}

\subsubsection{UC7 - Visualizzazione inviti in workspace}
\begin{itemize}
	\item \textbf{Attori principali:} 
	\item \textbf{Precondizioni:} 
	\begin{itemize}
		\item Il sistema è attivo e funzionante
		\item L'utente è autenticato nel sistema
        \item L'utente ha selezionato l'opzione per visualizzare i propri inviti
	\end{itemize}	\item \textbf{Postcondizioni:} Il sistema mostra la lista completa degli inviti ricevuti dall'utente

	\item \textbf{Scenario principale:}
	\begin{enumerate}
		\item L'utente accede alla sezione degli inviti ricevuti
		\item Il sistema recupera tutti gli inviti pendenti destinati all'utente e li mostra in una lista
		\item Per ogni invito vengono visualizzate le informazioni (UC7.1)
	\end{enumerate}
    \item \textbf{Inclusioni:} \hyperref[UC7.1]{UC7.1}
\end{itemize}

% TODO: Immagine mancante
% \begin{figure}[h]
% 	\centering
% 	\includegraphics[width=1\textwidth]{Figura 9.png}
% 	\caption{Figura 9: UC7 - Visualizzazione inviti in workspace}
% 	\label{fig:UC1}
% \end{figure}

% TODO: Immagine mancante
% \begin{figure}[h]
% 	\centering
% 	\includegraphics[width=1\textwidth]{Figura 10.png}
% 	\caption{Figura 10: Inclusioni di UC7.1: UC7.1.1,UC7.1.2}
% 	\label{fig:UC1}
% \end{figure}

\subsubsection{UC7.1 - Visualizzazione singolo invito in workspace}\label{UC7.1}
\begin{itemize}
	\item \textbf{Attori principali:} 
	\item \textbf{Precondizioni:} 
	\begin{itemize}
		\item Il sistema è attivo e funzionante
		\item L'utente è autenticato nel sistema
        \item L'utente ha selezionato l'opzione per visualizzare i propri inviti
        \item Esiste almeno un invito destinato all'utente
	\end{itemize}
    \item \textbf{Postcondizioni:} Il sistema mostra le informazioni complete di un singolo invito nella lista
	\item \textbf{Scenario principale:} Il sistema mostra un elemento della lista inviti contenente:
	\begin{itemize}
		\item Username del mittente dell'invito (\hyperref[UC7.1.1]{UC7.1.1})
		\item Username dell'owner del workspace (\hyperref[UC7.1.2]{UC7.1.2})
        \item Nome del workspace (\hyperref[UC7.1.3]{UC7.1.3})
	\end{itemize}
    \item \textbf{Inclusioni:} \hyperref[UC7.1.1]{UC7.1.1}, \hyperref[UC7.1.2]{UC7.1.2}, \hyperref[UC7.1.3]{UC7.1.3}
	
\end{itemize}

\subsubsection{UC7.1.1 - Visualizzazione username del mittente dell'invito ricevuto}\label{UC7.1.1}
\begin{itemize}
	\item \textbf{Attori principali:} 
	\item \textbf{Precondizioni:} 
	\begin{itemize}
		\item Il sistema è attivo e funzionante
		\item L'utente è autenticato nel sistema
        \item L'utente ha selezionato l'opzione per visualizzare i propri inviti
        \item L'utente sta visualizzando un singolo invito
	\end{itemize}
    \item \textbf{Postcondizioni:} Il sistema mostra l'username dell'utente che ha inviato l'invito
	\item \textbf{Scenario principale:} 
	\begin{enumerate}
	    \item Il sistema recupera l'username dell'utente mittente associato all'invito
        \item Il sistema visualizza l'username del mittente all'interno dell'elemento invito
	\end{enumerate}
\end{itemize}

\subsubsection{UC7.1.2 - Visualizzazione username dell'owner del workspace}\label{UC7.1.2}
\begin{itemize}
	\item \textbf{Attori principali:} 
	\item \textbf{Precondizioni:} 
	\begin{itemize}
		\item Il sistema è attivo e funzionante
		\item L'utente è autenticato nel sistema
        \item L'utente ha selezionato l'opzione per visualizzare i propri inviti
        \item L'utente sta visualizzando un singolo invito
	\end{itemize}
    \item \textbf{Postcondizioni:} Il sistema mostra l'username dell'owner del workspace a cui l'utente è stato invitato
	\item \textbf{Scenario principale:} 
	\begin{enumerate}
	    \item Il sistema recupera l'username dell'owner del workspace a cui l'utente è stato invitato
        \item Il sistema visualizza l'username dell'owner all'interno dell'elemento invito
	\end{enumerate}
	
\end{itemize}

\subsubsection{UC7.1.3 - Visualizzazione nome del workspace}\label{UC7.1.3}
\begin{itemize}
	\item \textbf{Attori principali:} 
	\item \textbf{Precondizioni:} 
	\begin{itemize}
		\item Il sistema è attivo e funzionante
		\item L'utente è autenticato nel sistema
        \item L'utente ha selezionato l'opzione per visualizzare i propri inviti
        \item L'utente sta visualizzando un singolo invito
	\end{itemize}
    \item \textbf{Postcondizioni:} Il sistema mostra il nome identificativo del workspace a cui l'utente è stato invitato
	\item \textbf{Scenario principale:} 
	\begin{enumerate}
	    \item Il sistema recupera il nome del workspace a cui l'utente è stato invitato
        \item Il sistema visualizza il nome del workspace all'interno dell'elemento invito
	\end{enumerate}
	
\end{itemize}

\subsubsection{UC8 - Gestione degli inviti}\label{UC8}
\begin{itemize}
	\item \textbf{Attori principali:} 
	\item \textbf{Precondizioni:} 
	\begin{itemize}
		\item Il sistema è attivo e funzionante
		\item L'utente è autenticato nel sistema
        \item L'utente ha selezionato un invito dalla lista degli inviti ricevuti
	\end{itemize}
    \item \textbf{Postcondizioni:} L'utente ha gestito l'invito selezionato
	\item \textbf{Scenario principale:}
    \begin{itemize}
        \item Il sistema mostra le opzioni disponibili per gestire l'invito
    \end{itemize}
    \item \textbf{UC che ereditano:} \hyperref[UC8.1]{UC8.1}, \hyperref[UC8.2]{UC8.2}
\end{itemize}

\subsubsection{UC8.1 - Accettazione invito}\label{UC8.1}
\begin{itemize}
	\item \textbf{Attori principali: Utente} 
	\item \textbf{Precondizioni:} 
	\begin{itemize}
		\item Il sistema è attivo e funzionante
		\item L'utente è stato identificato dal sistema come Utente
        \item L'utente ha selezionato un invito dalla lista degli inviti ricevuti
	\end{itemize}
    \item \textbf{Postcondizioni:} L'utente accetta l'invito selezionato e diventa nuovo membro del workspace con il ruolo specificato nell'invito
	\item \textbf{Scenario principale:}
    \begin{enumerate}
        \item L'utente seleziona l'opzione "Accetta" per l'invito
        \item Il sistema aggiunge l'utente al workspace con il ruolo specificato
        \item Il sistema rimuove l'invito dalla lista degli inviti pendenti
        \item Il sistema aggiorna la lista dei workspace dell'utente includendo il nuovo workspace
    \end{enumerate}
    \item \textbf{Eredita da:} \hyperref[UC8]{UC8}
\end{itemize}

\subsubsection{UC8.2 - Rifiuto invito}\label{UC8.2}
\begin{itemize}
	\item \textbf{Attori principali:} 
	\item \textbf{Precondizioni:} 
	\begin{itemize}
		\item Il sistema è attivo e funzionante
		\item L'utente è autenticato nel sistema
        \item L'utente ha selezionato un invito dalla lista degli inviti ricevuti
	\end{itemize}
    \item \textbf{Postcondizioni:} L'utente rifiuta l'invito selezionato e non diventa nuovo membro del workspace
	\item \textbf{Scenario principale:}
    \begin{enumerate}
        \item L'utente seleziona l'opzione "Rifiuta" per l'invito
        \item Il sistema rimuove l'invito dalla lista degli inviti pendenti
    \end{enumerate}
    \item \textbf{Eredita da:} \hyperref[UC8]{UC8}
\end{itemize}

\subsubsection{UC9 - Visualizzazione lista workspace}
\begin{itemize}
	\item \textbf{Attori principali:} 
	\item \textbf{Precondizioni:} 
	\begin{itemize}
		\item Il sistema è attivo e funzionante
		\item L'utente è autenticato nel sistema
        \item L'utente sta visualizzando la dashboard iniziale
	\end{itemize}
    \item \textbf{Postcondizioni:} Il sistema mostra la lista completa dei workspace di cui l'utente fa parte
	\item \textbf{Scenario principale:}
    \begin{enumerate}
        \item Il sistema recupera tutti i workspace a cui l'utente appartiene
        \item Il sistema mostra la lista dei workspace di cui l'utente fa parte
        \item La lista è composta da singoli elementi che rappresentano ciascuno un workspace (\hyperref[UC9.1]{UC9.1})
    \end{enumerate}
    \item \textbf{Inclusioni:} \hyperref[UC9.1]{UC9.1}
\end{itemize}

\subsubsection{UC9.1 - Visualizzazione elemento della lista di workspace}\label{UC9.1}
\begin{itemize}
	\item \textbf{Attori principali:} 
	\item \textbf{Precondizioni:} 
	\begin{itemize}
		\item Il sistema è attivo e funzionante
		\item L'utente è autenticato nel sistema
        \item L'utente sta visualizzando la dashboard iniziale
        \item Esiste almeno un workspace di cui l'utente fa parte
	\end{itemize}
    \item \textbf{Postcondizioni:} Il sistema mostra un singolo elemento della lista dei workspace rappresentante un workspace specifico
	\item \textbf{Scenario principale:} Il sistema mostra un elemento della lista contenente:
    \begin{itemize}
        \item Username dell'owner del workspace (\hyperref[UC9.1.1]{UC9.1.1})
        \item Nome del workspace (\hyperref[UC9.1.2]{UC9.1.2})
    \end{itemize}
    \item \textbf{Inclusioni:} \hyperref[UC9.1.1]{UC9.1.1}, \hyperref[UC9.1.2]{UC9.1.2}
\end{itemize}

\subsubsection{UC9.1.1 - Visualizzazione username dell'owner del workspace}\label{UC9.1.1}
\begin{itemize}
	\item \textbf{Attori principali:} 
	\item \textbf{Precondizioni:} 
	\begin{itemize}
		\item Il sistema è attivo e funzionante
		\item L'utente è autenticato nel sistema
        \item L'utente sta visualizzando la dashboard iniziale
	\end{itemize}
    \item \textbf{Postcondizioni:} Il sistema mostra l'username dell'owner di un workspace di cui l'utente fa parte
	\item \textbf{Scenario principale:}
    \begin{itemize}
        \item Il sistema recupera l'username dell'owner del singolo workspace selezionato dalla lista della dashboard
        \item Il sistema visualizza l'username dell'owner nell'elemento della lista
    \end{itemize}
\end{itemize}

\subsubsection{UC9.1.2 - Visualizzazione nome del workspace}\label{UC9.1.2}
\begin{itemize}
	\item \textbf{Attori principali:} 
	\item \textbf{Precondizioni:} 
	\begin{itemize}
		\item Il sistema è attivo e funzionante
		\item L'utente è autenticato nel sistema
        \item L'utente sta visualizzando la dashboard iniziale
	\end{itemize}
    \item \textbf{Postcondizioni:} Il sistema mostra il nome di un workspace di cui l'utente fa parte
	\item \textbf{Scenario principale:}
    \begin{itemize}
        \item Il sistema recupera il nome del singolo workspace selezionato dalla lista della dashboard
        \item Il sistema visualizza il nome nell'elemento della lista
    \end{itemize}
\end{itemize}

\subsubsection{UC10 - Ricerca workspace}
\begin{itemize}
	\item \textbf{Attori principali:} 
	\item \textbf{Precondizioni:} 
	\begin{itemize}
		\item Il sistema è attivo e funzionante
		\item L'utente è autenticato nel sistema
        \item L'utente sta visualizzando la lista dei workspace di cui fa parte nella dashboard
	\end{itemize}
    \item \textbf{Postcondizioni:} Il sistema mostra i workspace che corrispondono ai criteri di ricerca inseriti
	\item \textbf{Scenario principale:}
    \begin{enumerate}
        \item L'utente dalla dashboard principale accede alla funzione di ricerca workspace
        \item L'utente inserisce il nome (o parte del nome) del workspace che vuole cercare (\hyperref[UC10.1]{UC10.1})
        \item Il sistema filtra in tempo reale i workspace mostrati in base al testo inserito
    \end{enumerate}
    \item \textbf{Inclusioni:} 
    \item \textbf{Scenari alternativi:}
    \begin{itemize}
		\item Nessun workspace corrisponde alla ricerca, il sistema mostra un messaggio: "Nessun workspace trovato con questo nome"
	\end{itemize}
\end{itemize}

\subsubsection{UC10.1 - Inserimento nome del workspace da cercare}\label{UC10.1}
\begin{itemize}
	\item \textbf{Attori principali:} 
	\item \textbf{Precondizioni:} 
	\begin{itemize}
		\item Il sistema è attivo e funzionante
		\item L'utente è autenticato nel sistema
        \item L'utente sta eseguendo un'operazione che richiede l'inserimento del nome di un workspace
	\end{itemize}
    \item \textbf{Postcondizioni:} Il sistema riceve il nome (o parte del nome) del workspace
	\item \textbf{Scenario principale:}
    \begin{enumerate}
        \item Il sistema presenta un campo di input per l'inserimento del nome del workspace
        \item L'utente inserisce il nome (o parte del nome) del workspace 
        \item Il sistema memorizza il testo inserito
    \end{enumerate}
\end{itemize}

\subsubsection{UC11 - Creazione workspace}
\begin{itemize}
	\item \textbf{Attori principali:} 
	\item \textbf{Precondizioni:} 
	\begin{itemize}
		\item Il sistema è attivo e funzionante
		\item L'utente è autenticato nel sistema
        \item L'utente si trova nella dashboard iniziale
	\end{itemize}
    \item \textbf{Postcondizioni:} Un nuovo workspace viene creato con successo
	\item \textbf{Scenario principale:}
    \begin{enumerate}
        \item L'utente seleziona l'opzione per creare un nuovo workspace
        \item Il sistema presenta un form per l'inserimento dei dati del workspace
        \item L'utente inserisce il nome del workspace (\hyperref[UC11.1]{UC11.1})
        \item L'utente conferma la creazione
        \item Il sistema assegna automaticamente il ruolo di owner all'utente creatore (\hyperref[UC11.2]{UC11.2})
    \end{enumerate}
    \item \textbf{Inclusioni:} \hyperref[UC11.1]{UC11.1}, \hyperref[UC11.2]{UC11.2}
    \item \textbf{Estensioni:} \hyperref[UC11.3]{UC11.3}
\end{itemize}

\subsubsection{UC11.1 - Inserimento nome del workspace}\label{UC11.1}
\begin{itemize}
	\item \textbf{Attori principali:} 
	\item \textbf{Precondizioni:} 
	\begin{itemize}
		\item Il sistema è attivo e funzionante
		\item L'utente è autenticato nel sistema
        \item L'utente si trova nella dashboard iniziale
        \item L'utente sta creando un nuovo workspace 
	\end{itemize}
    \item \textbf{Postcondizioni:} Il workspace ha il nome assegnatogli dall'utente creatore
	\item \textbf{Scenario principale:}
    \begin{enumerate}
        \item L'utente inserisce il nome del workspace
        \item Il sistema valida il nome inserito 
    \end{enumerate}
\end{itemize}

\subsubsection{UC11.2- Assegnazione ruolo di owner al creatore del workspace}\label{UC11.2}
\begin{itemize}
	\item \textbf{Attori principali:} 
	\item \textbf{Precondizioni:} 
	\begin{itemize}
		\item Il sistema è attivo e funzionante
		\item L'utente è autenticato nel sistema
        \item L'utente si trova nella dashboard iniziale
        \item L'utente sta creando un nuovo workspace 
	\end{itemize}
    \item \textbf{Postcondizioni:} Il sistema assegna automaticamente il ruolo di owner del workspace all'utente che lo sta creando
	\item \textbf{Scenario principale:}
    \begin{enumerate}
        \item Il sistema crea automaticamente il ruolo di owner per il nuovo workspace
        \item Il sistema associa l'utente creatore al workspace con il ruolo di owner
        \item Il sistema registra l'utente come proprietario ufficiale del workspace
    \end{enumerate}
	
\end{itemize}

\subsubsection{UC11.3 - Creazione del workspace non riuscita}\label{UC11.3}
\begin{itemize}
	\item \textbf{Attori principali:} 
	\item \textbf{Precondizioni:} 
	\begin{itemize}
		\item Il sistema è attivo e funzionante
		\item L'utente è autenticato nel sistema
        \item L'utente si trova nella dashboard iniziale
        \item L'utente ha tentato la creazione di un nuovo workspace
	\end{itemize}
    \item \textbf{Postcondizioni:} 
    \begin{itemize}
        \item La creazione del workspace viene annullata
        \item Nessun nuovo workspace viene aggiunto al sistema
    \end{itemize}
	\item \textbf{Scenario principale:}
    \begin{enumerate}
        \item Il sistema non permette di terminare il processo di creazione di un workspace poichè esiste già un workspace con lo stesso nome creato dal medesimo utente o il nome inserito non è valido
        \item Il sistema mantiene aperto il form di creazione permettendo all'utente di correggere l'errore o annullare l'operazione
    \end{enumerate}
\end{itemize}

\subsubsection{UC13 - Visualizzazione lista dei ruoli di un workspace}
\begin{itemize}
	\item \textbf{Attori principali:} 
	\item \textbf{Precondizioni:} 
	\begin{itemize}
		\item Il sistema è attivo e funzionante
		\item L'utente è autenticato nel sistema
        \item L'utente desidera vedere la lista di tutti i ruoli presenti in un workspace
	\end{itemize}
    \item \textbf{Postcondizioni:} L'utente visualizza una lista con l'elenco di tutti i ruoli di un workspace
	\item \textbf{Scenario principale:}
        \item L'utente apre la lista con l'elenco di tutti i ruoli presenti al momento all'interno del workspace
        \item L'utente un singolo ruolo dalla lista per visualizzarne la descrizione (\hyperref[UC13.1]{UC13.1})
    \item \textbf{Inclusioni:} \hyperref[UC13.1]{UC13.1}
\end{itemize}

\subsubsection{UC13.1 - Visualizzazione singolo elemento lista ruoli di un workspace}\label{UC13.1}
\begin{itemize}
	\item \textbf{Attori principali:} 
	\item \textbf{Precondizioni:} 
	\begin{itemize}
		\item Il sistema è attivo e funzionante
		\item L'utente è autenticato nel sistema
        \item L'utente ha selezionato un singolo ruolo dalla lista di tutti i ruoli presenti in un workspace
	\end{itemize}
    \item \textbf{Postcondizioni:} L'utente visualizza singolo ruolo di un workspace
	\item \textbf{Scenario principale:} L'utente visualizza nome e relativa descrizione di un singolo ruolo presente all'interno del workspace	
\end{itemize}

\subsubsection{UC14 - Visualizzazione lista utenti di un workspace}
\begin{itemize}
	\item \textbf{Attori principali:} 
	\item \textbf{Precondizioni:} 
	\begin{itemize}
		\item Il sistema è attivo e funzionante
		\item L'utente è autenticato nel sistema
        \item L'utente vuole vedere la lista di tutti gli utenti presenti in un workspace
	\end{itemize}
    \item \textbf{Postcondizioni:} L'utente visualizza la lista di utenti di un singolo workspace
	\item \textbf{Scenario principale:} L'utente visualizza l'username di ogni singolo utente presente all'interno del workspace
	
\end{itemize}

\subsubsection{UC15 - Rimozione di un utente da un workspace}
\begin{itemize}
	\item \textbf{Attori principali:} 
	\item \textbf{Precondizioni:} 
	\begin{itemize}
		\item Il sistema è attivo e funzionante
		\item L'utente è autenticato nel sistema
        \item L'utente sta visualizzando la lista di tutti gli utenti presenti in un workspace
        \item L'utente ha selezionato un utente specifico dalla lista
        \item Esiste almeno un altro utente nel workspace oltre all'owner
        \item L'utente da rimuovere non è l'owner del workspace
	\end{itemize}
    \item \textbf{Postcondizioni:} 
    \begin{itemize}
        \item L'utente selezionato viene rimosso dal workspace
        \item L'utente rimosso perde l'accesso al workspace e a tutte le sue risorse
        \item Tutti i ruoli assegnati all'utente nel workspace vengono revocati
    \end{itemize}
	\item \textbf{Scenario principale:}
    \begin{enumerate}
        \item L'utente seleziona un utente specifico dalla lista 
        \item L'utente seleziona l'opzione "Rimuovi utente dal workspace"
        \item L'utente conferma l'operazione di rimozione
        \item Il sistema rimuove l'utente dal workspace e aggiorna la lista degli utenti del workspace
    \end{enumerate}
\end{itemize}

\subsubsection{UC19 - Visualizza lista tag raccolta}
\begin{itemize}
	\item \textbf{Attori principali:} Utente
	\item \textbf{Precondizioni:} 
	\begin{itemize}
		\item Il sistema è attivo e funzionante
		\item L'utente è stato riconosciuto dal sistema come Utente
		\item L'utente ha selezionato un workspace
	\end{itemize}
	\item \textbf{Postcondizioni:} Il sistema mostra a schermo l'elenco dei tag raccolta definiti nel workspace corrente
	\item \textbf{Scenario principale:}
	\begin{enumerate}
		\item L'utente seleziona l'opzione per visualizzare i tag raccolta del workspace corrente
		\item Il sistema recupera l'elenco dei tag raccolta associati al workspace selezionato
		\item Il sistema mostra a schermo la lista dei tag raccolta
	\end{enumerate}
	\item \textbf{Scenario alternativo:}
	\begin{enumerate}
		\item Il workspace non contiene alcun tag raccolta configurato ->Vedi [UC Nessun tag raccolta presente oppure elenco vuoto], il sistema mostra un messaggio informativo: "Nessun tag raccolta presente nel workspace corrente"
	\end{enumerate}
	\item \textbf{Inclusioni:}
	\begin{enumerate}
		\item UC19.1 - Visualizza elemento lista tag raccolta repository
	\end{enumerate}
\end{itemize}


\subsubsection{UC19.1 - Visualizza elemento lista tag raccolta}
\begin{itemize}
	\item \textbf{Attori principali:} Utente
	\item \textbf{Precondizioni:} 
	\begin{itemize}
		\item Il sistema è attivo e funzionante
		\item L'utente è stato riconosciuto dal sistema come Utente
		\item L'utente sta visualizzando la lista dei tag raccolta del workspace
	\end{itemize}
	\item \textbf{Postcondizioni:} Viene visualizzato, nel singolo elemento della lista dei tag raccolta, il nome del tag raccolta
	\item \textbf{Scenario principale:}
	\begin{enumerate}
		\item Il sistema recupera le informazioni relative a un singolo tag raccolta presente nella lista
		\item Il sistema mostra il nome identificativo del tag raccolta per la visualizzazione nella lista
	\end{enumerate}
	\item \textbf{Inclusioni:}
	\begin{enumerate}
		\item UC19.1.1 - Visualizza Nome tag raccolta
	\end{enumerate}
\end{itemize}

\subsubsection{UC19.1.1 - Visualizza Nome tag raccolta}

\subsubsection{UC20 - Crea tag raccolta}
\begin{itemize}
	\item \textbf{Attori principali:} Utente
	\item \textbf{Precondizioni:} 
	\begin{itemize}
		\item Il sistema è attivo e funzionante
		\item L'utente è stato riconosciuto dal sistema come Utente
		\item L'utente ha selezionato un workspace
	\end{itemize}
	\item \textbf{Postcondizioni:} 
	\begin{itemize}
		\item Un nuovo tag raccolta è stato creato nel sistema
		\item Il tag raccolta è associato al workspace corrente
		\item Il nuovo tag raccolta è disponibile nell'elenco dei tag del workspace
	\end{itemize}
	\item \textbf{Scenario principale:}
	\begin{enumerate}
		\item L'utente seleziona l'opzione per creare un nuovo tag raccolta
		\item Il sistema richiede l'inserimento del nome del tag raccolta
		\item L'utente inserisce il nome del tag raccolta
		\item L'utente conferma la creazione
		\item Il sistema verifica che il nome inserito sia valido e non già esistente nel workspace
		\item Il sistema crea il nuovo tag raccolta e lo associa al workspace corrente
		\item Il sistema mostra un messaggio di conferma dell'avvenuta creazione del tag raccolta
	\end{enumerate}
	\item \textbf{Inclusioni:}
	\begin{itemize}
		\item UC 20.1 - Inserisci nome tag raccolta
	\end{itemize}
		\item \textbf{Scenario alternativo:}
	\begin{enumerate}
		\item Il tag raccolta non viene creato perchè il nome del tag raccolta inserito è già esistente nel workspace o non è valido (vuoto, caratteri non ammessi) ->[vedi UC21 - Creazione tag raccolta non riuscita]
	\end{enumerate}
\end{itemize}

\subsubsection{UC 20.1 - Inserisci nome tag raccolta}
\begin{itemize}
	\item \textbf{Attori principali:} Utente
	\item \textbf{Precondizioni:} 
	\begin{itemize}
		\item Il sistema è attivo e funzionante
		\item L'utente è stato riconosciuto dal sistema come Utente
		\item L'utente ha selezionato un workspace
	\end{itemize}
	\item \textbf{Postcondizioni:} 
	\begin{itemize}
		\item Il nome del tag raccolta è stato acquisito dal sistema ed è disponibile per la validazione e la creazione del tag
	\end{itemize}
	\item \textbf{Scenario principale:}
	\begin{enumerate}
		\item Il sistema richiede l'inserimento del nome del tag raccolta
		\item L'utente inserisce il nome del tag raccolta
		\item Il sistema acquisisce il valore inserito
	\end{enumerate}
\end{itemize}

\subsubsection{UC21 - Creazione tag raccolta non riuscita}
\begin{itemize}
	\item \textbf{Attori principali:} Utente
	\item \textbf{Precondizioni:} 
	\begin{itemize}
		\item L'utente ha avviato la procedura di creazione di un nuovo tag raccolta (UC20)
		\item L'utente ha tentato di confermare la creazione del tag raccolta
		\item Si è verificata una delle seguenti condizioni:
		\begin{itemize}
			\item il nome del tag raccolta inserito è già esistente nel workspace
			\item il nome del tag raccolta inserito non è valido (vuoto, contiene solo spazi, contiene caratteri non ammessi)
		\end{itemize}
	\end{itemize}
	\item \textbf{Postcondizioni:} 
	\begin{itemize}
		\item La creazione del tag raccolta viene annullata
		\item Nessun nuovo tag raccolta viene aggiunto al workspace corrente
		\item Il sistemafornisce all'utente un messaggio di errore esplicativo
	\end{itemize}
	\item \textbf{Scenario principale:}
	\begin{enumerate}
		\item Il sistema rileva un errore di validazione sul nome del tag (nome non valido o è già esistente nel workspace)
		\item Il sistema annulla l'operazione di creazione
		\item Il sistema mostra un messaggio di errore all'utente specificando la causa del fallimento
		\item Il sistema consente all'utente di correggere il nome inserito o annullare l'operazione	
	\end{enumerate}
\end{itemize}

\subsubsection{UC22 - Eliminazione tag raccolta dal workspace}
\begin{itemize}
	\item \textbf{Attori principali:} Utente
	\item \textbf{Precondizioni:} 
	\begin{itemize}
		\item Il sistema è attivo e funzionante
		\item L'utente è stato riconosciuto dal sistema come Utente
		\item L'utente ha selezionato un workspace
		\item Esiste almeno un tag raccolta nel workspace
	\end{itemize}
	\item \textbf{Postcondizioni:} 
	\begin{itemize}
		\item Il tag raccolta selezionato è stato rimosso dal workspace
		\item Il tag raccolta è stato dissociato da tutte le repository a cui era associato
	\end{itemize}
	\item \textbf{Scenario principale:}
	\begin{enumerate}
		\item L'utente visualizza la lista dei tag raccolta del workspace
		\item L'utente seleziona il tag raccolta da eliminare
		\item Il sistema mostra un messaggio di avvertimento indicando il numero di repository coinvolte
		\item Il sistema richiede la conferma dell'operazione
		\item L'utente conferma l'eliminazione
		\item Il sistema rimuove il tag raccolta dal workspace
		\item Il sistema dissocia il tag raccolta da tutte le repository a cui era associato
		\item Il sistema aggiorna la visualizzazione e mostra un messaggio di conferma dell'avvenuta eliminazione
	\end{enumerate}
		\item \textbf{Scenario alternativo:}
	\begin{itemize}
		\item \textbf{Annullamento dell'operazione:} 
			\begin{enumerate}
				\item \textbf{Condizione:}l'utente annulla l'operazione nella fase di conferma
				\item \textbf{Flusso:} 
				\begin{enumerate}
					\item Il sistema interrompe il processo di eliminazione del tag
					\item Il tag raccolta rimane invariato 
				\end{enumerate}
			\end{enumerate}
	\end{itemize}
\end{itemize}

\subsubsection{UC23 - Assegnazione tag raccolta a repository}
\begin{itemize}
	\item \textbf{Attori principali:} Utente
	\item \textbf{Precondizioni:} 
	\begin{itemize}
		\item Il sistema è attivo e funzionante
		\item L'utente è stato riconosciuto dal sistema come Utente
		\item L'utente ha selezionato un workspace
		\item Esiste almeno una repository nel workspace
		\item Esiste almeno un tag raccolta nel workspace
	\end{itemize}
	\item \textbf{Postcondizioni:} 
	\begin{itemize}
		\item Il tag raccolta selezionato è associato alla repository selezionata
		\item L'associazione è persistita dal sistema
		\item La repository riflette l'insieme aggiornato dei tag raccolta associati
	\end{itemize}
	\item \textbf{Scenario principale:}
	\begin{enumerate}
		\item L'utente visualizza i dettagli di una repository o la lista delle repository
		\item L'utente seleziona l'opzione per gestire i tag raccolta della repository
		\item Il sistema recupera e mostra l'elenco dei tag raccolta disponibili nel workspace
		\item L'utente seleziona uno o più tag raccolta da associare alla repository
		\item L'utente conferma l'operazione
		\item Il sistema associa i tag selezionati alla repository
		\item Il sistema aggiorna i tag assegnati alla repository e conferma l'operazione
	\end{enumerate}
		\item \textbf{Scenario alternativo:}
	\begin{enumerate}
		\item \textbf{Tag già associato alla repository:} 
			\begin{itemize}
				\item \textbf{Condizione:}uno o più tag selezionati risultano già associati alla repository
				\item \textbf{Flusso:} 
				\begin{enumerate}
					\item Il sistema ignora l'associazione duplicata
					\item Il sistema mostra il messaggio: “Uno o più tag selezionati sono già associati alla repository”
				\end{enumerate}
			\end{itemize}
			\item \textbf{Nessun tag raccolta disponibile nel workspace:} 
			\begin{itemize}
				\item \textbf{Condizione:}il workspace non contiene alcun tag raccolta
				\item \textbf{Flusso:} 
				\begin{enumerate}
					\item Il sistema interrompe l'operazione di assegnazione
					\item Il sistema mostra il messaggio: “Nessun tag raccolta disponibile. Creare un tag prima di procedere”
				\end{enumerate}
			\end{itemize}
	\end{enumerate}
\end{itemize}

\subsubsection{UC24 -Rimozione tag raccolta da repository}
\begin{itemize}
	\item \textbf{Attori principali:} Utente
	\item \textbf{Precondizioni:} 
	\begin{itemize}
		\item Il sistema è attivo e funzionante
		\item L'utente è stato riconosciuto dal sistema come Utente
		\item L'utente ha selezionato un workspace
		\item L'utente ha selezionato una repository del workspace
		\item La repository selezionata ha almeno un tag raccolta associatoe
	\end{itemize}
	\item \textbf{Postcondizioni:} 
	\begin{itemize}
		\item Il tag raccolta selezionato viene dissociato dalla repository selezionata
	\end{itemize}
	\item \textbf{Scenario principale:}
	\begin{enumerate}
		\item L'utente visualizza i dettagli di una repository tra cui i tag raccolta assegnati
		\item L'utente seleziona l'opzione per gestire i tag raccolta della repository
		\item Il sistema mostra la lista dei tag raccolta attualmente assegnati alla repository
		\item L'utente seleziona uno o più tag raccolta da rimuovere dalla repository
		\item L'utente conferma la rimozione
		\item Il sistema dissocia i tag selezionati dalla repository
		\item Il sistema aggiorna la repository e mostra i tag rimanenti
	\end{enumerate}
		\item \textbf{Scenario alternativo:}
	\begin{enumerate}
		\item \textbf{Nessun tag raccolta associato:} 
			\begin{itemize}
				\item \textbf{Condizione:}la repository non ha alcun tag raccolta associato
				\item \textbf{Flusso:} 
				\begin{enumerate}
					\item Il sistema interrompe l'operazione
					\item Il sistema mostra un messaggio informativo: "Nessun tag assegnato a questa repository"
				\end{enumerate}
			\end{itemize}
			\item \textbf{Annullamento dell'operazione:} 
			\begin{itemize}
				\item \textbf{Condizione:}l'utente annulla la rimozione nella fase di conferma
				\item \textbf{Flusso:} 
				\begin{enumerate}
					\item Il sistema interrompe l'operazione
					\item Le associazioni dei tag raccolta alla repository rimangono invariate
				\end{enumerate}
			\end{itemize}
	\end{enumerate}
\end{itemize}

\subsubsection{UC25 - Visualizza lista repository del workspace}
\begin{itemize}
	\item \textbf{Attori principali:} Utente
	\item \textbf{Precondizioni:} 
	\begin{itemize}
		\item Il sistema è attivo e funzionante
		\item L'utente è stato riconosciuto dal sistema come Utente
		\item L'utente ha selezionato un workspace
	\end{itemize}
	\item \textbf{Postcondizioni:} 
	\begin{itemize}
		\item Il sistema presenta all'utente l'elenco delle repository associate al workspace corrente
		\item Per ciascuna repository sono visualizzate delle informazioni sintetiche
	\end{itemize}
	\item \textbf{Scenario principale:}
	\begin{enumerate}
		\item L'utente seleziona l'opzione per visualizzare l'elenco delle repository del workspace corrente
		\item Il sistema recupera l'elenco delle repository associate al workspace 
		\item Il sistema mostra la lista delle repository
		\item Per ciascuna repository, il sistema visualizza delle informazioni sintetiche
	\end{enumerate}
	\item \textbf{Inclusioni:}
	\begin{itemize}
		\item UC 25.1 - Visualizza nome repository
		\item UC 25.2 - Visualizza data ultima scansione
		\item UC 25.3 - Visualizza presenza criticità
		\item UC 25.4 - Visualizza voto documentazione
		\item UC 25.5 - Visualizza percentuale test coverage
	\end{itemize}
	\item \textbf{Scenario alternativo:}
	\begin{enumerate}
		\item \textbf{Nessuna repository presente nel workspace:} 
			\begin{itemize}
				\item \textbf{Condizione:}il workspace corrente non contiene repository associate
				\item \textbf{Flusso:} 
				\begin{enumerate}
					\item Il sistema mostra un messaggio informativo:“Nessuna repository presente nel workspace corrente”
				\end{enumerate}
			\end{itemize}
			\item \textbf{Annullamento dell'operazione:} 
			\begin{itemize}
				\item \textbf{Condizione:}l'utente annulla la rimozione nella fase di conferma
				\item \textbf{Flusso:} 
				\begin{enumerate}
					\item Il sistema interrompe l'operazione
					\item Le associazioni dei tag raccolta alla repository rimangono invariate
				\end{enumerate}
			\end{itemize}
	\end{enumerate}
\end{itemize}

\subsubsection{UC 25.1 - Visualizza nome repository}
\begin{itemize}
	\item \textbf{Attori principali:} Utente
	\item \textbf{Precondizioni:} 
	\begin{itemize}
		\item Il sistema è attivo e funzionante
		\item L'utente è stato riconosciuto dal sistema come Utente
		\item L'utente ha selezionato un workspace 
		\item L'utente sta visualizzando la lista delle repository del workspace corrente
		\item La repository di riferimento esiste nel sistema
	\end{itemize}
	\item \textbf{Postcondizioni:} 
	\begin{itemize}
		\item Viene visualizzato il nome identificativo della repository
	\end{itemize}
	\item \textbf{Scenario principale:}
	\begin{enumerate}
		\item Il sistema recupera il nome identificativo della repository
		\item Il sistema mostra il nome della repository nella lista
	\end{enumerate}
\end{itemize}

\subsubsection{UC 25.2 - Visualizza data ultima scansione}
\begin{itemize}
	\item \textbf{Attori principali:} Utente
	\item \textbf{Precondizioni:} 
	\begin{itemize}
		\item Il sistema è attivo e funzionante
		\item L'utente è stato riconosciuto dal sistema come Utente
		\item L'utente ha selezionato un workspace 
		\item L'utente sta visualizzando la lista delle repository del workspace corrente
		\item La repository di riferimento esiste nel sistema
	\end{itemize}
	\item \textbf{Postcondizioni:} 
	\begin{itemize}
		\item Viene visualizzata la data e l'ora dell'ultima scansione effettuata sulla repository
	\end{itemize}
	\item \textbf{Scenario principale:}
	\begin{enumerate}
		\item Il sistema recupera la data e l'ora dell'ultima analisi completata sulla repository
		\item Il sistema mostra la data e l'ora dell'ultima scansione nella lista
	\end{enumerate}
	\item \textbf{Scenario alternativo:}
	\begin{enumerate}
		\item \textbf{Nessuna repository presente nel workspace:} 
			\begin{itemize}
				\item \textbf{Condizione:}La repository non ha mai subito una scansione
				\item \textbf{Flusso:} 
				\begin{enumerate}
					\item Il sistema mostra il messaggio:“Nessuna scansione effettuata”
				\end{enumerate}
			\end{itemize}
			\item \textbf{Annullamento dell'operazione:} 
			\begin{itemize}
				\item \textbf{Condizione:}l'utente annulla la rimozione nella fase di conferma
				\item \textbf{Flusso:} 
				\begin{enumerate}
					\item Il sistema interrompe l'operazione
					\item Le associazioni dei tag raccolta alla repository rimangono invariate
				\end{enumerate}
			\end{itemize}
	\end{enumerate}
\end{itemize}

\subsubsection{UC 25.3 - Visualizza presenza criticità}
\begin{itemize}
	\item \textbf{Attori principali:} Utente
	\item \textbf{Precondizioni:} 
	\begin{itemize}
		\item Il sistema è attivo e funzionante
		\item L'utente è stato riconosciuto dal sistema come Utente
		\item L'utente ha selezionato un workspace 
		\item L'utente sta visualizzando la lista delle repository del workspace corrente
		\item La repository di riferimento esiste nel sistema
		\item La repository ha subito almeno una scansione ?????
	\end{itemize}
	\item \textbf{Postcondizioni:} 
	\begin{itemize}
		\item Viene visualizzato un indicatore che rappresenta la presenza e la gravità massima delle criticità rilevate nella repository
	\end{itemize}
	\item \textbf{Scenario principale:}
	\begin{enumerate}
		\item Il sistema recupera l'esito dell'ultima analisi della repository
		\item Il sistema determina la gravità massima delle vulnerabilità rilevate
		\item Il sistema visualizza un indicatore sintetico basato sull'indice CVSS, che identifica i seguenti livelli di criticità: Critico, Alto, Medio, Basso, Nessuna criticità
	\end{enumerate}
		\item \textbf{Scenario alternativo:}
	\begin{itemize}
		\item \textbf{Analisi di sicurezza non disponibile:} 
			\begin{itemize}
				\item \textbf{Condizione:}l'analisi delle vulnerabilità di sicurezza non è stata eseguita o non è andata a buon fine
				\item \textbf{Flusso:} 
				\begin{enumerate}
					\item Il sistema mostra il messaggio: “Analisi sicurezza non disponibile”
				\end{enumerate}
			\end{itemize}
	\end{itemize}
\end{itemize}

\subsubsection{UC 25.4 - Visualizza voto documentazione}
\begin{itemize}
	\item \textbf{Attori principali:} Utente
	\item \textbf{Precondizioni:} 
	\begin{itemize}
		\item Il sistema è attivo e funzionante
		\item L'utente è stato riconosciuto dal sistema come Utente
		\item L'utente ha selezionato un workspace 
		\item L'utente sta visualizzando la lista delle repository del workspace corrente
		\item La repository di riferimento esiste nel sistema
		\item La repository ha subito almeno una scansione ?????
	\end{itemize}
	\item \textbf{Postcondizioni:} 
	\begin{itemize}
		\item Viene visualizzato il voto relativo alla qualità della documentazione della repository
	\end{itemize}
	\item \textbf{Scenario principale:}
	\begin{enumerate}
		\item Il sistema recupera l'esito dell'ultima analisi della repository
		\item Il sistema calcola o recupera il voto di qualità della documentazione sulla base di criteri quali: presenza del file README, presenza di documentazione tecnica, presenza e qualità dei commenti nel codice, conformità agli standard di documentazione definiti, 		Il sistema mostra un voto numerico/ indicatore qualitativo che rappresenta la completezza e la qualità della documentazione presente nella repository, calcolato in base alla presenza di file README, documentazione tecnica, commenti nel codice e conformità agli standard di documentazione definiti
		\item Il sistema visualizza un voto numerico che rappresenta la qualità complessiva della documentazione
	\end{enumerate}
	\item \textbf{Scenario alternativo:}
	\begin{itemize}
		\item \textbf{Documentazione assente:} 
			\begin{itemize}
				\item \textbf{Condizione:}La repository non contiene documentazione analizzabile
				\item \textbf{Flusso:} 
				\begin{enumerate}
					\item Il sistema mostra il messaggio: “Documentazione assente”
				\end{enumerate}
			\end{itemize}
		\item \textbf{Analisi documentazione non disponibile:} 
			\begin{itemize}
				\item \textbf{Condizione:}l'analisi della documentazione non è stata eseguita o non è andata a buon fine
				\item \textbf{Flusso:} 
				\begin{enumerate}
					\item Il sistema mostra il messaggio: “Analisi documentazione non disponibile”
				\end{enumerate}
			\end{itemize}
	\end{itemize}
\end{itemize}

\subsubsection{UC 25.5 - Visualizza percentuale test coverage}
\begin{itemize}
	\item \textbf{Attori principali:} Utente
	\item \textbf{Precondizioni:} 
	\begin{itemize}
		\item Il sistema è attivo e funzionante
		\item L'utente è stato riconosciuto dal sistema come Utente
		\item L'utente ha selezionato un workspace 
		\item L'utente sta visualizzando la lista delle repository del workspace corrente
		\item La repository di riferimento esiste nel sistema
		\item La repository ha subito almeno una scansione ?????
	\end{itemize}
	\item \textbf{Postcondizioni:} 
	\begin{itemize}
		\item Viene visualizzato il test coverage della repository
	\end{itemize}
	\item \textbf{Scenario principale:}
	\begin{enumerate}
		\item Il sistema recupera il test coverage calcolato durante l'ultima analisi della repository (il test coverage viene calcolato come rapporto tra linee di codice eseguite durante i test automatici e totale delle linee di codice eseguibili presenti nella repository)
		\item Il sistema mostra il valore percentuale del test coverage
	\end{enumerate}
		\item \textbf{Scenario alternativo:}
	\begin{itemize}
		\item \textbf{Nessun test configurato:} 
			\begin{itemize}
				\item \textbf{Condizione:}La repository non contiene file di test o configurazioni per l'esecuzione dei test
				\item \textbf{Flusso:} 
				\begin{enumerate}
					\item Il sistema mostra il messaggio: “Test coverage: 0\% - Nessun test configurato”
				\end{enumerate}
			\end{itemize}
		\item \textbf{Analisi test coverage non disponibile:} 
			\begin{itemize}
				\item \textbf{Condizione:}l'analisi del test coverage non è stata eseguita o non è andata a buon fine
				\item \textbf{Flusso:} 
				\begin{enumerate}
					\item Il sistema mostra il messaggio: “Test coverage non disponibile”
				\end{enumerate}
			\end{itemize}
		\item \textbf{Linguaggio non supportato:} 
			\begin{itemize}
				\item \textbf{Condizione:} il linguaggio di programmazione utilizzato nella repository non è supportato per l'analisi del test coverage
				\item \textbf{Flusso:} 
				\begin{enumerate}
					\item Il sistema mostra il messaggio: “Analisi test coverage non supportata per questo linguaggio”
				\end{enumerate}
			\end{itemize}
	\end{itemize}
\end{itemize}

\subsubsection{UC26 - Aggiungi repository}
\begin{itemize}
	\item \textbf{Attori principali:} Utente
	\item \textbf{Precondizioni:} 
	\begin{itemize}
		\item Il sistema è attivo e funzionante
		\item L'utente è stato riconosciuto dal sistema come Utente
		\item L'utente ha selezionato un workspace
	\end{itemize}
	\item \textbf{Postcondizioni:} 
	\begin{itemize}
		\item La repository indicata è associata al workspace corrente
		\item Se la repository è privata, il sistema ha validato e memorizzato un token di accesso valido
		\item La repository risulta disponibile per le successive operazioni di analisi e visualizzazione
		\item Le informazioni di accesso alla repository sono memorizzate dal sistema
	\end{itemize}
	\item \textbf{Scenario principale:}
	\begin{enumerate}
		\item L'utente seleziona l'opzione per aggiungere una nuova repository
		\item Il sistema mostra un form per l'inserimento dei dati della repository
		\item L'utente inserisce il link della repository GitHub
		\item (Opzionale) L'utente inserisce il PAT(Personal Access Token) per l'accesso alla repository GitHub, se la repository è privata
		\item L'utente conferma l'operazione di aggiunta
		\item Il sistema valida il formato del link della repository
		\item Il sistema verifica l'accesso alla repository tramite PAT, se privata
		\item Il sistema associa la repository al workspace corrente
		\item Il sistema mostra un messaggio di conferma dell'avvenuta aggiunta
	\end{enumerate}
	\item \textbf{Inclusioni:}
	\begin{itemize}
		\item UC 26.1 - Inserisci link repository GitHub
		\item UC 26.2 - Inserisci PAT repository GitHub
	\end{itemize}
	\item \textbf{Scenario alternativo:}
	\begin{itemize}
		\item \textbf{UC 27 - Errore aggiunta repository:} errore se i dati inseriti non sono validi oppure l'accesso alla repository non è consentito 
	\end{itemize}
\end{itemize}

\subsubsection{UC 26.1 - Inserisci link repository GitHub}
\begin{itemize}
	\item \textbf{Attori principali:} Utente
	\item \textbf{Precondizioni:} 
	\begin{itemize}
		\item Il sistema è attivo e funzionante
		\item L'utente è stato riconosciuto dal sistema come Utente
		\item L'utente ha selezionato un workspace
		\item L'utente ha avviato la procedura di aggiunta di una nuova repository
	\end{itemize}
	\item \textbf{Postcondizioni:} 
	\begin{itemize}
		\item Il link della repository GitHub è acquisito dal sistema ed è disponibile per le successive fasi di validazione
	\end{itemize}
	\item \textbf{Scenario principale:}
	\begin{enumerate}
		\item Il sistema mostra un campo di input per l'inserimento del link della repository GitHub
		\item L'utente inserisce l'URL completo della repository GitHub
		\item Il sistema acquisisce e memorizza il link inserito
	\end{enumerate}
\end{itemize}

\subsubsection{UC 26.2 - Inserisci PAT repository GitHub}
\begin{itemize}
	\item \textbf{Attori principali:} Utente
	\item \textbf{Precondizioni:} 
	\begin{itemize}
		\item Il sistema è attivo e funzionante
		\item L'utente è stato riconosciuto dal sistema come Utente
		\item L'utente ha selezionato un workspace
		\item L'utente ha avviato la procedura di aggiunta di una nuova repository
		\item L'utente ha inserito il link della repository GitHub
	\end{itemize}
	\item \textbf{Postcondizioni:} 
	\begin{itemize}
		\item Il token di accesso alla repository GitHub, è acquisito dal sistema ed è disponibile per la verifica dell'accesso, se richiesto
	\end{itemize}
	\item \textbf{Scenario principale:}
	\begin{enumerate}
		\item Il sistema mostra un campo di input per l'inserimento del token di accesso alla repository GitHub
		\item L'utente inserisce un Personal Access Token (PAT) valido per l'accesso alla repository
		\item Il sistema acquisisce e memorizza il token inserito
	\end{enumerate}
		\item \textbf{Scenario alternativo:}
	\begin{itemize}
		\item \textbf{Repository pubblica:} 
			\begin{itemize}
				\item \textbf{Condizione:} La repository è pubblica e non richiede autenticazione
				\item \textbf{Flusso:} 
				\begin{enumerate}
					\item l'utente può omettere l'inserimento del token e procedere con la conferma dell'operazione
				\end{enumerate}
			\end{itemize}
	\end{itemize}
\end{itemize}

\subsubsection{UC27 - Erore aggiunta repository}
\begin{itemize}
	\item \textbf{Attori principali:} Utente
	\item \textbf{Precondizioni:} 
	\begin{itemize}
		\item Il sistema è attivo e funzionante
		\item L'utente è stato riconosciuto dal sistema come Utente
		\item L'utente ha selezionato un workspace
		\item L'utente ha avviato la procedura di aggiunta di una nuova repository
		\item L'utente ha confermato l'aggiunta della repository
		\item Si è verificata almeno una delle seguenti condizioni:
		\begin{itemize}
			\item Il link della repository inserito non è valido (formato errato, repository inesistente)
			\item Il token inserito non è valido o non ha i permessi necessari
			\item La repository è già presente nel workspace
			\item Si è verificato un errore di connessione con GitHub
			\item Il branch develop non esiste
		\end{itemize}
	\end{itemize}
	\item \textbf{Postcondizioni:} 
	\begin{itemize}
		\item L'aggiunta della repository viene annullata
		\item Nessuna nuova repository viene aggiunta al workspace
		\item Il sistema mostra un messaggio di errore specifico all'utente
		\item Il form di aggiunta rimane aperto, permettendo di correggere l'errore o annullare l'operazione
	\end{itemize}
	\item \textbf{Scenario principale:}
	\begin{enumerate}
		\item Il sistema valida il link e il token della repository fornito
		\item Il sistema rileva un errore durante la validazione o la connessione
		\item Il sistema annulla l'operazione di aggiunta
		\item Il sistema mostra un messaggio di errore specifico, a seconda della causa:
		\begin{itemize}
			\item \textbf{Link repository non valido}: verifica il formato dell'URL
			\item \textbf{Repository non trovata}: verifica che l'URL sia corretto, che la repository esista
			\item \textbf{Token non valido}: verifica il PAT
			\item \textbf{Permessi insufficienti}: il PAT non ha accesso alla repository
			\item \textbf{Repository già presente}: repository già presente nel workspace
			\item \textbf{Errore di connessione}: rrrore di connessione con GitHub, riprova più tardi
			\item \textbf{Branch non trovato}: il branch develop non esiste nella repository
		\end{itemize}
		\item Il sistema mantiene aperto il form di aggiunta per permettere correzioni o annullamento
	\end{enumerate}
	\item \textbf{Scenario alternativo:}
	\begin{itemize}
		\item \textbf{Annullamento dell'operazione:} 
			\begin{itemize}
				\item \textbf{Condizione:} l'utente decide di chiudere il form senza correggere i dati
				\item \textbf{Flusso:} 
				\begin{enumerate}
					\item Il sistema chiude il form di aggiunta
					\item L'utente torna alla visualizzazione precedente del workspace
				\end{enumerate}
			\end{itemize}
	\end{itemize}
\end{itemize}


\newpage
% ============================================
% UC33 - VISUALIZZA DETTAGLIO REPOSITORY
% ============================================
\subsubsection{UC33 - Visualizza dettaglio repository} \label{UC33}

\begin{figure}[H]
	\centering
	\includegraphics[width=0.7\textwidth]{../Assets/AdR/UC33ext.png}
	\caption{UC33 - Visualizza dettaglio repository}
	\label{fig:UC33ext}
\end{figure}

\begin{itemize}
    \item \textbf{Attori principali:} Utente
    \item \textbf{Precondizioni:}
    \begin{itemize}
        \item Il sistema è attivo e funzionante
        \item L'utente è stato riconosciuto dal sistema come Utente
        \item L'utente ha selezionato un workspace
        \item L'utente ha selezionato una repository presente nel workspace
        \item La repository selezionata ha subito almeno una scansione completata con successo
    \end{itemize}
    \item \textbf{Postcondizioni:}
    \begin{itemize}
        \item Il sistema presenta all'utente la dashboard completa della repository
        \item L'utente può visualizzare tutte le sezioni di analisi disponibili (test, informazioni tecniche, sicurezza, documentazione, qualità del codice)
    \end{itemize}
    \item \textbf{Scenario principale:}
    \begin{enumerate}
        \item L'utente seleziona una repository dalla lista delle repository del workspace
        \item Il sistema recupera i dati dell'ultima scansione effettuata sulla repository
        \item Il sistema visualizza la sezione di analisi dei test (\hyperref[UC33.1]{UC33.1})
        \item Il sistema visualizza la sezione delle informazioni tecniche (\hyperref[UC33.2]{UC33.2})
        \item Il sistema visualizza la sezione di analisi della sicurezza (\hyperref[UC33.3]{UC33.3})
        \item Il sistema visualizza la sezione di analisi della documentazione (\hyperref[UC33.4]{UC33.4})
        \item Il sistema visualizza la sezione della qualità del codice (\hyperref[UC33.5]{UC33.5})
    \end{enumerate}
    \item \textbf{Scenario alternativo:}
    \begin{itemize}
        \item \textbf{Nessuna scansione disponibile:}
        \begin{itemize}
            \item \textbf{Condizione:} La repository non ha mai subito una scansione o l'ultima scansione non è andata a buon fine
            \item \textbf{Flusso:}
            \begin{enumerate}
                \item Il sistema mostra un messaggio informativo: "Nessuna scansione disponibile per questa repository"
                \item Il sistema suggerisce all'utente di avviare una nuova scansione
            \end{enumerate}
        \end{itemize}
		\item \textbf{Selezione Branch:}
		\begin{itemize}
			\item \textbf{Condizione:} L'utente vuole cambiare branch di cui vedere i dettagli
			\begin{enumerate}
				\item L'utente sceglie il branch di cui vedere i dettagli tra la lista dei branch disponibili
			\end{enumerate}
		\end{itemize}
    \end{itemize}
    \item \textbf{Inclusioni:} \hyperref[UC33.1]{UC33.1}, \hyperref[UC33.2]{UC33.2}, \hyperref[UC33.3]{UC33.3}, \hyperref[UC33.4]{UC33.4}, \hyperref[UC33.5]{UC33.5}
    \item \textbf{Estensioni:} \hyperref[UC34]{UC34}, \hyperref[UC35]{UC35}
\end{itemize}

Il caso d'Uso UC33 include ulteriori casi d'uso come rappresentato nella seguente immagine:
\begin{figure}[H]
	\centering
	\includegraphics[width=0.6\textwidth]{../Assets/AdR/UC33.png}
	\caption{Inclusioni di UC33: UC33.1, UC33.2, UC33.3, UC33.4, UC33.5}
	\label{fig:UC33}
\end{figure}

% ============================================
% UC33.1 - VISUALIZZA ANALISI DEI TEST
% ============================================

\paragraph{UC33.1 - Visualizza analisi dei test} \label{UC33.1}
\begin{figure}[H]
	\centering
	\includegraphics[width=0.7\textwidth]{../Assets/AdR/UC33-1ext.png}
	\caption{UC33.1 - Visualizza analisi dei test}
	\label{fig:UC33-1ext}
\end{figure}

\begin{itemize}
    \item \textbf{Attori principali:} Utente
    \item \textbf{Precondizioni:}
    \begin{itemize}
        \item Il sistema è attivo e funzionante
        \item L'utente è stato riconosciuto dal sistema come Utente
        \item L'utente sta visualizzando il dettaglio di una repository (\hyperref[UC33]{UC33})
        \item La repository ha subito almeno una scansione che include l'analisi dei test
    \end{itemize}
    \item \textbf{Postcondizioni:}
    \begin{itemize}
        \item Il sistema presenta all'utente la sezione completa di analisi dei test
        \item L'utente può visualizzare il coverage, i test con qualità insufficiente, la percentuale di test passati e l'elenco dei test non passati
    \end{itemize}
    \item \textbf{Scenario principale:}
    \begin{enumerate}
        \item Il sistema recupera i dati dei test dall'ultima scansione della repository
        \item Il sistema mostra la percentuale di test coverage (\hyperref[UC33.1.1]{UC33.1.1})
        \item Il sistema mostra l'elenco dei test con qualità insufficiente (\hyperref[UC33.1.2]{UC33.1.2})
        \item Il sistema mostra la percentuale di test passati (\hyperref[UC33.1.3]{UC33.1.3})
        \item Il sistema mostra l'elenco dei test non passati (\hyperref[UC33.1.4]{UC33.1.4})
    \end{enumerate}
    \item \textbf{Scenario alternativo:}
    \begin{itemize}
        \item \textbf{Nessun test rilevato (\hyperref[UC33.1.5]{UC33.1.5}):}
        \begin{itemize}
            \item \textbf{Condizione:} L'ultima scansione non contiene informazioni sui test o la repository non contiene test configurati
            \item \textbf{Flusso:}
            \begin{enumerate}
                \item Il sistema mostra un messaggio informativo: "Nessun test rilevato nella repository"
                \item Il sistema suggerisce all'utente di configurare i test nella repository
            \end{enumerate}
        \end{itemize}
    \end{itemize}
    \item \textbf{Inclusioni:} \hyperref[UC33.1.1]{UC33.1.1}, \hyperref[UC33.1.2]{UC33.1.2}, \hyperref[UC33.1.3]{UC33.1.3}, \hyperref[UC33.1.4]{UC33.1.4}
    \item \textbf{Estensioni:} \hyperref[UC33.1.5]{UC33.1.5}
\end{itemize}

Il caso d'uso UC33.1 include ulteriori casi d'uso come rappresentato nella seguente immagine:
\begin{figure}[H]
	\centering
	\includegraphics[width=0.6\textwidth]{../Assets/AdR/UC33-1.png}
	\caption{Inclusioni di UC33.1: UC33.1.1, UC33.1.2, UC33.1.3, UC33.1.4}
	\label{fig:UC33-1}
\end{figure}

\subparagraph{UC33.1.1 - Visualizza test coverage}\label{UC33.1.1}
\begin{itemize}
    \item \textbf{Attori principali:} Utente
    \item \textbf{Precondizioni:}
    \begin{itemize}
        \item Il sistema è attivo e funzionante
        \item L'utente è stato riconosciuto dal sistema come Utente
        \item L'utente sta visualizzando la sezione di analisi dei test (\hyperref[UC33.1]{UC33.1})
        \item L'analisi del test coverage è stata eseguita con successo
    \end{itemize}
    \item \textbf{Postcondizioni:}
    \begin{itemize}
        \item Viene visualizzata la percentuale di copertura del codice da parte dei test
    \end{itemize}
    \item \textbf{Scenario principale:}
    \begin{enumerate}
        \item Il sistema recupera il valore di test coverage calcolato durante l'ultima scansione
        \item Il sistema visualizza la percentuale di copertura del codice (rapporto tra linee di codice eseguite durante i test e totale delle linee eseguibili)
        \item Il sistema mostra un indicatore visivo che rappresenta il livello di copertura raggiunto
    \end{enumerate}
\end{itemize}

\subparagraph{UC33.1.2 - Visualizza test con qualità insufficiente}\label{UC33.1.2}
\begin{itemize}
    \item \textbf{Attori principali:} Utente
    \item \textbf{Precondizioni:}
    \begin{itemize}
        \item Il sistema è attivo e funzionante
        \item L'utente è stato riconosciuto dal sistema come Utente
        \item L'utente sta visualizzando la sezione di analisi dei test (\hyperref[UC33.1]{UC33.1})
        \item L'analisi della qualità dei test è stata eseguita con successo
    \end{itemize}
    \item \textbf{Postcondizioni:}
    \begin{itemize}
        \item Viene visualizzato l'elenco dei test che presentano problemi di qualità
    \end{itemize}
    \item \textbf{Scenario principale:}
    \begin{enumerate}
        \item Il sistema recupera l'elenco dei test che superano le soglie di tolleranza definite
        \item Il sistema elenca i test con qualità insufficiente, indicando per ciascuno:
        \begin{itemize}
            \item Nome del test
            \item Tipo di problema rilevato (es. durata eccessiva, asserzioni mancanti, test instabili)
            \item Dettagli sulla violazione della soglia
        \end{itemize}
    \end{enumerate}
    
\end{itemize}

\subparagraph{UC33.1.3 - Visualizza percentuale test passati}\label{UC33.1.3}
\begin{itemize}
    \item \textbf{Attori principali:} Utente
    \item \textbf{Precondizioni:}
    \begin{itemize}
        \item Il sistema è attivo e funzionante
        \item L'utente è stato riconosciuto dal sistema come Utente
        \item L'utente sta visualizzando la sezione di analisi dei test (\hyperref[UC33.1]{UC33.1})
        \item I test sono stati eseguiti durante l'ultima scansione
    \end{itemize}
    \item \textbf{Postcondizioni:}
    \begin{itemize}
        \item Viene visualizzata la percentuale di test passati rispetto al totale
    \end{itemize}
    \item \textbf{Scenario principale:}
    \begin{enumerate}
        \item Il sistema recupera il numero totale di test eseguiti e il numero di test superati
        \item Il sistema calcola e visualizza il rapporto percentuale tra test superati e test totali
        \item Il sistema mostra un indicatore visivo che rappresenta il tasso di successo dei test
    \end{enumerate}
\end{itemize}

\subparagraph{UC33.1.4 - Visualizza test non passati}\label{UC33.1.4}
\begin{itemize}
    \item \textbf{Attori principali:} Utente
    \item \textbf{Precondizioni:}
    \begin{itemize}
        \item Il sistema è attivo e funzionante
        \item L'utente è stato riconosciuto dal sistema come Utente
        \item L'utente sta visualizzando la sezione di analisi dei test (\hyperref[UC33.1]{UC33.1})
        \item I test sono stati eseguiti durante l'ultima scansione
    \end{itemize}
    \item \textbf{Postcondizioni:}
    \begin{itemize}
        \item Viene visualizzato l'elenco dei test falliti con i relativi dettagli
    \end{itemize}
    \item \textbf{Scenario principale:}
    \begin{enumerate}
        \item Il sistema recupera l'elenco dei test che non sono stati superati durante l'ultima scansione
        \item Il sistema mostra l'elenco dei test falliti, indicando per ciascuno:
        \begin{itemize}
            \item Nome del test
            \item Messaggio di errore
            \item Stack trace o dettagli del fallimento
            \item File e riga di codice coinvolti
        \end{itemize}
    \end{enumerate}
    
\end{itemize}

\subparagraph{UC33.1.5 - Nessun test rilevato}\label{UC33.1.5}
\begin{itemize}
    \item \textbf{Attori principali:} Utente
    \item \textbf{Precondizioni:}
    \begin{itemize}
        \item Il sistema è attivo e funzionante
        \item L'utente è stato riconosciuto dal sistema come Utente
        \item L'utente sta visualizzando la sezione di analisi dei test (\hyperref[UC33.1]{UC33.1})
        \item La repository non contiene test configurati o l'analisi dei test non è stata eseguita
    \end{itemize}
    \item \textbf{Postcondizioni:}
    \begin{itemize}
        \item Il sistema informa l'utente dell'assenza di dati sui test
    \end{itemize}
    \item \textbf{Scenario principale:}
    \begin{enumerate}
        \item Il sistema rileva che non sono presenti informazioni sui test per la repository
        \item Il sistema mostra un messaggio informativo: "Nessun test rilevato nella repository"
        \item Il sistema suggerisce all'utente di configurare i test nella repository per abilitare l'analisi
    \end{enumerate}
\end{itemize}


% ============================================
% UC33.2 - VISUALIZZA INFORMAZIONI TECNICHE
% ============================================
\paragraph{UC33.2 - Visualizza informazioni tecniche} \label{UC33.2}
\begin{figure}[H]
	\centering
	\includegraphics[width=0.6\textwidth]{../Assets/AdR/UC33-2.png}
	\caption{Inclusioni di UC33.2: UC33.2.1, UC33.2.2, UC33.2.3}
	\label{fig:UC33-2}
\end{figure}

\begin{itemize}
    \item \textbf{Attori principali:} Utente
    \item \textbf{Precondizioni:}
    \begin{itemize}
        \item Il sistema è attivo e funzionante
        \item L'utente è stato riconosciuto dal sistema come Utente
        \item L'utente sta visualizzando il dettaglio di una repository (\hyperref[UC33]{UC33})
        \item La repository ha subito almeno una scansione che include l'analisi delle informazioni tecniche
    \end{itemize}
    \item \textbf{Postcondizioni:}
    \begin{itemize}
        \item Il sistema presenta all'utente la sezione completa delle informazioni tecniche della repository
        \item L'utente può visualizzare i linguaggi, le librerie e i framework utilizzati nella repository
    \end{itemize}
    \item \textbf{Scenario principale:}
    \begin{enumerate}
        \item Il sistema recupera le informazioni tecniche dall'ultima scansione della repository
        \item Il sistema visualizza la lista dei linguaggi di programmazione utilizzati (\hyperref[UC33.2.1]{UC33.2.1})
        \item Il sistema visualizza la lista delle librerie utilizzate (\hyperref[UC33.2.2]{UC33.2.2})
        \item Il sistema visualizza la lista dei framework utilizzati (\hyperref[UC33.2.3]{UC33.2.3})
    \end{enumerate}
    \item \textbf{Inclusioni:} \hyperref[UC33.2.1]{UC33.2.1}, \hyperref[UC33.2.2]{UC33.2.2}, \hyperref[UC33.2.3]{UC33.2.3}
\end{itemize}



% --- LINGUAGGI ---
\subparagraph{UC33.2.1 - Visualizza lista linguaggi}\label{UC33.2.1}

\begin{figure}[H]
	\centering
	\includegraphics[width=0.6\textwidth]{../Assets/AdR/UC33-2-1ext.png}
	\caption{UC33.2.1 - Visualizza lista linguaggi}
	\label{fig:UC33-2-1ext}
\end{figure}

\begin{itemize}
    \item \textbf{Attori principali:} Utente
    \item \textbf{Precondizioni:}
    \begin{itemize}
        \item Il sistema è attivo e funzionante
        \item L'utente è stato riconosciuto dal sistema come Utente
        \item L'utente sta visualizzando la sezione delle informazioni tecniche (\hyperref[UC33.2]{UC33.2})
    \end{itemize}
    \item \textbf{Postcondizioni:}
    \begin{itemize}
        \item Viene visualizzata la lista dei linguaggi di programmazione rilevati nella repository
    \end{itemize}
    \item \textbf{Scenario principale:}
    \begin{enumerate}
        \item Il sistema recupera l'elenco dei linguaggi di programmazione rilevati durante l'ultima scansione
        \item Per ogni linguaggio presente nella lista, il sistema visualizza il dettaglio del singolo linguaggio (\hyperref[UC33.2.1.1]{UC33.2.1.1})
    \end{enumerate}
    \item \textbf{Scenario alternativo:}
    \begin{itemize}
        \item \textbf{Nessun linguaggio rilevato (\hyperref[UC33.2.1.3]{UC33.2.1.3}):}
        \begin{itemize}
            \item \textbf{Condizione:} La scansione non ha rilevato alcun linguaggio di programmazione
            \item \textbf{Flusso:}
            \begin{enumerate}
                \item Il sistema mostra un messaggio informativo: "Nessun linguaggio di programmazione rilevato"
            \end{enumerate}
        \end{itemize}
    \end{itemize}
    \item \textbf{Inclusioni:} \hyperref[UC33.2.1.1]{UC33.2.1.1}
    \item \textbf{Estensioni:} \hyperref[UC33.2.1.3]{UC33.2.1.3}
\end{itemize}

Il caso d'uso UC33.2.1 include ulteriori casi d'uso come rappresentato nella seguente immagine:
\begin{figure}[H]
	\centering
	\includegraphics[width=0.6\textwidth]{../Assets/AdR/UC33-2-1.png}
	\caption{UC33.2.1.1 - Visualizza linguaggio singolo}
	\label{fig:UC33-2-1}
\end{figure}

\subparagraph{UC33.2.1.1 - Visualizza linguaggio singolo}\label{UC33.2.1.1}
\begin{itemize}
    \item \textbf{Attori principali:} Utente
    \item \textbf{Precondizioni:}
    \begin{itemize}
        \item Il sistema è attivo e funzionante
        \item L'utente è stato riconosciuto dal sistema come Utente
        \item L'utente sta visualizzando la lista dei linguaggi (\hyperref[UC33.2.1]{UC33.2.1})
        \item Esiste almeno un linguaggio rilevato nella repository
    \end{itemize}
    \item \textbf{Postcondizioni:}
    \begin{itemize}
        \item Vengono visualizzati i dettagli del singolo linguaggio di programmazione
    \end{itemize}
    \item \textbf{Scenario principale:}
    \begin{enumerate}
        \item Il sistema recupera le informazioni relative al singolo linguaggio
        \item Il sistema mostra i dettagli del linguaggio:
        \begin{itemize}
            \item Nome del linguaggio (\hyperref[UC33.2.1.1.1]{UC33.2.1.1.1})
            \item Versione del linguaggio (\hyperref[UC33.2.1.1.2]{UC33.2.1.1.2})
        \end{itemize}
    \end{enumerate}
    \item \textbf{Inclusioni:} \hyperref[UC33.2.1.1.1]{UC33.2.1.1.1}, \hyperref[UC33.2.1.1.2]{UC33.2.1.1.2}
\end{itemize}

\subparagraph{UC33.2.1.1.1 - Visualizza nome linguaggio}\label{UC33.2.1.1.1}
\begin{itemize}
    \item \textbf{Attori principali:} Utente
    \item \textbf{Precondizioni:}
    \begin{itemize}
        \item Il sistema è attivo e funzionante
        \item L'utente sta visualizzando il dettaglio di un linguaggio (\hyperref[UC33.2.1.1]{UC33.2.1.1})
    \end{itemize}
    \item \textbf{Postcondizioni:}
    \begin{itemize}
        \item Viene visualizzato il nome del linguaggio di programmazione
    \end{itemize}
    \item \textbf{Scenario principale:}
    \begin{enumerate}
        \item Il sistema recupera il nome identificativo del linguaggio
        \item Il sistema visualizza il nome del linguaggio 
    \end{enumerate}
\end{itemize}

\subparagraph{UC33.2.1.1.2 - Visualizza versione linguaggio}\label{UC33.2.1.1.2}
\begin{itemize}
    \item \textbf{Attori principali:} Utente
    \item \textbf{Precondizioni:}
    \begin{itemize}
        \item Il sistema è attivo e funzionante
        \item L'utente sta visualizzando il dettaglio di un linguaggio (\hyperref[UC33.2.1.1]{UC33.2.1.1})
    \end{itemize}
    \item \textbf{Postcondizioni:}
    \begin{itemize}
        \item Viene visualizzata la versione del linguaggio di programmazione
    \end{itemize}
    \item \textbf{Scenario principale:}
    \begin{enumerate}
        \item Il sistema recupera la versione del linguaggio rilevata nella repository
        \item Il sistema visualizza la versione del linguaggio 
    \end{enumerate}
    \item \textbf{Scenario alternativo:}

\end{itemize}

\subparagraph{UC33.2.1.3 - Nessun linguaggio rilevato}\label{UC33.2.1.3}
\begin{itemize}
    \item \textbf{Attori principali:} Utente
    \item \textbf{Precondizioni:}
    \begin{itemize}
        \item Il sistema è attivo e funzionante
        \item L'utente è stato riconosciuto dal sistema come Utente
        \item L'utente sta visualizzando la sezione delle informazioni tecniche (\hyperref[UC33.2]{UC33.2})
        \item La scansione non ha rilevato alcun linguaggio di programmazione
    \end{itemize}
    \item \textbf{Postcondizioni:}
    \begin{itemize}
        \item Il sistema informa l'utente dell'assenza di linguaggi rilevati
    \end{itemize}
    \item \textbf{Scenario principale:}
    \begin{enumerate}
        \item Il sistema rileva che non sono presenti linguaggi di programmazione identificati nella repository
        \item Il sistema mostra un messaggio informativo: "Nessun linguaggio di programmazione rilevato"
    \end{enumerate}
\end{itemize}

% --- LIBRERIE ---
\subparagraph{UC33.2.2 - Visualizza lista librerie}\label{UC33.2.2}

\begin{figure}[H]
	\centering
	\includegraphics[width=0.6\textwidth]{../Assets/AdR/UC33-2-2ext.png}
	\caption{UC33.2.2 - Visualizza lista librerie}
	\label{fig:UC33-2-2ext}
\end{figure}

\begin{itemize}
    \item \textbf{Attori principali:} Utente
    \item \textbf{Precondizioni:}
    \begin{itemize}
        \item Il sistema è attivo e funzionante
        \item L'utente è stato riconosciuto dal sistema come Utente
        \item L'utente sta visualizzando la sezione delle informazioni tecniche (\hyperref[UC33.2]{UC33.2})
    \end{itemize}
    \item \textbf{Postcondizioni:}
    \begin{itemize}
        \item Viene visualizzata la lista delle librerie rilevate nella repository
    \end{itemize}
    \item \textbf{Scenario principale:}
    \begin{enumerate}
        \item Il sistema recupera l'elenco delle librerie rilevate durante l'ultima scansione
        \item Per ogni libreria presente nella lista, il sistema visualizza il dettaglio della singola libreria (\hyperref[UC33.2.2.1]{UC33.2.2.1})
    \end{enumerate}
    \item \textbf{Scenario alternativo:}
    \begin{itemize}
        \item \textbf{Nessuna libreria rilevata (\hyperref[UC33.2.2.3]{UC33.2.2.3}):}
        \begin{itemize}
            \item \textbf{Condizione:} La scansione non ha rilevato alcuna libreria
            \item \textbf{Flusso:}
            \begin{enumerate}
                \item Il sistema mostra un messaggio informativo: "Nessuna libreria rilevata"
            \end{enumerate}
        \end{itemize}
    \end{itemize}
    \item \textbf{Inclusioni:} \hyperref[UC33.2.2.1]{UC33.2.2.1}
    \item \textbf{Estensioni:} \hyperref[UC33.2.2.3]{UC33.2.2.3}
\end{itemize}

Il caso d'uso UC33.2.2 include ulteriori casi d'uso come rappresentato nella seguente immagine:
\begin{figure}[H]
	\centering
	\includegraphics[width=0.6\textwidth]{../Assets/AdR/UC33-2-2.png}
	\caption{UC33.2.2.1 - Visualizza libreria singola}
	\label{fig:UC33-2-2}
\end{figure}

\subparagraph{UC33.2.2.1 - Visualizza libreria singola}\label{UC33.2.2.1}
\begin{itemize}
    \item \textbf{Attori principali:} Utente
    \item \textbf{Precondizioni:}
    \begin{itemize}
        \item Il sistema è attivo e funzionante
        \item L'utente è stato riconosciuto dal sistema come Utente
        \item L'utente sta visualizzando la lista delle librerie (\hyperref[UC33.2.2]{UC33.2.2})
        \item Esiste almeno una libreria rilevata nella repository
    \end{itemize}
    \item \textbf{Postcondizioni:}
    \begin{itemize}
        \item Vengono visualizzati i dettagli della singola libreria
    \end{itemize}
    \item \textbf{Scenario principale:}
    \begin{enumerate}
        \item Il sistema recupera le informazioni relative alla singola libreria
        \item Il sistema mostra i dettagli della libreria:
        \begin{itemize}
            \item Nome della libreria (\hyperref[UC33.2.2.1.1]{UC33.2.2.1.1})
            \item Versione della libreria (\hyperref[UC33.2.2.1.2]{UC33.2.2.1.2})
        \end{itemize}
    \end{enumerate}
    \item \textbf{Inclusioni:} \hyperref[UC33.2.2.1.1]{UC33.2.2.1.1}, \hyperref[UC33.2.2.1.2]{UC33.2.2.1.2}
\end{itemize}

\subparagraph{UC33.2.2.1.1 - Visualizza nome libreria}\label{UC33.2.2.1.1}
\begin{itemize}
    \item \textbf{Attori principali:} Utente
    \item \textbf{Precondizioni:}
    \begin{itemize}
        \item Il sistema è attivo e funzionante
        \item L'utente sta visualizzando il dettaglio di una libreria (\hyperref[UC33.2.2.1]{UC33.2.2.1})
    \end{itemize}
    \item \textbf{Postcondizioni:}
    \begin{itemize}
        \item Viene visualizzato il nome della libreria
    \end{itemize}
    \item \textbf{Scenario principale:}
    \begin{enumerate}
        \item Il sistema recupera il nome identificativo della libreria
        \item Il sistema visualizza il nome della libreria 
    \end{enumerate}
\end{itemize}

\subparagraph{UC33.2.2.1.2 - Visualizza versione libreria}\label{UC33.2.2.1.2}
\begin{itemize}
    \item \textbf{Attori principali:} Utente
    \item \textbf{Precondizioni:}
    \begin{itemize}
        \item Il sistema è attivo e funzionante
        \item L'utente sta visualizzando il dettaglio di una libreria (\hyperref[UC33.2.2.1]{UC33.2.2.1})
    \end{itemize}
    \item \textbf{Postcondizioni:}
    \begin{itemize}
        \item Viene visualizzata la versione della libreria
    \end{itemize}
    \item \textbf{Scenario principale:}
    \begin{enumerate}
        \item Il sistema recupera la versione della libreria rilevata nella repository
        \item Il sistema visualizza la versione della libreria 
    \end{enumerate}
\end{itemize}

\subparagraph{UC33.2.2.3 - Nessuna libreria rilevata}\label{UC33.2.2.3}
\begin{itemize}
    \item \textbf{Attori principali:} Utente
    \item \textbf{Precondizioni:}
    \begin{itemize}
        \item Il sistema è attivo e funzionante
        \item L'utente è stato riconosciuto dal sistema come Utente
        \item L'utente sta visualizzando la sezione delle informazioni tecniche (\hyperref[UC33.2]{UC33.2})
        \item La scansione non ha rilevato alcuna libreria
    \end{itemize}
    \item \textbf{Postcondizioni:}
    \begin{itemize}
        \item Il sistema informa l'utente dell'assenza di librerie rilevate
    \end{itemize}
    \item \textbf{Scenario principale:}
    \begin{enumerate}
        \item Il sistema rileva che non sono presenti librerie identificate nella repository
        \item Il sistema mostra un messaggio informativo: "Nessuna libreria rilevata"
    \end{enumerate}
\end{itemize}

% --- FRAMEWORK ---
\subparagraph{UC33.2.3 - Visualizza lista framework}\label{UC33.2.3}

\begin{figure}[H]
	\centering
	\includegraphics[width=0.6\textwidth]{../Assets/AdR/UC33-2-3ext.png}
	\caption{UC33.2.3 - Visualizza lista framework}
	\label{fig:UC33-2-3ext}
\end{figure}

\begin{itemize}
    \item \textbf{Attori principali:} Utente
    \item \textbf{Precondizioni:}
    \begin{itemize}
        \item Il sistema è attivo e funzionante
        \item L'utente è stato riconosciuto dal sistema come Utente
        \item L'utente sta visualizzando la sezione delle informazioni tecniche (\hyperref[UC33.2]{UC33.2})
    \end{itemize}
    \item \textbf{Postcondizioni:}
    \begin{itemize}
        \item Viene visualizzata la lista dei framework rilevati nella repository
    \end{itemize}
    \item \textbf{Scenario principale:}
    \begin{enumerate}
        \item Il sistema recupera l'elenco dei framework rilevati durante l'ultima scansione
        \item Per ogni framework presente nella lista, il sistema visualizza il dettaglio del singolo framework (\hyperref[UC33.2.3.1]{UC33.2.3.1})
    \end{enumerate}
    \item \textbf{Scenario alternativo:}
    \begin{itemize}
        \item \textbf{Nessun framework rilevato (\hyperref[UC33.2.3.3]{UC33.2.3.3}):}
        \begin{itemize}
            \item \textbf{Condizione:} La scansione non ha rilevato alcun framework
            \item \textbf{Flusso:}
            \begin{enumerate}
                \item Il sistema mostra un messaggio informativo: "Nessun framework rilevato"
            \end{enumerate}
        \end{itemize}
    \end{itemize}
    \item \textbf{Inclusioni:} \hyperref[UC33.2.3.1]{UC33.2.3.1}
    \item \textbf{Estensioni:} \hyperref[UC33.2.3.3]{UC33.2.3.3}
\end{itemize}

Il caso d'uso UC33.2.3 include ulteriori casi d'uso come rappresentato nella seguente immagine:
\begin{figure}[H]
	\centering
	\includegraphics[width=0.6\textwidth]{../Assets/AdR/UC33-2-3.png}
	\caption{UC33.2.3.1 - Visualizza framework singolo}
	\label{fig:UC33-2-3}
\end{figure}

\subparagraph{UC33.2.3.1 - Visualizza framework singolo}\label{UC33.2.3.1}
\begin{itemize}
    \item \textbf{Attori principali:} Utente
    \item \textbf{Precondizioni:}
    \begin{itemize}
        \item Il sistema è attivo e funzionante
        \item L'utente è stato riconosciuto dal sistema come Utente
        \item L'utente sta visualizzando la lista dei framework (\hyperref[UC33.2.3]{UC33.2.3})
        \item Esiste almeno un framework rilevato nella repository
    \end{itemize}
    \item \textbf{Postcondizioni:}
    \begin{itemize}
        \item Vengono visualizzati i dettagli del singolo framework
    \end{itemize}
    \item \textbf{Scenario principale:}
    \begin{enumerate}
        \item Il sistema recupera le informazioni relative al singolo framework
        \item Il sistema mostra i dettagli del framework:
        \begin{itemize}
            \item Nome del framework (\hyperref[UC33.2.3.1.1]{UC33.2.3.1.1})
            \item Versione del framework (\hyperref[UC33.2.3.1.2]{UC33.2.3.1.2})
        \end{itemize}
    \end{enumerate}
    \item \textbf{Inclusioni:} \hyperref[UC33.2.3.1.1]{UC33.2.3.1.1}, \hyperref[UC33.2.3.1.2]{UC33.2.3.1.2}
\end{itemize}

\subparagraph{UC33.2.3.1.1 - Visualizza nome framework}\label{UC33.2.3.1.1}
\begin{itemize}
    \item \textbf{Attori principali:} Utente
    \item \textbf{Precondizioni:}
    \begin{itemize}
        \item Il sistema è attivo e funzionante
        \item L'utente sta visualizzando il dettaglio di un framework (\hyperref[UC33.2.3.1]{UC33.2.3.1})
    \end{itemize}
    \item \textbf{Postcondizioni:}
    \begin{itemize}
        \item Viene visualizzato il nome del framework
    \end{itemize}
    \item \textbf{Scenario principale:}
    \begin{enumerate}
        \item Il sistema recupera il nome identificativo del framework
        \item Il sistema visualizza il nome del framework
    \end{enumerate}
\end{itemize}

\subparagraph{UC33.2.3.1.2 - Visualizza versione framework}\label{UC33.2.3.1.2}
\begin{itemize}
    \item \textbf{Attori principali:} Utente
    \item \textbf{Precondizioni:}
    \begin{itemize}
        \item Il sistema è attivo e funzionante
        \item L'utente sta visualizzando il dettaglio di un framework (\hyperref[UC33.2.3.1]{UC33.2.3.1})
    \end{itemize}
    \item \textbf{Postcondizioni:}
    \begin{itemize}
        \item Viene visualizzata la versione del framework
    \end{itemize}
    \item \textbf{Scenario principale:}
    \begin{enumerate}
        \item Il sistema recupera la versione del framework rilevata nella repository
        \item Il sistema visualizza la versione del framework 
    \end{enumerate}

\end{itemize}

\subparagraph{UC33.2.3.3 - Nessun framework rilevato}\label{UC33.2.3.3}
\begin{itemize}
    \item \textbf{Attori principali:} Utente
    \item \textbf{Precondizioni:}
    \begin{itemize}
        \item Il sistema è attivo e funzionante
        \item L'utente è stato riconosciuto dal sistema come Utente
        \item L'utente sta visualizzando la sezione delle informazioni tecniche (\hyperref[UC33.2]{UC33.2})
        \item La scansione non ha rilevato alcun framework
    \end{itemize}
    \item \textbf{Postcondizioni:}
    \begin{itemize}
        \item Il sistema informa l'utente dell'assenza di framework rilevati
    \end{itemize}
    \item \textbf{Scenario principale:}
    \begin{enumerate}
        \item Il sistema rileva che non sono presenti framework identificati nella repository
        \item Il sistema mostra un messaggio informativo: "Nessun framework rilevato"
    \end{enumerate}
\end{itemize}


% ============================================
% UC35 - SELEZIONE BRANCH
% ============================================
\subsubsection{UC35 - Selezione branch}\label{UC35}
\begin{figure}[H]
	\centering
	\includegraphics[width=0.8\textwidth]{../Assets/AdR/UC35ext.png}
	\caption{UC35 - Selezione branch}
	\label{fig:UC35ext}
\end{figure}

\begin{itemize}
    \item \textbf{Attori principali:} Utente Autenticato
    \item \textbf{Precondizioni:}
    \begin{itemize}
        \item Il sistema è attivo e funzionante
        \item L'utente è autenticato nel sistema
        \item L'utente deve selezionare un branch di una determinata repository
    \end{itemize}
    \item \textbf{Postcondizioni:}
    \begin{itemize}
        \item Il sistema ha registrato un branch valido su una determinata repository
    \end{itemize}
    \item \textbf{Scenario principale:}
    \begin{enumerate}
        \item Il sistema mostra la lista dei branch di una determinata repository (\hyperref[UC35.1]{UC35.1})
        \item L'utente seleziona il branch desiderato
        \item Il sistema memorizza le informazioni relative al branch selezionato
    \end{enumerate}
    \item \textbf{Scenario alternativo:}
    \begin{itemize}
        \item \textbf{Selezione branch non riuscita (\hyperref[UC35.2]{UC35.2}):} L'utente non seleziona nessun branch valido
    \end{itemize}
    \item \textbf{Inclusioni:} \hyperref[UC35.1]{UC35.1}
    \item \textbf{Estensioni:} \hyperref[UC35.2]{UC35.2}
\end{itemize}

% --- UC35.1 - VISUALIZZA LISTA BRANCH ---
\paragraph{UC35.1 - Visualizza lista branch}\label{UC35.1}
\begin{figure}[H]
	\centering
	\includegraphics[width=0.7\textwidth]{../Assets/AdR/UC35-1.png}
	\caption{Inclusioni di UC35.1: UC35.1.1}
	\label{fig:UC35-1}
\end{figure}

\begin{itemize}
    \item \textbf{Attori principali:} Utente Autenticato
    \item \textbf{Precondizioni:}
    \begin{itemize}
        \item Il sistema è attivo e funzionante
        \item L'utente sta eseguendo UC Selezione branch (\hyperref[UC35]{UC35})
    \end{itemize}
    \item \textbf{Postcondizioni:}
    \begin{itemize}
        \item L'utente visualizza una lista di branch di una determinata repository
    \end{itemize}
    \item \textbf{Scenario principale:}
    \begin{enumerate}
        \item L'utente deve selezionare un branch di una repository
        \item Il sistema recupera la lista dei branch disponibili per la repository
        \item Per ogni branch presente nella lista, il sistema visualizza il dettaglio del singolo branch (\hyperref[UC35.1.1]{UC35.1.1})
    \end{enumerate}
    \item \textbf{Inclusioni:} \hyperref[UC35.1.1]{UC35.1.1}
\end{itemize}

% --- UC35.1.1 - VISUALIZZA SINGOLO BRANCH ---
\subparagraph{UC35.1.1 - Visualizza singolo branch}\label{UC35.1.1}
\begin{itemize}
    \item \textbf{Attori principali:} Utente Autenticato
    \item \textbf{Precondizioni:}
    \begin{itemize}
        \item Il sistema è attivo e funzionante
        \item L'utente è autenticato nel sistema
        \item L'utente sta visualizzando la lista dei branch (\hyperref[UC35.1]{UC35.1})
        \item Esiste almeno un branch nella repository
    \end{itemize}
    \item \textbf{Postcondizioni:}
    \begin{itemize}
        \item Vengono visualizzati i dettagli del singolo branch
    \end{itemize}
    \item \textbf{Scenario principale:}
    \begin{enumerate}
        \item Il sistema recupera le informazioni relative al singolo branch
        \item Il sistema mostra i dettagli del branch:
        \begin{itemize}
            \item Nome del branch (\hyperref[UC35.1.1.1]{UC35.1.1.1})
            \item Data ultima scansione (\hyperref[UC35.1.1.2]{UC35.1.1.2})
        \end{itemize}
    \end{enumerate}
    \item \textbf{Inclusioni:} \hyperref[UC35.1.1.1]{UC35.1.1.1}, \hyperref[UC35.1.1.2]{UC35.1.1.2}
\end{itemize}

\subparagraph{UC35.1.1.1 - Visualizza nome branch}\label{UC35.1.1.1}
\begin{itemize}
    \item \textbf{Attori principali:} Utente Autenticato
    \item \textbf{Precondizioni:}
    \begin{itemize}
        \item Il sistema è attivo e funzionante
        \item L'utente sta visualizzando il dettaglio di un branch (\hyperref[UC35.1.1]{UC35.1.1})
    \end{itemize}
    \item \textbf{Postcondizioni:}
    \begin{itemize}
        \item Viene visualizzato il nome del branch
    \end{itemize}
    \item \textbf{Scenario principale:}
    \begin{enumerate}
        \item Il sistema recupera il nome identificativo del branch
        \item Il sistema visualizza il nome del branch
    \end{enumerate}
\end{itemize}

\subparagraph{UC35.1.1.2 - Visualizza data ultima scansione}\label{UC35.1.1.2}
\begin{itemize}
    \item \textbf{Attori principali:} Utente Autenticato
    \item \textbf{Precondizioni:}
    \begin{itemize}
        \item Il sistema è attivo e funzionante
        \item L'utente sta visualizzando il dettaglio di un branch (\hyperref[UC35.1.1]{UC35.1.1})
    \end{itemize}
    \item \textbf{Postcondizioni:}
    \begin{itemize}
        \item Viene visualizzata la data dell'ultima scansione effettuata sul branch
    \end{itemize}
    \item \textbf{Scenario principale:}
    \begin{enumerate}
        \item Il sistema recupera la data dell'ultima scansione eseguita sul branch
        \item Il sistema visualizza la data dell'ultima scansione se esistente
    \end{enumerate}

\end{itemize}

% --- UC35.2 - SELEZIONE BRANCH NON RIUSCITA ---
\paragraph{UC35.2 - Selezione branch non riuscita}\label{UC35.2}
\begin{itemize}
    \item \textbf{Attori principali:} Utente Autenticato
    \item \textbf{Precondizioni:}
    \begin{itemize}
        \item Il sistema è attivo e funzionante
        \item Il sistema sta eseguendo UC Selezione branch (\hyperref[UC35]{UC35})
        \item Durante il recupero o la validazione dei branch non viene trovato alcun branch utilizzabile
    \end{itemize}
    \item \textbf{Postcondizioni:}
    \begin{itemize}
        \item Nessun branch viene associato alla repository
    \end{itemize}
    \item \textbf{Scenario principale:}
    \begin{enumerate}
        \item Il sistema tenta di recuperare i branch disponibili per la repository
        \item Il sistema non trova branch validi
        \item Il sistema notifica l'errore di selezione branch all'utente e annulla il processo in corso
    \end{enumerate}
\end{itemize}


% ============================================
% UC36 - VISIONE AGGREGATA
% ============================================
\subsubsection{UC36 - Visione aggregata}\label{UC36}
\begin{figure}[H]
	\centering
	\includegraphics[width=0.9\textwidth]{../Assets/AdR/UC36ext.png}
	\caption{UC36 - Visione aggregata}
	\label{fig:UC36ext}
\end{figure}

\begin{itemize}
    \item \textbf{Attori principali:} Utente
    \item \textbf{Precondizioni:}
    \begin{itemize}
        \item Il sistema è attivo e funzionante
        \item L'utente è autenticato presso il sistema
        \item L'utente ha accesso a un workspace ed è entrato in uno specifico workspace
        \item Ogni repository del workspace ha un branch di default valido
    \end{itemize}
    \item \textbf{Postcondizioni:}
    \begin{itemize}
        \item L'utente visualizza una vista sintetica e aggregata delle informazioni relative all'insieme di repository che fanno parte del workspace
    \end{itemize}
    \item \textbf{Scenario principale:}
    \begin{enumerate}
        \item L'utente richiede la visualizzazione aggregata delle repository del workspace
        \item Il sistema verifica la presenza di un branch di default per ciascuna repository
        \item Il sistema recupera i dati di analisi relativi alle repository del workspace
        \item Il sistema calcola le metriche aggregate (medie, minimi, massimi, distribuzioni)
        \item Il sistema visualizza la sezione test aggregata (\hyperref[UC36.2]{UC36.2})
        \item Il sistema visualizza la sezione informazioni tecniche aggregata (\hyperref[UC36.3]{UC36.3})
        \item Il sistema visualizza la sezione sicurezza aggregata (\hyperref[UC36.4]{UC36.4})
        \item Il sistema visualizza la sezione documentazione aggregata (\hyperref[UC36.5]{UC36.5})
    \end{enumerate}
    \item \textbf{Scenario alternativo:}
    \begin{itemize}
        \item \textbf{Filtra per tag (\hyperref[UC36.1]{UC36.1}):}
        \begin{itemize}
            \item \textbf{Condizione:} L'utente vuole filtrare le repository per tag 
            \item \textbf{Flusso:}
            \begin{enumerate}
                \item L'utente seleziona uno o più tag 
                \item Il sistema filtra le repository in base ai tag selezionati
                \item Il sistema aggiorna la visione aggregata con le repository filtrate
            \end{enumerate}
        \end{itemize}
        \item \textbf{Errore Visione Aggregata (\hyperref[UC36.6]{UC36.6}):}
        \begin{itemize}
            \item \textbf{Condizione:} Si è verificato un errore nella visione aggregata, come workspace vuoto o nessuna repository selezionata
            \item \textbf{Flusso:}
            \begin{enumerate}
                \item Il sistema mostra un messaggio di errore appropriato
            \end{enumerate}
        \end{itemize}
        \item \textbf{Scansione Workspace (\hyperref[UC38]{UC38}):}
        \begin{itemize}
            \item \textbf{Condizione:} L'utente vuole lanciare una scansione sul workspace visualizzato
            \item \textbf{Flusso:}
            \begin{enumerate}
                \item L'utente avvia la scansione del workspace
            \end{enumerate}
        \end{itemize}
    \end{itemize}
    \item \textbf{Inclusioni:} \hyperref[UC36.2]{UC36.2}, \hyperref[UC36.3]{UC36.3}, \hyperref[UC36.4]{UC36.4}, \hyperref[UC36.5]{UC36.5}
    \item \textbf{Estensioni:} \hyperref[UC36.1]{UC36.1}, \hyperref[UC36.6]{UC36.6}, \hyperref[UC38]{UC38}
\end{itemize}

Il caso d'uso UC36 include ulteriori casi d'uso come rappresentato nella seguente immagine:
\begin{figure}[H]
	\centering
	\includegraphics[width=0.7\textwidth]{../Assets/AdR/UC36.png}
	\caption{Inclusioni di UC36: UC36.2, UC36.3, UC36.4, UC36.5}
	\label{fig:UC36inc}
\end{figure}


% --- UC36.1 - FILTRA PER TAG ---
\paragraph{UC36.1 - Filtra per tag}\label{UC36.1}
\begin{itemize}
    \item \textbf{Attori principali:} Utente
    \item \textbf{Precondizioni:}
    \begin{itemize}
        \item Il sistema è attivo e funzionante
        \item L'utente è autenticato presso il sistema
        \item L'utente sta visualizzando la visione aggregata (\hyperref[UC36]{UC36})
        \item Esiste almeno un tag raccolta definito nel workspace
    \end{itemize}
    \item \textbf{Postcondizioni:}
    \begin{itemize}
        \item La visione aggregata viene aggiornata mostrando solo le repository associate ai tag selezionati
    \end{itemize}
    \item \textbf{Scenario principale:}
    \begin{enumerate}
        \item L'utente seleziona l'opzione per filtrare per tag
        \item Il sistema mostra la lista dei tag raccolta disponibili nel workspace
        \item L'utente seleziona uno o più tag
        \item Il sistema filtra le repository in base ai tag selezionati
        \item Il sistema ricalcola e aggiorna la visione aggregata
    \end{enumerate}
\end{itemize}


% ============================================
% UC36.2 - SEZIONE TEST AGGREGATA
% ============================================
\paragraph{UC36.2 - Sezione Test aggregata}\label{UC36.2}
\begin{figure}[H]
	\centering
	\includegraphics[width=0.7\textwidth]{../Assets/AdR/UC36-2.png}
	\caption{Inclusioni di UC36.2: UC36.2.1, UC36.2.2, UC36.2.3}
	\label{fig:UC36-2}
\end{figure}

\begin{itemize}
    \item \textbf{Attori principali:} Utente
    \item \textbf{Precondizioni:}
    \begin{itemize}
        \item Il sistema è attivo e funzionante
        \item L'utente è autenticato presso il sistema
        \item L'utente sta visualizzando la visione aggregata (\hyperref[UC36]{UC36})
        \item Almeno una repository del workspace ha subito un'analisi dei test
    \end{itemize}
    \item \textbf{Postcondizioni:}
    \begin{itemize}
        \item Il sistema presenta all'utente la sezione completa di analisi dei test aggregata
    \end{itemize}
    \item \textbf{Scenario principale:}
    \begin{enumerate}
        \item Il sistema recupera i dati dei test dalle repository del workspace
        \item Il sistema calcola le metriche aggregate sui test
        \item Il sistema visualizza il test coverage medio (\hyperref[UC36.2.1]{UC36.2.1})
        \item Il sistema visualizza il numero di repository con test coverage superiore a soglia (\hyperref[UC36.2.2]{UC36.2.2})
        \item Il sistema visualizza il test coverage minimo (\hyperref[UC36.2.3]{UC36.2.3})
    \end{enumerate}
    \item \textbf{Inclusioni:} \hyperref[UC36.2.1]{UC36.2.1}, \hyperref[UC36.2.2]{UC36.2.2}, \hyperref[UC36.2.3]{UC36.2.3}
\end{itemize}

% --- UC36.2.1 - TEST COVERAGE MEDIA ---
\subparagraph{UC36.2.1 - Visualizza test coverage media}\label{UC36.2.1}
\begin{itemize}
    \item \textbf{Attori principali:} Utente
    \item \textbf{Precondizioni:}
    \begin{itemize}
        \item Il sistema è attivo e funzionante
        \item L'utente sta visualizzando la sezione test aggregata (\hyperref[UC36.2]{UC36.2})
    \end{itemize}
    \item \textbf{Postcondizioni:}
    \begin{itemize}
        \item Viene visualizzato il valore medio del test coverage tra le repository del workspace
    \end{itemize}
    \item \textbf{Scenario principale:}
    \begin{enumerate}
        \item Il sistema calcola la media del test coverage tra tutte le repository del workspace
        \item Il sistema visualizza il valore percentuale del test coverage medio
    \end{enumerate}
\end{itemize}

% --- UC36.2.2 - TEST COVERAGE SUPERIORE A SOGLIA ---
\subparagraph{UC36.2.2 - Visualizza test coverage superiore a soglia}\label{UC36.2.2}
\begin{itemize}
    \item \textbf{Attori principali:} Utente
    \item \textbf{Precondizioni:}
    \begin{itemize}
        \item Il sistema è attivo e funzionante
        \item L'utente sta visualizzando la sezione test aggregata (\hyperref[UC36.2]{UC36.2})
    \end{itemize}
    \item \textbf{Postcondizioni:}
    \begin{itemize}
        \item Viene visualizzato il numero di repository che hanno un test coverage superiore alla soglia definita
    \end{itemize}
    \item \textbf{Scenario principale:}
    \begin{enumerate}
        \item Il sistema conta il numero di repository con test coverage superiore al parametro configurato (default: 70\%)
        \item Il sistema visualizza il conteggio delle repository che superano la soglia
    \end{enumerate}
\end{itemize}

% --- UC36.2.3 - TEST COVERAGE MINIMO ---
\subparagraph{UC36.2.3 - Visualizza test coverage minimo}\label{UC36.2.3}
\begin{itemize}
    \item \textbf{Attori principali:} Utente
    \item \textbf{Precondizioni:}
    \begin{itemize}
        \item Il sistema è attivo e funzionante
        \item L'utente sta visualizzando la sezione test aggregata (\hyperref[UC36.2]{UC36.2})
    \end{itemize}
    \item \textbf{Postcondizioni:}
    \begin{itemize}
        \item Viene visualizzato il valore minimo del test coverage tra le repository del workspace
    \end{itemize}
    \item \textbf{Scenario principale:}
    \begin{enumerate}
        \item Il sistema individua il valore minimo di test coverage tra tutte le repository del workspace
        \item Il sistema visualizza il valore percentuale del test coverage minimo
    \end{enumerate}
\end{itemize}


% ============================================
% UC36.3 - SEZIONE INFORMAZIONI TECNICHE AGGREGATA
% ============================================
\paragraph{UC36.3 - Sezione Informazioni Tecniche Aggregata}\label{UC36.3}
\begin{figure}[H]
	\centering
	\includegraphics[width=0.7\textwidth]{../Assets/AdR/UC36-3.png}
	\caption{Inclusioni di UC36.3: UC36.3.1, UC36.3.2}
	\label{fig:UC36-3}
\end{figure}

\begin{itemize}
    \item \textbf{Attori principali:} Utente
    \item \textbf{Precondizioni:}
    \begin{itemize}
        \item Il sistema è attivo e funzionante
        \item L'utente è autenticato presso il sistema
        \item L'utente sta visualizzando la visione aggregata (\hyperref[UC36]{UC36})
        \item Almeno una repository del workspace ha subito un'analisi delle informazioni tecniche
    \end{itemize}
    \item \textbf{Postcondizioni:}
    \begin{itemize}
        \item Il sistema presenta all'utente la sezione completa delle informazioni tecniche aggregate
    \end{itemize}
    \item \textbf{Scenario principale:}
    \begin{enumerate}
        \item Il sistema recupera le informazioni tecniche dalle repository del workspace
        \item Il sistema aggrega i dati su linguaggi e framework
        \item Il sistema visualizza il grafico distribuzione linguaggi (\hyperref[UC36.3.1]{UC36.3.1})
        \item Il sistema visualizza il grafico distribuzione framework (\hyperref[UC36.3.2]{UC36.3.2})
    \end{enumerate}
    \item \textbf{Inclusioni:} \hyperref[UC36.3.1]{UC36.3.1}, \hyperref[UC36.3.2]{UC36.3.2}
\end{itemize}

% --- UC36.3.1 - GRAFICO LINGUAGGI ---
\subparagraph{UC36.3.1 - Visualizza grafico linguaggi}\label{UC36.3.1}
\begin{itemize}
    \item \textbf{Attori principali:} Utente
    \item \textbf{Precondizioni:}
    \begin{itemize}
        \item Il sistema è attivo e funzionante
        \item L'utente sta visualizzando la sezione informazioni tecniche aggregata (\hyperref[UC36.3]{UC36.3})
    \end{itemize}
    \item \textbf{Postcondizioni:}
    \begin{itemize}
        \item Viene visualizzato un grafico a torta che mostra la distribuzione dei linguaggi di programmazione
    \end{itemize}
    \item \textbf{Scenario principale:}
    \begin{enumerate}
        \item Il sistema aggrega i dati sui linguaggi di programmazione utilizzati nelle repository del workspace
        \item Il sistema calcola la percentuale di utilizzo di ciascun linguaggio
        \item Il sistema visualizza un grafico a torta con la distribuzione dei linguaggi
    \end{enumerate}
\end{itemize}

% --- UC36.3.2 - GRAFICO FRAMEWORK ---
\subparagraph{UC36.3.2 - Visualizza grafico framework}\label{UC36.3.2}
\begin{itemize}
    \item \textbf{Attori principali:} Utente
    \item \textbf{Precondizioni:}
    \begin{itemize}
        \item Il sistema è attivo e funzionante
        \item L'utente sta visualizzando la sezione informazioni tecniche aggregata (\hyperref[UC36.3]{UC36.3})
    \end{itemize}
    \item \textbf{Postcondizioni:}
    \begin{itemize}
        \item Viene visualizzato un grafico a torta che mostra la distribuzione dei framework
    \end{itemize}
    \item \textbf{Scenario principale:}
    \begin{enumerate}
        \item Il sistema aggrega i dati sui framework utilizzati nelle repository del workspace
        \item Il sistema calcola la percentuale di utilizzo di ciascun framework
        \item Il sistema visualizza un grafico a torta con la distribuzione dei framework
    \end{enumerate}
\end{itemize}


% ============================================
% UC36.4 - SEZIONE SICUREZZA AGGREGATA
% ============================================
\paragraph{UC36.4 - Sezione Sicurezza Aggregata}\label{UC36.4}
\begin{figure}[H]
	\centering
	\includegraphics[width=0.7\textwidth]{../Assets/AdR/UC36-4.png}
	\caption{Inclusioni di UC36.4: UC36.4.1, UC36.4.2}
	\label{fig:UC36-4}
\end{figure}

\begin{itemize}
    \item \textbf{Attori principali:} Utente
    \item \textbf{Precondizioni:}
    \begin{itemize}
        \item Il sistema è attivo e funzionante
        \item L'utente è autenticato presso il sistema
        \item L'utente sta visualizzando la visione aggregata (\hyperref[UC36]{UC36})
        \item Almeno una repository del workspace ha subito un'analisi di sicurezza OWASP
    \end{itemize}
    \item \textbf{Postcondizioni:}
    \begin{itemize}
        \item Il sistema presenta all'utente la sezione completa di analisi della sicurezza aggregata
    \end{itemize}
    \item \textbf{Scenario principale:}
    \begin{enumerate}
        \item Il sistema recupera i dati di sicurezza dalle repository del workspace
        \item Il sistema aggrega le informazioni sulle vulnerabilità
        \item Il sistema visualizza la percentuale di repository sicure (\hyperref[UC36.4.1]{UC36.4.1})
        \item Il sistema visualizza il grafico delle vulnerabilità (\hyperref[UC36.4.2]{UC36.4.2})
    \end{enumerate}
    \item \textbf{Inclusioni:} \hyperref[UC36.4.1]{UC36.4.1}, \hyperref[UC36.4.2]{UC36.4.2}
\end{itemize}

% --- UC36.4.1 - PERCENTUALE REPOSITORY SICURE ---
\subparagraph{UC36.4.1 - Visualizza percentuale repository sicure}\label{UC36.4.1}
\begin{itemize}
    \item \textbf{Attori principali:} Utente
    \item \textbf{Precondizioni:}
    \begin{itemize}
        \item Il sistema è attivo e funzionante
        \item L'utente sta visualizzando la sezione sicurezza aggregata (\hyperref[UC36.4]{UC36.4})
    \end{itemize}
    \item \textbf{Postcondizioni:}
    \begin{itemize}
        \item Viene visualizzato il numero di repository che non presentano le Top 10 vulnerabilità OWASP
    \end{itemize}
    \item \textbf{Scenario principale:}
    \begin{enumerate}
        \item Il sistema conta il numero di repository che non presentano vulnerabilità tra le Top 10 OWASP
        \item Il sistema visualizza il conteggio delle repository sicure
    \end{enumerate}
\end{itemize}

% --- UC36.4.2 - GRAFICO VULNERABILITÀ ---
\subparagraph{UC36.4.2 - Visualizza grafico vulnerabilità}\label{UC36.4.2}
\begin{itemize}
    \item \textbf{Attori principali:} Utente
    \item \textbf{Precondizioni:}
    \begin{itemize}
        \item Il sistema è attivo e funzionante
        \item L'utente sta visualizzando la sezione sicurezza aggregata (\hyperref[UC36.4]{UC36.4})
    \end{itemize}
    \item \textbf{Postcondizioni:}
    \begin{itemize}
        \item Viene visualizzato un istogramma che mostra la distribuzione delle vulnerabilità per gravità
    \end{itemize}
    \item \textbf{Scenario principale:}
    \begin{enumerate}
        \item L'utente visualizza un istogramma con il numero di repository che presentano vulnerabilità per ciascun livello di gravità (alta, media, bassa)
    \end{enumerate}
\end{itemize}


% ============================================
% UC36.5 - SEZIONE DOCUMENTAZIONE AGGREGATA
% ============================================
\paragraph{UC36.5 - Sezione Documentazione Aggregata}\label{UC36.5}
\begin{figure}[H]
	\centering
	\includegraphics[width=0.7\textwidth]{../Assets/AdR/UC36-5.png}
	\caption{Inclusioni di UC36.5: UC36.5.1, UC36.5.2, UC36.5.3, UC36.5.4}
	\label{fig:UC36-5}
\end{figure}

\begin{itemize}
    \item \textbf{Attori principali:} Utente
    \item \textbf{Precondizioni:}
    \begin{itemize}
        \item Il sistema è attivo e funzionante
        \item L'utente è autenticato presso il sistema
        \item L'utente sta visualizzando la visione aggregata (\hyperref[UC36]{UC36})
        \item Almeno una repository del workspace ha subito un'analisi della documentazione
    \end{itemize}
    \item \textbf{Postcondizioni:}
    \begin{itemize}
        \item Il sistema presenta all'utente la sezione completa di analisi della documentazione aggregata
    \end{itemize}
    \item \textbf{Scenario principale:}
    \begin{enumerate}
        \item Il sistema recupera i dati sulla documentazione dalle repository del workspace
        \item Il sistema calcola le metriche aggregate sulla qualità della documentazione
        \item Il sistema visualizza il voto documentazione minimo (\hyperref[UC36.5.1]{UC36.5.1})
        \item Il sistema visualizza il voto documentazione medio (\hyperref[UC36.5.2]{UC36.5.2})
        \item Il sistema visualizza il voto documentazione massimo (\hyperref[UC36.5.3]{UC36.5.3})
        \item Il sistema visualizza il grafico distribuzione voti documentazione (\hyperref[UC36.5.4]{UC36.5.4})
    \end{enumerate}
    \item \textbf{Inclusioni:} \hyperref[UC36.5.1]{UC36.5.1}, \hyperref[UC36.5.2]{UC36.5.2}, \hyperref[UC36.5.3]{UC36.5.3}, \hyperref[UC36.5.4]{UC36.5.4}
\end{itemize}

% --- UC36.5.1 - VOTO DOCUMENTAZIONE MINIMO ---
\subparagraph{UC36.5.1 - Visualizza voto documentazione minimo}\label{UC36.5.1}
\begin{itemize}
    \item \textbf{Attori principali:} Utente
    \item \textbf{Precondizioni:}
    \begin{itemize}
        \item Il sistema è attivo e funzionante
        \item L'utente sta visualizzando la sezione documentazione aggregata (\hyperref[UC36.5]{UC36.5})
    \end{itemize}
    \item \textbf{Postcondizioni:}
    \begin{itemize}
        \item Viene visualizzato il voto minimo della qualità della documentazione tra le repository del workspace
    \end{itemize}
    \item \textbf{Scenario principale:}
    \begin{enumerate}
        \item Il sistema individua il voto minimo di qualità della documentazione tra tutte le repository del workspace
        \item Il sistema visualizza il voto minimo della documentazione
    \end{enumerate}
\end{itemize}

% --- UC36.5.2 - VOTO DOCUMENTAZIONE MEDIO ---
\subparagraph{UC36.5.2 - Visualizza voto documentazione medio}\label{UC36.5.2}
\begin{itemize}
    \item \textbf{Attori principali:} Utente
    \item \textbf{Precondizioni:}
    \begin{itemize}
        \item Il sistema è attivo e funzionante
        \item L'utente sta visualizzando la sezione documentazione aggregata (\hyperref[UC36.5]{UC36.5})
    \end{itemize}
    \item \textbf{Postcondizioni:}
    \begin{itemize}
        \item Viene visualizzato il voto medio della qualità della documentazione tra le repository del workspace
    \end{itemize}
    \item \textbf{Scenario principale:}
    \begin{enumerate}
        \item Il sistema calcola la media dei voti di qualità della documentazione tra tutte le repository del workspace
        \item Il sistema visualizza il voto medio della documentazione
    \end{enumerate}
\end{itemize}

% --- UC36.5.3 - VOTO DOCUMENTAZIONE MASSIMO ---
\subparagraph{UC36.5.3 - Visualizza voto documentazione massimo}\label{UC36.5.3}
\begin{itemize}
    \item \textbf{Attori principali:} Utente
    \item \textbf{Precondizioni:}
    \begin{itemize}
        \item Il sistema è attivo e funzionante
        \item L'utente sta visualizzando la sezione documentazione aggregata (\hyperref[UC36.5]{UC36.5})
    \end{itemize}
    \item \textbf{Postcondizioni:}
    \begin{itemize}
        \item Viene visualizzato il voto massimo della qualità della documentazione tra le repository del workspace
    \end{itemize}
    \item \textbf{Scenario principale:}
    \begin{enumerate}
        \item Il sistema individua il voto massimo di qualità della documentazione tra tutte le repository del workspace
        \item Il sistema visualizza il voto massimo della documentazione
    \end{enumerate}
\end{itemize}

% --- UC36.5.4 - GRAFICO VOTI DOCUMENTAZIONE ---
\subparagraph{UC36.5.4 - Visualizza grafico voti documentazione}\label{UC36.5.4}
\begin{itemize}
    \item \textbf{Attori principali:} Utente
    \item \textbf{Precondizioni:}
    \begin{itemize}
        \item Il sistema è attivo e funzionante
        \item L'utente sta visualizzando la sezione documentazione aggregata (\hyperref[UC36.5]{UC36.5})
    \end{itemize}
    \item \textbf{Postcondizioni:}
    \begin{itemize}
        \item Viene visualizzato un grafico a torta che mostra la distribuzione dei voti sulla documentazione
    \end{itemize}
    \item \textbf{Scenario principale:}
    \begin{enumerate}
        \item Il sistema raggruppa i voti di qualità della documentazione delle repository del workspace
        \item Il sistema calcola la distribuzione dei voti
        \item Il sistema visualizza un grafico a torta con la distribuzione dei voti sulla documentazione
    \end{enumerate}
\end{itemize}


% ============================================
% UC36.6 - ERRORE VISIONE AGGREGATA
% ============================================
\paragraph{UC36.6 - Errore Visione Aggregata}\label{UC36.6}
\begin{itemize}
    \item \textbf{Attori principali:} Utente
    \item \textbf{Precondizioni:}
    \begin{itemize}
        \item Il sistema è attivo e funzionante
        \item L'utente è autenticato presso il sistema
        \item L'utente sta tentando di visualizzare la visione aggregata (\hyperref[UC36]{UC36})
        \item Si è verificata una delle seguenti condizioni:
        \begin{itemize}
            \item L'insieme di repository del workspace è vuoto
            \item Il workspace non contiene repository
        \end{itemize}
    \end{itemize}
    \item \textbf{Postcondizioni:}
    \begin{itemize}
        \item Il sistema informa l'utente dell'impossibilità di visualizzare la visione aggregata
    \end{itemize}
    \item \textbf{Scenario principale:}
    \begin{enumerate}
        \item Il sistema rileva una condizione di errore
        \item Il sistema mostra un messaggio di errore appropriato: "Il workspace non contiene repository da analizzare" (se workspace vuoto)
    \end{enumerate}
\end{itemize}



\end{document}