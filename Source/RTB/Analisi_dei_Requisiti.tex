\documentclass[a4paper, 11pt]{article}

% ====== PACCHETTI NECESSARI ======
\usepackage[utf8]{inputenc}
\usepackage[T1]{fontenc}
\usepackage[italian]{babel}
\usepackage{geometry}
\usepackage{graphicx}
\usepackage[table]{xcolor}
\usepackage{tabularx}
\usepackage{array}
\usepackage{amssymb}
\usepackage{fancyhdr}
\setlength{\headheight}{14pt}
\usepackage{titlesec}
\usepackage{helvet}
\renewcommand{\familydefault}{\sfdefault}
\usepackage{lipsum}
\usepackage{longtable}
\usepackage{etoolbox}

\makeatletter
\renewcommand\LT@output{%
  \ifnum\outputpenalty <-\@Mi
    \ifnum\outputpenalty > -\LT@end@pen
      \LT@err{floats and marginpars not allowed in a longtable}\@ehc
    \else
      \setbox\z@\vbox{\unvbox\@cclv}%
      \ifdim \ht\LT@lastfoot>\ht\LT@foot
        \dimen@\pagegoal
        \advance\dimen@-\ht\LT@lastfoot
        \ifdim\dimen@<\ht\z@
          \setbox\@cclv\vbox{\unvbox\z@\copy\LT@foot\vss}%
          \@makecol
          \@outputpage
          \setbox\z@\vbox{\box\LT@head}%
        \fi
      \fi
      \global\@colroom\@colht
      \global\vsize\@colht
      \vbox
        {\unvbox\z@\box\ifvoid\LT@lastfoot\LT@foot\else\LT@lastfoot\fi}%
    \fi
  \else
    \setbox\@cclv\vbox{\unvbox\@cclv\copy\LT@foot\vss}%
    \@makecol
    \@outputpage
    \global\vsize\@colroom
    \copy\LT@head\nobreak
  \fi}
\makeatother
\usepackage{float}
\usepackage{booktabs}
\usepackage{enumitem}
\usepackage{titlecaps}
\usepackage[pdfencoding=auto]{hyperref}
\usepackage{bookmark}

\usepackage{titlesec}


\setcounter{secnumdepth}{6} % Numera fino ai subsubparagraph (livello 6)
\setcounter{tocdepth}{6}    % Mostra nell'indice fino ai subsubparagraph (livello 6)
% ====== IMPOSTAZIONI GLOBALI DI STILE ======

\makeatletter

% ====== PARAGRAPH come subsubsection ======
\renewcommand\paragraph{\@startsection{paragraph}{4}
  {0pt}
  {2.5ex plus 1ex minus .2ex}
  {1.5ex plus .2ex}
  {\normalfont\normalsize\bfseries\color{AccentDark}}}

% ====== SUBPARAGRAPH come paragraph ======
\renewcommand\subparagraph{\@startsection{subparagraph}{5}
  {0pt}
  {2ex plus 0.8ex minus .2ex}
  {1ex plus .2ex}
  {\normalfont\normalsize\bfseries\color{AccentColor}}}

% ====== SUBSUBPARAGRAPH (livello 6, custom) ======
\newcounter{subsubparagraph}[subparagraph]
\renewcommand{\thesubsubparagraph}{\thesubparagraph.\arabic{subsubparagraph}}
\newcommand\subsubparagraph{\@startsection{subsubparagraph}{6}
  {0pt}
  {1.5ex plus 0.5ex minus .2ex}
  {0.8ex plus .2ex}
  {\normalfont\small\bfseries\color{AccentColor}}}

% Definizione dei formati TOC per paragraph, subparagraph e subsubparagraph
\renewcommand*\l@paragraph{\@dottedtocline{4}{5.5em}{4.5em}}
\renewcommand*\l@subparagraph{\@dottedtocline{5}{7.5em}{5.5em}}
\newcommand*\l@subsubparagraph{\@dottedtocline{6}{9.5em}{6.5em}}

\makeatother


% 1. DEFINIZIONE COLORI BLU-VIOLA
\definecolor{AccentColor}{RGB}{80, 90, 180} % Blu-viola principale
\definecolor{AccentLight}{RGB}{80, 90, 180} % Versione più chiara
\definecolor{AccentDark}{RGB}{50, 60, 140} % Versione più scura
\definecolor{LightGray}{RGB}{245, 245, 250}
\definecolor{MediumGray}{RGB}{200, 200, 210}

% 2. IMPOSTAZIONE MARGINI
\geometry{a4paper, left=2.5cm, right=2.5cm, top=3.5cm, bottom=3.5cm}

% 3. STILE DEI TITOLI DI SEZIONE
\titleformat{\section}
  {\normalfont\sffamily\Large\bfseries\color{AccentColor}}
  {\thesection}
  {1em}
  {}
\titleformat{\subsection}
  {\normalfont\sffamily\large\bfseries\color{AccentDark}}
  {\thesubsection}
  {1em}
  {}

  \titleformat{\subsubsection}
  {\normalfont\sffamily\large\bfseries\color{AccentDark}}
  {\thesubsubsection}
  {1em}
  {}

% 4. IMPOSTAZIONE HEADER E FOOTER
\pagestyle{fancy}
\fancyhf{} 
\fancyhead[L]{\sffamily\bfseries\color{AccentColor}@BYTE HOLDERS}
\fancyhead[R]{\sffamily\color{AccentColor}\thepage}
\renewcommand{\headrulewidth}{0.8pt}
\renewcommand{\headrule}{\color{AccentColor}\hrule width\headwidth height\headrulewidth \vskip-\headrulewidth}

% 5. IMPOSTAZIONE LINK
\hypersetup{
	colorlinks=true,
	linkcolor=AccentColor,
	urlcolor=AccentLight,
	citecolor=AccentDark,
}

% 6. PERSONALIZZAZIONE ELENCHI
\setlist[itemize]{itemsep=2pt, topsep=4pt}
\setlist[enumerate]{itemsep=2pt, topsep=4pt}

% ====== COMANDI PERSONALIZZATI ======
\newcommand{\refterm}[1]{\textit{#1}\textsuperscript{G}}
\newcommand{\term}[1]{\subsubsection{\refterm{#1}}}

% ====== STILE TABELLE MIGLIORATO ======
\newcolumntype{Y}{>{\raggedright\arraybackslash}X} % Colonna giustificata a sinistra
\setlength{\arrayrulewidth}{0.4pt} % Linee più sottili
\setlength{\tabcolsep}{10pt} % Spaziatura interna celle
\renewcommand{\arraystretch}{1.4} % Altezza righe

% ====== INIZIO DEL DOCUMENTO ======
\begin{document}
	

\pagestyle{empty}

% ====== PAGINA DI TITOLO ======
\begin{titlepage}
    \centering
    
    \includegraphics[width=0.55\textwidth]{../Assets/ByteHolders1.png}\par\vspace{1.5cm}
    
    {\LARGE \sffamily \color{AccentColor}\bfseries Analisi dei Requisiti}\par
    \vspace{0.5cm}
    {\large \color{AccentColor}\sffamily Versione 1.0.0}\par
    
    \vfill
    
    \noindent\color{AccentColor}\rule{\textwidth}{1pt}\par
    \vspace{0.5cm}
    
    \begin{tabularx}{\textwidth}{@{} >{\centering\arraybackslash\bfseries\sffamily}X >{\centering\arraybackslash\sffamily}X @{}}
        Responsabile & \sffamily Giacomo Nalotto  \\
        \arrayrulecolor{MediumGray}\hline \\[-1.5ex]
        
        Verificatori & \sffamily Alessandro Morabito, Giulia Romanato, Lorenzo Grolla, Giacomo Nalotto\\
        \arrayrulecolor{MediumGray}\hline \\[-1.5ex]
        
        Redattori & \sffamily Alessandro Morabito, Giulia Romanato, Lorenzo Grolla, Giacomo Nalotto  \\
        \arrayrulecolor{MediumGray}\hline \\[-1.5ex]
        
        Destinatari & \sffamily Prof. Tullio Vardanega, Prof. Riccardo Cardin, Var Group \\
        \arrayrulecolor{MediumGray}\hline \\[-1.5ex]
        
        Uso & \sffamily Esterno \\
        \arrayrulecolor{MediumGray}\hline \\[-1.5ex]
    \end{tabularx}
    
    \vfill
\end{titlepage}

\newpage

% ====== TABELLA DI VERSIONAMENTO ======
{\normalfont\sffamily\huge\bfseries\color{AccentColor} Registro delle versioni}
\vspace{1cm}

\renewcommand{\arraystretch}{1.0}
\rowcolors{2}{LightGray}{white}
\begin{longtable}{>{\centering\arraybackslash}p{1.5cm} >{\centering\arraybackslash}p{2cm} >{\raggedright\arraybackslash}p{2.5cm} >{\raggedright\arraybackslash}p{6.5cm}}
	\rowcolor{AccentColor}
	\textcolor{white}{\textbf{Versione}} &
	\textcolor{white}{\textbf{Data}} &
	\multicolumn{1}{c}{\textcolor{white}{\textbf{Autore}}} &
	\multicolumn{1}{c}{\textcolor{white}{\textbf{Descrizione delle modifiche}}} \\
	\endfirsthead
	\rowcolor{AccentColor}
	\textcolor{white}{\textbf{Versione}} &
	\textcolor{white}{\textbf{Data}} &
	\multicolumn{1}{c}{\textcolor{white}{\textbf{Autore}}} &
	\multicolumn{1}{c}{\textcolor{white}{\textbf{Descrizione delle modifiche}}} \\
	\endhead
	1.0.0 & 22/02/2026 & Lorenzo Grolla & Approvazione del documento \\
	0.19.0 & 16/02/2026 & Lorenzo Grolla & Fix errori formato \\
	0.18.0 & 15/02/2026 & Giacomo Nalotto & Aggiunta tabelle riassuntive dei requisiti \\
	0.17.0 & 14/02/2026 & Lorenzo Grolla & Aggiornamento UC14 e UC33.1.2 \\
	0.16.0 & 14/02/2026 & Giacomo Nalotto & Inizio stesura tabelle dei requisiti \\
	0.15.0 & 13/02/2026 & Giulia Romanato & Inserito UC25.1 Visualizza singolo elemento lista repository del workspace, sistemati grafici, cambiato numero Aggiorna repository (UC33->UC37) e aggiunta Aggiorna repository come extend a UC35, UC33, UC36\\
	0.14.0 & 13/02/2026 & Giacomo Nalotto & Sistemazione pagina di titolo \\
	0.13.0 & 12/02/2026 & Giulia Romanato & Cambio numero: (UC37->UC15, UC38->UC16, UC39->UC17) e unificato stile delle subsubsection \\
	0.12.0 & 12/02/2026 & Alessandro Morabito & Verifica del documento \\
	0.11.0 & 11/02/2026 & Giacomo Nalotto & Sistemazione paragrafi introduttivi da 1 a 3.2 \\
	0.10.0 & 08/02/2026 & Giulia Romanato & Stesura UC Backend (UC39 Esegui analisi con relativi sottocasi d'uso) \\
	0.9.0 & 08/02/2026 & Giulia Romanato & Stesura UC Analisi (da UC37 a UC38) e inserimento immagini da UC19 a UC27 e UC37 e UC38 \\
	0.8.0 & 08/02/2026 & Lorenzo Grolla & Stesura UC Verifica aggiornamento scansione \\
	0.7.0 & 06/02/2026 & Giacomo Nalotto & Stesura UC Visione analisi sicurezza, analisi documentazione e analisi qualità del codice\\
	0.6.0 & 04/02/2026 & Lorenzo Grolla & Stesura UC Visione Aggregata\\
	0.5.0 & 25/01/2026 & Lorenzo Grolla & Stesura UC Dettagli Repository e Selezione Branch\\
	0.4.0 & 19/01/2026 & Giulia Romanato & Stesura UC Gestione Respository e tag (da UC19 a UC27)\\
	0.3.0 & 18/01/2026 & Giacomo Nalotto & Stesura UC Gestione Workspace\\
	0.2.0 & 14/01/2026 & Lorenzo Grolla & Stesura UC Autenticazione\\
	0.1.0 & 19/12/2025 & Alessandro Morabito & Inizio stesura\\
\end{longtable}
\renewcommand{\arraystretch}{1.4}

% ====== INDICE ======
\pagestyle{fancy}
\newpage
\tableofcontents
\newpage

% ====== INTRODUZIONE ======
\section{Introduzione}
\subsection{Scopo del documento}
Il presente documento ha lo scopo di formalizzare i requisiti funzionali e non funzionali del sistema CodeGuardian, sviluppato dal gruppo Byte Holders. \\
La stesura dell'elaborato segue le linee guida dello standard IEEE 830-1998 e si rivolge ai seguenti stakeholder:
\begin{itemize}
	\item Var Group, azienda proponente e destinataria finale del prodotto software
	\item Byte Holders, team di sviluppo che utilizzerà il documento come riferimento tecnico
	\item Prof. Tullio Vardanega e Prof. Riccardo Cardin, committenti accademici e supervisori del progetto
\end{itemize}
\\
La struttura del documento prevede una descrizione generale del prodotto nella Sezione \ref{sec:descrizione generale}, seguita dalla specifica dettagliata dei casi d'uso nella Sezione \ref{sec:casi d'uso}.

\subsection{Scopo del prodotto}
Il sistema CodeGuardian è progettato per condurre analisi approfondite sulla qualità di repository GitHub, garantendo un controllo rigoroso sulla gestione degli accessi e sull'autorizzazione all'avvio delle procedure di scansione. 
\\
La piattaforma si propone come uno strumento strategico per team di sviluppo multidisciplinari, consentendo il monitoraggio centralizzato dello stato del software e la generazione di metriche aggregate su molteplici progetti analizzati.

\subsection{Glossario}
Al fine di garantire la massima chiarezza terminologica ed evitare ambiguità, i termini tecnici presenti in questo documento trovano una definizione puntuale nel \href{https://byte-holders.github.io/Documentazione/RTB/Glossario.pdf}{\refterm{Glossario}}. Tali lemmi sono identificati graficamente dall'uso del corsivo e dalla presenza della lettera \textit{G} posta in apice (ad es. \refterm{formula in glossario}). La validità del riferimento è estesa a tutte le possibili declinazioni e coniugazioni dei termini indicati.

% Dal momento che il glossario è un documento interno, possiamo mettere qua dentro tutti i termini che ci servono
\subsection{Definizioni, acronimi e abbreviazioni}
\term{Caso d'uso}
Un Caso d'uso rappresenta una descrizione coerente delle funzionalità che il sistema offre per soddisfare un obiettivo specifico di un utente. Formalmente, consiste in un insieme di scenari (principali, alternativi o di errore) che descrivono le interazioni tra gli attori e il sistema stesso, finalizzate al raggiungimento di un risultato di valore per l'interessato.

\term{Scenario}
Uno Scenario è una specifica sequenza di eventi e azioni che descrive un singolo percorso di esecuzione all'interno di un caso d'uso. Esso elenca i passi logici compiuti dagli attori e le risposte fornite dal sistema. Ogni caso d'uso comprende tipicamente uno Scenario Principale (il percorso ideale di successo) e uno o più Scenari Alternativi (che gestiscono varianti o condizioni di errore).

\term{Attore}
Un Attore è un'entità esterna al sistema che interagisce con esso per scambiare informazioni o attivare funzionalità. Un attore può essere un essere umano, un componente hardware o un altro sistema software. Gli attori vengono classificati in primari, se danno inizio al caso d'uso per raggiungere un obiettivo, o secondari, se forniscono un servizio al sistema durante l'esecuzione dello scenario.

\subsection{Riferimenti}
\subsubsection{Riferimenti normativi}
\begin{itemize}
	\item 
		Norme Di Progetto\\
		\url{https://byte-holders.github.io/Documentazione/RTB/Norme_Di_Progetto.pdf}
	\item
		830-1998 - IEEE Recommended Practice for Software Requirements Specifications\\
		\url{https://ieeexplore.ieee.org/document/720574}
	\item
		Capitolato\\
		\url{https://www.math.unipd.it/~tullio/IS-1/2025/Progetto/C2.pdf}
\end{itemize}
\subsubsection{Riferimenti informativi}
\begin{itemize}
	\item
		\refterm{Glossario}\\
		\url{https://byte-holders.github.io/Documentazione/RTB/Glossario.pdf}
	\item
		Specifica UML 2.5.1\\
		\url{https://www.omg.org/spec/UML/2.5.1/PDF}
\end{itemize}

\section{Descrizione generale} \label{sec:descrizione generale}
\subsection{Prospettiva del prodotto}
Il progetto CodeGuardian, sviluppato dal gruppo Byte Holders, si colloca come una soluzione software avanzata per la verifica automatizzata e la remediation di repository software. Il sistema è progettato come una piattaforma web basata su un'architettura a microservizi e agenti intelligenti (Multi-Agent System), in grado di interfacciarsi direttamente con repository GitHub.
\\ \\
L'obiettivo primario è fornire un'analisi approfondita su tre assi fondamentali: qualità del codice, sicurezza (con particolare attenzione agli standard OWASP) e documentazione. A differenza dei tradizionali strumenti di analisi statica, CodeGuardian non si limita a segnalare le criticità, ma sfrutta l'intelligenza artificiale generativa per proporre soluzioni pratiche (remediation) e generare reportistica dettagliata.
\\ \\
Il sistema gestisce l'accesso e la visibilità delle risorse tramite una logica a Workspace, permettendo a diverse tipologie di utenti di collaborare sugli stessi progetti con livelli di permessi granulari.

\subsection{Funzioni del prodotto}
Le principali funzionalità offerte dal sistema CodeGuardian sono riassumibili nelle seguenti macro-aree:
\begin{itemize}
    \item Gestione workspace e repository: creazione di spazi di lavoro condivisi, collegamento con account GitHub e selezione dei branch da analizzare
    \item Analisi multi-agente: orchestrazione di agenti IA specializzati per eseguire scansioni parallele su:
    \begin{itemize}
        \item Sicurezza
        \item Qualità del codice
        \item Documentazione
    \end{itemize}
    \item Dashboard e visualizzazione dati: visualizzazione di metriche sintetiche, grafici storici e liste di vulnerabilità filtrabili
    \item Reportistica: generazione ed esportazione di report sullo stato di salute del progetto
    \item Gestione utenti e ruoli: sistema con accesso basato sui ruoli RBAC per definire chi può avviare scansioni, visualizzare risultati o gestire il team
\end{itemize}

\subsection{Caratteristiche dell'utente}
Il software CodeGuardian si rivolge a team di sviluppo moderni, dove la collaborazione tra la componente gestionale e quella tecnica è essenziale per il successo di un progetto. L'utilità del prodotto risiede nella sua capacità di supportare le diverse figure professionali coinvolte nel ciclo di vita del software e di offrire una visione trasparente e condivisa dello stato di salute di quest'ultimo. 
\\
Un'azienda che adotta CodeGuardian mette a disposizione dei propri Project Manager uno strumento di controllo e organizzazione, permettendo loro di gestire il perimetro del team, amministrare i permessi nei vari spazi di lavoro e monitorare l'andamento generale dei progetti attraverso dati sintetici e report chiari, senza la necessità di possedere competenze di programmazione avanzate.
\\
Parallelamente, il sistema offre un supporto concreto a chi ha la responsabilità di guidare le scelte tecnologiche, come il Tech Lead, fornendo gli strumenti necessari per impostare standard di sicurezza elevati e validare la robustezza delle repository attraverso l'avvio di scansioni mirate. 
\\
Infine, CodeGuardian si rivela un assistente indispensabile per il Developer, che può concentrarsi sulla risoluzione rapida delle criticità grazie ai suggerimenti diretti forniti dall'intelligenza artificiale, riducendo i tempi di manutenzione e aumentando la sicurezza del codice scritto quotidianamente.

\section{Casi d'uso} \label{sec:casi d'uso}
\subsection{Lista degli Attori}
Nella definizione dei casi d'uso sono stati individuati i seguenti attori principali:
\subsubsection{Attori di Frontend}
\begin{itemize}
    	\item \textbf{Utente non autenticato}: un utente che non ha ancora effettuato l'accesso al sistema, le sue interazioni sono limitate alle procedure di autenticazione e registrazione
    
        \item \textbf{Utente}: rappresenta la generalizzazione di qualsiasi utente che ha effettuato l'accesso alla piattaforma, è l'attore base da cui ereditano tutti i ruoli specifici all'interno di un workspace
        
        \item \textbf{Project Manager}: è l'attore con privilegi gestionali sul workspace; si occupa di invitare i membri del team, assegnare i ruoli e configurare i permessi, la sua interazione con il sistema è focalizzata sulla consultazione di dashboard riepilogative e report manageriali per monitorare l'avanzamento del progetto, senza entrare nei dettagli tecnici del codice
        
        \item \textbf{Tech Lead}: è l'attore con i privilegi tecnici più elevati; oltre a visualizzare tutte le analisi, ha la responsabilità di configurare le repository e detiene il permesso critico per l'avvio delle scansioni, supervisiona la qualità generale del codice e definisce gli standard per il team di sviluppo
        
        \item \textbf{Developer}: è l'attore operativo che lavora quotidianamente sul codice, ha accesso completo ai dettagli tecnici delle analisi (sicurezza, qualità, documentazione) e utilizza attivamente gli agenti AI per ottenere suggerimenti di remediation e risolvere le vulnerabilità individuate
\end{itemize}
\subsubsection{Attori di Backend}
    \begin{itemize}
        \item \textbf{Sistema}: attore per i processi automatici e di backend, rappresenta l'entità logica che esegue le operazioni di analisi, orchestrazione e gestione dei dati senza intervento umano diretto durante l'esecuzione
    
        \item \textbf{Orchestratore}: è il componente centrale del backend che coordina l'intero flusso di lavoro, riceve le richieste dagli utenti (o dai timer), istanzia gli Agenti necessari, raccoglie i loro report parziali, aggrega i risultati finali e gestisce il salvataggio dei dati

        \item \textbf{Agente}: componente software autonomo basato su AI specializzato in uno specifico dominio di analisi (es. Agente sicurezza, Agente qualità, Agente documentazione), riceve porzioni di codice dall'Orchestratore, esegue l'analisi e restituisce un report parziale
        
        \item \textbf{Database}: attore passivo responsabile della persistenza dei dati, interagisce con l'Orchestratore per il salvataggio dei report di scansione, delle configurazioni dei workspace e dei profili utente

        \item \textbf{Sistema Esterno (GitHub)}: rappresenta la piattaforma esterna con cui il sistema interagisce per clonare le repository, recuperare i metadati dei branch e sincronizzare le modifiche al codice
    \end{itemize}

\subsection{Struttura generale di un caso d'uso}
Si è deciso di descrivere ciascun \refterm{caso d'uso} seguendo la seguente struttura (\underline{sottolineati} i campi sempre popolati):

\begin{tabular}{|p{.25\linewidth} p{.6\linewidth}|}
	\hline
	\textbf{\underline{Codice}} 
	&
	Codice identificativo utilizzato per far riferimento al \refterm{caso d'uso} corrente\\
	
	\textbf{\underline{Titolo}} 
	&
	Titolo del \refterm{caso d'uso} corrente\\
	
	\textbf{\underline{Attori principali}} 
	&
	Attori che agiscono sul sistema dando inizio allo scenario\\
	
	\textbf{\underline{Attori secondari}}
	&
	Attori di supporto che agiscono in risposta a stimoli del sistema\\
	
	\textbf{\underline{Precondizioni}}
	&
	Condizioni necessarie per l'esecuzione del \refterm{caso d'uso} corrente\\
	
	\textbf{\underline{Postcondizioni}} 
	&
	Condizioni in cui viene lasciato il sistema al termine dello \refterm{scenario principale}\\
	
	\textbf{\underline{Scenario principale}}
	&
	Descrizione degli eventi che avvengono all'interno dello \refterm{scenario principale}\\
	
	\textbf{\underline{Inclusioni}} 
	&
	Lista dei riferimenti a \refterm{casi d'uso} terzi \refterm{inclusi} dal \refterm{caso d'uso} corrente e al quale si fa riferimento nella sezione \textit{Scenario principale}\\
	
	\textbf{\underline{Scenari alternativi}} 
	&
	Descrizione delle situazione che portano a \refterm{scenari alternativi}\\
	
	\textbf{\underline{Eredita da}} 
	&
	Codice del \refterm{caso d'uso} terzo da cui eredita il \refterm{caso d'uso} corrente (non ammettiamo ereditarietà multipla)\\
	
	\hline
\end{tabular}

\newpage
\subsection{Lista dei \refterm{casi d'uso}}

% UC1: REGISTRAZIONE UTENTE
\subsubsection{UC1 - Registrazione}\label{UC1}
\begin{figure}[h]
	\centering
	\includegraphics[width=0.8\textwidth]{../Assets/AdR/UC1ext.png}
	\caption{UC1 - Registrazione}
	\label{fig:UC1ext}
\end{figure}

\begin{itemize}
	\item \textbf{Attori principali:} Utente non Autenticato
	\item \textbf{Precondizioni:} 
	\begin{itemize}
		\item Il sistema è attivo e funzionante
		\item L'utente non possiede ancora un account attivo nel sistema
	\end{itemize}
	\item \textbf{Postcondizioni:} Viene creato un nuovo profilo utente nel sistema CodeGuardian con stato "da confermare".
	\item \textbf{Scenario principale:}
	\begin{enumerate}
		\item L'utente accede alla pagina di registrazione
		\item L'utente inserisce l'username (\hyperref[UC1.1]{UC1.1})
		\item L'utente inserisce l'email (\hyperref[UC1.2]{UC1.2})
		\item L'utente inserisce la password (\hyperref[UC1.3]{UC1.3})
		\item L'utente conferma la registrazione
		\item Il sistema valida i dati e crea l'utente su Amazon Cognito
		\item Il sistema invia un codice OTP all'email fornita
	\end{enumerate}
	\item \textbf{Scenario alternativo:}
	\begin{enumerate}
		\item L'utente ha inserito un errore nei dati inseriti come username già esistente, email già registrata oppure password non valida (\hyperref[UC1.4]{UC1.4})
	\end{enumerate}
	\item \textbf{Inclusioni:} \hyperref[UC1.1]{UC1.1}, \hyperref[UC1.2]{UC1.2}, \hyperref[UC1.3]{UC1.3},

	\item \textbf{Estensioni:} \hyperref[UC1.4]{UC1.4}
	
\end{itemize}
\newpage

Il caso d'Uso UC1 include ulteriori casi d'uso come rappresentato nella seguente immagine:
\begin{figure}[H]
	\centering
	\includegraphics[width=0.6\textwidth]{../Assets/AdR/UC1.png}
	\caption{Inclusioni di UC1: UC1.1,UC1.2,UC1.3}
	\label{fig:UC1}
\end{figure}

% --- Sottocasi di UC1 ---
\paragraph{UC1.1 - Inserimento username}\label{UC1.1}
\begin{itemize}
	\item \textbf{Attori principali:} Utente non Autenticato
	\item \textbf{Precondizioni:} 
	\begin{itemize}
		\item Il sistema è attivo e funzionante
		\item L'utente si trova nella pagina di registrazione
	\end{itemize}
	\item \textbf{Postcondizioni:} L'username è inserito nel sistema.
	\item \textbf{Scenario principale:} L'utente digita l'username scelto nell'apposito campo.
\end{itemize}

\paragraph{UC1.2 - Inserimento email}\label{UC1.2}
\begin{itemize}
	\item \textbf{Attori principali:} Utente non Autenticato
	\item \textbf{Precondizioni:} 
	\begin{itemize}
		\item Il sistema è attivo e funzionante
		\item L'utente si trova nella pagina di registrazione
	\end{itemize}
	\item \textbf{Postcondizioni:} L'email è inserita nel sistema.
	\item \textbf{Scenario principale:} L'utente digita il proprio indirizzo email nell'apposito campo.
\end{itemize}

\paragraph{UC1.3 - Inserimento password}\label{UC1.3}
\begin{itemize}
	\item \textbf{Attori principali:} Utente non Autenticato
	\item \textbf{Precondizioni:} 
	\begin{itemize}
		\item Il sistema è attivo e funzionante
		\item L'utente si trova nella pagina di registrazione
	\end{itemize}
	\item \textbf{Postcondizioni:} La password è inserita nel sistema.
	\item \textbf{Scenario principale:} L'utente digita la password desiderata nell'apposito campo.
\end{itemize}

\paragraph{UC1.4 - Registrazione fallita}\label{UC1.4}
\begin{itemize}
	\item \textbf{Attori principali:} Utente non Autenticato
	\item \textbf{Precondizioni:} 
	\begin{itemize}
		\item Il sistema è attivo e funzionante
		\item L'utente ha tentato la conferma della registrazione con dati non validi
	\end{itemize}
	\item \textbf{Postcondizioni:} La registrazione non viene completata; l'utente rimane nella pagina di registrazione.
	\item \textbf{Scenario principale:}
	\begin{enumerate}
		\item Il sistema rileva un errore nei dati inseriti (username già esistente, email già registrata oppure password non conforme ai requisiti di sicurezza)
		\item Il sistema mostra un messaggio di errore specifico all'utente
		\item Il sistema permette all'utente di correggere i dati mantenendo quelli validi
	\end{enumerate}
\end{itemize}


% UC2: CONFERMA REGISTRAZIONE 
\subsubsection{UC2 - Conferma registrazione}\label{UC2}
\begin{figure}[H]
	\centering
	\includegraphics[width=0.7\textwidth]{../Assets/AdR/UC2.png}
	\caption{UC2 - Conferma Registrazione}
	\label{fig:UC2}
\end{figure}

\begin{itemize}
	\item \textbf{Attori principali:} Utente non Autenticato
	\item \textbf{Precondizioni:} 
		\begin{itemize}
			\item Il sistema è attivo e funzionante
			\item L'utente ha completato la prima fase di registrazione (\hyperref[UC1]{UC1})
			\item L'utente ha ricevuto il codice OTP via email
		\end{itemize}
	\item \textbf{Postcondizioni:} L'account viene attivato e l'utente può effettuare il login.
	\item \textbf{Scenario principale:}
	\begin{enumerate}
		\item L'utente accede alla pagina di conferma registrazione
		\item Il sistema richiede il codice di verifica
		\item L'utente inserisce il codice OTP (\hyperref[UC2.1]{UC2.1})
		\item L'utente conferma l'invio
		\item Il sistema verifica il codice e attiva l'account
	\end{enumerate}
	\item \textbf{Scenario alternativo:}
	\begin{enumerate}
		\item L'utente ha inserito un codice OTP non valido o scaduto(\hyperref[UC2.2]{UC2.2})
	\end{enumerate}
	\item \textbf{Inclusioni:} \hyperref[UC2.1]{UC2.1}
	\item \textbf{Estensioni:} \hyperref[UC2.2]{UC2.2}
\end{itemize}

% --- Sottocasi di UC2 ---
\paragraph{UC2.1 - Inserimento codice OTP}\label{UC2.1}
\begin{itemize}
	\item \textbf{Attori principali:} Utente non Autenticato
	\item \textbf{Precondizioni:} 
	\begin{itemize}
		\item Il sistema è attivo e funzionante
		\item L'utente si trova nella pagina di conferma registrazione
	\end{itemize}
	\item \textbf{Postcondizioni:} Il codice OTP è inserito nel sistema.
	\item \textbf{Scenario principale:} L'utente inserisce il codice numerico ricevuto via email nell'apposito campo.
\end{itemize}

\paragraph{UC2.2 - Verifica fallita}\label{UC2.2}
\begin{itemize}
	\item \textbf{Attori principali:} Utente non Autenticato
	\item \textbf{Precondizioni:} 
	\begin{itemize}
		\item Il sistema è attivo e funzionante
		\item L'utente ha inserito un codice OTP errato o scaduto
	\end{itemize}
	\item \textbf{Postcondizioni:} L'account rimane nello stato "da confermare".
	\item \textbf{Scenario principale:}
	\begin{enumerate}
		\item Il sistema rileva che il codice non è valido o è scaduto
		\item Il sistema mostra un messaggio di errore "Codice non valido o scaduto"
		\item Il sistema offre l'opzione per richiedere un nuovo codice OTP
	\end{enumerate}
\end{itemize}


% UC3: LOGIN 
\subsubsection{UC3 - Login} \label{UC3}
\begin{figure}[H]
	\centering
	\includegraphics[width=0.7\textwidth]{../Assets/AdR/UC3ext.png}
	\caption{UC3 - Login}
	\label{fig:UC3ext}
\end{figure}
\begin{itemize}
	\item \textbf{Attori principali:} Utente non Autenticato
	\item \textbf{Precondizioni:} 
		\begin{itemize}
			\item Il sistema è attivo e funzionante
			\item L'utente possiede un account attivo nel sistema
		\end{itemize}
	\item \textbf{Postcondizioni:} L'utente è autenticato e accede alla home del sistema
	\item \textbf{Scenario principale:}
	\begin{enumerate}
		\item L'utente accede alla pagina di login
		\item L'utente inserisce l'username (\hyperref[UC3.1]{UC3.1})
		\item L'utente inserisce la password (\hyperref[UC3.2]{UC3.2})
		\item L'utente conferma l'accesso
		\item Il sistema valida le credenziali tramite Amazon Cognito
		\item Il sistema reindirizza l'utente alla home
	\end{enumerate}
	\item \textbf{Scenari Alternativi}
	\begin{enumerate}
		\item L'utente ha inserito credenziali errate o inesistenti(\hyperref[UC3.3]{UC3.3})
	\end{enumerate}
	\item \textbf{Inclusioni:} \hyperref[UC3.1]{UC3.1}, \hyperref[UC3.2]{UC3.2}
	\item \textbf{Estensioni:} \hyperref[UC3.3]{UC3.3}
	
\end{itemize}

% --- Sottocasi di UC3 ---
Il caso d'Uso UC3 include ulteriori casi d'uso come rappresentato nella seguente immagine:
\begin{figure}[H]
	\centering
	\includegraphics[width=0.6\textwidth]{../Assets/AdR/UC3.png}
	\caption{Inclusioni di UC3: UC3.1,UC3.2}
	\label{fig:UC3}
\end{figure}


\paragraph{UC3.1 - Inserimento username o email}\label{UC3.1}
\begin{itemize}
	\item \textbf{Attori principali:} Utente non Autenticato
	\item \textbf{Precondizioni:} 
	\begin{itemize}
		\item Il sistema è attivo e funzionante
		\item L'utente si trova nella pagina di login
	\end{itemize}
	\item \textbf{Postcondizioni:} L'username è inserito nel sistema.
	\item \textbf{Scenario principale:} L'utente inserisce il proprio username o email nell'apposito campo.
\end{itemize}

\paragraph{UC3.2 - Inserimento password}\label{UC3.2}
\begin{itemize}
	\item \textbf{Attori principali:} Utente non Autenticato
	\item \textbf{Precondizioni:} 
	\begin{itemize}
		\item Il sistema è attivo e funzionante
		\item L'utente si trova nella pagina di login
	\end{itemize}
	\item \textbf{Postcondizioni:} La password è inserita nel sistema.
	\item \textbf{Scenario principale:} L'utente inserisce la propria password nell'apposito campo.
\end{itemize}

\paragraph{UC3.3 - Login fallito}\label{UC3.3}
\begin{itemize}
	\item \textbf{Attori principali:} Utente non Autenticato
	\item \textbf{Precondizioni:} 
	\begin{itemize}
		\item Il sistema è attivo e funzionante
		\item L'utente ha inviato credenziali non valide
	\end{itemize}
	\item \textbf{Postcondizioni:} L'utente rimane non autenticato.
	\item \textbf{Scenario principale:}
	\begin{enumerate}
		\item Il sistema verifica che le credenziali non corrispondono a nessun account attivo
		\item Il sistema mostra il messaggio "Username o password errati"
		\item Il sistema permette di riprovare l'inserimento delle credenziali
	\end{enumerate}
\end{itemize}

% UC4: RECUPERO PASSWORD 
\subsubsection{UC4 - Recupero password}\label{UC4}
\begin{figure}[H]
	\centering
	\includegraphics[width=0.7\textwidth]{../Assets/AdR/UC4ext.png}
	\caption{UC4}
	\label{fig:UC4ext}
\end{figure}

\begin{itemize}
	\item \textbf{Attori principali:} Utente non Autenticato
	\item \textbf{Precondizioni:} 
	\begin{itemize}
		\item Il sistema è attivo e funzionante
		\item L'utente possiede un account registrato
		\item L'utente necessita di reimpostare la propria password
        \item L'utente ha selezionato l'opzione "Recupera password"
	\end{itemize}
	\item \textbf{Postcondizioni:} La password viene reimpostata con successo.
	\item \textbf{Scenario principale:}
	\begin{enumerate}
		\item L'utente accede alla funzionalità di recupero password dalla pagina di login
		\item L'utente inserisce l'email associata al proprio account (\hyperref[UC4.1]{UC4.1})
		\item L'utente conferma la richiesta
		\item Il sistema valida l'email e genera un codice OTP
		\item Il sistema invia il codice OTP via email
		\item L'utente inserisce il codice OTP ricevuto (\hyperref[UC4.2]{UC4.2})
		\item L'utente inserisce la nuova password (\hyperref[UC4.3]{UC4.3})
		\item L'utente conferma il cambio password
		\item Il sistema valida il codice OTP, verifica che la nuova password rispetti i requisiti di sicurezza e aggiorna la password
		\item Il sistema conferma l'aggiornamento e reindirizza alla pagina di login
	\end{enumerate}
	\item \textbf{Scenario alternativo:}
	\begin{enumerate}
		\item L'utente ha inserito credenziali errate oppure il codice OTP è scaduto o errato(\hyperref[UC4.4]{UC4.4})
	\end{enumerate}
	\item \textbf{Inclusioni:} \hyperref[UC4.1]{UC4.1}, \hyperref[UC4.2]{UC4.2}, \hyperref[UC4.3]{UC4.3}
	\item \textbf{Estensioni:} \hyperref[UC4.4]{UC4.4}
\end{itemize}

% --- Sottocasi di UC4 ---
Il caso d'Uso UC4 include ulteriori casi d'uso come rappresentato nella seguente immagine:
\begin{figure}[h]
	\centering
	\includegraphics[width=0.6\textwidth]{../Assets/AdR/UC4.png}
	\caption{Inclusioni di UC4: UC4.1,UC4.2,UC4.3}
	\label{fig:UC4}
\end{figure}

\paragraph{UC4.1 - Inserimento email per recupero}\label{UC4.1}
\begin{itemize}
	\item \textbf{Attori principali:} Utente non Autenticato
	\item \textbf{Precondizioni:} 
	\begin{itemize}
		\item Il sistema è attivo e funzionante
		\item L'utente si trova nella pagina di recupero password
	\end{itemize}
	\item \textbf{Postcondizioni:} Il sistema invia il codice OTP all'email fornita.
	\item \textbf{Scenario principale:} L'utente inserisce l'email associata all'account nell'apposito campo e conferma la richiesta.
\end{itemize}

\paragraph{UC4.2 - Inserimento codice OTP}\label{UC4.2}
\begin{itemize}
	\item \textbf{Attori principali:} Utente non Autenticato
	\item \textbf{Precondizioni:} 
	\begin{itemize}
		\item Il sistema è attivo e funzionante
		\item L'utente ha ricevuto il codice OTP via email
		\item L'utente si trova nella pagina di reset password
	\end{itemize}
	\item \textbf{Postcondizioni:} Il codice OTP è inserito nel sistema.
	\item \textbf{Scenario principale:} L'utente inserisce il codice OTP ricevuto via email nell'apposito campo.
\end{itemize}

\paragraph{UC4.3 - Inserimento nuova password}\label{UC4.3}
\begin{itemize}
	\item \textbf{Attori principali:} Utente non Autenticato
	\item \textbf{Precondizioni:} 
	\begin{itemize}
		\item Il sistema è attivo e funzionante
		\item L'utente ha inserito il codice OTP
		\item L'utente si trova nella pagina di reset password
	\end{itemize}
	\item \textbf{Postcondizioni:} La nuova password è inserita nel sistema.
	\item \textbf{Scenario principale:} L'utente inserisce la nuova password desiderata nell'apposito campo.
\end{itemize}

\paragraph{UC4.4 - Ripristino fallito}\label{UC4.4}
\begin{itemize}
	\item \textbf{Attori principali:} Utente non Autenticato
	\item \textbf{Precondizioni:} 
	\begin{itemize}
		\item Il sistema è attivo e funzionante
		\item L'utente ha inserito un codice OTP non valido o scaduto, oppure una password non conforme ai requisiti
	\end{itemize}
	\item \textbf{Postcondizioni:} La password rimane invariata.
	\item \textbf{Scenario principale:}
	\begin{enumerate}
		\item Il sistema rileva che il codice OTP è errato o scaduto, oppure che la password non rispetta i requisiti di sicurezza
		\item Il sistema mostra un messaggio di errore specifico
		\item Il sistema impedisce il cambio password e permette di richiedere un nuovo codice o correggere la password
	\end{enumerate}
\end{itemize}

\subsubsection{UC5 - Invito utente in workspace}\label{UC5}

\begin{figure}[h]
	\centering
	\includegraphics[width=0.8\textwidth]{../Assets/AdR/UC5ext.png}
	\caption{UC5 - Invito utente in workspace}
	\label{fig:UC5ext}
\end{figure}

\begin{itemize}
	\item \textbf{Attori principali:} Project Manager
	\item \textbf{Precondizioni:} 
	\begin{itemize}
		\item Il sistema è attivo e funzionante
		\item L'utente è riconosciuto dal sistema come Project Manager
        \item L'utente fa parte del workspace all'interno del quale invita un altro utente
	\end{itemize}
    \item \textbf{Postcondizioni:} L'invito viene inviato ed un utente esterno al workspace lo riceve
	\item \textbf{Scenario principale:}
	\begin{enumerate}
		\item Un utente che fa parte di un workspace vuole invitare un altro utente che non ne fa parte
		\item Inserisce l'username dell'utente che vuole invitare (\hyperref[UC5.1]{UC5.1})
		\item Seleziona al nuovo utente il ruolo scegliendo dalla lista dei ruoli del workspace (\hyperref[UC5.2]{UC5.2})
	\end{enumerate}
    \item \textbf{Inclusioni:} \hyperref[UC5.1]{UC5.1}, \hyperref[UC5.2]{UC5.2}
    \item \textbf{Estensioni:} \hyperref[UC5.3]{UC5.3}
	
\end{itemize}

Il caso d'Uso UC5 include ulteriori casi d'uso come rappresentato nella seguente immagine:
\begin{figure}[h]
	\centering
	\includegraphics[width=0.8\textwidth]{../Assets/AdR/UC5.png}
	\caption{Inclusioni di UC5: UC5.1, UC5.2, UC5.2.1}
	\label{fig:UC5}
\end{figure}

\paragraph{UC5.1 - Inserimento username nuovo utente}\label{UC5.1}
\begin{itemize}
	\item \textbf{Attori principali:} Project Manager
	\item \textbf{Precondizioni:} 
	\begin{itemize}
		\item Il sistema è attivo e funzionante
		\item L'utente è autenticato nel sistema
        \item L'utente fa parte del workspace all'interno del quale invita un altro utente
        \item L'utente ha avviato la procedura di invito di un nuovo utente
	\end{itemize}
    \item \textbf{Postcondizioni:} Il sistema riceve l'username dell'utente da invitare
	\item \textbf{Scenario principale:} Il sistema presenta un campo di input per l'inserimento dell'username, l'utente digita l'username dell'utente che desidera invitare
	
\end{itemize}

\paragraph{UC5.2 - Selezione ruolo da assegnare al nuovo utente}\label{UC5.2}
\begin{itemize}
	\item \textbf{Attori principali:} Project Manager
	\item \textbf{Precondizioni:} 
	\begin{itemize}
		\item Il sistema è attivo e funzionante
		\item L'utente è autenticato nel sistema
        \item L'utente fa parte del workspace all'interno del quale invita un altro utente
        \item L'utente ha avviato la procedura di invito di un nuovo utente
	\end{itemize}
    	\item \textbf{Postcondizioni:} Il sistema riceve la selezione del ruolo da assegnare all'utente invitato
	\item \textbf{Scenario principale:} 
    \begin{enumerate}
    \item Il sistema recupera la lista dei ruoli disponibili nel workspace
    \item Il sistema presenta una lista a tendina con i ruoli disponibili
    \item L'utente seleziona il ruolo desiderato dalla lista
    \item Il sistema memorizza il ruolo selezionato
    \end{enumerate}
    \item \textbf{Inclusioni:} \hyperref[UC11]{UC11}
\end{itemize}

\paragraph{UC5.3 - Invito non riuscito}\label{UC5.3}
\begin{itemize}
	\item \textbf{Attori principali:} Project Manager
	\item \textbf{Precondizioni:} 
	\begin{itemize}
		\item Il sistema è attivo e funzionante
		\item L'utente è autenticato nel sistema
        \item L'utente ha tentato l'invio di un invito
        \item L'utente ha avviato la procedura di invito di un nuovo utente
	\end{itemize}
    \item \textbf{Postcondizioni:} 
    \begin{itemize}
    \item Il sistema annulla il tentativo di invito e nessun invito viene creato o inviato
    \item Viene mostrato a schermo un messaggio di errore specifico
    \end{itemize}
	\item \textbf{Scenario principale:} 
    \begin{enumerate}
    \item Il sistema tenta di validare l'username inserito dall'utente
    \item Il sistema rileva che l'username è già presente nel workspace o non esiste
    \end{enumerate}
	
\end{itemize}

\subsubsection{UC6 - Visualizzazione inviti in workspace}\label{UC6}

\begin{figure}[h]
	\centering
	\includegraphics[width=0.8\textwidth]{../Assets/AdR/UC6.png}
	\caption{UC6 e UC6.1}
	\label{fig:UC6}
\end{figure}

\begin{itemize}
	\item \textbf{Attori principali:} Utente
	\item \textbf{Precondizioni:} 
	\begin{itemize}
		\item Il sistema è attivo e funzionante
		\item L'utente è autenticato nel sistema
        \item L'utente ha selezionato l'opzione per visualizzare i propri inviti
	\end{itemize}	\item \textbf{Postcondizioni:} Il sistema mostra la lista completa degli inviti ricevuti dall'utente

	\item \textbf{Scenario principale:}
	\begin{enumerate}
		\item L'utente accede alla sezione degli inviti ricevuti
		\item Il sistema recupera tutti gli inviti pendenti destinati all'utente e li mostra in una lista
		\item Per ogni invito vengono visualizzate le informazioni (\hyperref[UC6.1]{UC6.1})
	\end{enumerate}
    \item \textbf{Inclusioni:} \hyperref[UC6.1]{UC6.1}
\end{itemize}

\paragraph{UC6.1 - Visualizzazione singolo invito in workspace}\label{UC6.1}
\begin{itemize}
	\item \textbf{Attori principali:} Utente
	\item \textbf{Precondizioni:} 
	\begin{itemize}
		\item Il sistema è attivo e funzionante
		\item L'utente è autenticato nel sistema
        \item L'utente ha selezionato l'opzione per visualizzare i propri inviti
        \item Esiste almeno un invito destinato all'utente
	\end{itemize}
    \item \textbf{Postcondizioni:} Il sistema mostra le informazioni complete di un singolo invito nella lista
	\item \textbf{Scenario principale:} Il sistema mostra un elemento della lista inviti contenente:
	\begin{itemize}
		\item Username del mittente dell'invito (\hyperref[UC6.1.1]{UC6.1.1})
		\item Username dell'owner del workspace (\hyperref[UC6.1.2]{UC6.1.2})
        \item Nome del workspace (\hyperref[UC6.1.3]{UC6.1.3})
	\end{itemize}
    \item \textbf{Inclusioni:} \hyperref[UC6.1.1]{UC6.1.1}, \hyperref[UC6.1.2]{UC6.1.2}, \hyperref[UC6.1.3]{UC6.1.3}
	
\end{itemize}

Il caso d'Uso UC6.1 include ulteriori casi d'uso come rappresentato nella seguente immagine:
\begin{figure}[h]
	\centering
	\includegraphics[width=0.8\textwidth]{../Assets/AdR/UC6-1.png}
	\caption{Figura 10: Inclusioni di UC6.1: UC6.1.1, UC6.1.2, UC6.1.3}
	\label{fig:UC6-1}
\end{figure}

\subparagraph{UC6.1.1 - Visualizzazione username del mittente dell'invito ricevuto}\label{UC6.1.1}
\begin{itemize}
	\item \textbf{Attori principali:} Utente
	\item \textbf{Precondizioni:} 
	\begin{itemize}
		\item Il sistema è attivo e funzionante
		\item L'utente è autenticato nel sistema
        \item L'utente ha selezionato l'opzione per visualizzare i propri inviti
        \item L'utente sta visualizzando un singolo invito
	\end{itemize}
    \item \textbf{Postcondizioni:} Il sistema mostra l'username dell'utente che ha inviato l'invito
	\item \textbf{Scenario principale:} 
	\begin{enumerate}
	    \item Il sistema recupera l'username dell'utente mittente associato all'invito
        \item Il sistema visualizza l'username del mittente all'interno dell'elemento invito
	\end{enumerate}
\end{itemize}

\subparagraph{UC6.1.2 - Visualizzazione username dell'owner del workspace}\label{UC6.1.2}
\begin{itemize}
	\item \textbf{Attori principali:} Utente
	\item \textbf{Precondizioni:} 
	\begin{itemize}
		\item Il sistema è attivo e funzionante
		\item L'utente è autenticato nel sistema
        \item L'utente ha selezionato l'opzione per visualizzare i propri inviti
        \item L'utente sta visualizzando un singolo invito
	\end{itemize}
    \item \textbf{Postcondizioni:} Il sistema mostra l'username dell'owner del workspace a cui l'utente è stato invitato
	\item \textbf{Scenario principale:} 
	\begin{enumerate}
	    \item Il sistema recupera l'username dell'owner del workspace a cui l'utente è stato invitato
        \item Il sistema visualizza l'username dell'owner all'interno dell'elemento invito
	\end{enumerate}
	
\end{itemize}

\subparagraph{UC6.1.3 - Visualizzazione nome del workspace}\label{UC6.1.3}
\begin{itemize}
	\item \textbf{Attori principali:} Utente
	\item \textbf{Precondizioni:} 
	\begin{itemize}
		\item Il sistema è attivo e funzionante
		\item L'utente è autenticato nel sistema
        \item L'utente ha selezionato l'opzione per visualizzare i propri inviti
        \item L'utente sta visualizzando un singolo invito
	\end{itemize}
    \item \textbf{Postcondizioni:} Il sistema mostra il nome identificativo del workspace a cui l'utente è stato invitato
	\item \textbf{Scenario principale:} 
	\begin{enumerate}
	    \item Il sistema recupera il nome del workspace a cui l'utente è stato invitato
        \item Il sistema visualizza il nome del workspace all'interno dell'elemento invito
	\end{enumerate}
	
\end{itemize}

\subsubsection{UC7 - Gestione degli inviti}\label{UC7}

\begin{figure}[h]
	\centering
	\includegraphics[width=0.8\textwidth]{../Assets/AdR/UC7.png}
	\caption{UC7 - Gestione degli inviti}
	\label{fig:UC7}
\end{figure}

\begin{itemize}
	\item \textbf{Attori principali:} Utente
	\item \textbf{Precondizioni:} 
	\begin{itemize}
		\item Il sistema è attivo e funzionante
		\item L'utente è autenticato nel sistema
        \item L'utente ha selezionato un invito dalla lista degli inviti ricevuti
	\end{itemize}
    \item \textbf{Postcondizioni:} L'utente ha gestito l'invito selezionato
	\item \textbf{Scenario principale:}
    \begin{itemize}
        \item Il sistema mostra le opzioni disponibili per gestire l'invito
    \end{itemize}
    \item \textbf{UC che ereditano:} \hyperref[UC7.1]{UC7.1}, \hyperref[UC7.2]{UC7.2}
\end{itemize}

\paragraph{UC7.1 - Accettazione invito}\label{UC7.1}
\begin{itemize}
	\item \textbf{Attori principali:} Utente
	\item \textbf{Precondizioni:} 
	\begin{itemize}
		\item Il sistema è attivo e funzionante
		\item L'utente è stato identificato dal sistema come Utente
        \item L'utente ha selezionato un invito dalla lista degli inviti ricevuti
	\end{itemize}
    \item \textbf{Postcondizioni:} L'utente accetta l'invito selezionato e diventa nuovo membro del workspace con il ruolo specificato nell'invito
	\item \textbf{Scenario principale:}
    \begin{enumerate}
        \item L'utente seleziona l'opzione "Accetta" per l'invito
        \item Il sistema aggiunge l'utente al workspace con il ruolo specificato
        \item Il sistema rimuove l'invito dalla lista degli inviti pendenti
        \item Il sistema aggiorna la lista dei workspace dell'utente includendo il nuovo workspace
    \end{enumerate}
    \item \textbf{Eredita da:} \hyperref[UC7]{UC7}
\end{itemize}

\paragraph{UC7.2 - Rifiuto invito}\label{UC7.2}
\begin{itemize}
	\item \textbf{Attori principali:} Utente
	\item \textbf{Precondizioni:} 
	\begin{itemize}
		\item Il sistema è attivo e funzionante
		\item L'utente è autenticato nel sistema
        \item L'utente ha selezionato un invito dalla lista degli inviti ricevuti
	\end{itemize}
    \item \textbf{Postcondizioni:} L'utente rifiuta l'invito selezionato e non diventa nuovo membro del workspace
	\item \textbf{Scenario principale:}
    \begin{enumerate}
        \item L'utente seleziona l'opzione "Rifiuta" per l'invito
        \item Il sistema rimuove l'invito dalla lista degli inviti pendenti
    \end{enumerate}
    \item \textbf{Eredita da:} \hyperref[UC7]{UC7}
\end{itemize}

\subsubsection{UC8 - Visualizzazione lista workspace}\label{UC8}

\begin{figure}[h]
	\centering
	\includegraphics[width=0.8\textwidth]{../Assets/AdR/UC8.png}
	\caption{UC8 e UC8.1}
	\label{fig:UC8}
\end{figure}

\begin{itemize}
	\item \textbf{Attori principali:} Utente
	\item \textbf{Precondizioni:} 
	\begin{itemize}
		\item Il sistema è attivo e funzionante
		\item L'utente è autenticato nel sistema
        \item L'utente sta visualizzando la dashboard iniziale
	\end{itemize}
    \item \textbf{Postcondizioni:} Il sistema mostra la lista completa dei workspace di cui l'utente fa parte
	\item \textbf{Scenario principale:}
    \begin{enumerate}
        \item Il sistema recupera tutti i workspace a cui l'utente appartiene
        \item Il sistema mostra la lista dei workspace di cui l'utente fa parte
        \item La lista è composta da singoli elementi che rappresentano ciascuno un workspace (\hyperref[UC8.1]{UC8.1})
    \end{enumerate}
    \item \textbf{Inclusioni:} \hyperref[UC8.1]{UC8.1}
\end{itemize}

Il caso d'Uso UC8.1 include ulteriori casi d'uso come rappresentato nella seguente immagine:
\begin{figure}[h]
	\centering
	\includegraphics[width=0.7\textwidth]{../Assets/AdR/UC8-1.png}
	\caption{Inclusioni di UC8.1: UC8.1.1, UC8.1.2}
	\label{fig:UC8-1}
\end{figure}

\paragraph{UC8.1 - Visualizzazione elemento della lista di workspace}\label{UC8.1}
\begin{itemize}
	\item \textbf{Attori principali:} Utente
	\item \textbf{Precondizioni:} 
	\begin{itemize}
		\item Il sistema è attivo e funzionante
		\item L'utente è autenticato nel sistema
        \item L'utente sta visualizzando la dashboard iniziale
        \item Esiste almeno un workspace di cui l'utente fa parte
	\end{itemize}
    \item \textbf{Postcondizioni:} Il sistema mostra un singolo elemento della lista dei workspace rappresentante un workspace specifico
	\item \textbf{Scenario principale:} Il sistema mostra un elemento della lista contenente:
    \begin{itemize}
        \item Username dell'owner del workspace (\hyperref[UC8.1.1]{UC8.1.1})
        \item Nome del workspace (\hyperref[UC8.1.2]{UC8.1.2})
    \end{itemize}
    \item \textbf{Inclusioni:} \hyperref[UC8.1.1]{UC8.1.1}, \hyperref[UC8.1.2]{UC8.1.2}
\end{itemize}

\subparagraph{UC8.1.1 - Visualizzazione username dell'owner del workspace}\label{UC8.1.1}
\begin{itemize}
	\item \textbf{Attori principali:} Utente
	\item \textbf{Precondizioni:} 
	\begin{itemize}
		\item Il sistema è attivo e funzionante
		\item L'utente è autenticato nel sistema
        \item L'utente sta visualizzando la dashboard iniziale
	\end{itemize}
    \item \textbf{Postcondizioni:} Il sistema mostra l'username dell'owner di un workspace di cui l'utente fa parte
	\item \textbf{Scenario principale:}
    \begin{itemize}
        \item Il sistema recupera l'username dell'owner del singolo workspace selezionato dalla lista della dashboard
        \item Il sistema visualizza l'username dell'owner nell'elemento della lista
    \end{itemize}
\end{itemize}

\subparagraph{UC8.1.2 - Visualizzazione nome del workspace}\label{UC8.1.2}
\begin{itemize}
	\item \textbf{Attori principali:} Utente
	\item \textbf{Precondizioni:} 
	\begin{itemize}
		\item Il sistema è attivo e funzionante
		\item L'utente è autenticato nel sistema
        \item L'utente sta visualizzando la dashboard iniziale
	\end{itemize}
    \item \textbf{Postcondizioni:} Il sistema mostra il nome di un workspace di cui l'utente fa parte
	\item \textbf{Scenario principale:}
    \begin{itemize}
        \item Il sistema recupera il nome del singolo workspace selezionato dalla lista della dashboard
        \item Il sistema visualizza il nome nell'elemento della lista
    \end{itemize}
\end{itemize}

\subsubsection{UC9 - Ricerca workspace}\label{UC9}

\begin{figure}[h]
	\centering
	\includegraphics[width=0.8\textwidth]{../Assets/AdR/UC9.png}
	\caption{UC9 e UC9.1}
	\label{fig:UC9}
\end{figure}

\begin{itemize}
	\item \textbf{Attori principali:} Utente
	\item \textbf{Precondizioni:} 
	\begin{itemize}
		\item Il sistema è attivo e funzionante
		\item L'utente è autenticato nel sistema
        \item L'utente sta visualizzando la lista dei workspace di cui fa parte nella dashboard
	\end{itemize}
    \item \textbf{Postcondizioni:} Il sistema mostra i workspace che corrispondono ai criteri di ricerca inseriti
	\item \textbf{Scenario principale:}
    \begin{enumerate}
        \item L'utente dalla dashboard principale accede alla funzione di ricerca workspace
        \item L'utente inserisce il nome (o parte del nome) del workspace che vuole cercare (\hyperref[UC9.1]{UC9.1})
        \item Il sistema filtra in tempo reale i workspace mostrati in base al testo inserito
    \end{enumerate}
    \item \textbf{Scenari alternativi:}
    \begin{itemize}
		\item Nessun workspace corrisponde alla ricerca, il sistema mostra un messaggio: "Nessun workspace trovato con questo nome"
	\end{itemize}
\end{itemize}

\paragraph{UC9.1 - Inserimento nome del workspace da cercare}\label{UC9.1}
\begin{itemize}
	\item \textbf{Attori principali:} Utente
	\item \textbf{Precondizioni:} 
	\begin{itemize}
		\item Il sistema è attivo e funzionante
		\item L'utente è autenticato nel sistema
        \item L'utente sta eseguendo un'operazione che richiede l'inserimento del nome di un workspace
	\end{itemize}
    \item \textbf{Postcondizioni:} Il sistema riceve il nome (o parte del nome) del workspace
	\item \textbf{Scenario principale:}
    \begin{enumerate}
        \item Il sistema presenta un campo di input per l'inserimento del nome del workspace
        \item L'utente inserisce il nome (o parte del nome) del workspace 
        \item Il sistema memorizza il testo inserito
    \end{enumerate}
\end{itemize}

\subsubsection{UC10 - Creazione workspace}\label{UC10}

\begin{figure}[h]
	\centering
	\includegraphics[width=0.8\textwidth]{../Assets/AdR/UC10ext.png}
	\caption{UC10 - Creazione workspace}
	\label{fig:UC10ext}
\end{figure}

\begin{itemize}
	\item \textbf{Attori principali:} Utente
	\item \textbf{Precondizioni:} 
	\begin{itemize}
		\item Il sistema è attivo e funzionante
		\item L'utente è autenticato nel sistema
        \item L'utente si trova nella dashboard iniziale
	\end{itemize}
    \item \textbf{Postcondizioni:} Un nuovo workspace viene creato con successo
	\item \textbf{Scenario principale:}
    \begin{enumerate}
        \item L'utente seleziona l'opzione per creare un nuovo workspace
        \item Il sistema presenta un form per l'inserimento dei dati del workspace
        \item L'utente inserisce il nome del workspace (\hyperref[UC10.1]{UC10.1})
        \item L'utente conferma la creazione
        \item Il sistema assegna automaticamente il ruolo di owner all'utente creatore (\hyperref[UC10.2]{UC10.2})
    \end{enumerate}
    \item \textbf{Inclusioni:} \hyperref[UC10.1]{UC10.1}, \hyperref[UC10.2]{UC10.2}
    \item \textbf{Estensioni:} \hyperref[UC10.3]{UC10.3}
\end{itemize}

Il caso d'Uso UC10 include ulteriori casi d'uso come rappresentato nella seguente immagine:
\begin{figure}[h]
	\centering
	\includegraphics[width=0.8\textwidth]{../Assets/AdR/UC10.png}
	\caption{Inclusioni di UC10: UC10.1, UC10.2}
	\label{fig:UC10}
\end{figure}

\paragraph{UC10.1 - Inserimento nome del workspace}\label{UC10.1}
\begin{itemize}
	\item \textbf{Attori principali:} Utente
	\item \textbf{Precondizioni:} 
	\begin{itemize}
		\item Il sistema è attivo e funzionante
		\item L'utente è autenticato nel sistema
        \item L'utente si trova nella dashboard iniziale
        \item L'utente sta creando un nuovo workspace 
	\end{itemize}
    \item \textbf{Postcondizioni:} Il workspace ha il nome assegnatogli dall'utente creatore
	\item \textbf{Scenario principale:}
    \begin{enumerate}
        \item L'utente inserisce il nome del workspace
        \item Il sistema valida il nome inserito 
    \end{enumerate}
\end{itemize}

\paragraph{UC10.2 - Assegnazione ruolo di owner al creatore del workspace}\label{UC10.2}
\begin{itemize}
	\item \textbf{Attori principali:} Utente
	\item \textbf{Precondizioni:} 
	\begin{itemize}
		\item Il sistema è attivo e funzionante
		\item L'utente è autenticato nel sistema
        \item L'utente si trova nella dashboard iniziale
        \item L'utente sta creando un nuovo workspace 
	\end{itemize}
    \item \textbf{Postcondizioni:} Il sistema assegna automaticamente il ruolo di owner del workspace all'utente che lo sta creando
	\item \textbf{Scenario principale:}
    \begin{enumerate}
        \item Il sistema crea automaticamente il ruolo di owner per il nuovo workspace
        \item Il sistema associa l'utente creatore al workspace con il ruolo di owner
        \item Il sistema registra l'utente come proprietario ufficiale del workspace
        \item Il sistema riconosce l'utente come Project Manager
    \end{enumerate}
	
\end{itemize}

\paragraph{UC10.3 - Creazione del workspace non riuscita}\label{UC10.3}
\begin{itemize}
	\item \textbf{Attori principali:} Utente
	\item \textbf{Precondizioni:} 
	\begin{itemize}
		\item Il sistema è attivo e funzionante
		\item L'utente è autenticato nel sistema
        \item L'utente si trova nella dashboard iniziale
        \item L'utente ha tentato la creazione di un nuovo workspace
	\end{itemize}
    \item \textbf{Postcondizioni:} 
    \begin{itemize}
        \item La creazione del workspace viene annullata
        \item Nessun nuovo workspace viene aggiunto al sistema
    \end{itemize}
	\item \textbf{Scenario principale:}
    \begin{enumerate}
        \item Il sistema non permette di terminare il processo di creazione di un workspace poichè esiste già un workspace con lo stesso nome creato dal medesimo utente o il nome inserito non è valido
        \item Il sistema mantiene aperto il form di creazione permettendo all'utente di correggere l'errore o annullare l'operazione
    \end{enumerate}
\end{itemize}

\subsubsection{UC11 - Visualizzazione lista dei ruoli di un workspace}\label{UC11}

\begin{figure}[h]
	\centering
	\includegraphics[width=0.8\textwidth]{../Assets/AdR/UC11.png}
	\caption{UC11 e UC11.1}
	\label{fig:UC11}
\end{figure}

\begin{itemize}
	\item \textbf{Attori principali:} Utente
	\item \textbf{Precondizioni:} 
	\begin{itemize}
		\item Il sistema è attivo e funzionante
		\item L'utente è autenticato nel sistema
        \item L'utente desidera vedere la lista di tutti i ruoli presenti in un workspace
	\end{itemize}
    \item \textbf{Postcondizioni:} L'utente visualizza una lista con l'elenco di tutti i ruoli di un workspace
	\item \textbf{Scenario principale:}
	\begin{enumerate}
        \item L'utente apre la lista con l'elenco di tutti i ruoli presenti al momento all'interno del workspace
        \item L'utente seleziona un singolo ruolo dalla lista per visualizzarne la descrizione (\hyperref[UC11.1]{UC11.1})
	\end{enumerate}
    \item \textbf{Inclusioni:} \hyperref[UC11.1]{UC11.1}
\end{itemize}

\paragraph{UC11.1 - Visualizzazione singolo elemento lista ruoli di un workspace}\label{UC11.1}
\begin{itemize}
	\item \textbf{Attori principali:} Utente
	\item \textbf{Precondizioni:} 
	\begin{itemize}
		\item Il sistema è attivo e funzionante
		\item L'utente è autenticato nel sistema
        \item L'utente ha selezionato un singolo ruolo dalla lista di tutti i ruoli presenti in un workspace
	\end{itemize}
    \item \textbf{Postcondizioni:} L'utente visualizza singolo ruolo di un workspace
	\item \textbf{Scenario principale:} L'utente visualizza nome e relativa descrizione di un singolo ruolo presente all'interno del workspace	
\end{itemize}

\subsubsection{UC12 - Visualizzazione lista utenti di un workspace}\label{UC12}

\begin{figure}[h]
	\centering
	\includegraphics[width=0.8\textwidth]{../Assets/AdR/UC12.png}
	\caption{UC12 e UC12.1}
	\label{fig:UC12}
\end{figure}

\begin{itemize}
	\item \textbf{Attori principali:} Utente
	\item \textbf{Precondizioni:} 
	\begin{itemize}
		\item Il sistema è attivo e funzionante
		\item L'utente è autenticato nel sistema
        \item L'utente vuole vedere la lista di tutti gli utenti presenti in un workspace
	\end{itemize}
    \item \textbf{Postcondizioni:} L'utente visualizza la lista di utenti di un singolo workspace
	\item \textbf{Scenario principale:} L'utente visualizza l'username di ogni singolo utente presente all'interno del workspace
	\item \textbf{Inclusioni:} \hyperref[UC12.1]{UC12.1}
\end{itemize}

\paragraph{UC12.1 - Visualizzazione singolo elemento lista utenti di un workspace}\label{UC12.1}
\begin{itemize}
	\item \textbf{Attori principali:} Utente
	\item \textbf{Precondizioni:} 
	\begin{itemize}
		\item Il sistema è attivo e funzionante
		\item L'utente è autenticato nel sistema
        \item L'utente ha selezionato un singolo utente dalla lista di tutti gli utenti presenti in un workspace
	\end{itemize}
    \item \textbf{Postcondizioni:} L'utente visualizza singolo utente di un workspace
	\item \textbf{Scenario principale:} L'utente visualizza username e relativo ruolo di un singolo utente presente all'interno del workspace	
\end{itemize}

\subsubsection{UC13 - Rimozione di un utente da un workspace}\label{UC13}

\begin{figure}[h]
	\centering
	\includegraphics[width=0.7\textwidth]{../Assets/AdR/UC13.png}
	\caption{UC13 - Rimozione di un utente da un workspace}
	\label{fig:UC13}
\end{figure}

\begin{itemize}
	\item \textbf{Attori principali:} Project Manager
	\item \textbf{Precondizioni:} 
	\begin{itemize}
		\item Il sistema è attivo e funzionante
		\item L'utente è autenticato nel sistema
        \item L'utente sta visualizzando la lista di tutti gli utenti presenti in un workspace
        \item L'utente ha selezionato un utente specifico dalla lista
        \item Esiste almeno un altro utente nel workspace oltre all'owner
        \item Esiste un altro utente con il ruolo di Project Manager
        \item L'utente da rimuovere non è l'owner del workspace
	\end{itemize}
    \item \textbf{Postcondizioni:} 
    \begin{itemize}
        \item L'utente selezionato viene rimosso dal workspace
        \item L'utente rimosso perde l'accesso al workspace e a tutte le sue risorse
        \item Tutti i ruoli assegnati all'utente nel workspace vengono revocati
    \end{itemize}
	\item \textbf{Scenario principale:}
    \begin{enumerate}
        \item L'utente seleziona un utente specifico dalla lista 
        \item L'utente seleziona l'opzione "Rimuovi utente dal workspace"
        \item L'utente conferma l'operazione di rimozione
        \item Il sistema rimuove l'utente dal workspace e aggiorna la lista degli utenti del workspace
    \end{enumerate}
\end{itemize}

% ====== UC14 - VERIFICA AGGIORNAMENTO SCANSIONE REPOSITORY (BACKEND) ======

\subsubsection{UC14 - Verifica aggiornamento scansione repository} \label{UC14}

\begin{figure}[H]
	\centering
	\includegraphics[width=1.0\textwidth]{../Assets/AdR/UC14.png}
	\caption{UC14 - Verifica aggiornamento scansione repository}
	\label{fig:UC14}
\end{figure}

\begin{itemize}
	\item \textbf{Attori principali:} Sistema
	\item \textbf{Attori secondari:} GitHub
	\item \textbf{Precondizioni:}
	\begin{itemize}
		\item Il sistema è attivo e funzionante
		\item È stata richiesta la visualizzazione dei dettagli di una repository specifica di un workspace
		\item La repository è associata a un link GitHub valido
	\end{itemize}
	\item \textbf{Postcondizioni:}
	\begin{itemize}
		\item Lo stato di aggiornamento della repository è stato determinato
		\item Se l'ultima scansione è precedente all'ultimo push, la repository è stata marcata come "non aggiornata" nel sistema
	\end{itemize}
	\item \textbf{Scenario principale:}
	\begin{enumerate}
		\item Il sistema esegue la verifica dello stato di aggiornamento della repository
		\item Il sistema recupera la data dell'ultimo push dalla repository GitHub (\hyperref[UC14.1]{UC14.1 - Recupero data ultimo push da GitHub})
		\item Il sistema recupera la data dell'ultima scansione completata sulla repository dal database interno
		\item Vengono confrontate la data dell'ultimo push con la data dell'ultima scansione
		\item La data dell'ultimo push è successiva alla data dell'ultima scansione
		\item Il sistema aggiorna lo stato della repository a "non aggiornata" (\hyperref[UC14.2]{UC14.2 - Aggiornamento stato repository})
	\end{enumerate}
	\item \textbf{Inclusioni:} \hyperref[UC14.1]{UC14.1}, \hyperref[UC14.2]{UC14.2}
\end{itemize}


% ------ UC14.1 ------
\paragraph{UC14.1 - Recupero data ultimo push da GitHub} \label{UC14.1}

\begin{figure}[H]
	\centering
	\includegraphics[width=1.0\textwidth]{../Assets/AdR/UC14-1ext.png}
	\caption{UC14.1 - Recupero data ultimo push da GitHub}
	\label{fig:UC14.1}
\end{figure}

\begin{itemize}
	\item \textbf{Attori principali:} Sistema
	\item \textbf{Attori secondari:} GitHub 
	\item \textbf{Precondizioni:}
	\begin{itemize}
		\item Il sistema è attivo e funzionante
		\item La repository è presente nel workspace e il relativo link GitHub è valido
		\item Se la repository è privata, è disponibile un PAT valido con i permessi necessari
	\end{itemize}
	\item \textbf{Postcondizioni:}
	\begin{itemize}
		\item Il sistema ha acquisito la data e l'ora dell'ultimo push effettuato sulla repository GitHub
	\end{itemize}
	\item \textbf{Scenario principale:}
	\begin{enumerate}
		\item Il sistema invia una richiesta alle GitHub per ottenere le informazioni della repository
		\item GitHub restituisce i metadati della repository, incluso il campo \texttt{pushed\_at}
		\item Il sistema estrae e memorizza la data dell'ultimo push
	\end{enumerate}

	\item \textbf{Scenario alternativo:}
	\begin{itemize}
		\item \textbf{Errore connessione GitHub:} (\hyperref[UC14.3]{UC14.3})
		\begin{itemize}
			\item \textbf{Condizione:} Il sistema non riesce a comunicare con le GitHub per recuperare la data dell'ultimo push
			\item \textbf{Flusso:}
			\begin{enumerate}
				\item Il sistema rileva l'errore di connessione
				\item Il sistema imposta lo stato della repository come "verifica non disponibile"
			\end{enumerate}
		\end{itemize}
		\item \textbf{PAT o URL non esistente: } (\hyperref[UC18]{UC18})
		\begin{itemize}
			\item \textbf{Condizione:} L’URL della repository fornito non è esistente oppure il PAT non è esistente o è scaduto
			\item \textbf{Flusso: }
			\begin{enumerate}
				\item GitHub rileva un errore dovuto all'url o PAT non corretti
				\item Il sistema imposta lo stato della repository come "verifica non disponibile"
			\end{enumerate}
		\end{itemize}
	\end{itemize}
	\item \textbf{Estensioni:} \hyperref[UC14.3]{UC14.3} \hyperref[UC18]{UC18}

\end{itemize}


% ------ UC14.2 ------
\paragraph{UC14.2 - Aggiornamento stato repository} \label{UC14.2}


\begin{itemize}
	\item \textbf{Attori principali:} Sistema
	\item \textbf{Precondizioni:}
	\begin{itemize}
		\item Il sistema è attivo e funzionante
		\item Il sistema ha determinato che la data dell'ultimo push è successiva alla data dell'ultima scansione completata
	\end{itemize}
	\item \textbf{Postcondizioni:}
	\begin{itemize}
		\item Lo stato della repository è stato aggiornato a "non aggiornata" nel database
	\end{itemize}
	\item \textbf{Scenario principale:}
	\begin{enumerate}
		\item Il sistema aggiorna lo stato della repository a "non aggiornata" nel database
	\end{enumerate}
\end{itemize}


% ------ UC14.3 ------
\paragraph{UC14.3 - Errore connessione GitHub} \label{UC14.3}

\begin{itemize}
	\item \textbf{Attori principali:} Sistema
	\item \textbf{Precondizioni:}
	\begin{itemize}
		\item Il sistema è attivo e funzionante
		\item Il sistema sta eseguendo \hyperref[UC14.1]{UC14.1}
		\item Si è verificato un errore di comunicazione con GitHub, non riuscendo a raggiungerlo
	\end{itemize}
	\item \textbf{Postcondizioni:}
	\begin{itemize}
		\item Lo stato della repository è stato impostato come "verifica non disponibile"
	\end{itemize}
	\item \textbf{Scenario principale:}
	\begin{enumerate}
		\item Il sistema non riesce a comunicare con GitHub
		\item Il sistema imposta lo stato di verifica della repository come "verifica non disponibile"
	\end{enumerate}
\end{itemize}

% ============================================
% UC15 - AVVIA SCANSIONE BACKEND
% ============================================

\subsubsection{UC15 - Avvia scansione}\label{UC15}

\begin{figure}[H]
	\centering
	\includegraphics[width=0.9\textwidth]{../Assets/AdR/UC15.png}
	\caption{UC15 - Avvia scansione}
	\label{fig:UC37}
\end{figure}

\begin{itemize}
	\item \textbf{Attori principali:} Utente
	\item \textbf{Precondizioni:}
	\begin{itemize}
		\item Il sistema è attivo e funzionante
		\item L'utente è stato riconosciuto dal sistema come Utente
		\item L'utente ha selezionato un workspace
		\item Esiste almeno una repository nel workspace
	\end{itemize}
	\item \textbf{Postcondizioni:} 
	\begin{itemize}
		\item Una nuova scansione viene avviata su una o più repository, su un branch valido, selezionate dall'utente
		\item Le repository sottoposte a scansione vengono aggiornate con lo stato "Scansione in corso" fino al completamento della scansione
		\item Al completamento della scansione, le repository vengono aggiornate con i risultati dell'analisi e lo stato "Scansione completata"
	\end{itemize}
	\item \textbf{Scenario principale:}
	\begin{enumerate}
		\item L'utente seleziona l'opzione "Avvia scansione"
		\item Il sistema richiede all'utente di selezionare una o più repository su cui avviare la scansione
		\item L'utente seleziona le repository su cui avviare la scansione
		\item Il sistema verifica che le repository selezionate abbiano un branch valido (es. develop)
		\item Il sistema avvia l'esecuzione dell'analisi sulle repository selezionate (\hyperref[UC17]{UC17 - Esegui analisi})
		\item Il sistema aggiorna lo stato delle repository sottoposte a scansione a "Scansione in corso"
		\item Al completamento della scansione, il sistema aggiorna le repository con i risultati dell'analisi e lo stato "Scansione completata"	
	\end{enumerate}
	\item \textbf{Scenario alternativo:}
	\begin{itemize}
		\item \textbf{Annullamento scansione:} \hyperref[UC16]{UC16}
			\begin{itemize}
				\item \textbf{Condizione:} l'utente decide di annullare l'operazione di avvio scansione prima del suo completamento
				\item \textbf{Flusso:} 
				\begin{enumerate}
					\item L'utente seleziona l'opzione "Annulla scansione" durante la procedura di avvio scansione
					\item Il sistema interrompe la procedura di avvio scansione
					\item Il sistema non avvia alcuna nuova scansione e se sono già state scansionate alcune repository, i dati di queste repository vengono riprestinati allo stato precedente all'avvio della scansione
				\end{enumerate}
			\end{itemize}
		\item \textbf{Scansione non riuscita:} 
			\begin{itemize}
				\item \textbf{Condizione:} si verifica un errore tecnico durante l'avvio della scansione o durante la scansione stessa
				\item \textbf{Flusso:} 
				\begin{enumerate}
					\item Il sistema rileva un errore durante l'avvio o l'esecuzione della scansione
					\item Il sistema interrompe la scansione in corso
					\item Il sistema mantiene o ripristina lo stato precedente delle repository
					\item Il sistema mostra un messaggio informativo: "Errore durante la scansione. Riprova più tardi." 
				\end{enumerate}
			\end{itemize}
		\item \textbf{Scansione in corso:} 
			\begin{itemize}
				\item \textbf{Condizione:} è attualmente in corso una scansione su una o più repository selezionate per la scansione
				\item \textbf{Flusso:} 
				\begin{enumerate}
					\item Il sistema recupera i nomi delle repository attualmente sottoposte a scansione
					\item Il sistema mostra un messaggio informativo: "Scansione in corso nelle seguenti repository: [nomi delle repository], prova più tardi." 
					\item Il sistema avvia la scansione solo sulle repository selezionate che non sono attualmente sottoposte a scansione
				\end{enumerate}
			\end{itemize}
		\item \textbf{Branch non valido}: 
			\begin{itemize}
				\item \textbf{Condizione:} una o più repository selezionate per la scansione non hanno un branch valido (es. develop)
				\item \textbf{Flusso:} 
				\begin{enumerate}
					\item Il sistema identifica le repository senza un branch valido
					\item Il sistema mostra un messaggio informativo: "Le seguenti repository non hanno un branch valido e non saranno sottoposte a scansione: [nomi delle repository]. Assicurati che il branch develop esista e riprova." 
					\item Il sistema avvia la scansione solo sulle repository selezionate che hanno un branch valido
				\end{enumerate}
			\end{itemize}
	\end{itemize}
	\item \textbf{Inclusioni}: \hyperref[UC17]{UC17}
	\item \textbf{Estensioni}: \hyperref[UC16]{UC16}
	\item \textbf{UC che ereditano}:
	\begin{itemize}
		\item \hyperref[UC15.1]{UC15.1 - Scansione workspace}
		\item \hyperref[UC15.2]{UC15.2 - Scansione singola repository}
		\item \hyperref[UC15.3]{UC15.3 - Scansione insieme di repository}
	\end{itemize}

    Il caso d'uso è generalizzato nei casi d'uso riportati nel diagramma seguente:
	\begin{figure}[H]
	\centering
	\includegraphics[width=0.9\textwidth]{../Assets/AdR/UC15ge.png}
	\caption{UC15 - Avvia scansione} 
	\label{fig:Generalizzazzione UC15: }
	\end{figure}
    
\end{itemize}

\paragraph{UC15.1 - Scansione workspace} \label{UC15.1}
\begin{itemize}
	\item \textbf{Attori principali:} Utente
	\item \textbf{Precondizioni:}
	\begin{itemize}
		\item Il sistema è attivo e funzionante
		\item L'utente è stato riconosciuto dal sistema come Utente
		\item L'utente ha selezionato un workspace
		\item Esiste almeno una repository nel workspace
		\item L'utente ha selezionato l'opzione di avvio scansione su tutto il workspace
	\end{itemize}
	\item \textbf{Postcondizioni:} 
	\begin{itemize}
		\item Una scansione viene avviata su tutte le repository del workspace
		\item Il branch develop è utilizzato per tutte le repository
	\end{itemize}
	\item \textbf{Scenario principale:}
	\begin{enumerate}
		\item L'utente seleziona l'opzione "Scansione workspace"
		\item Il sistema seleziona automaticamente tutte le repository appartenenti al workspace
		\item Il sistema seleziona automaticamente il branch develop per tutte le repository
		\item Il sistema esegue l'analisi (\hyperref[UC17]{UC17 - Esegui analisi})
	\end{enumerate}
	\item \textbf{Inclusioni:} \hyperref[UC17]{UC17}
	\item \textbf{Eredita da}: \hyperref[UC15]{UC15 - Avvia scansione}
\end{itemize}

\paragraph{UC15.2 - Scansione singola repository} \label{UC15.2}

\begin{figure}[H]
	\centering
	\includegraphics[width=0.7\textwidth]{../Assets/AdR/UC15_2.png}
	\caption{UC15.2 - Scansione singola repository}
	\label{fig:UC15-2.2}
\end{figure}

\begin{itemize}
	\item \textbf{Attori principali:} Utente
	\item \textbf{Precondizioni:}
	\begin{itemize}
		\item Il sistema è attivo e funzionante
		\item L'utente è stato riconosciuto dal sistema come Utente
		\item L'utente ha selezionato un workspace
		\item Esiste almeno una repository nel workspace
		\item L'utente ha selezionato una singola repository su cui avviare la scansione
	\end{itemize}
	\item \textbf{Postcondizioni:} 
	\begin{itemize}
		\item Una scansione viene avviata sulla repository selezionata
		\item Il branch selezionato dall'utente viene analizzato
	\end{itemize}
	\item \textbf{Scenario principale:}
	\begin{enumerate}
		\item l'utente seleziona una repository del workspace
		\item L'utente seleziona l'opzione "Avvio scansione"
		\item Il sistema richiede all'utente di selezionare un branch valido su cui avviare la scansione (\hyperref[UC35]{UC35 - Selezione branch})
		\item L'utente seleziona un branch valido
		\item Il sistema esegue l'analisi (\hyperref[UC17]{UC17 - Esegui analisi})
	\end{enumerate}
	\item \textbf{Inclusioni}: \hyperref[UC35]{UC35}
	\item \textbf{Eredita da}: \hyperref[UC15]{UC15 - Avvia scansione}
\end{itemize}

\paragraph{UC15.3 - Scansione insieme di repository} \label{UC15.3}
\begin{itemize}
	\item \textbf{Attori principali:} Utente
	\item \textbf{Precondizioni:}
	\begin{itemize}
		\item Il sistema è attivo e funzionante
		\item L'utente è stato riconosciuto dal sistema come Utente
		\item L'utente ha selezionato un workspace
		\item Esiste almeno una repository nel workspace
		\item L'utente ha selezionato almeno due repository 
	\end{itemize}
	\item \textbf{Postcondizioni:} 
	\begin{itemize}
		\item Una scansione viene avviata su tutte le repository selezionate
		\item Il branch develop è utilizzato per tutte le repository
	\end{itemize}
	\item \textbf{Scenario principale:}
	\begin{enumerate}
		\item l'utente seleziona due o più repository del workspace
		\item L'utente seleziona l'opzione "Avvio scansione"
		\item Il sistema seleziona automaticamente il branch develop
		\item Il sistema esegue l'analisi (\hyperref[UC17]{UC17 - Esegui analisi}) 
	\end{enumerate}
	\item \textbf{Eredita da}: \hyperref[UC15]{UC15 - Avvia scansione}
\end{itemize}

\subsubsection{UC16 - Annullamento scansione} \label{UC16}
\begin{itemize}
	\item \textbf{Attori principali:} Utente
	\item \textbf{Precondizioni:}
	\begin{itemize}
		\item Il sistema è attivo e funzionante
		\item L'utente è stato riconosciuto dal sistema come Utente
		\item L'utente ha selezionato un workspace
		\item Esiste almeno una repository nel workspace
		\item L'utente ha avviato una scansione
	\end{itemize}
	\item \textbf{Postcondizioni:} 
	\begin{itemize}
		\item La scansione in corso viene annullata
		\item Le repository sottoposte a scansione vengono ripristinate allo stato precedente all'avvio della scansione
	\end{itemize}
	\item \textbf{Scenario principale:}
	\begin{enumerate}
		\item l'utente seleziona l'opzione "Annulla scansione"
		\item Il sistema interrompe la scansione in corso
		\item Il sistema ripristina le repository sottoposte a scansione allo stato precedente all'avvio della scansione
		\item Il sistema mostra un messaggio informativo: "Scansione annullata. Ripristino delle informazioni delle repository precedentemente scansionate."
	\end{enumerate}
\end{itemize}

\subsubsection{UC17 - Esegui analisi} \label{UC17}

\begin{figure}[H]
	\centering
	\includegraphics[width=1.0\textwidth]{../Assets/AdR/UC17.png}
	\caption{UC17 - Esegui analisi}
	\label{fig:UC17}
\end{figure}

\begin{itemize}
	\item \textbf{Attori principali:} Sistema
	\item \textbf{Attori secondari:} Orchestratore, Agenti, GitHub, Database
	\item \textbf{Precondizioni:}
	\begin{itemize}
		\item Il sistema è attivo e funzionante
		\item È stata avviata una scansione (\hyperref[UC15]{UC15})
		\item Esiste almeno una repository da analizzare
		\item Per ogni repository sono noti:
		\begin{itemize}
			\item URL GitHub valido
			\item Branch da analizzare
			\item Eventuale PAT valida, in caso di repository private
		\end{itemize}
	\end{itemize}
	\item \textbf{Postcondizioni:} 
	\begin{itemize}
		\item Per ogni repository analizzata:
		\begin{itemize}
			\item è stato generato un report completo
			\item il report è stato salvato nel database
		\end{itemize}
		\item Lo stato della scansione viene aggiornato a “Scansione completata” o “Scansione fallita”
	\end{itemize}
	\item \textbf{Scenario principale:}
	\begin{enumerate}
		\item Il sistema recupera la lista delle repository da analizzare e i branch associati
		\item Il sistema clona le repository in locale (\hyperref[UC17.1]{UC17.1 - Clona repository})
		\item Il sistema passa le repository clonate all'orchestratore
		\item L'orchestratore coordina l'assegnazione delle repository agli agenti (\hyperref[UC17.2]{UC17.2 - Coordina analisi repository})
		\item Gli agenti eseguono le analisi parziali sulle repository assegnate (\hyperref[UC17.3]{UC17.3 - Esegui analisi parziale})
		\item L'orchestratore raccoglie i report parziali e genera il report completo (\hyperref[UC17.4]{UC17.4 - Aggrega report e salva risultati})
	\end{enumerate}
	\item \textbf{Scenario alternativo:}
	\begin{itemize}
		\item \textbf{Errore durante la clonazione:} 
			\begin{itemize}
				\item \textbf{Condizione:} avviene un errore durante la clonazione di una repository
				\item \textbf{Flusso:} 
				\begin{enumerate}
					\item Il sistema esclude la repository dall'analisi
					\item Il sistema segnala all'utente il fallimento nell'analisi della repository
				\end{enumerate}
			\end{itemize}
		\item \textbf{Errore durante l'analisi di un agente:} 
			\begin{itemize}
				\item \textbf{Condizione:} si verifica un errore durante la creazione del report parziale da parte di un agente per una repository
				\item \textbf{Flusso:} 
				\begin{enumerate}
					\item Il sistema rileva un errore durante la creazione del report da parte di un agente su una repository
					\item La repository non analizzabile viene marcata come “Analisi fallita”
					\item Il sistema continua l'analisi per le repository valide
					\item Il sistema conclude la scansione con stato “Completata con errori”
				\end{enumerate}
			\end{itemize}
	\end{itemize}
	\item \textbf{Inclusioni:} \hyperref[UC17.1]{UC17.1}, \hyperref[UC17.2]{UC17.2}, \hyperref[UC17.3]{UC17.3}, \hyperref[UC17.4]{UC17.4}
\end{itemize}

\paragraph{UC17.1 - Clona repository} \label{UC17.1}
\begin{itemize}
	\item \textbf{Attori principali:} Sistema
	\item \textbf{Attori secondari:} GitHub
	\item \textbf{Precondizioni:}
	\begin{itemize}
		\item Il sistema è attivo e funzionante
		\item È stata avviata una scansione (\hyperref[UC15]{UC15})
		\item Esiste almeno una repository da analizzare
		\item Sono disponibili:
		\begin{itemize}
			\item URL GitHub valido
			\item Branch da analizzare
			\item Eventuale PAT valida, in caso di repository private
		\end{itemize}
	\end{itemize}
	\item \textbf{Postcondizioni:} 
	\begin{itemize}
		\item La repository github viene clonata in locale sul server
		\item Viene restituito il path della directory locale della repository
	\end{itemize}
	\item \textbf{Scenario principale:}
	\begin{enumerate}
		\item Il sistema recupera l'URL della repository
		\item Il sistema recupera il branch da analizzare
		\item Se la repository è privata, il sistema recupera e utilizza una PAT valida
		\item Il sistema clona il branch selezionato della repository su server locale
		\item Il sistema salva il path della repository clonata
	\end{enumerate}
	\item \textbf{Scenario alternativo:}
	\begin{itemize}
		\item \textbf{URL o PAT non esistente:} 
			\begin{itemize}
				\item \textbf{Condizione:} l'URL GitHub o la PAT non esistente
				\item \textbf{Flusso:} 
				\begin{enumerate}
					\item Il sistema tenta la connessione alla repository
					\item La connessione fallisce
					\item Il sistema comunica all'utente l'errore
					\item La repository viene marcata come “Clonazione fallita” e viene esclusa dall'analisi
				\end{enumerate}
			\end{itemize}
		\item \textbf{Spazio insufficiente sul server:} 
			\begin{itemize}
				\item \textbf{Condizione:} errore durante la clonazione dovuto all'insufficienza di spazio nel server
				\item \textbf{Flusso:} 
				\begin{enumerate}
					\item Il sistema rileva che lo spazio disponibile è insufficiente
					\item Il sistema interrompe la clonazione
					\item L'errore viene registrato e comunicato all'utente
					\item La repository viene marcata come “Clonazione fallita” e viene esclusa dall'analisi
				\end{enumerate}
			\end{itemize}
	\end{itemize}
\end{itemize}

\paragraph{UC17.2 - Coordina analisi repository} \label{UC17.2}
\begin{itemize}
	\item \textbf{Attori principali:} Orchestratore
	\item \textbf{Attori secondari:} Agenti
	\item \textbf{Precondizioni:}
	\begin{itemize}
		\item Il sistema è attivo e funzionante
		\item Sono disponibili i path delle repository clonate
		\item Gli agenti sono registrati (noti al sistema) e disponibili (attivi e non occupati in altre analisi)
	\end{itemize}
	\item \textbf{Postcondizioni:} 
	\begin{itemize}
		\item Ogni repository viene assegnata a tutti gli agenti
		\item Nessun agente analizza più volte la stessa repository nella stessa scansione
	\end{itemize}
	\item \textbf{Scenario principale:}
	\begin{enumerate}
		\item L'orchestratore riceve la lista delle repository clonate (path alla directory della repository locale)
		\item L'orchestratore verifica la disponibilità degli agenti (controlla quali agenti sono liberi e quali no)
		\item L'orchestratore, per ogni repository, assegna la repository a ciascun agente disponibile
		\item L'orchestratore monitora l'avanzamento delle analisi
		\item Quando tutti gli agenti hanno analizzato una repository, l'orchestratore la rimuove dalla lista di scansione
	\end{enumerate}
	\item \textbf{Scenario alternativo:}
	\begin{itemize}
		\item \textbf{Nessun agente disponibile:} 
			\begin{itemize}
				\item \textbf{Condizione:} tutti gli agenti sono occupati o non raggiungibili
				\item \textbf{Flusso:} 
				\begin{enumerate}
					\item L'orchestratore verifica la disponibilità degli agenti
					\item Nessun agente risulta disponibile
					\item L'orchestratore mette le repository in coda
					\item L'analisi riprende quando un agente diventa disponibile
				\end{enumerate}
			\end{itemize}
	\end{itemize}
\end{itemize}

\paragraph{UC17.3 - Esegui analisi parziale} \label{UC17.3}
\begin{itemize}
	\item \textbf{Attori principali:} Agente
	\item \textbf{Attori secondari:} Orchestratore 
	\item \textbf{Precondizioni:}
	\begin{itemize}
		\item Il sistema è attivo e funzionante
		\item L'agente ha ricevuto il path della repository locale
	\end{itemize}
	\item \textbf{Postcondizioni:} 
	\begin{itemize}
		\item Viene generato un report parziale relativo a uno specifico tipo di analisi
		\item Il report parziale viene inviato all'orchestratore
	\end{itemize}
	\item \textbf{Scenario principale:}
	\begin{enumerate}
		\item L'agente riceve il path della repository locale
		\item L'agente esegue l'analisi di competenza (codice, documentazione o OWASP)
		\item L'agente genera un report parziale
		\item L'agente invia il report all'orchestratore
	\end{enumerate}
	\item \textbf{Scenario alternativo:}
	\begin{itemize}
		\item \textbf{Errore durante l'analisi:} 
			\begin{itemize}
				\item \textbf{Condizione:} errore interno dell'agente (tool fallisce, crash, eccezione)
				\item \textbf{Flusso:} 
				\begin{enumerate}
					\item L'agente avvia l'analisi
					\item Si verifica un errore
					\item L'agente genera un report di errore
					\item Il report viene inviato all'orchestratore
				\end{enumerate}
			\end{itemize}
		\item \textbf{Nessun dato rilevante:} 
			\begin{itemize}
				\item \textbf{Condizione:} non ci sono dati utili all'analisi
				\item \textbf{Flusso:} 
				\begin{enumerate}
					\item L'agente avvia l'analisi
					\item Non vengono trovati dati utili all'analisi
					\item L'agente invia un report vuoto o informativo
					\item L'orchestratore considera l'analisi parziale completata
				\end{enumerate}
			\end{itemize}
	\end{itemize}
\end{itemize}

\paragraph{UC17.4 - Aggrega report e salva risultati} \label{UC17.4}
\begin{itemize}
	\item \textbf{Attori principali:} Orchestratore
	\item \textbf{Attori secondari:}  Database
	\item \textbf{Precondizioni:}
	\begin{itemize}
		\item Il sistema è attivo e funzionante
		\item Sono disponibili tutti i report parziali per una repository
	\end{itemize}
	\item \textbf{Postcondizioni:} 
	\begin{itemize}
		\item Viene generato un report completo per il branch analizzato della repository
		\item Il report completo viene salvato nel database
	\end{itemize}
	\item \textbf{Scenario principale:}
	\begin{enumerate}
		\item L'orchestratore riceve i report parziali dagli agenti
		\item L'orchestratore associa i report alla repository corretta
		\item Quando tutti i report previsti sono disponibili: aggrega i report parziali e genera il report completo
		\item Il report completo viene salvato nel database
		\item La repository viene marcata come analizzata
	\end{enumerate}
	\item \textbf{Scenario alternativo:}
	\begin{itemize}
		\item \textbf{Report parziale non valido:} 
			\begin{itemize}
				\item \textbf{Condizione:} un report ricevuto è incompleto o segnala un errore
				\item \textbf{Flusso:} 
				\begin{enumerate}
					\item L'orchestratore valida i report ricevuti
					\item Un report risulta non valido
					\item Il sistema registra l'errore
				\end{enumerate}
			\end{itemize}
		\item \textbf{Spazio insufficiente sul server:} 
			\begin{itemize}
				\item \textbf{Condizione:} errore durante il salvataggio nel database dovuto all'insufficienza di spazio nel server
				\item \textbf{Flusso:} 
				\begin{enumerate}
					\item Il sistema tenta di salvare il report finale
					\item Il salvataggio fallisce per mancanza di spazio nel server
					\item Il sistema registra l'errore e segnala il fallimento
				\end{enumerate}
			\end{itemize}
	\end{itemize}
\end{itemize}

\subsubsection{UC18 - PAT o URL non esistente} \label{UC18}
\begin{itemize}
    \item \textbf{Attori principali:} Sistema
    \item \textbf{Attori secondari:} GitHub
    \item \textbf{Precondizioni:}
        \begin{itemize}
            \item Il sistema è attivo e funzionante
            \item Il sistema sta eseguendo il caso d'uso \hyperref[UC14.1]{UC14.1} oppure \hyperref[UC17.1]{UC17.1}
            \item L'URL della repository fornito non è esistente oppure il PAT non è esistente o è scaduto
        \end{itemize}
    \item \textbf{Postcondizioni:}
        \begin{itemize}
            \item L'operazione in corso viene interrotta
            \item Il sistema comunica all'utente il motivo del fallimento
            \item La repository viene esclusa dalle operazioni successive
        \end{itemize}
    \item \textbf{Scenario Principale:}
        \begin{enumerate}
            \item Il sistema invia una richiesta a GitHub per accedere alla repository utilizzando l'URL salvato e l'eventuale PAT associato
            \item GitHub respinge la connessione restituendo un errore 
            \item Il sistema riceve l'errore e interrompe l'operazione sulla repository specifica
            \item Il sistema notifica l'errore all'utente
        \end{enumerate}
\end{itemize}


\subsubsection{UC19 - Visualizza lista tag raccolta} \label{UC19}

\begin{figure}[H]
	\centering
	\includegraphics[width=1.0\textwidth]{../Assets/AdR/UC19.png}
	\caption{UC19 - Visualizza lista tag raccolta}
	\label{fig:UC19}
\end{figure}

\begin{itemize}
	\item \textbf{Attori principali:} Utente
	\item \textbf{Precondizioni:} 
	\begin{itemize}
		\item Il sistema è attivo e funzionante
		\item L'utente è stato riconosciuto dal sistema come Utente
		\item L'utente ha selezionato un workspace
	\end{itemize}
	\item \textbf{Postcondizioni:} Il sistema mostra a schermo l'elenco dei tag raccolta definiti nel workspace o repository corrente
	\item \textbf{Scenario principale:}
	\begin{enumerate}
		\item L'utente seleziona l'opzione per visualizzare i tag raccolta
		\item Il sistema determina il contesto di visualizzazione (workspace o repository selezionata)
		\item Il sistema recupera l'elenco dei tag raccolta associati al contesto selezionato
		\item Il sistema mostra a schermo la lista dei tag raccolta (\hyperref[UC19.1]{UC19.1 - Visualizza elemento lista tag raccolta repository})
	\end{enumerate}
	\item \textbf{Scenario alternativo:}
    \begin{itemize}
        \item \textbf{Nessun tag raccolta presente:}
        \begin{itemize}
            \item \textbf{Condizione:} Il workspace selezionato non contiene alcun tag raccolta
            \item \textbf{Flusso:}
            \begin{enumerate}
                \item Il sistema mostra un messaggio informativo all'utente indicando che non sono presenti tag raccolta nel workspace
            \end{enumerate}
        \end{itemize}
    \end{itemize}
	\item \textbf{Inclusioni:} \hyperref[UC19.1]{UC19.1}
	\item \textbf{UC che ereditano:} 
	\begin{itemize}
		\item \hyperref[UC19.1]{UC19.1 - Visualizza lista tag raccolta repository}
		\item \hyperref[UC19.2]{UC19.2 - Visualizza lista tag raccolta workspace}
	\end{itemize}
\end{itemize}

\paragraph{UC19.1 - Visualizza lista tag raccolta repository} \label{UC19.1}

\begin{figure}[H]
	\centering
	\includegraphics[width=0.9\textwidth]{../Assets/AdR/UC19_1.png}
	\caption{UC19.1 - Visualizza lista tag raccolta repository}
	\label{fig:UC19.1}
\end{figure}

\begin{itemize}
	\item \textbf{Attori principali:} Utente
	\item \textbf{Precondizioni:} 
	\begin{itemize}
		\item Il sistema è attivo e funzionante
		\item L'utente è stato riconosciuto dal sistema come Utente
		\item L'utente ha selezionato una repository appartenente al workspace
	\end{itemize}
	\item \textbf{Postcondizioni:} Il sistema mostra a schermo l'elenco dei tag raccolta associati alla repository selezionata
	\item \textbf{Scenario principale:}
	\begin{enumerate}
		\item L'utente seleziona una repository appartenente al workspace
		\item Il sistema recupera l'elenco dei tag raccolta associati alla repository selezionata
		\item Il sistema mostra la lista dei tag raccolta della repository (\hyperref[UC19.1.1]{UC19.1.1 - Visualizza elemento lista tag raccolta})
	\end{enumerate}
	\item \textbf{Inclusioni:} \hyperref[UC19.1.1]{UC19.1.1}
	\item \textbf{Eredita da:} \hyperref[UC19]{UC19 - Visualizza lista tag raccolta}
\end{itemize}

\paragraph{UC19.2 - Visualizza lista tag raccolta workspace} \label{UC19.2}

\begin{figure}[H]
	\centering
	\includegraphics[width=0.9\textwidth]{../Assets/AdR/UC19_2.png}
	\caption{UC19.2 - Visualizza lista tag raccolta workspace}
	\label{fig:UC19.2}
\end{figure}

\begin{itemize}
	\item \textbf{Attori principali:} Utente
	\item \textbf{Precondizioni:} 
	\begin{itemize}
		\item Il sistema è attivo e funzionante
		\item L'utente è stato riconosciuto dal sistema come Utente
		\item L'utente ha selezionato un workspace
	\end{itemize}
	\item \textbf{Postcondizioni:} Il sistema mostra a schermo l'elenco dei tag raccolta associati al workspace selezionato
	\item \textbf{Scenario principale:}
	\begin{enumerate}
		\item L'utente seleziona l'opzione “Gestione tag raccolta” del workspace
		\item Il sistema recupera l'elenco dei tag raccolta associati al workspace selezionato
		\item Il sistema mostra la lista dei tag raccolta del workspace (\hyperref[UC19.1.1]{UC19.1.1 - Visualizza elemento lista tag raccolta})
	\end{enumerate}
	\item \textbf{Inclusioni:} \hyperref[UC19.1.1]{UC19.1.1}
	\item \textbf{Eredita da:} \hyperref[UC19]{UC19 - Visualizza lista tag raccolta}
\end{itemize}

\subparagraph{UC19.1.1 - Visualizza elemento lista tag raccolta} \label{UC19.1.1}
\begin{itemize}
	\item \textbf{Attori principali:} Utente
	\item \textbf{Precondizioni:} 
	\begin{itemize}
		\item Il sistema è attivo e funzionante
		\item L'utente è stato riconosciuto dal sistema come Utente
		\item L'utente sta visualizzando la lista dei tag raccolta del workspace
	\end{itemize}
	\item \textbf{Postcondizioni:} Viene visualizzato, nel singolo elemento della lista dei tag raccolta, il nome del tag raccolta
	\item \textbf{Scenario principale:}
	\begin{enumerate}
		\item Il sistema recupera le informazioni relative a un singolo tag raccolta presente nella lista
		\item Il sistema mostra il nome identificativo del tag raccolta per la visualizzazione nella lista  ( \hyperref[UC19.1.1.1]{UC19.1.1.1 - Visualizza Nome tag raccolta})
	\end{enumerate}
	\item \textbf{Inclusioni:} \hyperref[UC19.1.1.1]{UC19.1.1.1}
\end{itemize}

\subparagraph{UC19.1.1.1 - Visualizza Nome tag raccolta} \label{UC19.1.1.1}
\begin{itemize}
	\item \textbf{Attori principali:} Utente
	\item \textbf{Precondizioni:} 
	\begin{itemize}
		\item Il sistema è attivo e funzionante
		\item L'utente è stato riconosciuto dal sistema come Utente
		\item L'utente sta visualizzando la lista dei tag raccolta del workspace o di un repository
	\end{itemize}
	\item \textbf{Postcondizioni:} Viene visualizzato il nome del tag raccolta
	\item \textbf{Scenario principale:}
	\begin{enumerate}
		\item Il sistema recupera il nome del tag raccolta da mostrare nella lista
		\item Il sistema mostra il nome del tag raccolta nell'elemento della lista
	\end{enumerate}
\end{itemize}

\subsubsection{UC20 - Crea tag raccolta}\label{UC20}

\begin{figure}[H]
	\centering
	\includegraphics[width=1.0\textwidth]{../Assets/AdR/UC20.png}
	\caption{UC20 - Crea tag raccolta}
	\label{fig:UC20}
\end{figure}

\begin{itemize}
	\item \textbf{Attori principali:} Utente
	\item \textbf{Precondizioni:} 
	\begin{itemize}
		\item Il sistema è attivo e funzionante
		\item L'utente è stato riconosciuto dal sistema come Utente
		\item L'utente ha selezionato un workspace
	\end{itemize}
	\item \textbf{Postcondizioni:} 
	\begin{itemize}
		\item Un nuovo tag raccolta è stato creato nel sistema
		\item Il tag raccolta è associato al workspace corrente
		\item Il nuovo tag raccolta è disponibile nell'elenco dei tag del workspace
	\end{itemize}
	\item \textbf{Scenario principale:}
	\begin{enumerate}
		\item L'utente seleziona l'opzione per creare un nuovo tag raccolta
		\item Il sistema richiede l'inserimento del nome del tag raccolta
		\item L'utente inserisce il nome del tag raccolta (\hyperref[UC20.1]{UC 20.1 - Inserisci nome tag raccolta})
		\item L'utente conferma la creazione
		\item Il sistema verifica che il nome inserito sia valido e non già esistente nel workspace
		\item Il sistema crea il nuovo tag raccolta e lo associa al workspace corrente
		\item Il sistema mostra un messaggio di conferma dell'avvenuta creazione del tag raccolta
	\end{enumerate}
	\item \textbf{Inclusioni:} \hyperref[UC20.1]{UC 20.1}
		\item \textbf{Scenario alternativo:}
    \begin{itemize}
        \item \textbf{Nome non valido o già esistente:} (\hyperref[UC21]{UC21 - Creazione tag raccolta non riuscita})
        \begin{itemize}
            \item \textbf{Condizione:} il nome inserito è vuoto, contiene caratteri non ammessi o è già presente nel workspace
            \item \textbf{Flusso:}
            \begin{enumerate}
                \item Il sistema non crea il tag raccolta
                \item Il sistema informa l'utente dell'errore e consente la modifica del nome o l'annullamento dell'operazione
            \end{enumerate}
        \end{itemize}
    \end{itemize}
\end{itemize}

\paragraph{UC 20.1 - Inserisci nome tag raccolta} \label{UC20.1}
\begin{itemize}
	\item \textbf{Attori principali:} Utente
	\item \textbf{Precondizioni:} 
	\begin{itemize}
		\item Il sistema è attivo e funzionante
		\item L'utente è stato riconosciuto dal sistema come Utente
		\item L'utente ha selezionato un workspace
	\end{itemize}
	\item \textbf{Postcondizioni:} 
	\begin{itemize}
		\item Il nome del tag raccolta è stato acquisito dal sistema ed è disponibile per la validazione e la creazione del tag
	\end{itemize}
	\item \textbf{Scenario principale:}
	\begin{enumerate}
		\item Il sistema richiede l'inserimento del nome del tag raccolta
		\item L'utente inserisce il nome del tag raccolta
		\item Il sistema acquisisce il valore inserito
	\end{enumerate}
\end{itemize}

\subsubsection{UC21 - Creazione tag raccolta non riuscita} \label{UC21}
\begin{itemize}
	\item \textbf{Attori principali:} Utente
	\item \textbf{Precondizioni:} 
	\begin{itemize}
		\item L'utente ha avviato la procedura di creazione di un nuovo tag raccolta (UC20)
		\item L'utente ha tentato di confermare la creazione del tag raccolta
		\item Si è verificata una delle seguenti condizioni:
		\begin{itemize}
			\item il nome del tag raccolta inserito è già esistente nel workspace
			\item il nome del tag raccolta inserito non è valido (vuoto, contiene solo spazi, contiene caratteri non ammessi)
		\end{itemize}
	\end{itemize}
	\item \textbf{Postcondizioni:} 
	\begin{itemize}
		\item La creazione del tag raccolta viene annullata
		\item Nessun nuovo tag raccolta viene aggiunto al workspace corrente
		\item Il sistema fornisce all'utente un messaggio di errore esplicativo
	\end{itemize}
	\item \textbf{Scenario principale:}
	\begin{enumerate}
		\item Il sistema rileva un errore di validazione sul nome del tag (nome non valido o è già esistente nel workspace)
		\item Il sistema annulla l'operazione di creazione
		\item Il sistema mostra un messaggio di errore all'utente specificando la causa del fallimento
		\item Il sistema consente all'utente di correggere il nome inserito o annullare l'operazione	
	\end{enumerate}
\end{itemize}

\subsubsection{UC22 - Eliminazione tag raccolta dal workspace}\label{UC22}

\begin{figure}[H]
	\centering
	\includegraphics[width=0.8\textwidth]{../Assets/AdR/UC22.png}
	\caption{UC22 - Eliminazione tag raccolta dal workspace}
	\label{fig:UC22}
\end{figure}

\begin{itemize}
	\item \textbf{Attori principali:} Utente
	\item \textbf{Precondizioni:} 
	\begin{itemize}
		\item Il sistema è attivo e funzionante
		\item L'utente è stato riconosciuto dal sistema come Utente
		\item L'utente ha selezionato un workspace
		\item Esiste almeno un tag raccolta nel workspace
	\end{itemize}
	\item \textbf{Postcondizioni:} 
	\begin{itemize}
		\item Il tag raccolta selezionato è stato rimosso dal workspace
		\item Il tag raccolta è stato dissociato da tutte le repository a cui era associato
		\item Viene visualizzata la lista tag raccolta aggiornata
	\end{itemize}
	\item \textbf{Scenario principale:}
	\begin{enumerate}
		\item L'utente visualizza la lista dei tag raccolta del workspace
		\item L'utente seleziona il tag raccolta da eliminare
		\item Il sistema mostra un messaggio di avvertimento indicando il numero di repository coinvolte
		\item Il sistema richiede la conferma dell'operazione
		\item L'utente conferma l'eliminazione
		\item Il sistema rimuove il tag raccolta dal workspace
		\item Il sistema dissocia il tag raccolta da tutte le repository a cui era associato
		\item Il sistema aggiorna la visualizzazione e mostra un messaggio di conferma dell'avvenuta eliminazione
	\end{enumerate}
		\item \textbf{Scenario alternativo:}
	\begin{itemize}
		\item \textbf{Annullamento dell'operazione:} 
			\begin{enumerate}
				\item \textbf{Condizione:}l'utente annulla l'operazione nella fase di conferma
				\item \textbf{Flusso:} 
				\begin{enumerate}
					\item Il sistema interrompe il processo di eliminazione del tag
					\item Il tag raccolta rimane invariato 
				\end{enumerate}
			\end{enumerate}
	\end{itemize}
\end{itemize}

\subsubsection{UC23 - Assegnazione tag raccolta a repository}\label{UC23}

\begin{figure}[H]
	\centering
	\includegraphics[width=0.7\textwidth]{../Assets/AdR/UC23.png}
	\caption{UC23 - Assegnazione tag raccolta a repository}
	\label{fig:UC23}
\end{figure}


\begin{itemize}
	\item \textbf{Attori principali:} Utente
	\item \textbf{Precondizioni:} 
	\begin{itemize}
		\item Il sistema è attivo e funzionante
		\item L'utente è stato riconosciuto dal sistema come Utente
		\item L'utente ha selezionato un workspace
		\item Esiste almeno una repository nel workspace
		\item Esiste almeno un tag raccolta nel workspace
	\end{itemize}
	\item \textbf{Postcondizioni:} 
	\begin{itemize}
		\item Il tag raccolta selezionato è associato alla repository selezionata
		\item L'associazione è persistita dal sistema
		\item La repository riflette l'insieme aggiornato dei tag raccolta associati
	\end{itemize}
	\item \textbf{Scenario principale:}
	\begin{enumerate}
		\item L'utente visualizza i dettagli di una repository o la lista delle repository
		\item L'utente seleziona l'opzione per gestire i tag raccolta della repository
		\item Il sistema recupera e mostra l'elenco dei tag raccolta disponibili nel workspace
		\item L'utente seleziona uno o più tag raccolta da associare alla repository
		\item L'utente conferma l'operazione
		\item Il sistema associa i tag selezionati alla repository
		\item Il sistema aggiorna i tag assegnati alla repository e conferma l'operazione
	\end{enumerate}
		\item \textbf{Scenario alternativo:}
	\begin{enumerate}
		\item \textbf{Tag già associato alla repository:} 
			\begin{itemize}
				\item \textbf{Condizione:}uno o più tag selezionati risultano già associati alla repository
				\item \textbf{Flusso:} 
				\begin{enumerate}
					\item Il sistema ignora l'associazione duplicata
					\item Il sistema mostra il messaggio: “Uno o più tag selezionati sono già associati alla repository”
				\end{enumerate}
			\end{itemize}
			\item \textbf{Nessun tag raccolta disponibile nel workspace:} 
			\begin{itemize}
				\item \textbf{Condizione:}il workspace non contiene alcun tag raccolta
				\item \textbf{Flusso:} 
				\begin{enumerate}
					\item Il sistema interrompe l'operazione di assegnazione
					\item Il sistema mostra il messaggio: “Nessun tag raccolta disponibile. Creare un tag prima di procedere”
				\end{enumerate}
			\end{itemize}
	\end{enumerate}
\end{itemize}

\subsubsection{UC24 -Rimozione tag raccolta da repository}\label{UC24}

\begin{figure}[H]
	\centering
	\includegraphics[width=0.7\textwidth]{../Assets/AdR/UC24.png}
	\caption{UC24 -Rimozione tag raccolta da repository}
	\label{fig:UC24}
\end{figure}

\begin{itemize}
	\item \textbf{Attori principali:} Utente
	\item \textbf{Precondizioni:} 
	\begin{itemize}
		\item Il sistema è attivo e funzionante
		\item L'utente è stato riconosciuto dal sistema come Utente
		\item L'utente ha selezionato un workspace
		\item L'utente ha selezionato una repository del workspace
		\item La repository selezionata ha almeno un tag raccolta associato
	\end{itemize}
	\item \textbf{Postcondizioni:} 
	\begin{itemize}
		\item Il tag raccolta selezionato viene dissociato dalla repository selezionata
	\end{itemize}
	\item \textbf{Scenario principale:}
	\begin{enumerate}
		\item L'utente visualizza i dettagli di una repository tra cui i tag raccolta assegnati
		\item L'utente seleziona l'opzione per gestire i tag raccolta della repository
		\item Il sistema mostra la lista dei tag raccolta attualmente assegnati alla repository
		\item L'utente seleziona uno o più tag raccolta da rimuovere dalla repository
		\item L'utente conferma la rimozione
		\item Il sistema dissocia i tag selezionati dalla repository
		\item Il sistema aggiorna la repository e mostra i tag rimanenti
	\end{enumerate}
		\item \textbf{Scenario alternativo:}
	\begin{enumerate}
		\item \textbf{Nessun tag raccolta associato:} 
			\begin{itemize}
				\item \textbf{Condizione:}la repository non ha alcun tag raccolta associato
				\item \textbf{Flusso:} 
				\begin{enumerate}
					\item Il sistema interrompe l'operazione
					\item Il sistema mostra un messaggio informativo: "Nessun tag assegnato a questa repository"
				\end{enumerate}
			\end{itemize}
			\item \textbf{Annullamento dell'operazione:} 
			\begin{itemize}
				\item \textbf{Condizione:}l'utente annulla la rimozione nella fase di conferma
				\item \textbf{Flusso:} 
				\begin{enumerate}
					\item Il sistema interrompe l'operazione
					\item Le associazioni dei tag raccolta alla repository rimangono invariate
				\end{enumerate}
			\end{itemize}
	\end{enumerate}
\end{itemize}

\subsubsection{UC25 - Visualizza lista repository del workspace}\label{UC25}

\begin{figure}[H]
	\centering
	\includegraphics[width=0.9\textwidth]{../Assets/AdR/UC25.png}
	\caption{UC25 - Visualizza lista repository del workspace}
	\label{fig:UC25}
\end{figure}

\begin{itemize}
	\item \textbf{Attori principali:} Utente
	\item \textbf{Precondizioni:} 
	\begin{itemize}
		\item Il sistema è attivo e funzionante
		\item L'utente è stato riconosciuto dal sistema come Utente
		\item L'utente ha selezionato un workspace
	\end{itemize}
	\item \textbf{Postcondizioni:} 
	\begin{itemize}
		\item Il sistema presenta all'utente l'elenco delle repository associate al workspace corrente
		\item Per ciascuna repository sono visualizzate delle informazioni sintetiche
	\end{itemize}
	\item \textbf{Scenario principale:}
	\begin{enumerate}
		\item L'utente seleziona l'opzione per visualizzare l'elenco delle repository del workspace corrente
		\item Il sistema recupera l'elenco delle repository associate al workspace 
		\item Il sistema mostra la lista delle repository
		\item Per ciascuna repository, il sistema visualizza delle informazioni sintetiche (\hyperref[UC25.1]{UC25.1})
	\end{enumerate}
	
	\item \textbf{Scenario alternativo:}
	\begin{itemize}
		\item \textbf{Nessuna repository presente nel workspace:} 
			\begin{itemize}
				\item \textbf{Condizione:}il workspace corrente non contiene repository associate
				\item \textbf{Flusso:} 
				\begin{enumerate}
					\item Il sistema mostra un messaggio informativo:“Nessuna repository presente nel workspace corrente”
				\end{enumerate}
			\end{itemize}
        \item \textbf{Annullamento dell'operazione:} 
		\begin{itemize}
			\item \textbf{Condizione:}l'utente annulla la rimozione nella fase di conferma
			\item \textbf{Flusso:} 
			\begin{enumerate}
				\item Il sistema interrompe l'operazione
				\item Le associazioni dei tag raccolta alla repository rimangono invariate
			\end{enumerate}
		\end{itemize}
		\item \textbf{Aggiorna visualizzazione (\hyperref[UC37]{UC37 - Aggiorna repository}):}
        \begin{itemize}
            \item \textbf{Condizione:} L'utente seleziona “Aggiorna”
            \item \textbf{Flusso:}
            \begin{enumerate}
                \item Il sistema controlla se ci sono scansioni più recenti
                \item Il sistema aggiorna i dati visualizzati con quelli dell'ultima scansione
            \end{enumerate}
        \end{itemize}
	\end{itemize}
	\item \textbf{Inclusioni:} \hyperref[UC25.1]{UC25.1}
	\item \textbf{Estensioni:} \hyperref[UC37]{UC37}
\end{itemize}

\paragraph{UC25.1 - Visualizza singolo elemento lista repository del workspace}\label{UC25.1}

\begin{figure}[H]
	\centering
	\includegraphics[width=0.9\textwidth]{../Assets/AdR/UC25_1.png}
	\caption{UC25.1 - Visualizza singolo elemento lista repository del workspace}
	\label{fig:UC25.1}
\end{figure}

\begin{itemize}
	\item \textbf{Attori principali:} Utente
	\item \textbf{Precondizioni:} 
	\begin{itemize}
		\item Il sistema è attivo e funzionante
		\item L'utente è stato riconosciuto dal sistema come Utente
		\item L'utente ha selezionato un workspace
		\item È stata richiesta la visualizzazione della lista delle repository (\hyperref[UC25]{UC25})
		\item La repository di riferimento esiste ed è associata al workspace corrente
	\end{itemize}
	\item \textbf{Postcondizioni:} 
	\begin{itemize}
		\item Il sistema visualizza le informazioni sintetiche relative alla repository selezionata
	\end{itemize}
	\item \textbf{Scenario principale:}
	\begin{enumerate}
		\item Il sistema identifica la repository da visualizzare nella lista
		\item Il sistema recupera le informazioni sintetiche associate alla repository
		\item Il sistema visualizza nella lista:
		\begin{itemize}
			\item Il nome della repository (\hyperref[UC25.1.1]{UC25.1.1})
			\item La data dell'ultima scansione (\hyperref[UC25.1.2]{UC25.1.2})
			\item L'indicatore di presenza criticità (\hyperref[UC25.1.3]{UC25.1.3})
			\item Il voto della documentazione (\hyperref[UC25.1.4]{UC25.1.4})
			\item La percentuale di test coverage (\hyperref[UC25.1.5]{UC25.1.5})
		\end{itemize}
	\end{enumerate}
	\item \textbf{Scenario alternativo:}
	\begin{itemize}
		\item \textbf{Dati non disponibili per la repository}
		\begin{itemize}
			\item \textbf{Condizione:} alcune informazioni sintetiche non sono disponibili
			\item \textbf{Flusso:}
			\begin{enumerate}
				\item Il sistema visualizza solo le informazioni disponibili
				\item Per i dati mancanti mostra un messaggio informativo (es. "Dato non disponibile")
			\end{enumerate}
		\end{itemize}
	\end{itemize}
	\item \textbf{Inclusioni:} \hyperref[UC25.1.1]{UC25.1.1}, \hyperref[UC25.1.2]{UC25.1.2}, \hyperref[UC25.1.3]{UC25.1.3}, \hyperref[UC25.1.4]{UC25.1.4}, \hyperref[UC25.1.5]{UC25.1.5}
\end{itemize}

\paragraph{UC25.1.1 - Visualizza nome repository}\label{UC25.1.1}
\begin{itemize}
	\item \textbf{Attori principali:} Utente
	\item \textbf{Precondizioni:} 
	\begin{itemize}
		\item Il sistema è attivo e funzionante
		\item L'utente è stato riconosciuto dal sistema come Utente
		\item L'utente ha selezionato un workspace 
		\item L'utente sta visualizzando la lista delle repository del workspace corrente
		\item La repository di riferimento esiste nel sistema
	\end{itemize}
	\item \textbf{Postcondizioni:} 
	\begin{itemize}
		\item Viene visualizzato il nome identificativo della repository
	\end{itemize}
	\item \textbf{Scenario principale:}
	\begin{enumerate}
		\item Il sistema recupera il nome identificativo della repository
		\item Il sistema mostra il nome della repository nella lista
	\end{enumerate}
\end{itemize}

\paragraph{UC25.1.2 - Visualizza data ultima scansione} \label{UC25.1.2}
\begin{itemize}
	\item \textbf{Attori principali:} Utente
	\item \textbf{Precondizioni:} 
	\begin{itemize}
		\item Il sistema è attivo e funzionante
		\item L'utente è stato riconosciuto dal sistema come Utente
		\item L'utente ha selezionato un workspace 
		\item L'utente sta visualizzando la lista delle repository del workspace corrente
		\item La repository di riferimento esiste nel sistema
	\end{itemize}
	\item \textbf{Postcondizioni:} 
	\begin{itemize}
		\item Viene visualizzata la data e l'ora dell'ultima scansione effettuata sulla repository
	\end{itemize}
	\item \textbf{Scenario principale:}
	\begin{enumerate}
		\item Il sistema recupera la data e l'ora dell'ultima analisi completata sulla repository
		\item Il sistema mostra la data e l'ora dell'ultima scansione nella lista
	\end{enumerate}
	\item \textbf{Scenario alternativo:}
	\begin{enumerate}
		\item \textbf{Nessuna scansione:} 
			\begin{itemize}
				\item \textbf{Condizione:}La repository non ha mai subito una scansione
				\item \textbf{Flusso:} 
				\begin{enumerate}
					\item Il sistema mostra il messaggio:“Nessuna scansione effettuata”
				\end{enumerate}
			\end{itemize}
	\end{enumerate}
\end{itemize}

\paragraph{UC25.1.3 - Visualizza presenza criticità} \label{UC25.1.3}
\begin{itemize}
	\item \textbf{Attori principali:} Utente
	\item \textbf{Precondizioni:} 
	\begin{itemize}
		\item Il sistema è attivo e funzionante
		\item L'utente è stato riconosciuto dal sistema come Utente
		\item L'utente ha selezionato un workspace 
		\item L'utente sta visualizzando la lista delle repository del workspace corrente
		\item La repository di riferimento esiste nel sistema
		\item La repository ha subito almeno una scansione
	\end{itemize}
	\item \textbf{Postcondizioni:} 
	\begin{itemize}
		\item Viene visualizzato un indicatore che rappresenta la presenza e la gravità massima delle criticità rilevate nella repository
	\end{itemize}
	\item \textbf{Scenario principale:}
	\begin{enumerate}
		\item Il sistema recupera l'esito dell'ultima analisi della repository
		\item Il sistema determina la gravità massima delle vulnerabilità rilevate
		\item Il sistema visualizza un indicatore sintetico basato sull'indice CVSS, che identifica i seguenti livelli di criticità: Critico, Alto, Medio, Basso, Nessuna criticità
	\end{enumerate}
		\item \textbf{Scenario alternativo:}
	\begin{itemize}
		\item \textbf{Analisi di sicurezza non disponibile:} 
			\begin{itemize}
				\item \textbf{Condizione:}l'analisi delle vulnerabilità di sicurezza non è stata eseguita o non è andata a buon fine
				\item \textbf{Flusso:} 
				\begin{enumerate}
					\item Il sistema mostra il messaggio: “Analisi sicurezza non disponibile”
				\end{enumerate}
			\end{itemize}
	\end{itemize}
\end{itemize}

\paragraph{UC25.1.4 - Visualizza voto documentazione} \label{UC25.1.4}
\begin{itemize}
	\item \textbf{Attori principali:} Utente
	\item \textbf{Precondizioni:} 
	\begin{itemize}
		\item Il sistema è attivo e funzionante
		\item L'utente è stato riconosciuto dal sistema come Utente
		\item L'utente ha selezionato un workspace 
		\item L'utente sta visualizzando la lista delle repository del workspace corrente
		\item La repository di riferimento esiste nel sistema
		\item La repository ha subito almeno una scansione
	\end{itemize}
	\item \textbf{Postcondizioni:} 
	\begin{itemize}
		\item Viene visualizzato il voto relativo alla qualità della documentazione della repository
	\end{itemize}
	\item \textbf{Scenario principale:}
	\begin{enumerate}
		\item Il sistema recupera l'esito dell'ultima analisi della repository
		\item Il sistema calcola o recupera il voto di qualità della documentazione sulla base di criteri quali: presenza del file README, presenza di documentazione tecnica, presenza e qualità dei commenti nel codice, conformità agli standard di documentazione definiti, 		Il sistema mostra un voto numerico/ indicatore qualitativo che rappresenta la completezza e la qualità della documentazione presente nella repository, calcolato in base alla presenza di file README, documentazione tecnica, commenti nel codice e conformità agli standard di documentazione definiti
		\item Il sistema visualizza un voto numerico che rappresenta la qualità complessiva della documentazione
	\end{enumerate}
	\item \textbf{Scenario alternativo:}
	\begin{itemize}
		\item \textbf{Documentazione assente:} 
			\begin{itemize}
				\item \textbf{Condizione:}La repository non contiene documentazione analizzabile
				\item \textbf{Flusso:} 
				\begin{enumerate}
					\item Il sistema mostra il messaggio: “Documentazione assente”
				\end{enumerate}
			\end{itemize}
		\item \textbf{Analisi documentazione non disponibile:} 
			\begin{itemize}
				\item \textbf{Condizione:}l'analisi della documentazione non è stata eseguita o non è andata a buon fine
				\item \textbf{Flusso:} 
				\begin{enumerate}
					\item Il sistema mostra il messaggio: “Analisi documentazione non disponibile”
				\end{enumerate}
			\end{itemize}
	\end{itemize}
\end{itemize}

\paragraph{UC25.1.5 - Visualizza percentuale test coverage} \label{UC25.1.5}
\begin{itemize}
	\item \textbf{Attori principali:} Utente
	\item \textbf{Precondizioni:} 
	\begin{itemize}
		\item Il sistema è attivo e funzionante
		\item L'utente è stato riconosciuto dal sistema come Utente
		\item L'utente ha selezionato un workspace 
		\item L'utente sta visualizzando la lista delle repository del workspace corrente
		\item La repository di riferimento esiste nel sistema
		\item La repository ha subito almeno una scansione 
	\end{itemize}
	\item \textbf{Postcondizioni:} 
	\begin{itemize}
		\item Viene visualizzato il test coverage della repository
	\end{itemize}
	\item \textbf{Scenario principale:}
	\begin{enumerate}
		\item Il sistema recupera il test coverage calcolato durante l'ultima analisi della repository (il test coverage viene calcolato come rapporto tra linee di codice eseguite durante i test automatici e totale delle linee di codice eseguibili presenti nella repository)
		\item Il sistema mostra il valore percentuale del test coverage
	\end{enumerate}
		\item \textbf{Scenario alternativo:}
	\begin{itemize}
		\item \textbf{Nessun test configurato:} 
			\begin{itemize}
				\item \textbf{Condizione:}La repository non contiene file di test o configurazioni per l'esecuzione dei test
				\item \textbf{Flusso:} 
				\begin{enumerate}
					\item Il sistema mostra il messaggio: “Test coverage: 0\% - Nessun test configurato”
				\end{enumerate}
			\end{itemize}
		\item \textbf{Analisi test coverage non disponibile:} 
			\begin{itemize}
				\item \textbf{Condizione:}l'analisi del test coverage non è stata eseguita o non è andata a buon fine
				\item \textbf{Flusso:} 
				\begin{enumerate}
					\item Il sistema mostra il messaggio: “Test coverage non disponibile”
				\end{enumerate}
			\end{itemize}
		\item \textbf{Linguaggio non supportato:} 
			\begin{itemize}
				\item \textbf{Condizione:} il linguaggio di programmazione utilizzato nella repository non è supportato per l'analisi del test coverage
				\item \textbf{Flusso:} 
				\begin{enumerate}
					\item Il sistema mostra il messaggio: “Analisi test coverage non supportata per questo linguaggio”
				\end{enumerate}
			\end{itemize}
	\end{itemize}
\end{itemize}

\subsubsection{UC26 - Aggiungi repository}\label{UC26}

\begin{figure}[H]
	\centering
	\includegraphics[width=0.9\textwidth]{../Assets/AdR/UC26.png}
	\caption{UC26 - Aggiungi repository}
	\label{fig:UC26}
\end{figure}

\begin{itemize}
	\item \textbf{Attori principali:} Utente
	\item \textbf{Precondizioni:} 
	\begin{itemize}
		\item Il sistema è attivo e funzionante
		\item L'utente è stato riconosciuto dal sistema come Utente
		\item L'utente ha selezionato un workspace
	\end{itemize}
	\item \textbf{Postcondizioni:} 
	\begin{itemize}
		\item La repository indicata è associata al workspace corrente
		\item Se la repository è privata, il sistema ha validato e memorizzato un token di accesso valido
		\item La repository risulta disponibile per le successive operazioni di analisi e visualizzazione
		\item Le informazioni di accesso alla repository sono memorizzate dal sistema
	\end{itemize}
	\item \textbf{Scenario principale:}
	\begin{enumerate}
		\item L'utente seleziona l'opzione per aggiungere una nuova repository
		\item Il sistema mostra un form per l'inserimento dei dati della repository
		\item L'utente inserisce il link della repository GitHub (\hyperref[UC26.1]{UC 26.1 - Inserisci link repository GitHub})
		\item (Opzionale) L'utente inserisce il PAT(Personal Access Token) per l'accesso alla repository GitHub, se la repository è privata
		\item L'utente conferma l'operazione di aggiunta
		\item Il sistema valida il formato del link della repository
		\item Il sistema verifica l'accesso alla repository tramite PAT, se privata (\hyperref[UC26.2]{UC 26.2 - Inserisci PAT repository GitHub})
		\item Il sistema associa la repository al workspace corrente
		\item Il sistema mostra un messaggio di conferma dell'avvenuta aggiunta
	\end{enumerate}
	\item \textbf{Scenario alternativo:}
	\begin{itemize}
		\item \textbf{UC 27 - Errore aggiunta repository:} errore se i dati inseriti non sono validi oppure l'accesso alla repository non è consentito 
	\end{itemize}
	\item \textbf{Inclusioni:} \hyperref[UC26.1]{UC26.1}, \hyperref[UC26.2]{UC26.2}
	\item \textbf{Estensioni:} \hyperref[UC27]{UC27}
\end{itemize}

\paragraph{UC26.1 - Inserisci link repository GitHub}\label{UC26.1}
\begin{itemize}
	\item \textbf{Attori principali:} Utente
	\item \textbf{Precondizioni:} 
	\begin{itemize}
		\item Il sistema è attivo e funzionante
		\item L'utente è stato riconosciuto dal sistema come Utente
		\item L'utente ha selezionato un workspace
		\item L'utente ha avviato la procedura di aggiunta di una nuova repository
	\end{itemize}
	\item \textbf{Postcondizioni:} 
	\begin{itemize}
		\item Il link della repository GitHub è acquisito dal sistema ed è disponibile per le successive fasi di validazione
	\end{itemize}
	\item \textbf{Scenario principale:}
	\begin{enumerate}
		\item Il sistema mostra un campo di input per l'inserimento del link della repository GitHub
		\item L'utente inserisce l'URL completo della repository GitHub
		\item Il sistema acquisisce e memorizza il link inserito
	\end{enumerate}
\end{itemize}

\paragraph{UC26.2 - Inserisci PAT repository GitHub}\label{UC26.2}
\begin{itemize}
	\item \textbf{Attori principali:} Utente
	\item \textbf{Precondizioni:} 
	\begin{itemize}
		\item Il sistema è attivo e funzionante
		\item L'utente è stato riconosciuto dal sistema come Utente
		\item L'utente ha selezionato un workspace
		\item L'utente ha avviato la procedura di aggiunta di una nuova repository
		\item L'utente ha inserito il link della repository GitHub
	\end{itemize}
	\item \textbf{Postcondizioni:} 
	\begin{itemize}
		\item Il token di accesso alla repository GitHub, è acquisito dal sistema ed è disponibile per la verifica dell'accesso, se richiesto
	\end{itemize}
	\item \textbf{Scenario principale:}
	\begin{enumerate}
		\item Il sistema mostra un campo di input per l'inserimento del token di accesso alla repository GitHub
		\item L'utente inserisce un Personal Access Token (PAT) valido per l'accesso alla repository
		\item Il sistema acquisisce e memorizza il token inserito
	\end{enumerate}
		\item \textbf{Scenario alternativo:}
	\begin{itemize}
		\item \textbf{Repository pubblica:} 
			\begin{itemize}
				\item \textbf{Condizione:} La repository è pubblica e non richiede autenticazione
				\item \textbf{Flusso:} 
				\begin{enumerate}
					\item l'utente può omettere l'inserimento del token e procedere con la conferma dell'operazione
				\end{enumerate}
			\end{itemize}
	\end{itemize}
\end{itemize}

\subsubsection{UC27 - Errore aggiunta repository}\label{UC27}
\begin{itemize}
	\item \textbf{Attori principali:} Utente
	\item \textbf{Precondizioni:} 
	\begin{itemize}
		\item Il sistema è attivo e funzionante
		\item L'utente è stato riconosciuto dal sistema come Utente
		\item L'utente ha selezionato un workspace
		\item L'utente ha avviato la procedura di aggiunta di una nuova repository
		\item L'utente ha confermato l'aggiunta della repository
		\item Si è verificata almeno una delle seguenti condizioni:
		\begin{itemize}
			\item Il link della repository inserito non è valido (formato errato, repository inesistente)
			\item Il token inserito non è valido o non ha i permessi necessari
			\item La repository è già presente nel workspace
			\item Si è verificato un errore di connessione con GitHub
			\item Il branch develop non esiste
		\end{itemize}
	\end{itemize}
	\item \textbf{Postcondizioni:} 
	\begin{itemize}
		\item L'aggiunta della repository viene annullata
		\item Nessuna nuova repository viene aggiunta al workspace
		\item Il sistema mostra un messaggio di errore specifico all'utente
		\item Il form di aggiunta rimane aperto, permettendo di correggere l'errore o annullare l'operazione
	\end{itemize}
	\item \textbf{Scenario principale:}
	\begin{enumerate}
		\item Il sistema valida il link e il token della repository fornito
		\item Il sistema rileva un errore durante la validazione o la connessione
		\item Il sistema annulla l'operazione di aggiunta
		\item Il sistema mostra un messaggio di errore specifico, a seconda della causa:
		\begin{itemize}
			\item \textbf{Link repository non valido}: verifica il formato dell'URL
			\item \textbf{Repository non trovata}: verifica che l'URL sia corretto, che la repository esista
			\item \textbf{Token non valido}: verifica il PAT
			\item \textbf{Permessi insufficienti}: il PAT non ha accesso alla repository
			\item \textbf{Repository già presente}: repository già presente nel workspace
			\item \textbf{Errore di connessione}: errore di connessione con GitHub, riprova più tardi
			\item \textbf{Branch non trovato}: il branch develop non esiste nella repository
		\end{itemize}
		\item Il sistema mantiene aperto il form di aggiunta per permettere correzioni o annullamento
	\end{enumerate}
	\item \textbf{Scenario alternativo:}
	\begin{itemize}
		\item \textbf{Annullamento dell'operazione:} 
			\begin{itemize}
				\item \textbf{Condizione:} l'utente decide di chiudere il form senza correggere i dati
				\item \textbf{Flusso:} 
				\begin{enumerate}
					\item Il sistema chiude il form di aggiunta
					\item L'utente torna alla visualizzazione precedente del workspace
				\end{enumerate}
			\end{itemize}
	\end{itemize}
\end{itemize}

\subsubsection{UC28 - Rimuovi repository} \label{UC28}

\begin{figure}[H]
	\centering
	\includegraphics[width=0.9\textwidth]{../Assets/AdR/UC28.png}
	\caption{UC28 - Rimuovi repository}
	\label{fig:UC28}
\end{figure}

\begin{itemize}
	\item \textbf{Attori principali:} Utente
	\item \textbf{Precondizioni:} 
	\begin{itemize}
		\item Il sistema è attivo e funzionante
		\item L'utente è stato riconosciuto dal sistema come Utente
		\item L'utente ha selezionato un workspace
		\item L'utente ha selezionato una repository
	\end{itemize}
	\item \textbf{Postcondizioni:} 
	\begin{itemize}
		\item La repository non è più associata al workspace corrente
		\item Se la repository non è associata ad altri workspace: tutti i dati di analisi, scansioni, configurazioni e associazioni (inclusi i tag raccolta) vengono eliminati dal sistema
		\item Se la repository è associata ad altri workspace: vengono rimossi solo i dati e le associazioni relative al workspace corrente
	\end{itemize}
	\item \textbf{Scenario principale:}
	\begin{enumerate}
		\item L'utente seleziona l'opzione “Rimuovi repository” dalla repository selezionata
		\item Il sistema mostra una finestra di conferma che informa l'utente che:
		\begin{itemize}
			\item la repository verrà rimossa dal workspace
			\item tutti i dati di analisi associati verranno eliminati
			\item l'operazione è irreversibile
		\end{itemize}
		\item L'utente conferma l'operazione di rimozione
		\item Il sistema verifica che la repository sia associata esclusivamente al workspace corrente
		\item Il sistema rimuove la repository dal workspace
		\item Il sistema elimina tutti i dati di analisi, scansioni, configurazioni associate alla repository
		\item Il sistema aggiorna la visualizzazione e mostra un messaggio di conferma dell'avvenuta rimozione
	\end{enumerate}
	\item \textbf{Scenario alternativo:}
	\begin{itemize}
		\item \textbf{Annullamento rimozione repository:} (\hyperref[UC28.1]{UC28.1 - Annullamento rimozione repository})
			\begin{itemize}
				\item \textbf{Condizione:} l'utente annulla l'operazione nella finestra di conferma
				\item \textbf{Flusso:} 
				\begin{enumerate}
					\item Il sistema annulla l'operazione
					\item La repository rimane invariata nel workspace
				\end{enumerate}
			\end{itemize}
		\item \textbf{Repository con scansioni in corso:} (\hyperref[UC28.2]{UC28.2 - Repository con scansioni in corso})
			\begin{itemize}
				\item \textbf{Condizione:} sono presenti scansioni attive sulla repository
				\item \textbf{Flusso:} 
				\begin{enumerate}
					\item Il sistema mostra un avviso: “Sono presenti scansioni in corso che verranno interrotte”
					\item Il sistema richiede una conferma esplicita aggiuntiva
					\item Se l'utente conferma, il flusso riprende dal punto 4 dello scenario principale
				\end{enumerate}
			\end{itemize}
		\item \textbf{Errore durante la rimozione:} 
			\begin{itemize}
				\item \textbf{Condizione:} si verifica un errore tecnico durante la rimozione
				\item \textbf{Flusso:} 
				\begin{enumerate}
					\item Il sistema interrompe l'operazione
					\item Il sistema mostra un messaggio di errore: “Errore durante la rimozione della repository”
				\end{enumerate}
			\end{itemize}
		\item \textbf{Repository appartenente a più workspace}
			\begin{itemize}
				\item \textbf{Condizione:} la repository da analizzare appartiene ad altri workspace
				\item \textbf{Flusso:} 
				\begin{enumerate}
					\item Il sistema rileva che la repository è condivisa con altri workspace
					\item Il sistema mostra un messaggio informativo: “La repository è associata ad altri workspace. Verrà rimossa solo dal workspace corrente.”
					\item L'utente conferma l'operazione
					\item Il sistema rimuove l'associazione tra la repository e il workspace corrente
					\item Il sistema elimina esclusivamente:
					\begin{itemize}
						\item i dati di analisi relativi al workspace corrente
						\item le configurazioni specifiche del workspace
						\item le associazioni ai tag raccolta del workspace corrente
					\end{itemize}
					\item La repository e i relativi dati restano disponibili negli altri workspace
				\end{enumerate}
			\end{itemize}
	\end{itemize}
	\item \textbf{Estensioni:} \hyperref[UC28.1]{UC28.1}, \hyperref[UC28.2]{UC28.2}
\end{itemize}

\paragraph{UC28.1 - Annullamento rimozione repository} \label{UC28.1}
\begin{itemize}
	\item \textbf{Attori principali:} Utente
	\item \textbf{Precondizioni:} 
	\begin{itemize}
		\item Il sistema è attivo e funzionante
		\item L'utente è stato riconosciuto dal sistema come Utente
		\item L'utente ha selezionato un workspace
		\item L'utente ha selezionato una repository
		\item L'utente ha avviato il caso d'uso UC 28 - Rimuovi repository (\hyperref[UC28]{UC28})
		\item Il sistema ha mostrato la richiesta di conferma della rimozione
	\end{itemize}
	\item \textbf{Postcondizioni:} 
	\begin{itemize}
		\item La repository selezionata non viene rimossa dal workspace
		\item Nessun dato di analisi, scansioni o configurazione viene eliminato
		\item Il sistema torna alla visualizzazione precedente alla richiesta di rimozione
	\end{itemize}
	\item \textbf{Scenario principale:}
	\begin{enumerate}
		\item L'utente seleziona l'opzione “Rimuovi repository” dalla repository selezionata
		\item Il sistema mostra una finestra di conferma che informa l'utente che:
		\begin{itemize}
			\item la repository verrà rimossa dal workspace
			\item tutti i dati di analisi associati verranno eliminati
			\item l'operazione è irreversibile
		\end{itemize}
		\item L'utente visualizza il messaggio di conferma della rimozione della repository
		\item L'utente seleziona l'opzione “Annulla”
		\item Il sistema interrompe la procedura di rimozione
		\item Il sistema ritorna alla visualizzazione precedente
	\end{enumerate}
\end{itemize}

\paragraph{UC28.2 - Repository con scansioni in corso} \label{UC28.2}
\begin{itemize}
	\item \textbf{Attori principali:} Utente
	\item \textbf{Precondizioni:} 
	\begin{itemize}
		\item Il sistema è attivo e funzionante
		\item L'utente è stato riconosciuto dal sistema come Utente
		\item L'utente ha selezionato un workspace
		\item L'utente ha selezionato una repository
		\item L'utente ha avviato il caso d'uso UC 28 - Rimuovi repository (\hyperref[UC28]{UC28})
		\item Il sistema rileva la presenza di scansioni in corso sulla repository
	\end{itemize}
	\item \textbf{Postcondizioni:} 
	\begin{itemize}
		\item Le scansioni in corso vengono interrotte e la repository viene rimossa solo se l'utente conferma esplicitamente la richiesta
		\item In caso di annullamento, nessuna modifica viene applicata
	\end{itemize}
	\item \textbf{Scenario principale:}
	\begin{enumerate}
		\item Il sistema rileva che la repository ha scansioni attive
		\item Il sistema mostra un messaggio: “Sono presenti scansioni in corso che verranno interrotte. Vuoi procedere?”
		\item L'utente conferma la rimozione nonostante le scansioni attive
		\item Il sistema interrompe le scansioni in corso
		\item Il sistema procede con la rimozione della repository come descritto nello scenario principale di UC 28 - Rimuovi repository (\hyperref[UC28]{UC28}) 
	\end{enumerate}
\end{itemize}


\subsubsection{UC29 - Ricerca repository}\label{UC29}

\begin{figure}[H]
	\centering
	\includegraphics[width=0.8\textwidth]{../Assets/AdR/UC29.png}
	\caption{UC 29 - Ricerca repository}
	\label{fig:UC29}
\end{figure}


\begin{itemize}
	\item \textbf{Attori principali:} Utente
	\item \textbf{Precondizioni:} 
	\begin{itemize}
		\item Il sistema è attivo e funzionante
		\item L'utente è stato riconosciuto dal sistema come Utente
		\item L'utente ha selezionato un workspace
	\end{itemize}
	\item \textbf{Postcondizioni:} 
	\begin{itemize}
		\item Il sistema mostra la lista della lista delle repository del workspace che soddisfano i criteri di ricerca specificati dall'utente
	\end{itemize}
	\item \textbf{Scenario principale:}
	\begin{enumerate}
		\item L'utente seleziona l'opzione di ricerca repository
		\item Il sistema acquisisce i criteri di ricerca dall'utente (tramite barra di ricerca o filtri)
		\item Il sistema filtra la lista delle repository del workspace in base ai criteri specificati
		\item Il sistema mostra la lista delle repository che corrispondono ai criteri di ricerca
	\end{enumerate}
	\item \textbf{UC che ereditano:} 
	\begin{itemize}
		\item \hyperref[UC30]{UC30 - Ricerca con barra}
		\item \hyperref[UC31]{UC31 - Ricerca con filtri}
	\end{itemize}
\end{itemize}

\subsubsection{UC30 - Ricerca con barra}\label{UC30}
\begin{itemize}
	\item \textbf{Attori principali:} Utente
	\item \textbf{Precondizioni:} 
	\begin{itemize}
		\item Il sistema è attivo e funzionante
		\item L'utente è stato riconosciuto dal sistema come Utente
		\item L'utente ha selezionato un workspace
		\item L'utente ha selezionato l'opzione di ricerca repository attraverso barra di ricerca
	\end{itemize}
	\item \textbf{Postcondizioni:} 
	\begin{itemize}
		\item Il sistema mostra la lista della lista delle repository del workspace che soddisfano i criteri di ricerca specificati dall'utente nella barra di ricerca
	\end{itemize}
	\item \textbf{Scenario principale:}
	\begin{enumerate}
		\item L'utente digita una parola chiave o una frase nella barra di ricerca
		\item L'utente conferma l'operazione di ricerca
		\item Il sistema filtra la lista delle repository del workspace in base alla corrispondenza con la parola chiave o frase inserita (nome repository, linguaggio di programmazione, tag raccolta o username github)
		\item Il sistema mostra la lista delle repository che corrispondono ai criteri di ricerca
	\end{enumerate}
\end{itemize}

\subsubsection{UC31 - Ricerca con filtri}\label{UC31}

\begin{figure}[H]
 	\centering
 	\includegraphics[width=0.8\textwidth]{../Assets/AdR/UC31.png}
 	\caption{UC 31 - Ricerca con filtri}
 	\label{fig:UC31}
\end{figure}

\begin{itemize}
	\item \textbf{Attori principali:} Utente
	\item \textbf{Precondizioni:} 
	\begin{itemize}
		\item Il sistema è attivo e funzionante
		\item L'utente è stato riconosciuto dal sistema come Utente
		\item L'utente ha selezionato un workspace
		\item L'utente ha selezionato l'opzione di ricerca repository con filtri
	\end{itemize}
	\item \textbf{Postcondizioni:} 
	\begin{itemize}
		\item Il sistema mostra la lista della lista delle repository del workspace che soddisfano i criteri di ricerca specificati dall'utente tramite filtri
	\end{itemize}
	\item \textbf{Scenario principale:}
	\begin{enumerate}
		\item L'utente seleziona uno o più filtri di ricerca (\hyperref[UC31.1]{UC31.1 - Selezione filtri di ricerca})
		\item L'utente conferma l'operazione di ricerca
		\item Il sistema filtra la lista delle repository del workspace in base ai criteri specificati tramite i filtri selezionati
		\item Il sistema mostra la lista delle repository che corrispondono ai criteri di ricerca
	\end{enumerate}
	\item \textbf{Inclusioni}: \hyperref[UC31.1]{UC31.1}
\end{itemize}

\paragraph{UC31.1 Selezione filtri di ricerca} \label{UC31.1}
\begin{itemize}
	\item \textbf{Attori principali:} Utente
	\item \textbf{Precondizioni:}
	\begin{itemize}
		\item Il sistema è attivo e funzionante
		\item L'utente è stato riconosciuto dal sistema come Utente
		\item L'utente ha selezionato un workspace
		\item L'utente ha avviato una ricerca con filtri
	\end{itemize}
	\item \textbf{Postcondizioni:} 
	\begin{itemize}
		\item I filtri di ricerca selezionati dall'utente sono acquisiti dal sistema
	\end{itemize}
	\item \textbf{Scenario principale:}
	\begin{enumerate}
		\item Il sistema mostra una lista di filtri di ricerca disponibili 
		\item L'utente può selezionare uno o più filtri dalla lista:
		\begin{itemize}
			\item linguaggi di programmazione
			\item tag raccolta
			\item nome del repository
			\item username dell'utente che ha aggiunto la repository
			\item username github di una persona che lavora su una repository 
			\item intervallo di test coverage
			\item intervallo di voto della documentazione
		\end{itemize}
		\item L'utente conferma la selezione dei filtri
		\item Il sistema acquisisce i filtri selezionati
	\end{enumerate}
	\item \textbf{Scenario alternativo:}
	\begin{itemize}
		\item \textbf{Nessun filtro selezionato:} 
			\begin{itemize}
				\item \textbf{Condizione:} l'utente non seleziona alcun filtro
				\item \textbf{Flusso:} 
				\begin{enumerate}
					\item Il sistema considera tutti i repository del workspace come risultati della ricerca
				\end{enumerate}
			\end{itemize}
		\item \textbf{Rimozione filtri selezionati:} 
			\begin{itemize}
				\item \textbf{Condizione:} l'utente decide di rimuovere uno o più filtri selezionati
				\item \textbf{Flusso:} 
				\begin{enumerate}
					\item L'utente deseleziona i filtri che non vuole più utilizzare
					\item Il sistema aggiorna la lista dei filtri selezionati
				\end{enumerate}
			\end{itemize}
		\item \textbf{Inserimento di valori non validi}:
		\begin{itemize}
				\item \textbf{Condizione:} l'utente inserisce valori non validi
				\item \textbf{Flusso:} 
				\begin{enumerate}
					\item il sistema segnala l'errore
					\item l'utente può correggere i dati inseriti
				\end{enumerate}
		\end{itemize}
	\end{itemize}
\end{itemize}

\subsubsection{UC32 - Ordinamento lista repository}\label{UC32}

\begin{figure}[H]
	\centering
	\includegraphics[width=0.7\textwidth]{../Assets/AdR/UC32.png}
	\caption{UC32 - Ordinamento lista repository}
	\label{fig:UC32}
\end{figure}

\begin{itemize}
	\item \textbf{Attori principali:} Utente
	\item \textbf{Precondizioni:}
	\begin{itemize}
		\item Il sistema è attivo e funzionante
		\item L'utente è stato riconosciuto dal sistema come Utente
		\item L'utente ha selezionato un workspace
		\item L'utente sta visualizzando una lista di repository del workspace
	\end{itemize}
	\item \textbf{Postcondizioni:} 
	\begin{itemize}
		\item Il sistema mostra la lista delle repository ordinata secondo il criterio e la direzione selezionati
	\end{itemize}
	\item \textbf{Scenario principale:}
	\begin{enumerate}
		\item L'utente seleziona l'opzione di ordinamento della lista repository
		\item Il sistema mostra il criterio di ordinamento attualmente attivo
		\item Il sistema mostra una lista di criteri di ordinamento disponibili:
		\begin{itemize}
			\item Nome repository (A-Z, Z-A)
			\item Data di aggiunta al workspace (più recente, più vecchia)
			\item tempo dall'ultima scansione (più recente, più vecchia)
			\item Test coverage (crescente, decrescente)
			\item Voto della documentazione (crescente, decrescente)
		\end{itemize}
		\item L'utente seleziona un criterio di ordinamento dalla lista
		\item L'utente seleziona la direzione di ordinamento: 
		\begin{itemize}
			\item crescente: A-Z, da più vecchia a più recente, numero crescente
			\item decrescente: Z-A, da più recente a più vecchia, numero decrescente
		\end{itemize}
		\item Il sistema ordina la lista delle repository in base al criterio e alla direzione selezionati
		\item Il sistema mostra la lista delle repository ordinata
	\end{enumerate}
	\item \textbf{Scenario alternativo:}
	\begin{itemize}
		\item \textbf{Nessun criterio selezionato:} 
			\begin{itemize}
				\item \textbf{Condizione:} l'utente non seleziona alcun criterio di ordinamento
				\item \textbf{Flusso:} 
				\begin{enumerate}
					\item Il sistema mantiene l'ordinamento attuale della lista delle repository
				\end{enumerate}
			\end{itemize}
	\end{itemize}
\end{itemize}

\newpage
% ============================================
% UC33 - VISUALIZZA DETTAGLIO REPOSITORY
% ============================================
\subsubsection{UC33 - Visualizza dettaglio repository} \label{UC33}

\begin{figure}[H]
	\centering
	\includegraphics[width=0.9\textwidth]{../Assets/AdR/UC33ext.png}
	\caption{UC33 - Visualizza dettaglio repository}
	\label{fig:UC33ext}
\end{figure}

\begin{itemize}
    \item \textbf{Attori principali:} Utente
    \item \textbf{Precondizioni:}
    \begin{itemize}
        \item Il sistema è attivo e funzionante
        \item L'utente è stato riconosciuto dal sistema come Utente
        \item L'utente ha selezionato un workspace
        \item L'utente ha selezionato una repository presente nel workspace
        \item La repository selezionata ha subito almeno una scansione completata con successo
    \end{itemize}
    \item \textbf{Postcondizioni:}
    \begin{itemize}
        \item Il sistema presenta all'utente la dashboard completa della repository
        \item L'utente può visualizzare tutte le sezioni di analisi disponibili (test, informazioni tecniche, sicurezza, documentazione, qualità del codice)
    \end{itemize}
    \item \textbf{Scenario principale:}
    \begin{enumerate}
        \item L'utente seleziona una repository dalla lista delle repository del workspace
        \item Il sistema recupera i dati dell'ultima scansione effettuata sulla repository
        \item Il sistema visualizza la sezione di analisi dei test (\hyperref[UC33.1]{UC33.1})
        \item Il sistema visualizza la sezione delle informazioni tecniche (\hyperref[UC33.2]{UC33.2})
        \item Il sistema visualizza la sezione di analisi della sicurezza semplice (\hyperref[UC33.3]{UC33.3})
        \item Il sistema visualizza la sezione di analisi della sicurezza completa (\hyperref[UC33.4]{UC33.4})
        \item Il sistema visualizza la sezione di analisi della documentazione (\hyperref[UC33.5]{UC33.5})
        \item Il sistema visualizza la sezione della qualità del codice (\hyperref[UC33.6]{UC33.6})
    \end{enumerate}
    \item \textbf{Scenario alternativo:}
    \begin{itemize}
        \item \textbf{Nessuna scansione disponibile:}
        \begin{itemize}
            \item \textbf{Condizione:} La repository non ha mai subito una scansione o l'ultima scansione non è andata a buon fine
            \item \textbf{Flusso:}
            \begin{enumerate}
                \item Il sistema mostra un messaggio informativo: "Nessuna scansione disponibile per questa repository"
                \item Il sistema suggerisce all'utente di avviare una nuova scansione
            \end{enumerate}
        \end{itemize}
		\item \textbf{Selezione Branch:}
		\begin{itemize}
			\item \textbf{Condizione:} L'utente vuole cambiare branch di cui vedere i dettagli
            \item  \textbf{Flusso:}
			\begin{enumerate}
				\item L'utente sceglie il branch di cui vedere i dettagli tra la lista dei branch disponibili
			\end{enumerate}
		\end{itemize}
        \item \textbf{Aggiorna visualizzazione (\hyperref[UC37]{UC37 - Aggiorna repository}):}
        \begin{itemize}
            \item \textbf{Condizione:} L'utente seleziona “Aggiorna”
            \item \textbf{Flusso:}
            \begin{enumerate}
                \item Il sistema controlla se ci sono scansioni più recenti
                \item Il sistema aggiorna i dati visualizzati con quelli dell'ultima scansione
            \end{enumerate}
        \end{itemize}
    \end{itemize}
    \item \textbf{Inclusioni:} \hyperref[UC33.1]{UC33.1}, \hyperref[UC33.2]{UC33.2}, \hyperref[UC33.3]{UC33.3}, \hyperref[UC33.4]{UC33.4}, \hyperref[UC33.5]{UC33.5}, \hyperref[UC33.6]{UC33.6}
    \item \textbf{Estensioni:} \hyperref[UC34]{UC34}, \hyperref[UC35]{UC35}, \hyperref[UC37]{UC37}
\end{itemize}

Il caso d'Uso UC33 include ulteriori casi d'uso come rappresentato nella seguente immagine:
\begin{figure}[H]
	\centering
	\includegraphics[width=0.6\textwidth]{../Assets/AdR/UC33.png}
	\caption{Inclusioni di UC33: UC33.1, UC33.2, UC33.3, UC33.4, UC33.5, UC33.6}
	\label{fig:UC33}
\end{figure}

% ============================================
% UC33.1 - VISUALIZZA ANALISI DEI TEST
% ============================================

\paragraph{UC33.1 - Visualizza analisi dei test} \label{UC33.1}
\begin{figure}[H]
	\centering
	\includegraphics[width=0.9\textwidth]{../Assets/AdR/UC33-1ext.png}
	\caption{UC33.1 - Visualizza analisi dei test}
	\label{fig:UC33-1ext}
\end{figure}

\begin{itemize}
    \item \textbf{Attori principali:} Utente
    \item \textbf{Precondizioni:}
    \begin{itemize}
        \item Il sistema è attivo e funzionante
        \item L'utente è stato riconosciuto dal sistema come Utente
        \item L'utente sta visualizzando il dettaglio di una repository (\hyperref[UC33]{UC33})
        \item La repository ha subito almeno una scansione che include l'analisi dei test
    \end{itemize}
    \item \textbf{Postcondizioni:}
    \begin{itemize}
        \item Il sistema presenta all'utente la sezione completa di analisi dei test
        \item L'utente può visualizzare il coverage, i test con qualità insufficiente, la percentuale di test passati e l'elenco dei test non passati
    \end{itemize}
    \item \textbf{Scenario principale:}
    \begin{enumerate}
        \item Il sistema recupera i dati dei test dall'ultima scansione della repository
        \item Il sistema mostra la percentuale di test coverage (\hyperref[UC33.1.1]{UC33.1.1})
        \item Il sistema mostra l'elenco dei test con qualità insufficiente (\hyperref[UC33.1.2]{UC33.1.2})
        \item Il sistema mostra la percentuale di test passati (\hyperref[UC33.1.3]{UC33.1.3})
        \item Il sistema mostra l'elenco dei test non passati (\hyperref[UC33.1.4]{UC33.1.4})
    \end{enumerate}
    \item \textbf{Scenario alternativo:}
    \begin{itemize}
        \item \textbf{Nessun test rilevato (\hyperref[UC33.1.5]{UC33.1.5}):}
        \begin{itemize}
            \item \textbf{Condizione:} L'ultima scansione non contiene informazioni sui test o la repository non contiene test configurati
            \item \textbf{Flusso:}
            \begin{enumerate}
                \item Il sistema mostra un messaggio informativo: "Nessun test rilevato nella repository"
                \item Il sistema suggerisce all'utente di configurare i test nella repository
            \end{enumerate}
        \end{itemize}
    \end{itemize}
    \item \textbf{Inclusioni:} \hyperref[UC33.1.1]{UC33.1.1}, \hyperref[UC33.1.2]{UC33.1.2}, \hyperref[UC33.1.3]{UC33.1.3}, \hyperref[UC33.1.4]{UC33.1.4}
    \item \textbf{Estensioni:} \hyperref[UC33.1.5]{UC33.1.5}
\end{itemize}

Il caso d'uso UC33.1 include ulteriori casi d'uso come rappresentato nella seguente immagine:
\begin{figure}[H]
	\centering
	\includegraphics[width=0.6\textwidth]{../Assets/AdR/UC33-1.png}
	\caption{Inclusioni di UC33.1: UC33.1.1, UC33.1.2, UC33.1.3, UC33.1.4}
	\label{fig:UC33-1}
\end{figure}

\subparagraph{UC33.1.1 - Visualizza test coverage}\label{UC33.1.1}
\begin{itemize}
    \item \textbf{Attori principali:} Utente
    \item \textbf{Precondizioni:}
    \begin{itemize}
        \item Il sistema è attivo e funzionante
        \item L'utente è stato riconosciuto dal sistema come Utente
        \item L'utente sta visualizzando la sezione di analisi dei test (\hyperref[UC33.1]{UC33.1})
        \item L'analisi del test coverage è stata eseguita con successo
    \end{itemize}
    \item \textbf{Postcondizioni:}
    \begin{itemize}
        \item Viene visualizzata la percentuale di copertura del codice da parte dei test
    \end{itemize}
    \item \textbf{Scenario principale:}
    \begin{enumerate}
        \item Il sistema recupera il valore di test coverage calcolato durante l'ultima scansione
        \item Il sistema visualizza la percentuale di copertura del codice (rapporto tra linee di codice eseguite durante i test e totale delle linee eseguibili)
        \item Il sistema mostra un indicatore visivo che rappresenta il livello di copertura raggiunto
    \end{enumerate}
\end{itemize}

\subparagraph{UC33.1.2 - Visualizza lista test con qualità insufficiente}\label{UC33.1.2}
\begin{figure}[H]
	\centering
	\includegraphics[width=0.9\textwidth]{../Assets/AdR/UC33-1-2.png}
	\caption{UC33.1.2 - Visualizza lista test con qualità insufficiente}
	\label{fig:UC33-1-2}
\end{figure}

\begin{itemize}
    \item \textbf{Attori principali:} Utente
    \item \textbf{Precondizioni:}
    \begin{itemize}
        \item Il sistema è attivo e funzionante
        \item L'utente è stato riconosciuto dal sistema come Utente
        \item L'utente sta visualizzando la sezione di analisi dei test (\hyperref[UC33.1]{UC33.1})
        \item Sono stati rilevati test con qualità insufficiente durante l'ultima scansione
    \end{itemize}
    \item \textbf{Postcondizioni:}
    \begin{itemize}
        \item Viene visualizzata la lista dei test che non rispettano gli standard di qualità definiti
    \end{itemize}
    \item \textbf{Scenario principale:}
    \begin{enumerate}
        \item Il sistema recupera l'elenco dei test unitari valutati come insufficienti
        \item Il sistema visualizza la lista completa dei test problematici
        \item Per ogni test nella lista, il sistema visualizza il singolo elemento contenente le informazioni di dettaglio (\hyperref[UC33.1.2.1]{UC33.1.2.1})
    \end{enumerate}
    \item \textbf{Inclusioni:} \hyperref[UC33.1.2.1]{UC33.1.2.1}
\end{itemize}

\subsubparagraph{UC33.1.2.1 - Visualizza test singolo insufficiente}\label{UC33.1.2.1}
\begin{itemize}
    \item \textbf{Attori principali:} Utente
    \item \textbf{Precondizioni:}
    \begin{itemize}
        \item Il sistema è attivo e funzionante
        \item L'utente sta visualizzando la lista dei test con qualità insufficiente (\hyperref[UC33.1.2]{UC33.1.2})
    \end{itemize}
    \item \textbf{Postcondizioni:}
    \begin{itemize}
        \item Vengono visualizzati i dati identificativi e i dettagli dell'errore per un singolo test
    \end{itemize}
    \item \textbf{Scenario principale:}
    \begin{enumerate}
        \item Il sistema identifica il singolo test all'interno della lista
        \item Il sistema visualizza il nome del test unitario (\hyperref[UC33.1.2.1.1]{UC33.1.2.1.1})
        \item Il sistema visualizza i dettagli della violazione o del problema di qualità riscontrato (\hyperref[UC33.1.2.1.2]{UC33.1.2.1.2})
    \end{enumerate}
    \item \textbf{Inclusioni:} \hyperref[UC33.1.2.1.1]{UC33.1.2.1.1}, \hyperref[UC33.1.2.1.2]{UC33.1.2.1.2}
\end{itemize}

\subsubparagraph{UC33.1.2.1.1 - Visualizza nome test unitario}\label{UC33.1.2.1.1}
\begin{itemize}
    \item \textbf{Attori principali:} Utente
    \item \textbf{Precondizioni:}
    \begin{itemize}
        \item Il sistema è attivo e funzionante
        \item L'utente sta visualizzando un singolo test insufficiente (\hyperref[UC33.1.2.1]{UC33.1.2.1})
    \end{itemize}
    \item \textbf{Postcondizioni:}
    \begin{itemize}
        \item Viene mostrato il nome identificativo o la firma del metodo del test unitario
    \end{itemize}
    \item \textbf{Scenario principale:}
    \begin{enumerate}
        \item Il sistema recupera il nome del test unitario dal report di scansione
        \item Il sistema mostra a schermo il nome del test per identificarlo univocamente
    \end{enumerate}
\end{itemize}

\subsubparagraph{UC33.1.2.1.2 - Visualizza dettagli violazione}\label{UC33.1.2.1.2}
\begin{itemize}
    \item \textbf{Attori principali:} Utente
    \item \textbf{Precondizioni:}
    \begin{itemize}
        \item Il sistema è attivo e funzionante
        \item L'utente sta visualizzando un singolo test insufficiente (\hyperref[UC33.1.2.1]{UC33.1.2.1})
    \end{itemize}
    \item \textbf{Postcondizioni:}
    \begin{itemize}
        \item Vengono mostrati i motivi specifici per cui la qualità del test è ritenuta insufficiente
    \end{itemize}
    \item \textbf{Scenario principale:}
    \begin{enumerate}
        \item Il sistema recupera i metadati relativi alla violazione 
        \item Il sistema mostra una descrizione testuale che spiega la criticità rilevata
    \end{enumerate}
\end{itemize}

\subparagraph{UC33.1.3 - Visualizza percentuale test passati}\label{UC33.1.3}
\begin{itemize}
    \item \textbf{Attori principali:} Utente
    \item \textbf{Precondizioni:}
    \begin{itemize}
        \item Il sistema è attivo e funzionante
        \item L'utente è stato riconosciuto dal sistema come Utente
        \item L'utente sta visualizzando la sezione di analisi dei test (\hyperref[UC33.1]{UC33.1})
        \item I test sono stati eseguiti durante l'ultima scansione
    \end{itemize}
    \item \textbf{Postcondizioni:}
    \begin{itemize}
        \item Viene visualizzata la percentuale di test passati rispetto al totale
    \end{itemize}
    \item \textbf{Scenario principale:}
    \begin{enumerate}
        \item Il sistema recupera il numero totale di test eseguiti e il numero di test superati
        \item Il sistema calcola e visualizza il rapporto percentuale tra test superati e test totali
        \item Il sistema mostra un indicatore visivo che rappresenta il tasso di successo dei test
    \end{enumerate}
\end{itemize}

\subparagraph{UC33.1.4 - Visualizza test non passati}\label{UC33.1.4}
\begin{itemize}
    \item \textbf{Attori principali:} Utente
    \item \textbf{Precondizioni:}
    \begin{itemize}
        \item Il sistema è attivo e funzionante
        \item L'utente è stato riconosciuto dal sistema come Utente
        \item L'utente sta visualizzando la sezione di analisi dei test (\hyperref[UC33.1]{UC33.1})
        \item I test sono stati eseguiti durante l'ultima scansione
    \end{itemize}
    \item \textbf{Postcondizioni:}
    \begin{itemize}
        \item Viene visualizzato l'elenco dei test falliti con i relativi dettagli
    \end{itemize}
    \item \textbf{Scenario principale:}
    \begin{enumerate}
        \item Il sistema recupera l'elenco dei test che non sono stati superati durante l'ultima scansione
        \item Il sistema mostra l'elenco dei test falliti, indicando per ciascuno:
        \begin{itemize}
            \item Nome del test
            \item Messaggio di errore
            \item Stack trace o dettagli del fallimento
            \item File e riga di codice coinvolti
        \end{itemize}
    \end{enumerate}
    
\end{itemize}

\subparagraph{UC33.1.5 - Nessun test rilevato}\label{UC33.1.5}
\begin{itemize}
    \item \textbf{Attori principali:} Utente
    \item \textbf{Precondizioni:}
    \begin{itemize}
        \item Il sistema è attivo e funzionante
        \item L'utente è stato riconosciuto dal sistema come Utente
        \item L'utente sta visualizzando la sezione di analisi dei test (\hyperref[UC33.1]{UC33.1})
        \item La repository non contiene test configurati o l'analisi dei test non è stata eseguita
    \end{itemize}
    \item \textbf{Postcondizioni:}
    \begin{itemize}
        \item Il sistema informa l'utente dell'assenza di dati sui test
    \end{itemize}
    \item \textbf{Scenario principale:}
    \begin{enumerate}
        \item Il sistema rileva che non sono presenti informazioni sui test per la repository
        \item Il sistema mostra un messaggio informativo: "Nessun test rilevato nella repository"
        \item Il sistema suggerisce all'utente di configurare i test nella repository per abilitare l'analisi
    \end{enumerate}
\end{itemize}


% ============================================
% UC33.2 - VISUALIZZA INFORMAZIONI TECNICHE
% ============================================
\paragraph{UC33.2 - Visualizza informazioni tecniche} \label{UC33.2}
\begin{figure}[H]
	\centering
	\includegraphics[width=0.6\textwidth]{../Assets/AdR/UC33-2.png}
	\caption{Inclusioni di UC33.2: UC33.2.1, UC33.2.2, UC33.2.3}
	\label{fig:UC33-2}
\end{figure}

\begin{itemize}
    \item \textbf{Attori principali:} Utente
    \item \textbf{Precondizioni:}
    \begin{itemize}
        \item Il sistema è attivo e funzionante
        \item L'utente è stato riconosciuto dal sistema come Utente
        \item L'utente sta visualizzando il dettaglio di una repository (\hyperref[UC33]{UC33})
        \item La repository ha subito almeno una scansione che include l'analisi delle informazioni tecniche
    \end{itemize}
    \item \textbf{Postcondizioni:}
    \begin{itemize}
        \item Il sistema presenta all'utente la sezione completa delle informazioni tecniche della repository
        \item L'utente può visualizzare i linguaggi, le librerie e i framework utilizzati nella repository
    \end{itemize}
    \item \textbf{Scenario principale:}
    \begin{enumerate}
        \item Il sistema recupera le informazioni tecniche dall'ultima scansione della repository
        \item Il sistema visualizza la lista dei linguaggi di programmazione utilizzati (\hyperref[UC33.2.1]{UC33.2.1})
        \item Il sistema visualizza la lista delle librerie utilizzate (\hyperref[UC33.2.2]{UC33.2.2})
        \item Il sistema visualizza la lista dei framework utilizzati (\hyperref[UC33.2.3]{UC33.2.3})
    \end{enumerate}
    \item \textbf{Inclusioni:} \hyperref[UC33.2.1]{UC33.2.1}, \hyperref[UC33.2.2]{UC33.2.2}, \hyperref[UC33.2.3]{UC33.2.3}
\end{itemize}



% --- LINGUAGGI ---
\subparagraph{UC33.2.1 - Visualizza lista linguaggi}\label{UC33.2.1}

\begin{figure}[H]
	\centering
	\includegraphics[width=0.8\textwidth]{../Assets/AdR/UC33-2-1ext.png}
	\caption{UC33.2.1 - Visualizza lista linguaggi}
	\label{fig:UC33-2-1ext}
\end{figure}

\begin{itemize}
    \item \textbf{Attori principali:} Utente
    \item \textbf{Precondizioni:}
    \begin{itemize}
        \item Il sistema è attivo e funzionante
        \item L'utente è stato riconosciuto dal sistema come Utente
        \item L'utente sta visualizzando la sezione delle informazioni tecniche (\hyperref[UC33.2]{UC33.2})
    \end{itemize}
    \item \textbf{Postcondizioni:}
    \begin{itemize}
        \item Viene visualizzata la lista dei linguaggi di programmazione rilevati nella repository
    \end{itemize}
    \item \textbf{Scenario principale:}
    \begin{enumerate}
        \item Il sistema recupera l'elenco dei linguaggi di programmazione rilevati durante l'ultima scansione
        \item Per ogni linguaggio presente nella lista, il sistema visualizza il dettaglio del singolo linguaggio (\hyperref[UC33.2.1.1]{UC33.2.1.1})
    \end{enumerate}
    \item \textbf{Scenario alternativo:}
    \begin{itemize}
        \item \textbf{Nessun linguaggio rilevato (\hyperref[UC33.2.1.3]{UC33.2.1.3}):}
        \begin{itemize}
            \item \textbf{Condizione:} La scansione non ha rilevato alcun linguaggio di programmazione
            \item \textbf{Flusso:}
            \begin{enumerate}
                \item Il sistema mostra un messaggio informativo: "Nessun linguaggio di programmazione rilevato"
            \end{enumerate}
        \end{itemize}
    \end{itemize}
    \item \textbf{Inclusioni:} \hyperref[UC33.2.1.1]{UC33.2.1.1}
    \item \textbf{Estensioni:} \hyperref[UC33.2.1.3]{UC33.2.1.3}
\end{itemize}

Il caso d'uso UC33.2.1 include ulteriori casi d'uso come rappresentato nella seguente immagine:
\begin{figure}[H]
	\centering
	\includegraphics[width=0.8\textwidth]{../Assets/AdR/UC33-2-1.png}
	\caption{UC33.2.1.1 - Visualizza linguaggio singolo}
	\label{fig:UC33-2-1}
\end{figure}

\subsubparagraph{UC33.2.1.1 - Visualizza linguaggio singolo}\label{UC33.2.1.1}
\begin{itemize}
    \item \textbf{Attori principali:} Utente
    \item \textbf{Precondizioni:}
    \begin{itemize}
        \item Il sistema è attivo e funzionante
        \item L'utente è stato riconosciuto dal sistema come Utente
        \item L'utente sta visualizzando la lista dei linguaggi (\hyperref[UC33.2.1]{UC33.2.1})
        \item Esiste almeno un linguaggio rilevato nella repository
    \end{itemize}
    \item \textbf{Postcondizioni:}
    \begin{itemize}
        \item Vengono visualizzati i dettagli del singolo linguaggio di programmazione
    \end{itemize}
    \item \textbf{Scenario principale:}
    \begin{enumerate}
        \item Il sistema recupera le informazioni relative al singolo linguaggio
        \item Il sistema mostra i dettagli del linguaggio:
        \begin{itemize}
            \item Nome del linguaggio (\hyperref[UC33.2.1.1.1]{UC33.2.1.1.1})
            \item Versione del linguaggio (\hyperref[UC33.2.1.1.2]{UC33.2.1.1.2})
        \end{itemize}
    \end{enumerate}
    \item \textbf{Inclusioni:} \hyperref[UC33.2.1.1.1]{UC33.2.1.1.1}, \hyperref[UC33.2.1.1.2]{UC33.2.1.1.2}
\end{itemize}

\subsubparagraph{UC33.2.1.1.1 - Visualizza nome linguaggio}\label{UC33.2.1.1.1}
\begin{itemize}
    \item \textbf{Attori principali:} Utente
    \item \textbf{Precondizioni:}
    \begin{itemize}
        \item Il sistema è attivo e funzionante
        \item L'utente sta visualizzando il dettaglio di un linguaggio (\hyperref[UC33.2.1.1]{UC33.2.1.1})
    \end{itemize}
    \item \textbf{Postcondizioni:}
    \begin{itemize}
        \item Viene visualizzato il nome del linguaggio di programmazione
    \end{itemize}
    \item \textbf{Scenario principale:}
    \begin{enumerate}
        \item Il sistema recupera il nome identificativo del linguaggio
        \item Il sistema visualizza il nome del linguaggio 
    \end{enumerate}
\end{itemize}

\subsubparagraph{UC33.2.1.1.2 - Visualizza versione linguaggio}\label{UC33.2.1.1.2}
\begin{itemize}
    \item \textbf{Attori principali:} Utente
    \item \textbf{Precondizioni:}
    \begin{itemize}
        \item Il sistema è attivo e funzionante
        \item L'utente sta visualizzando il dettaglio di un linguaggio (\hyperref[UC33.2.1.1]{UC33.2.1.1})
    \end{itemize}
    \item \textbf{Postcondizioni:}
    \begin{itemize}
        \item Viene visualizzata la versione del linguaggio di programmazione
    \end{itemize}
    \item \textbf{Scenario principale:}
    \begin{enumerate}
        \item Il sistema recupera la versione del linguaggio rilevata nella repository
        \item Il sistema visualizza la versione del linguaggio 
    \end{enumerate}
\end{itemize}

\subsubparagraph{UC33.2.1.3 - Nessun linguaggio rilevato}\label{UC33.2.1.3}
\begin{itemize}
    \item \textbf{Attori principali:} Utente
    \item \textbf{Precondizioni:}
    \begin{itemize}
        \item Il sistema è attivo e funzionante
        \item L'utente è stato riconosciuto dal sistema come Utente
        \item L'utente sta visualizzando la sezione delle informazioni tecniche (\hyperref[UC33.2]{UC33.2})
        \item La scansione non ha rilevato alcun linguaggio di programmazione
    \end{itemize}
    \item \textbf{Postcondizioni:}
    \begin{itemize}
        \item Il sistema informa l'utente dell'assenza di linguaggi rilevati
    \end{itemize}
    \item \textbf{Scenario principale:}
    \begin{enumerate}
        \item Il sistema rileva che non sono presenti linguaggi di programmazione identificati nella repository
        \item Il sistema mostra un messaggio informativo: "Nessun linguaggio di programmazione rilevato"
    \end{enumerate}
\end{itemize}

% --- LIBRERIE ---
\subparagraph{UC33.2.2 - Visualizza lista librerie}\label{UC33.2.2}

\begin{figure}[H]
	\centering
	\includegraphics[width=0.8\textwidth]{../Assets/AdR/UC33-2-2ext.png}
	\caption{UC33.2.2 - Visualizza lista librerie}
	\label{fig:UC33-2-2ext}
\end{figure}

\begin{itemize}
    \item \textbf{Attori principali:} Utente
    \item \textbf{Precondizioni:}
    \begin{itemize}
        \item Il sistema è attivo e funzionante
        \item L'utente è stato riconosciuto dal sistema come Utente
        \item L'utente sta visualizzando la sezione delle informazioni tecniche (\hyperref[UC33.2]{UC33.2})
    \end{itemize}
    \item \textbf{Postcondizioni:}
    \begin{itemize}
        \item Viene visualizzata la lista delle librerie rilevate nella repository
    \end{itemize}
    \item \textbf{Scenario principale:}
    \begin{enumerate}
        \item Il sistema recupera l'elenco delle librerie rilevate durante l'ultima scansione
        \item Per ogni libreria presente nella lista, il sistema visualizza il dettaglio della singola libreria (\hyperref[UC33.2.2.1]{UC33.2.2.1})
    \end{enumerate}
    \item \textbf{Scenario alternativo:}
    \begin{itemize}
        \item \textbf{Nessuna libreria rilevata (\hyperref[UC33.2.2.3]{UC33.2.2.3}):}
        \begin{itemize}
            \item \textbf{Condizione:} La scansione non ha rilevato alcuna libreria
            \item \textbf{Flusso:}
            \begin{enumerate}
                \item Il sistema mostra un messaggio informativo: "Nessuna libreria rilevata"
            \end{enumerate}
        \end{itemize}
    \end{itemize}
    \item \textbf{Inclusioni:} \hyperref[UC33.2.2.1]{UC33.2.2.1}
    \item \textbf{Estensioni:} \hyperref[UC33.2.2.3]{UC33.2.2.3}
\end{itemize}

Il caso d'uso UC33.2.2 include ulteriori casi d'uso come rappresentato nella seguente immagine:
\begin{figure}[H]
	\centering
	\includegraphics[width=0.8\textwidth]{../Assets/AdR/UC33-2-2.png}
	\caption{UC33.2.2.1 - Visualizza libreria singola}
	\label{fig:UC33-2-2}
\end{figure}

\subsubparagraph{UC33.2.2.1 - Visualizza libreria singola}\label{UC33.2.2.1}
\begin{itemize}
    \item \textbf{Attori principali:} Utente
    \item \textbf{Precondizioni:}
    \begin{itemize}
        \item Il sistema è attivo e funzionante
        \item L'utente è stato riconosciuto dal sistema come Utente
        \item L'utente sta visualizzando la lista delle librerie (\hyperref[UC33.2.2]{UC33.2.2})
        \item Esiste almeno una libreria rilevata nella repository
    \end{itemize}
    \item \textbf{Postcondizioni:}
    \begin{itemize}
        \item Vengono visualizzati i dettagli della singola libreria
    \end{itemize}
    \item \textbf{Scenario principale:}
    \begin{enumerate}
        \item Il sistema recupera le informazioni relative alla singola libreria
        \item Il sistema mostra i dettagli della libreria:
        \begin{itemize}
            \item Nome della libreria (\hyperref[UC33.2.2.1.1]{UC33.2.2.1.1})
            \item Versione della libreria (\hyperref[UC33.2.2.1.2]{UC33.2.2.1.2})
        \end{itemize}
    \end{enumerate}
    \item \textbf{Inclusioni:} \hyperref[UC33.2.2.1.1]{UC33.2.2.1.1}, \hyperref[UC33.2.2.1.2]{UC33.2.2.1.2}
\end{itemize}

\subsubparagraph{UC33.2.2.1.1 - Visualizza nome libreria}\label{UC33.2.2.1.1}
\begin{itemize}
    \item \textbf{Attori principali:} Utente
    \item \textbf{Precondizioni:}
    \begin{itemize}
        \item Il sistema è attivo e funzionante
        \item L'utente sta visualizzando il dettaglio di una libreria (\hyperref[UC33.2.2.1]{UC33.2.2.1})
    \end{itemize}
    \item \textbf{Postcondizioni:}
    \begin{itemize}
        \item Viene visualizzato il nome della libreria
    \end{itemize}
    \item \textbf{Scenario principale:}
    \begin{enumerate}
        \item Il sistema recupera il nome identificativo della libreria
        \item Il sistema visualizza il nome della libreria 
    \end{enumerate}
\end{itemize}

\subsubparagraph{UC33.2.2.1.2 - Visualizza versione libreria}\label{UC33.2.2.1.2}
\begin{itemize}
    \item \textbf{Attori principali:} Utente
    \item \textbf{Precondizioni:}
    \begin{itemize}
        \item Il sistema è attivo e funzionante
        \item L'utente sta visualizzando il dettaglio di una libreria (\hyperref[UC33.2.2.1]{UC33.2.2.1})
    \end{itemize}
    \item \textbf{Postcondizioni:}
    \begin{itemize}
        \item Viene visualizzata la versione della libreria
    \end{itemize}
    \item \textbf{Scenario principale:}
    \begin{enumerate}
        \item Il sistema recupera la versione della libreria rilevata nella repository
        \item Il sistema visualizza la versione della libreria 
    \end{enumerate}
\end{itemize}

\subsubparagraph{UC33.2.2.3 - Nessuna libreria rilevata}\label{UC33.2.2.3}
\begin{itemize}
    \item \textbf{Attori principali:} Utente
    \item \textbf{Precondizioni:}
    \begin{itemize}
        \item Il sistema è attivo e funzionante
        \item L'utente è stato riconosciuto dal sistema come Utente
        \item L'utente sta visualizzando la sezione delle informazioni tecniche (\hyperref[UC33.2]{UC33.2})
        \item La scansione non ha rilevato alcuna libreria
    \end{itemize}
    \item \textbf{Postcondizioni:}
    \begin{itemize}
        \item Il sistema informa l'utente dell'assenza di librerie rilevate
    \end{itemize}
    \item \textbf{Scenario principale:}
    \begin{enumerate}
        \item Il sistema rileva che non sono presenti librerie identificate nella repository
        \item Il sistema mostra un messaggio informativo: "Nessuna libreria rilevata"
    \end{enumerate}
\end{itemize}

% --- FRAMEWORK ---
\subparagraph{UC33.2.3 - Visualizza lista framework}\label{UC33.2.3}

\begin{figure}[H]
	\centering
	\includegraphics[width=0.8\textwidth]{../Assets/AdR/UC33-2-3ext.png}
	\caption{UC33.2.3 - Visualizza lista framework}
	\label{fig:UC33-2-3ext}
\end{figure}

\begin{itemize}
    \item \textbf{Attori principali:} Utente
    \item \textbf{Precondizioni:}
    \begin{itemize}
        \item Il sistema è attivo e funzionante
        \item L'utente è stato riconosciuto dal sistema come Utente
        \item L'utente sta visualizzando la sezione delle informazioni tecniche (\hyperref[UC33.2]{UC33.2})
    \end{itemize}
    \item \textbf{Postcondizioni:}
    \begin{itemize}
        \item Viene visualizzata la lista dei framework rilevati nella repository
    \end{itemize}
    \item \textbf{Scenario principale:}
    \begin{enumerate}
        \item Il sistema recupera l'elenco dei framework rilevati durante l'ultima scansione
        \item Per ogni framework presente nella lista, il sistema visualizza il dettaglio del singolo framework (\hyperref[UC33.2.3.1]{UC33.2.3.1})
    \end{enumerate}
    \item \textbf{Scenario alternativo:}
    \begin{itemize}
        \item \textbf{Nessun framework rilevato (\hyperref[UC33.2.3.3]{UC33.2.3.3}):}
        \begin{itemize}
            \item \textbf{Condizione:} La scansione non ha rilevato alcun framework
            \item \textbf{Flusso:}
            \begin{enumerate}
                \item Il sistema mostra un messaggio informativo: "Nessun framework rilevato"
            \end{enumerate}
        \end{itemize}
    \end{itemize}
    \item \textbf{Inclusioni:} \hyperref[UC33.2.3.1]{UC33.2.3.1}
    \item \textbf{Estensioni:} \hyperref[UC33.2.3.3]{UC33.2.3.3}
\end{itemize}

Il caso d'uso UC33.2.3 include ulteriori casi d'uso come rappresentato nella seguente immagine:
\begin{figure}[H]
	\centering
	\includegraphics[width=0.8\textwidth]{../Assets/AdR/UC33-2-3.png}
	\caption{UC33.2.3.1 - Visualizza framework singolo}
	\label{fig:UC33-2-3}
\end{figure}

\subsubparagraph{UC33.2.3.1 - Visualizza framework singolo}\label{UC33.2.3.1}
\begin{itemize}
    \item \textbf{Attori principali:} Utente
    \item \textbf{Precondizioni:}
    \begin{itemize}
        \item Il sistema è attivo e funzionante
        \item L'utente è stato riconosciuto dal sistema come Utente
        \item L'utente sta visualizzando la lista dei framework (\hyperref[UC33.2.3]{UC33.2.3})
        \item Esiste almeno un framework rilevato nella repository
    \end{itemize}
    \item \textbf{Postcondizioni:}
    \begin{itemize}
        \item Vengono visualizzati i dettagli del singolo framework
    \end{itemize}
    \item \textbf{Scenario principale:}
    \begin{enumerate}
        \item Il sistema recupera le informazioni relative al singolo framework
        \item Il sistema mostra i dettagli del framework:
        \begin{itemize}
            \item Nome del framework (\hyperref[UC33.2.3.1.1]{UC33.2.3.1.1})
            \item Versione del framework (\hyperref[UC33.2.3.1.2]{UC33.2.3.1.2})
        \end{itemize}
    \end{enumerate}
    \item \textbf{Inclusioni:} \hyperref[UC33.2.3.1.1]{UC33.2.3.1.1}, \hyperref[UC33.2.3.1.2]{UC33.2.3.1.2}
\end{itemize}

\subsubparagraph{UC33.2.3.1.1 - Visualizza nome framework}\label{UC33.2.3.1.1}
\begin{itemize}
    \item \textbf{Attori principali:} Utente
    \item \textbf{Precondizioni:}
    \begin{itemize}
        \item Il sistema è attivo e funzionante
        \item L'utente sta visualizzando il dettaglio di un framework (\hyperref[UC33.2.3.1]{UC33.2.3.1})
    \end{itemize}
    \item \textbf{Postcondizioni:}
    \begin{itemize}
        \item Viene visualizzato il nome del framework
    \end{itemize}
    \item \textbf{Scenario principale:}
    \begin{enumerate}
        \item Il sistema recupera il nome identificativo del framework
        \item Il sistema visualizza il nome del framework
    \end{enumerate}
\end{itemize}

\subsubparagraph{UC33.2.3.1.2 - Visualizza versione framework}\label{UC33.2.3.1.2}
\begin{itemize}
    \item \textbf{Attori principali:} Utente
    \item \textbf{Precondizioni:}
    \begin{itemize}
        \item Il sistema è attivo e funzionante
        \item L'utente sta visualizzando il dettaglio di un framework (\hyperref[UC33.2.3.1]{UC33.2.3.1})
    \end{itemize}
    \item \textbf{Postcondizioni:}
    \begin{itemize}
        \item Viene visualizzata la versione del framework
    \end{itemize}
    \item \textbf{Scenario principale:}
    \begin{enumerate}
        \item Il sistema recupera la versione del framework rilevata nella repository
        \item Il sistema visualizza la versione del framework 
    \end{enumerate}

\end{itemize}

\subsubparagraph{UC33.2.3.3 - Nessun framework rilevato}\label{UC33.2.3.3}
\begin{itemize}
    \item \textbf{Attori principali:} Utente
    \item \textbf{Precondizioni:}
    \begin{itemize}
        \item Il sistema è attivo e funzionante
        \item L'utente è stato riconosciuto dal sistema come Utente
        \item L'utente sta visualizzando la sezione delle informazioni tecniche (\hyperref[UC33.2]{UC33.2})
        \item La scansione non ha rilevato alcun framework
    \end{itemize}
    \item \textbf{Postcondizioni:}
    \begin{itemize}
        \item Il sistema informa l'utente dell'assenza di framework rilevati
    \end{itemize}
    \item \textbf{Scenario principale:}
    \begin{enumerate}
        \item Il sistema rileva che non sono presenti framework identificati nella repository
        \item Il sistema mostra un messaggio informativo: "Nessun framework rilevato"
    \end{enumerate}
\end{itemize}





\paragraph{UC33.3 - Visualizzazione analisi sicurezza semplice}\label{UC33.3}

\begin{figure}[H]
	\centering
	\includegraphics[width=0.6\textwidth]{../Assets/AdR/UC33-3.png}
	\caption{UC33.3 - Visualizzazione analisi sicurezza semplice}
	\label{fig:UC33-3}
\end{figure}

\begin{itemize}
	\item \textbf{Attori principali:} Utente
	\item \textbf{Precondizioni:} 
    \begin{itemize}
        \item L'utente è autenticato presso il sistema
        \item L’utente ha selezionato una specifica repository
        \item L’utente ha il permesso di visualizzare la sezione OWASP semplice all’interno del workspace di cui fa parte la repository selezionata
        \item L’utente ha selezionato uno specifico branch della repository da visualizzare
    \end{itemize}
	\item \textbf{Postcondizioni:} 
    \begin{itemize}
    \item L’utente ha una visione dell’analisi OWASP semplificata dello stato di sicurezza della repository
    \end{itemize}
	\item \textbf{Scenario principale:}
	\begin{enumerate}
		\item L’utente seleziona una repository appartenente ad un workspace a cui ha accesso
		\item Il sistema recupera le informazioni relative alle vulnerabilità OWASP (Top 10 e altre)
        \item Il sistema visualizza un istogramma con il numero di vulnerabilità suddivise per livello di criticità (alta, media, bassa)
	\end{enumerate}
    \item \textbf{Scenario alternativo:}
	\begin{enumerate}
		\item Analisi OWASP non disponibile: se la repository non presenta file utili all'analisi, il sistema informa l'utente con il messaggio: "Nessun file utile per l'analisi OWASP"
		\item Analisi OWASP non eseguibile: in caso di errore del motore di analisi o formato dei file non supportato, il sistema informa l'utente con l'errore: "Errore nell'esecuzione dell'analisi OWASP"
		\item Nessuna vulnerabilità riscontrata: se l'analisi termina correttamente ma non rileva minacce, il sistema mostra la percentuale di test superati e il messaggio: "Nessuna vulnerabilità rilevata"   
	\end{enumerate}
\end{itemize}

\paragraph{UC33.4 - Visualizzazione analisi sicurezza completa}\label{UC33.4}

\begin{figure}[H]
	\centering
	\includegraphics[width=0.9\textwidth]{../Assets/AdR/UC33-4.png}
	\caption{UC33.4 - Visualizzazione analisi sicurezza completa}
	\label{fig:UC33-4}
\end{figure}

\begin{itemize}
    \item \textbf{Attori principali:} 
    \begin{itemize}
        \item Tech Lead
        \item Developer
    \end{itemize}
    \item \textbf{Precondizioni:} 
    \begin{itemize}
        \item Il sistema è attivo e funzionante
        \item L'utente è autenticato presso il sistema
        \item L'utente ha selezionato una specifica repository e un relativo branch
    \end{itemize}
    \item \textbf{Postcondizioni:}
    \begin{itemize}
    \item L'utente ottiene una visione dettagliata dell'analisi di sicurezza e delle eventuali vulnerabilità rilevate nella repository
    \end{itemize}
    \item \textbf{Scenario principale:}
    \begin{enumerate}
        \item L'utente seleziona la sezione relativa all'analisi OWASP completa
        \item Il sistema recupera le informazioni sulle top 10 criticità OWASP
        \item Il sistema indica l'eventuale lista delle vulnerabilità riscontrate nella repository (\hyperref[UC33.4.1]{UC33.4.1})
        \item L'utente può selezionare una singola vulnerabilità dalla lista per consultarne i dettagli e le possibili remediation (\hyperref[UC33.4.2]{UC33.4.2})
    \end{enumerate}
    \item \textbf{Scenario alternativo:}
	\begin{enumerate}
		\item Analisi OWASP non disponibile: se la repository non presenta file utili all'analisi, il sistema informa l'utente con il messaggio: "Nessun file utile per l'analisi OWASP"
		\item Analisi OWASP non eseguibile: in caso di errore del motore di analisi o formato dei file non supportato, il sistema informa l'utente con l'errore: "Errore nell'esecuzione dell'analisi OWASP"
		\item Nessuna vulnerabilità riscontrata: se l'analisi termina correttamente ma non rileva minacce, il sistema mostra la percentuale di test superati e il messaggio: "Nessuna vulnerabilità rilevata"
	\end{enumerate}
    \item \textbf{Inclusioni:} \hyperref[UC33.4.1]{UC33.4.1}, \hyperref[UC33.4.2]{UC33.4.2}
\end{itemize}

\subparagraph{UC33.4.1 - Visualizzazione elenco vulnerabilità riscontrate}\label{UC33.4.1}
\begin{itemize}
    \item \textbf{Attori principali:}
    \begin{itemize}
        \item Tech Lead
        \item Developer
    \end{itemize}
    \item \textbf{Precondizioni:} 
    \begin{itemize}
        \item L'utente è autenticato presso il sistema
        \item L’utente ha selezionato una specifica repository
        \item Il sistema ha individuato una serie di vulnerabilità nella repository scansionata
    \end{itemize}
    \item \textbf{Postcondizioni:} 
    \begin{itemize}
        \item L'utente visualizza la lista ordinata di tutte le vulnerabilità identificate e le categorie di criticità OWASP di appartenenza
    \end{itemize}
    \item \textbf{Scenario principale:}
    \begin{enumerate}
        \item Il sistema recupera i dati riguardanti le vulnerabilità della repository
        \item Il sistema visualizza un elenco ordinato in modo decrescente per gravità         
        \item Per ciascun elemento della lista, il sistema visualizza le informazioni sintetiche (\hyperref[UC33.4.1.1]{UC33.4.1.1})
    \end{enumerate}
    \item \textbf{Scenari alternativi:}
    \begin{itemize}
        \item Nessuna vulnerabilità rilevata: il sistema mostra il messaggio: ``Nessuna vulnerabilità rilevata''
    \end{itemize}
    \item \textbf{Inclusioni:} \hyperref[UC33.4.1.1]{UC33.4.1.1}
\end{itemize}

\subparagraph{UC33.4.1.1 - Visualizzazione singolo elemento della lista di vulnerabilità}\label{UC33.4.1.1}
\begin{itemize}
    \item \textbf{Attori principali:} 
    \begin{itemize}
        \item Tech Lead
        \item Developer
    \end{itemize}
    \item \textbf{Precondizioni:} 
    \begin{itemize}
        \item L'utente è autenticato presso il sistema
        \item L’utente ha selezionato una specifica repository
        \item L'utente sta consultando l'elenco delle vulnerabilità
    \end{itemize}
    \item \textbf{Postcondizioni:}
    \begin{itemize}
        \item L'utente visualizza i dati identificativi di una specifica vulnerabilità
    \end{itemize}
    \item \textbf{Scenario principale:}
    \begin{enumerate}
        \item Per la singola vulnerabilità, il sistema mostra: 
            \begin{itemize}
                \item nome/tipo della vulnerabilità
                \item il file di origine
                \item categoria di criticità OWASP di appartenenza
            \end{itemize}
        \item L'utente può scegliere di approfondire la singola vulnerabilità per accedere alla visione completa
    \end{enumerate}
\end{itemize}

\subparagraph{UC33.4.2 - Approfondimento criticità}\label{UC33.4.2}
\begin{itemize}
    \item \textbf{Attori principali:} 
    \begin{itemize}
        \item Tech Lead
        \item Developer
    \end{itemize}
    \item \textbf{Precondizioni:} 
    \begin{itemize}
        \item L'utente è autenticato presso il sistema
        \item L’utente ha selezionato una specifica repository
        \item L'utente ha selezionato una specifica vulnerabilità dall'elenco delle vulnerabilità
    \end{itemize}
    \item \textbf{Postcondizioni:} 
    \begin{itemize}
        \item L’utente visualizza i dettagli completi della vulnerabilità selezionata
        \item Il sistema fornisce, se disponibile, una proposta di remediation
    \end{itemize}
    \item \textbf{Scenario principale:}
    \begin{enumerate}
        \item Il sistema mostra il livello di gravità e una breve descrizione tecnica del tipo di vulnerabilità
        \item Il sistema mostra il file coinvolto e il frammento di codice che genera la vulnerabilità
        \item Il sistema fornisce un suggerimento testuale per la risoluzione del problema
        \item Il sistema mostra una possibile remediation (codice alternativo)
    \end{enumerate}
    \item \textbf{Scenari alternativi:}
    \begin{itemize}
        \item Codice non individuato: se il sistema non individua il frammento, mostra: ``Codice sorgente non disponibile per questa vulnerabilità''
        \item Suggerimento non disponibile: se manca la descrizione risolutiva, mostra: ``Suggerimento non disponibile''
        \item Remediation non disponibile: se non è generabile codice alternativo, mostra: ``Remediation non disponibile''
    \end{itemize}
\end{itemize}

\paragraph{UC33.5 - Visualizzazione analisi documentazione di una repository} \label{UC33.5}

\begin{figure}[H]
	\centering
	\includegraphics[width=0.9\textwidth]{../Assets/AdR/UC33-5.png}
	\caption{UC33.5 - Visualizzazione analisi documentazione di una repository}
	\label{fig:UC33-5}
\end{figure}

\begin{itemize}
    \item \textbf{Attori principali:} Utente 
    \item \textbf{Precondizioni:}
    \begin{itemize}
        \item Il sistema è attivo e funzionante
        \item È stata effettuata un'analisi della documentazione su una specifica repository
    \end{itemize}
    \item \textbf{Postcondizioni:}
    \begin{itemize}
        \item L'utente visualizza la valutazione della qualità della documentazione e i consigli per migliorarla
    \end{itemize}
    \item \textbf{Scenario principale:}
    \begin{enumerate}
        \item Il sistema mostra il voto complessivo assegnato alla documentazione (\hyperref[UC33.5.1]{UC33.5.1})
        \item Il sistema fornisce consigli specifici per il miglioramento del file README (\hyperref[UC33.5.2]{UC33.5.2})
        \item Il sistema elenca le sezioni mancanti o incomplete (\hyperref[UC33.5.3]{UC33.5.3})
    \end{enumerate}
    \item \textbf{Inclusioni:} \hyperref[UC33.5.1]{UC33.5.1}, \hyperref[UC33.5.2]{UC33.5.2}, \hyperref[UC33.5.3]{UC33.5.3}
\end{itemize}

\subparagraph{UC33.5.1 - Visualizzazione voto documentazione}\label{UC33.5.1}
\begin{itemize}
    \item \textbf{Attori principali:} Utente 
    \item \textbf{Precondizioni:} 
    \begin{itemize}
        \item Il sistema è attivo e funzionante
        \item L'utente sta visualizzando la sezione documentazione di una specifica repository
    \end{itemize}
    \item \textbf{Postcondizioni:} 
    \begin{itemize}
        \item Viene visualizzato un voto sintetico sulla qualità documentale
    \end{itemize}
    \item \textbf{Scenario principale:}
    \begin{itemize}
        \item Il sistema mostra un voto numerico che rappresenta la completezza e la qualità della documentazione presente nella repository, calcolato in base alla presenza di file README, documentazione tecnica e commenti nel codice
    \end{itemize}
    \item \textbf{Scenario alternativo:}
    \begin{itemize}
        \item La repository non contiene documentazione analizzabile, il sistema mostra "Documentazione assente"
        \item L'analisi della documentazione non è stata eseguita, il sistema mostra "Analisi documentazione non disponibile"
    \end{itemize}
\end{itemize}

\subparagraph{UC33.5.2 - Visualizzazione consigli sul README}\label{UC33.5.2}
\begin{itemize}
    \item \textbf{Attori principali:} Utente 
    \item \textbf{Precondizioni:} 
    \begin{itemize}
        \item Il sistema è attivo e funzionante
        \item Il file README è presente nella repository
        \item L'utente sta visualizzando la sezione documentazione di una specifica repository
    \end{itemize}
    \item \textbf{Postcondizioni:} 
    \begin{itemize}
        \item Vengono mostrati suggerimenti testuali per il miglioramento del README
    \end{itemize}
    \item \textbf{Scenario principale:}
    \begin{enumerate}
        \item Il sistema analizza il contenuto del README della repository
        \item Il sistema suggerisce l'aggiunta di sezioni utili per arricchirne il contenuto  
    \end{enumerate}
\end{itemize}

\subparagraph{UC33.5.3 - Visualizza sezioni mancanti}\label{UC33.5.3}
\begin{itemize}
    \item \textbf{Attori principali:} Utente 
    \item \textbf{Precondizioni:} 
    \begin{itemize}
        \item Il sistema è attivo e funzionante
        \item L'analisi ha rilevato l'assenza di documenti standard
    \end{itemize}
    \item \textbf{Postcondizioni:} 
    \begin{itemize}
        \item Viene mostrata una lista dei documenti assenti
    \end{itemize}
    \item \textbf{Scenario principale:}
    \begin{enumerate}
        \item Il sistema verifica la presenza di documenti standard
        \item Il sistema elenca le sezioni o i file documentali che non sono stati trovati nella repository 
    \end{enumerate}
\end{itemize}

% ============================================
% UC33.5 - VISUALIZZA QUALITÀ DEL CODICE
% ============================================
\paragraph{UC33.6 - Visualizzazione qualità del codice di una repository} \label{UC33.6}

\begin{figure}[H]
	\centering
	\includegraphics[width=0.8\textwidth]{../Assets/AdR/UC33-6.png}
	\caption{UC33.6 e UC33.6.1}
	\label{fig:UC33-6}
\end{figure}

\begin{itemize}
    \item \textbf{Attori principali:} Utente 
    \item \textbf{Precondizioni:}
    \begin{itemize}
        \item Il sistema è attivo e funzionante
        \item È stata effettuata un'analisi statica del codice
    \end{itemize}
    \item \textbf{Postcondizioni:}
    \begin{itemize}
        \item L'utente riceve suggerimenti per migliorare la qualità del codice sorgente
    \end{itemize}
    \item \textbf{Scenario principale:}
    \begin{enumerate}
        \item Il sistema recupera i risultati dell'analisi sulla qualità del codice
        \item Il sistema mostra una lista di suggerimenti basati sulle best practices di programmazione (\hyperref[UC33.6.1]{UC33.6.1})
    \end{enumerate}
    \item \textbf{Inclusioni:} \hyperref[UC33.6.1]{UC33.6.1}
\end{itemize}

\subparagraph{UC33.6.1 - Visualizzazione lista suggerimenti sulla qualità del codice}\label{UC33.6.1}
\begin{itemize}
    \item \textbf{Attori principali:} Utente 
    \item \textbf{Precondizioni:} 
    \begin{itemize}
        \item Il sistema è attivo e funzionante
        \item Sono state rilevate violazioni di best practices o code smells
    \end{itemize}
    \item \textbf{Postcondizioni:} 
    \begin{itemize}
        \item Viene mostrato l'elenco dei suggerimenti tecnici per migliorare il codice
    \end{itemize}
    \item \textbf{Scenario principale:}
    \begin{enumerate}
        \item Il sistema presenta suggerimenti pratici per il refactoring del codice 
        \item L'utente può consultare i dettagli per applicare le best practices suggerite 
    \end{enumerate}
\end{itemize}

\subsubsection{UC34 - Nessuna scansione effettuata}\label{UC34}

\begin{figure}[h]
	\centering
	\includegraphics[width=0.7\textwidth]{../Assets/AdR/UC34.png}
	\caption{UC34 - Nessuna scansione effettuata}
	\label{fig:UC34}
\end{figure}

\begin{itemize}
    \item \textbf{Attori principali:} Utente 
    \item \textbf{Precondizioni:}
    \begin{itemize}
        \item L'utente tenta di visualizzare i dettagli di una repository
        \item Non è mai stata effettuata una scansione sulla repository selezionata, oppure l'ultima scansione non ha prodotto dati validi 
    \end{itemize}
    \item \textbf{Postcondizioni:}
    \begin{itemize}
        \item Viene mostrato un messaggio informativo che notifica l'assenza di dati
    \end{itemize}
    \item \textbf{Scenario principale:}
    \begin{enumerate}
        \item Il sistema verifica lo stato delle analisi per la repository selezionata
        \item Il sistema rileva che non esistono dati di scansione disponibili
        \item Il sistema mostra il messaggio: "Nessuna scansione effettuata" o "Dati non disponibili" 
        \item Il sistema può proporre all'utente di avviare una nuova analisi
    \end{enumerate}
\end{itemize}

% ============================================
% UC35 - SELEZIONE BRANCH
% ============================================
\subsubsection{UC35 - Selezione branch}\label{UC35}
\begin{figure}[H]
	\centering
	\includegraphics[width=0.8\textwidth]{../Assets/AdR/UC35ext.png}
	\caption{UC35 - Selezione branch}
	\label{fig:UC35ext}
\end{figure}

\begin{itemize}
    \item \textbf{Attori principali:} Utente
    \item \textbf{Precondizioni:}
    \begin{itemize}
        \item Il sistema è attivo e funzionante
        \item L'utente è autenticato nel sistema
        \item L'utente deve selezionare un branch di una determinata repository
    \end{itemize}
    \item \textbf{Postcondizioni:}
    \begin{itemize}
        \item Il sistema ha associato un branch valido su una determinata repository
    \end{itemize}
    \item \textbf{Scenario principale:}
    \begin{enumerate}
        \item Il sistema mostra la lista dei branch di una determinata repository (\hyperref[UC35.1]{UC35.1})
        \item L'utente seleziona il branch desiderato
        \item Il sistema memorizza le informazioni relative al branch selezionato
    \end{enumerate}
    \item \textbf{Scenario alternativo:}
    \begin{itemize}
        \item \textbf{Selezione branch non riuscita (\hyperref[UC35.2]{UC35.2}):} L'utente non seleziona nessun branch valido
    \end{itemize}
    \item \textbf{Inclusioni:} \hyperref[UC35.1]{UC35.1}
    \item \textbf{Estensioni:} \hyperref[UC35.2]{UC35.2}
\end{itemize}

% --- UC35.1 - VISUALIZZA LISTA BRANCH ---
\paragraph{UC35.1 - Visualizza lista branch}\label{UC35.1}
\begin{figure}[H]
	\centering
	\includegraphics[width=0.7\textwidth]{../Assets/AdR/UC35-1.png}
	\caption{Inclusioni di UC35.1: UC35.1.1}
	\label{fig:UC35-1}
\end{figure}

\begin{itemize}
    \item \textbf{Attori principali:} Utente 
    \item \textbf{Precondizioni:}
    \begin{itemize}
        \item Il sistema è attivo e funzionante
        \item L'utente sta eseguendo UC Selezione branch (\hyperref[UC35]{UC35})
    \end{itemize}
    \item \textbf{Postcondizioni:}
    \begin{itemize}
        \item L'utente visualizza una lista di branch di una determinata repository
    \end{itemize}
    \item \textbf{Scenario principale:}
    \begin{enumerate}
        \item L'utente deve selezionare un branch di una repository
        \item Il sistema recupera la lista dei branch disponibili per la repository
        \item Per ogni branch presente nella lista, il sistema visualizza il dettaglio del singolo branch (\hyperref[UC35.1.1]{UC35.1.1})
    \end{enumerate}
    \item \textbf{Inclusioni:} \hyperref[UC35.1.1]{UC35.1.1}
\end{itemize}

% --- UC35.1.1 - VISUALIZZA SINGOLO BRANCH ---
\subparagraph{UC35.1.1 - Visualizza singolo branch}\label{UC35.1.1}
\begin{itemize}
    \item \textbf{Attori principali:} Utente 
    \item \textbf{Precondizioni:}
    \begin{itemize}
        \item Il sistema è attivo e funzionante
        \item L'utente è autenticato nel sistema
        \item L'utente sta visualizzando la lista dei branch (\hyperref[UC35.1]{UC35.1})
        \item Esiste almeno un branch nella repository
    \end{itemize}
    \item \textbf{Postcondizioni:}
    \begin{itemize}
        \item Vengono visualizzati i dettagli del singolo branch
    \end{itemize}
    \item \textbf{Scenario principale:}
    \begin{enumerate}
        \item Il sistema recupera le informazioni relative al singolo branch
        \item Il sistema mostra i dettagli del branch:
        \begin{itemize}
            \item Nome del branch (\hyperref[UC35.1.1.1]{UC35.1.1.1})
            \item Data ultima scansione (\hyperref[UC35.1.1.2]{UC35.1.1.2})
        \end{itemize}
    \end{enumerate}
    \item \textbf{Inclusioni:} \hyperref[UC35.1.1.1]{UC35.1.1.1}, \hyperref[UC35.1.1.2]{UC35.1.1.2}
\end{itemize}

\subsubparagraph{UC35.1.1.1 - Visualizza nome branch}\label{UC35.1.1.1}
\begin{itemize}
    \item \textbf{Attori principali:} Utente 
    \item \textbf{Precondizioni:}
    \begin{itemize}
        \item Il sistema è attivo e funzionante
        \item L'utente sta visualizzando il dettaglio di un branch (\hyperref[UC35.1.1]{UC35.1.1})
    \end{itemize}
    \item \textbf{Postcondizioni:}
    \begin{itemize}
        \item Viene visualizzato il nome del branch
    \end{itemize}
    \item \textbf{Scenario principale:}
    \begin{enumerate}
        \item Il sistema recupera il nome identificativo del branch
        \item Il sistema visualizza il nome del branch
    \end{enumerate}
\end{itemize}

\subsubparagraph{UC35.1.1.2 - Visualizza data ultima scansione}\label{UC35.1.1.2}
\begin{itemize}
    \item \textbf{Attori principali:} Utente 
    \item \textbf{Precondizioni:}
    \begin{itemize}
        \item Il sistema è attivo e funzionante
        \item L'utente sta visualizzando il dettaglio di un branch (\hyperref[UC35.1.1]{UC35.1.1})
    \end{itemize}
    \item \textbf{Postcondizioni:}
    \begin{itemize}
        \item Viene visualizzata la data dell'ultima scansione effettuata sul branch
    \end{itemize}
    \item \textbf{Scenario principale:}
    \begin{enumerate}
        \item Il sistema recupera la data dell'ultima scansione eseguita sul branch
        \item Il sistema visualizza la data dell'ultima scansione se esistente
    \end{enumerate}

\end{itemize}

% --- UC35.2 - SELEZIONE BRANCH NON RIUSCITA ---
\paragraph{UC35.2 - Selezione branch non riuscita}\label{UC35.2}
\begin{itemize}
    \item \textbf{Attori principali:} Utente
    \item \textbf{Precondizioni:}
    \begin{itemize}
        \item Il sistema è attivo e funzionante
        \item Il sistema sta eseguendo UC Selezione branch (\hyperref[UC35]{UC35})
        \item Durante il recupero o la validazione dei branch non viene trovato alcun branch utilizzabile
    \end{itemize}
    \item \textbf{Postcondizioni:}
    \begin{itemize}
        \item Nessun branch viene associato alla repository
    \end{itemize}
    \item \textbf{Scenario principale:}
    \begin{enumerate}
        \item Il sistema tenta di recuperare i branch disponibili per la repository
        \item Il sistema non trova branch validi
        \item Il sistema notifica l'errore di selezione branch all'utente e annulla il processo in corso
    \end{enumerate}
\end{itemize}


% ============================================
% UC36 - VISIONE AGGREGATA
% ============================================
\subsubsection{UC36 - Visione aggregata}\label{UC36}
\begin{figure}[H]
	\centering
	\includegraphics[width=0.9\textwidth]{../Assets/AdR/UC36ext.png}
	\caption{UC36 - Visione aggregata}
	\label{fig:UC36ext}
\end{figure}

\begin{itemize}
    \item \textbf{Attori principali:} Utente
    \item \textbf{Precondizioni:}
    \begin{itemize}
        \item Il sistema è attivo e funzionante
        \item L'utente è autenticato presso il sistema
        \item L'utente ha accesso a un workspace ed è entrato in uno specifico workspace
        \item Ogni repository del workspace ha un branch di default valido
    \end{itemize}
    \item \textbf{Postcondizioni:}
    \begin{itemize}
        \item L'utente visualizza una vista sintetica e aggregata delle informazioni relative all'insieme di repository che fanno parte del workspace
    \end{itemize}
    \item \textbf{Scenario principale:}
    \begin{enumerate}
        \item L'utente richiede la visualizzazione aggregata delle repository del workspace
        \item Il sistema verifica la presenza di un branch di default per ciascuna repository
        \item Il sistema recupera i dati di analisi relativi alle repository del workspace
        \item Il sistema calcola le metriche aggregate (medie, minimi, massimi, distribuzioni)
        \item Il sistema visualizza la sezione test aggregata (\hyperref[UC36.2]{UC36.2})
        \item Il sistema visualizza la sezione informazioni tecniche aggregata (\hyperref[UC36.3]{UC36.3})
        \item Il sistema visualizza la sezione sicurezza aggregata (\hyperref[UC36.4]{UC36.4})
        \item Il sistema visualizza la sezione documentazione aggregata (\hyperref[UC36.5]{UC36.5})
    \end{enumerate}
    \item \textbf{Scenario alternativo:}
    \begin{itemize}
        \item \textbf{Filtra per tag (\hyperref[UC36.1]{UC36.1}):}
        \begin{itemize}
            \item \textbf{Condizione:} L'utente vuole filtrare le repository per tag 
            \item \textbf{Flusso:}
            \begin{enumerate}
                \item L'utente seleziona uno o più tag 
                \item Il sistema filtra le repository in base ai tag selezionati
                \item Il sistema aggiorna la visione aggregata con le repository filtrate
            \end{enumerate}
        \end{itemize}
        \item \textbf{Errore Visione Aggregata (\hyperref[UC36.6]{UC36.6}):}
        \begin{itemize}
            \item \textbf{Condizione:} Si è verificato un errore nella visione aggregata, come workspace vuoto o nessuna repository selezionata
            \item \textbf{Flusso:}
            \begin{enumerate}
                \item Il sistema mostra un messaggio di errore appropriato
            \end{enumerate}
        \end{itemize}
        \item \textbf{Scansione Workspace (\hyperref[UC15.1]{UC15.1}):}
        \begin{itemize}
            \item \textbf{Condizione:} L'utente vuole lanciare una scansione sul workspace visualizzato
            \item \textbf{Flusso:}
            \begin{enumerate}
                \item L'utente avvia la scansione del workspace
            \end{enumerate}
            \end{itemize}
		\item \textbf{Aggiorna visualizzazione (\hyperref[UC37]{UC37}):}
        \begin{itemize}
            \item \textbf{Condizione:} L'utente seleziona “Aggiorna”
            \item \textbf{Flusso:}
            \begin{enumerate}
                \item Il sistema controlla se ci sono scansioni più recenti
                \item Il sistema aggiorna i dati visualizzati con quelli dell'ultima scansione
            \end{enumerate}
        \end{itemize}
    \end{itemize}
    \item \textbf{Inclusioni:} \hyperref[UC36.2]{UC36.2}, \hyperref[UC36.3]{UC36.3}, \hyperref[UC36.4]{UC36.4}, \hyperref[UC36.5]{UC36.5}
    \item \textbf{Estensioni:} \hyperref[UC36.1]{UC36.1}, \hyperref[UC36.6]{UC36.6}, \hyperref[UC15.1]{UC15.1}, \hyperref[UC37]{UC37} 
\end{itemize}

Il caso d'uso UC36 include ulteriori casi d'uso come rappresentato nella seguente immagine:
\begin{figure}[H]
	\centering
	\includegraphics[width=0.7\textwidth]{../Assets/AdR/UC36.png}
	\caption{Inclusioni di UC36: UC36.2, UC36.3, UC36.4, UC36.5}
	\label{fig:UC36inc}
\end{figure}


% --- UC36.1 - FILTRA PER TAG ---
\paragraph{UC36.1 - Filtra per tag}\label{UC36.1}
\begin{itemize}
    \item \textbf{Attori principali:} Utente
    \item \textbf{Precondizioni:}
    \begin{itemize}
        \item Il sistema è attivo e funzionante
        \item L'utente è autenticato presso il sistema
        \item L'utente sta visualizzando la visione aggregata (\hyperref[UC36]{UC36})
        \item Esiste almeno un tag raccolta definito nel workspace
    \end{itemize}
    \item \textbf{Postcondizioni:}
    \begin{itemize}
        \item La visione aggregata viene aggiornata mostrando solo le repository associate ai tag selezionati
    \end{itemize}
    \item \textbf{Scenario principale:}
    \begin{enumerate}
        \item L'utente seleziona l'opzione per filtrare per tag
        \item Il sistema mostra la lista dei tag raccolta disponibili nel workspace
        \item L'utente seleziona uno o più tag
        \item Il sistema filtra le repository in base ai tag selezionati
        \item Il sistema ricalcola e aggiorna la visione aggregata
    \end{enumerate}
\end{itemize}


% ============================================
% UC36.2 - SEZIONE TEST AGGREGATA
% ============================================
\paragraph{UC36.2 - Sezione Test aggregata}\label{UC36.2}
\begin{figure}[H]
	\centering
	\includegraphics[width=0.7\textwidth]{../Assets/AdR/UC36-2.png}
	\caption{Inclusioni di UC36.2: UC36.2.1, UC36.2.2, UC36.2.3}
	\label{fig:UC36-2}
\end{figure}

\begin{itemize}
    \item \textbf{Attori principali:} Utente
    \item \textbf{Precondizioni:}
    \begin{itemize}
        \item Il sistema è attivo e funzionante
        \item L'utente è autenticato presso il sistema
        \item L'utente sta visualizzando la visione aggregata (\hyperref[UC36]{UC36})
        \item Almeno una repository del workspace ha subito un'analisi dei test
    \end{itemize}
    \item \textbf{Postcondizioni:}
    \begin{itemize}
        \item Il sistema presenta all'utente la sezione completa di analisi dei test aggregata
    \end{itemize}
    \item \textbf{Scenario principale:}
    \begin{enumerate}
        \item Il sistema recupera i dati dei test dalle repository del workspace
        \item Il sistema calcola le metriche aggregate sui test
        \item Il sistema visualizza il test coverage medio (\hyperref[UC36.2.1]{UC36.2.1})
        \item Il sistema visualizza il numero di repository con test coverage superiore a soglia (\hyperref[UC36.2.2]{UC36.2.2})
        \item Il sistema visualizza il test coverage minimo (\hyperref[UC36.2.3]{UC36.2.3})
    \end{enumerate}
    \item \textbf{Inclusioni:} \hyperref[UC36.2.1]{UC36.2.1}, \hyperref[UC36.2.2]{UC36.2.2}, \hyperref[UC36.2.3]{UC36.2.3}
\end{itemize}

% --- UC36.2.1 - TEST COVERAGE MEDIA ---
\subparagraph{UC36.2.1 - Visualizza test coverage media}\label{UC36.2.1}
\begin{itemize}
    \item \textbf{Attori principali:} Utente
    \item \textbf{Precondizioni:}
    \begin{itemize}
        \item Il sistema è attivo e funzionante
        \item L'utente sta visualizzando la sezione test aggregata (\hyperref[UC36.2]{UC36.2})
    \end{itemize}
    \item \textbf{Postcondizioni:}
    \begin{itemize}
        \item Viene visualizzato il valore medio del test coverage tra le repository del workspace
    \end{itemize}
    \item \textbf{Scenario principale:}
    \begin{enumerate}
        \item Il sistema calcola la media del test coverage tra tutte le repository del workspace
        \item Il sistema visualizza il valore percentuale del test coverage medio
    \end{enumerate}
\end{itemize}

% --- UC36.2.2 - TEST COVERAGE SUPERIORE A SOGLIA ---
\subparagraph{UC36.2.2 - Visualizza test coverage superiore a soglia}\label{UC36.2.2}
\begin{itemize}
    \item \textbf{Attori principali:} Utente
    \item \textbf{Precondizioni:}
    \begin{itemize}
        \item Il sistema è attivo e funzionante
        \item L'utente sta visualizzando la sezione test aggregata (\hyperref[UC36.2]{UC36.2})
    \end{itemize}
    \item \textbf{Postcondizioni:}
    \begin{itemize}
        \item Viene visualizzato il numero di repository che hanno un test coverage superiore alla soglia definita
    \end{itemize}
    \item \textbf{Scenario principale:}
    \begin{enumerate}
        \item Il sistema conta il numero di repository con test coverage superiore al parametro configurato (default: 70\%)
        \item Il sistema visualizza il conteggio delle repository che superano la soglia
    \end{enumerate}
\end{itemize}

% --- UC36.2.3 - TEST COVERAGE MINIMO ---
\subparagraph{UC36.2.3 - Visualizza test coverage minimo}\label{UC36.2.3}
\begin{itemize}
    \item \textbf{Attori principali:} Utente
    \item \textbf{Precondizioni:}
    \begin{itemize}
        \item Il sistema è attivo e funzionante
        \item L'utente sta visualizzando la sezione test aggregata (\hyperref[UC36.2]{UC36.2})
    \end{itemize}
    \item \textbf{Postcondizioni:}
    \begin{itemize}
        \item Viene visualizzato il valore minimo del test coverage tra le repository del workspace
    \end{itemize}
    \item \textbf{Scenario principale:}
    \begin{enumerate}
        \item Il sistema individua il valore minimo di test coverage tra tutte le repository del workspace
        \item Il sistema visualizza il valore percentuale del test coverage minimo
    \end{enumerate}
\end{itemize}


% ============================================
% UC36.3 - SEZIONE INFORMAZIONI TECNICHE AGGREGATA
% ============================================
\paragraph{UC36.3 - Sezione Informazioni Tecniche Aggregata}\label{UC36.3}
\begin{figure}[H]
	\centering
	\includegraphics[width=0.7\textwidth]{../Assets/AdR/UC36-3.png}
	\caption{Inclusioni di UC36.3: UC36.3.1, UC36.3.2}
	\label{fig:UC36-3}
\end{figure}

\begin{itemize}
    \item \textbf{Attori principali:} Utente
    \item \textbf{Precondizioni:}
    \begin{itemize}
        \item Il sistema è attivo e funzionante
        \item L'utente è autenticato presso il sistema
        \item L'utente sta visualizzando la visione aggregata (\hyperref[UC36]{UC36})
        \item Almeno una repository del workspace ha subito un'analisi delle informazioni tecniche
    \end{itemize}
    \item \textbf{Postcondizioni:}
    \begin{itemize}
        \item Il sistema presenta all'utente la sezione completa delle informazioni tecniche aggregate
    \end{itemize}
    \item \textbf{Scenario principale:}
    \begin{enumerate}
        \item Il sistema recupera le informazioni tecniche dalle repository del workspace
        \item Il sistema aggrega i dati su linguaggi e framework
        \item Il sistema visualizza il grafico distribuzione linguaggi (\hyperref[UC36.3.1]{UC36.3.1})
        \item Il sistema visualizza il grafico distribuzione framework (\hyperref[UC36.3.2]{UC36.3.2})
    \end{enumerate}
    \item \textbf{Inclusioni:} \hyperref[UC36.3.1]{UC36.3.1}, \hyperref[UC36.3.2]{UC36.3.2}
\end{itemize}

% --- UC36.3.1 - GRAFICO LINGUAGGI ---
\subparagraph{UC36.3.1 - Visualizza grafico linguaggi}\label{UC36.3.1}
\begin{itemize}
    \item \textbf{Attori principali:} Utente
    \item \textbf{Precondizioni:}
    \begin{itemize}
        \item Il sistema è attivo e funzionante
        \item L'utente sta visualizzando la sezione informazioni tecniche aggregata (\hyperref[UC36.3]{UC36.3})
    \end{itemize}
    \item \textbf{Postcondizioni:}
    \begin{itemize}
        \item Viene visualizzato un grafico a torta che mostra la distribuzione dei linguaggi di programmazione
    \end{itemize}
    \item \textbf{Scenario principale:}
    \begin{enumerate}
        \item Il sistema aggrega i dati sui linguaggi di programmazione utilizzati nelle repository del workspace
        \item Il sistema calcola la percentuale di utilizzo di ciascun linguaggio
        \item Il sistema visualizza un grafico a torta con la distribuzione dei linguaggi
    \end{enumerate}
\end{itemize}

% --- UC36.3.2 - GRAFICO FRAMEWORK ---
\subparagraph{UC36.3.2 - Visualizza grafico framework}\label{UC36.3.2}
\begin{itemize}
    \item \textbf{Attori principali:} Utente
    \item \textbf{Precondizioni:}
    \begin{itemize}
        \item Il sistema è attivo e funzionante
        \item L'utente sta visualizzando la sezione informazioni tecniche aggregata (\hyperref[UC36.3]{UC36.3})
    \end{itemize}
    \item \textbf{Postcondizioni:}
    \begin{itemize}
        \item Viene visualizzato un grafico a torta che mostra la distribuzione dei framework
    \end{itemize}
    \item \textbf{Scenario principale:}
    \begin{enumerate}
        \item Il sistema aggrega i dati sui framework utilizzati nelle repository del workspace
        \item Il sistema calcola la percentuale di utilizzo di ciascun framework
        \item Il sistema visualizza un grafico a torta con la distribuzione dei framework
    \end{enumerate}
\end{itemize}


% ============================================
% UC36.4 - SEZIONE SICUREZZA AGGREGATA
% ============================================
\paragraph{UC36.4 - Sezione Sicurezza Aggregata}\label{UC36.4}
\begin{figure}[H]
	\centering
	\includegraphics[width=0.7\textwidth]{../Assets/AdR/UC36-4.png}
	\caption{Inclusioni di UC36.4: UC36.4.1, UC36.4.2}
	\label{fig:UC36-4}
\end{figure}

\begin{itemize}
    \item \textbf{Attori principali:} Utente
    \item \textbf{Precondizioni:}
    \begin{itemize}
        \item Il sistema è attivo e funzionante
        \item L'utente è autenticato presso il sistema
        \item L'utente sta visualizzando la visione aggregata (\hyperref[UC36]{UC36})
        \item Almeno una repository del workspace ha subito un'analisi di sicurezza OWASP
    \end{itemize}
    \item \textbf{Postcondizioni:}
    \begin{itemize}
        \item Il sistema presenta all'utente la sezione completa di analisi della sicurezza aggregata
    \end{itemize}
    \item \textbf{Scenario principale:}
    \begin{enumerate}
        \item Il sistema recupera i dati di sicurezza dalle repository del workspace
        \item Il sistema aggrega le informazioni sulle vulnerabilità
        \item Il sistema visualizza la percentuale di repository sicure (\hyperref[UC36.4.1]{UC36.4.1})
        \item Il sistema visualizza il grafico delle vulnerabilità (\hyperref[UC36.4.2]{UC36.4.2})
    \end{enumerate}
    \item \textbf{Inclusioni:} \hyperref[UC36.4.1]{UC36.4.1}, \hyperref[UC36.4.2]{UC36.4.2}
\end{itemize}

% --- UC36.4.1 - PERCENTUALE REPOSITORY SICURE ---
\subparagraph{UC36.4.1 - Visualizza percentuale repository sicure}\label{UC36.4.1}
\begin{itemize}
    \item \textbf{Attori principali:} Utente
    \item \textbf{Precondizioni:}
    \begin{itemize}
        \item Il sistema è attivo e funzionante
        \item L'utente sta visualizzando la sezione sicurezza aggregata (\hyperref[UC36.4]{UC36.4})
    \end{itemize}
    \item \textbf{Postcondizioni:}
    \begin{itemize}
        \item Viene visualizzato il numero di repository che non presentano le Top 10 vulnerabilità OWASP
    \end{itemize}
    \item \textbf{Scenario principale:}
    \begin{enumerate}
        \item Il sistema conta il numero di repository che non presentano vulnerabilità tra le Top 10 OWASP
        \item Il sistema visualizza il conteggio delle repository sicure
    \end{enumerate}
\end{itemize}

% --- UC36.4.2 - GRAFICO VULNERABILITÀ ---
\subparagraph{UC36.4.2 - Visualizza grafico vulnerabilità}\label{UC36.4.2}
\begin{itemize}
    \item \textbf{Attori principali:} Utente
    \item \textbf{Precondizioni:}
    \begin{itemize}
        \item Il sistema è attivo e funzionante
        \item L'utente sta visualizzando la sezione sicurezza aggregata (\hyperref[UC36.4]{UC36.4})
    \end{itemize}
    \item \textbf{Postcondizioni:}
    \begin{itemize}
        \item Viene visualizzato un istogramma che mostra la distribuzione delle vulnerabilità per gravità
    \end{itemize}
    \item \textbf{Scenario principale:}
    \begin{enumerate}
        \item L'utente visualizza un istogramma con il numero di repository che presentano vulnerabilità per ciascun livello di gravità (alta, media, bassa)
    \end{enumerate}
\end{itemize}


% ============================================
% UC36.5 - SEZIONE DOCUMENTAZIONE AGGREGATA
% ============================================
\paragraph{UC36.5 - Sezione Documentazione Aggregata}\label{UC36.5}
\begin{figure}[H]
	\centering
	\includegraphics[width=0.7\textwidth]{../Assets/AdR/UC36-5.png}
	\caption{Inclusioni di UC36.5: UC36.5.1, UC36.5.2, UC36.5.3, UC36.5.4}
	\label{fig:UC36-5}
\end{figure}

\begin{itemize}
    \item \textbf{Attori principali:} Utente
    \item \textbf{Precondizioni:}
    \begin{itemize}
        \item Il sistema è attivo e funzionante
        \item L'utente è autenticato presso il sistema
        \item L'utente sta visualizzando la visione aggregata (\hyperref[UC36]{UC36})
        \item Almeno una repository del workspace ha subito un'analisi della documentazione
    \end{itemize}
    \item \textbf{Postcondizioni:}
    \begin{itemize}
        \item Il sistema presenta all'utente la sezione completa di analisi della documentazione aggregata
    \end{itemize}
    \item \textbf{Scenario principale:}
    \begin{enumerate}
        \item Il sistema recupera i dati sulla documentazione dalle repository del workspace
        \item Il sistema calcola le metriche aggregate sulla qualità della documentazione
        \item Il sistema visualizza il voto documentazione minimo (\hyperref[UC36.5.1]{UC36.5.1})
        \item Il sistema visualizza il voto documentazione medio (\hyperref[UC36.5.2]{UC36.5.2})
        \item Il sistema visualizza il voto documentazione massimo (\hyperref[UC36.5.3]{UC36.5.3})
        \item Il sistema visualizza il grafico distribuzione voti documentazione (\hyperref[UC36.5.4]{UC36.5.4})
    \end{enumerate}
    \item \textbf{Inclusioni:} \hyperref[UC36.5.1]{UC36.5.1}, \hyperref[UC36.5.2]{UC36.5.2}, \hyperref[UC36.5.3]{UC36.5.3}, \hyperref[UC36.5.4]{UC36.5.4}
\end{itemize}

% --- UC36.5.1 - VOTO DOCUMENTAZIONE MINIMO ---
\subparagraph{UC36.5.1 - Visualizza voto documentazione minimo}\label{UC36.5.1}
\begin{itemize}
    \item \textbf{Attori principali:} Utente
    \item \textbf{Precondizioni:}
    \begin{itemize}
        \item Il sistema è attivo e funzionante
        \item L'utente sta visualizzando la sezione documentazione aggregata (\hyperref[UC36.5]{UC36.5})
    \end{itemize}
    \item \textbf{Postcondizioni:}
    \begin{itemize}
        \item Viene visualizzato il voto minimo della qualità della documentazione tra le repository del workspace
    \end{itemize}
    \item \textbf{Scenario principale:}
    \begin{enumerate}
        \item Il sistema individua il voto minimo di qualità della documentazione tra tutte le repository del workspace
        \item Il sistema visualizza il voto minimo della documentazione
    \end{enumerate}
\end{itemize}

% --- UC36.5.2 - VOTO DOCUMENTAZIONE MEDIO ---
\subparagraph{UC36.5.2 - Visualizza voto documentazione medio}\label{UC36.5.2}
\begin{itemize}
    \item \textbf{Attori principali:} Utente
    \item \textbf{Precondizioni:}
    \begin{itemize}
        \item Il sistema è attivo e funzionante
        \item L'utente sta visualizzando la sezione documentazione aggregata (\hyperref[UC36.5]{UC36.5})
    \end{itemize}
    \item \textbf{Postcondizioni:}
    \begin{itemize}
        \item Viene visualizzato il voto medio della qualità della documentazione tra le repository del workspace
    \end{itemize}
    \item \textbf{Scenario principale:}
    \begin{enumerate}
        \item Il sistema calcola la media dei voti di qualità della documentazione tra tutte le repository del workspace
        \item Il sistema visualizza il voto medio della documentazione
    \end{enumerate}
\end{itemize}

% --- UC36.5.3 - VOTO DOCUMENTAZIONE MASSIMO ---
\subparagraph{UC36.5.3 - Visualizza voto documentazione massimo}\label{UC36.5.3}
\begin{itemize}
    \item \textbf{Attori principali:} Utente
    \item \textbf{Precondizioni:}
    \begin{itemize}
        \item Il sistema è attivo e funzionante
        \item L'utente sta visualizzando la sezione documentazione aggregata (\hyperref[UC36.5]{UC36.5})
    \end{itemize}
    \item \textbf{Postcondizioni:}
    \begin{itemize}
        \item Viene visualizzato il voto massimo della qualità della documentazione tra le repository del workspace
    \end{itemize}
    \item \textbf{Scenario principale:}
    \begin{enumerate}
        \item Il sistema individua il voto massimo di qualità della documentazione tra tutte le repository del workspace
        \item Il sistema visualizza il voto massimo della documentazione
    \end{enumerate}
\end{itemize}

% --- UC36.5.4 - GRAFICO VOTI DOCUMENTAZIONE ---
\subparagraph{UC36.5.4 - Visualizza grafico voti documentazione}\label{UC36.5.4}
\begin{itemize}
    \item \textbf{Attori principali:} Utente
    \item \textbf{Precondizioni:}
    \begin{itemize}
        \item Il sistema è attivo e funzionante
        \item L'utente sta visualizzando la sezione documentazione aggregata (\hyperref[UC36.5]{UC36.5})
    \end{itemize}
    \item \textbf{Postcondizioni:}
    \begin{itemize}
        \item Viene visualizzato un grafico a torta che mostra la distribuzione dei voti sulla documentazione
    \end{itemize}
    \item \textbf{Scenario principale:}
    \begin{enumerate}
        \item Il sistema raggruppa i voti di qualità della documentazione delle repository del workspace
        \item Il sistema calcola la distribuzione dei voti
        \item Il sistema mostra un grafico a torta con la distribuzione dei voti sulla documentazione
    \end{enumerate}
\end{itemize}


% ============================================
% UC36.6 - ERRORE VISIONE AGGREGATA
% ============================================
\paragraph{UC36.6 - Errore Visione Aggregata}\label{UC36.6}
\begin{itemize}
    \item \textbf{Attori principali:} Utente
    \item \textbf{Precondizioni:}
    \begin{itemize}
        \item Il sistema è attivo e funzionante
        \item L'utente è autenticato presso il sistema
        \item L'utente sta tentando di visualizzare la visione aggregata (\hyperref[UC36]{UC36})
        \item Si è verificata una delle seguenti condizioni:
        \begin{itemize}
            \item L'insieme di repository del workspace è vuoto
            \item Il workspace non contiene repository
        \end{itemize}
    \end{itemize}
    \item \textbf{Postcondizioni:}
    \begin{itemize}
        \item Il sistema informa l'utente dell'impossibilità di visualizzare la visione aggregata
    \end{itemize}
    \item \textbf{Scenario principale:}
    \begin{enumerate}
        \item Il sistema rileva una condizione di errore
        \item Il sistema mostra un messaggio di errore appropriato: "Il workspace non contiene repository da analizzare" (se workspace vuoto)
    \end{enumerate}
\end{itemize}


\subsubsection{UC37 - Aggiorna repository} \label{UC37}
\begin{itemize}
	\item \textbf{Attori principali:} Utente
	\item \textbf{Precondizioni:}
	\begin{itemize}
		\item Il sistema è attivo e funzionante
		\item L'utente è stato riconosciuto dal sistema come Utente
		\item L'utente ha selezionato un workspace
		\item L'utente sta visualizzando una lista di repository del workspace (\hyperref[UC25]{UC25}), una singola repository (\hyperref[UC33]{UC33}) o la visione aggregata del workspace (\hyperref[UC36]{UC36})
	\end{itemize}
	\item \textbf{Postcondizioni:} 
	\begin{itemize}
		\item Le informazioni mostrate sulle repository sono aggiornate in base all'ultima scansione completata
	\end{itemize}
	\item \textbf{Scenario principale:}
	\begin{enumerate}
		\item L'utente seleziona l'azione "Aggiorna"
		\item Il sistema verifica se, dall'ultima visualizzazione, sono state completate nuove scansioni
		\item Se sono state completate nuove scansioni, il sistema recupera i dati aggiornati delle repository
		\item Il sistema aggiorna le informazioni mostrate sulle repository in base ai risultati dell'ultima scansione completata
	\end{enumerate}
	\item \textbf{Scenario alternativo:}
	\begin{itemize}
		\item \textbf{Nessuna nuova scansione:} 
			\begin{itemize}
				\item \textbf{Condizione:} non sono presenti nuove scansioni completate dall'ultima visualizzazione delle repository
				\item \textbf{Flusso:} 
				\begin{enumerate}
					\item Il sistema mostra un messaggio informativo: "Nessuna nuova scansione completata. Le informazioni mostrate sono aggiornate." 
				\end{enumerate}
			\end{itemize}
		\item \textbf{Scansione in corso:} 
			\begin{itemize}
				\item \textbf{Condizione:} è attualmente in corso una scansione su una o più repository
				\item \textbf{Flusso:} 
				\begin{enumerate}
					\item Il sistema mostra un messaggio informativo: "Scansione in corso, provare ad aggiornare le informazioni più tardi." 
				\end{enumerate}
			\end{itemize}
	\end{itemize}
\end{itemize}






\newpage
\section{Requisiti} \label{sec:requisiti}

In questa sezione vengono elencati tutti i requisiti del sistema, classificati per tipologia.
La nomenclatura utilizzata per i codici dei requisiti segue il formato \textbf{R-X-Y-Z}, dove:
\begin{itemize}
    \item \textbf{X}: numero progressivo del requisito
    \item \textbf{Y}: tipo del requisito (\textbf{F} = Funzionale, \textbf{Q} = Qualità, \textbf{V} = Vincolo)
    \item \textbf{Z}: classe del requisito (\textbf{Ob} = Obbligatorio, \textbf{De} = Desiderabile, \textbf{Op} = Opzionale)
\end{itemize}

I requisiti sono così raggruppati: 
\begin{itemize}
    \item Funzionali: requisiti che rappresentano qualcosa che il Sistema sviluppato deve
avere per soddisfare un’aspettativa
    \item Qualità: requisiti che devono essere soddisfatti per accertare la qualità di
quanto realizzato
    \item Vincolo: restrizioni poste al Sistema, quali, a titolo di esempio, sull’uso di alcune
tecnologie
\end{itemize}

\subsection{Requisiti Funzionali}
\renewcommand{\arraystretch}{1.0}
\rowcolors{2}{lightgray!20}{white}
\begin{longtable}[c]{ >{\centering\arraybackslash}m{2.5cm} >{\centering\arraybackslash}m{8.3cm} >{\centering\arraybackslash}m{3.5cm} }
    \rowcolor{AccentDark}
    \color{white}\textbf{Codice} & \color{white}\textbf{Descrizione} & \color{white}\textbf{Fonti} \\
    \endhead

    R-1-F-Ob & L'utente non autenticato deve poter registrarsi al sistema inserendo username, email e password & \hyperref[UC1]{UC1}, Capitolato \\
    R-2-F-De & Il sistema deve inviare un codice OTP via email per la conferma della registrazione & \hyperref[UC2]{UC2} \\
    R-3-F-Ob & L'utente non autenticato deve poter effettuare il login tramite username/email e password & \hyperref[UC3]{UC3}, Capitolato \\
    R-4-F-Ob & Il sistema deve mostrare un messaggio di errore in caso di credenziali errate & \hyperref[UC3.3]{UC3.3} \\
    R-5-F-Ob & L'utente deve poter recuperare la password & \hyperref[UC4]{UC4} \\
    R-6-F-Ob & Il sistema deve autenticare gli utenti tramite Amazon Cognito & \hyperref[UC1]{UC1}, \hyperref[UC3]{UC3}, Capitolato \\
    R-7-F-Ob & L'utente autenticato deve poter visualizzare la lista dei workspace a cui appartiene & \hyperref[UC8]{UC8} \\
    R-8-F-Ob & L'utente autenticato deve poter creare un nuovo workspace inserendone il nome & \hyperref[UC10]{UC10} \\
    R-9-F-Ob & L'utente deve poter cercare workspace per nome & \hyperref[UC9]{UC9} \\
    R-10-F-Ob & Il Project Manager deve poter invitare utenti in un workspace selezionando un ruolo & \hyperref[UC5]{UC5} \\
    R-11-F-Ob & L'utente deve poter visualizzare la lista degli inviti ricevuti con dettagli & \hyperref[UC6]{UC6} \\
    R-12-F-Ob & L'utente deve poter accettare o rifiutare un invito a un workspace & \hyperref[UC7]{UC7} \\
    R-13-F-De & L'utente deve poter visualizzare la lista dei ruoli di un workspace & \hyperref[UC11]{UC11} \\
    R-14-F-Ob & L'utente deve poter visualizzare la lista degli utenti di un workspace & \hyperref[UC12]{UC12} \\
    R-15-F-Ob & Il Project Manager deve poter rimuovere un utente da un workspace & \hyperref[UC13]{UC13} \\
    R-16-F-De & Il sistema deve verificare automaticamente lo stato di aggiornamento di una repository & \hyperref[UC14]{UC14} \\
    R-17-F-De & L'utente deve poter visualizzare la lista dei tag raccolta di un workspace & \hyperref[UC19]{UC19} \\
    R-18-F-De & L'utente deve poter creare un tag raccolta all'interno di un workspace & \hyperref[UC20]{UC20} \\
    R-19-F-De & L'utente deve poter eliminare un tag raccolta dal workspace & \hyperref[UC22]{UC22} \\
    R-20-F-De & L'utente deve poter assegnare tag raccolta a una repository & \hyperref[UC23]{UC23} \\
    R-21-F-De & L'utente deve poter rimuovere tag raccolta da una repository & \hyperref[UC24]{UC24} \\
    R-22-F-Ob & L'utente deve poter visualizzare la lista delle repository del workspace & \hyperref[UC25]{UC25} \\
    R-23-F-Ob & L'utente deve poter aggiungere una repository GitHub al workspace & \hyperref[UC26]{UC26} \\
    R-24-F-Ob & Il sistema deve validare il link della repository e verificare l'accesso & \hyperref[UC27]{UC27} \\
    R-25-F-Ob & L'utente deve poter rimuovere una repository dal workspace & \hyperref[UC28]{UC28} \\
    R-26-F-Ob & L'utente deve poter effettuare ricerche di repository & \hyperref[UC29]{UC29} \\
    R-27-F-De & L'utente deve poter filtrare le repository tramite barra di ricerca e filtri & \hyperref[UC30]{UC30}, \hyperref[UC31]{UC31} \\
    R-28-F-De & L'utente deve poter ordinare la lista delle repository & \hyperref[UC32]{UC32} \\
    R-29-F-Ob & L'utente deve poter aggiornare le informazioni sulle repository in base all'ultima scansione & \hyperref[UC37]{UC37} \\
    R-30-F-Ob & L'utente deve poter visualizzare il dettaglio di una repository & \hyperref[UC33]{UC33} \\
    R-31-F-Ob & Il sistema deve permettere la selezione del branch per la visualizzazione & \hyperref[UC35]{UC35} \\
    R-32-F-De & Il sistema deve mostrare la percentuale di test coverage & \hyperref[UC33.1.1]{UC33.1.1} \\
    R-33-F-Ob & Il sistema deve mostrare l'elenco dei test con qualità insufficiente & \hyperref[UC33.1.2]{UC33.1.2}, Capitolato \\
    R-34-F-De & Il sistema deve mostrare la percentuale di test passati & \hyperref[UC33.1.3]{UC33.1.3} \\
    R-35-F-De & Il sistema deve mostrare l'elenco dei test non passati & \hyperref[UC33.1.4]{UC33.1.4} \\
    R-36-F-De & Il sistema deve mostrare un messaggio informativo se nessun test è rilevato & \hyperref[UC33.1.5]{UC33.1.5} \\
    R-37-F-Ob & Il sistema deve mostrare la lista dei linguaggi rilevati & \hyperref[UC33.2.1]{UC33.2.1}, Capitolato \\
    R-38-F-Ob & Il sistema deve mostrare la lista delle librerie rilevate & \hyperref[UC33.2.2]{UC33.2.2}, Capitolato \\
    R-39-F-Ob & Il sistema deve mostrare la lista dei framework rilevati & \hyperref[UC33.2.3]{UC33.2.3}, Capitolato \\
    R-40-F-Ob & Il sistema deve mostrare grafici semplificati dei risultati OWASP & \hyperref[UC33.3]{UC33.3}, Capitolato \\
    R-41-F-Ob & Il sistema deve mostrare un elenco ordinato delle vulnerabilità OWASP & \hyperref[UC33.4]{UC33.4}, Capitolato \\
    R-42-F-De & Il sistema deve fornire i dettagli per ogni vulnerabilità & \hyperref[UC33.4.2]{UC33.4.2}, Capitolato \\
    R-43-F-Ob & Il sistema deve mostrare un voto complessivo della qualità documentale & \hyperref[UC33.5.1]{UC33.5.1}, Capitolato \\
    R-44-F-De & Il sistema deve mostrare consigli per il miglioramento del README & \hyperref[UC33.5.2]{UC33.5.2} \\
    R-45-F-Ob & Il sistema deve elencare sezioni o documenti standard mancanti & \hyperref[UC33.5.3]{UC33.5.3} \\
    R-46-F-De & Il sistema deve mostrare suggerimenti sulla qualità del codice & \hyperref[UC33.6]{UC33.6}, Capitolato \\
    R-47-F-Ob & L'utente deve poter avviare una scansione & \hyperref[UC15]{UC15}, Capitolato \\
    R-48-F-Ob & Il sistema deve mostrare lo stato della scansione in corso & \hyperref[UC15]{UC15} \\
    R-49-F-Ob & L'utente deve poter annullare una scansione in corso & \hyperref[UC16]{UC16} \\
    R-50-F-Ob & Il sistema deve notificare errori durante la scansione & \hyperref[UC15]{UC15} \\
    R-51-F-Ob & L'utente deve poter visualizzare una visione aggregata del workspace & \hyperref[UC36]{UC36}, Capitolato \\
    R-52-F-De & L'utente deve poter filtrare la visione aggregata per tag raccolta & \hyperref[UC36.1]{UC36.1} \\
    R-53-F-Ob & Il sistema deve analizzare una o più repository & \hyperref[UC17]{UC17} \\
    R-54-F-Ob & Il sistema deve distribuire l'analisi tra più componenti specializzati & \hyperref[UC17.2]{UC17.2} \\
    R-55-F-Ob & Il sistema deve aggregare i risultati & \hyperref[UC17.4]{UC17.4} \\
    R-56-F-Ob & Il sistema deve salvare un report finale & \hyperref[UC17.4]{UC17.4} \\
    \\
    R-57-F-Op & Integrazione con CI/CD (GitHub Actions) & Capitolato sez. N2H \\
    R-58-F-Op & Analisi storica e confronto tra versioni & Capitolato sez. N2H \\
    R-59-F-Op & Ranking dei progetti per qualità complessiva & Capitolato sez. N2H \\
    R-60-F-Op & Remediation interattiva con anteprima & Capitolato sez. N2H \\
    R-61-F-Op & Notifiche integrate (Slack/Teams/Email) & Capitolato sez. N2H \\
    R-62-F-Op & Plugin system per nuovi agenti di analisi & Capitolato sez. N2H \\
\end{longtable}

\subsection{Requisiti di Qualità}
\renewcommand{\arraystretch}{1.0}
\rowcolors{2}{lightgray!20}{white}
\begin{longtable}[c]{ >{\centering\arraybackslash}m{2.5cm} >{\centering\arraybackslash}m{8.3cm} >{\centering\arraybackslash}m{3.5cm} }
    \rowcolor{AccentDark}
    \color{white}\textbf{Codice} & \color{white}\textbf{Descrizione} & \color{white}\textbf{Fonti} \\
    \endhead
    R-1-Q-Ob & Coverage minimo dei test di unità del 70\% tramite test automatizzati & Capitolato \\
    R-2-Q-Ob & Presenza di documentazione tecnica completa & Capitolato \\
    R-3-Q-Ob & Le API devono essere documentate tramite Swagger/OpenAPI & Capitolato \\
    R-4-Q-Ob & Deve essere presente un documento di Bug Reporting & Capitolato \\
    R-5-Q-Ob & L'applicazione deve essere sviluppata in ottica modulare ed estensibile & Capitolato \\
    R-6-Q-De & L'interfaccia utente deve essere responsive e fruibile da vari dispositivi & Capitolato \\
    R-7-Q-De & Il sistema deve garantire tempi di risposta ragionevoli per le scansioni & Capitolato \\
\end{longtable}

\subsection{Requisiti di Vincolo}
\renewcommand{\arraystretch}{1.0}
\rowcolors{2}{lightgray!20}{white}
\begin{longtable}[c]{ >{\centering\arraybackslash}m{2.5cm} >{\centering\arraybackslash}m{8.3cm} >{\centering\arraybackslash}m{3.5cm} }
    \rowcolor{AccentDark}
    \color{white}\textbf{Codice} & \color{white}\textbf{Descrizione} & \color{white}\textbf{Fonti} \\
    \endhead
    R-1-V-Ob & Backend e Orchestratore sviluppati in Node.js o Python & Capitolato \\
    R-2-V-Ob & Frontend sviluppato utilizzando la libreria React.js & Capitolato \\
    R-3-V-Ob & Database supportati: MongoDB o PostgreSQL & Capitolato \\
    R-4-V-Ob & L'infrastruttura cloud deve essere ospitata su Amazon Web Services (AWS) & Capitolato \\
    R-5-V-Ob & L'autenticazione deve essere gestita tramite Amazon Cognito & Capitolato \\
    R-6-V-Ob & Il codice sorgente deve essere versionato tramite Git & Capitolato \\
    R-7-V-De & L'integrazione CI/CD deve utilizzare GitHub Actions & Capitolato \\
    R-8-V-Ob & L'analisi di sicurezza deve basarsi sullo standard OWASP Top 10 & Capitolato \\
    R-9-V-Ob & Architettura ad agenti coordinati da un orchestratore autocratico & Capitolato, \hyperref[UC17]{UC17} \\
\end{longtable}


\subsection{Tracciamento Casi d'Uso e riepilogo}

\renewcommand{\arraystretch}{1.0}
\rowcolors{2}{lightgray!20}{white}

\begin{longtable}[c]{ >{\centering\arraybackslash}m{2.5cm} >{\centering\arraybackslash}m{10cm} }
    \rowcolor{AccentDark}
    \color{white}\textbf{Caso d'Uso} & \color{white}\textbf{Requisiti} \\
    \endhead

    \hyperref[UC1]{UC1} & R-1-F-Ob, R-6-F-Ob \\
    \hyperref[UC2]{UC2} & R-2-F-De \\
    \hyperref[UC3]{UC3} & R-3-F-Ob, R-4-F-Ob, R-6-F-Ob \\
    \hyperref[UC3.3]{UC3.3} & R-4-F-Ob \\
    \hyperref[UC4]{UC4} & R-5-F-Ob \\
    \hyperref[UC5]{UC5} & R-10-F-Ob \\
    \hyperref[UC6]{UC6} & R-11-F-Ob \\
    \hyperref[UC7]{UC7} & R-12-F-Ob \\
    \hyperref[UC8]{UC8} & R-7-F-Ob \\
    \hyperref[UC9]{UC9} & R-9-F-Ob \\
    \hyperref[UC10]{UC10} & R-8-F-Ob \\
    \hyperref[UC11]{UC11} & R-13-F-De \\
    \hyperref[UC12]{UC12} & R-14-F-Ob \\
    \hyperref[UC13]{UC13} & R-15-F-Ob \\
    \hyperref[UC14]{UC14} & R-16-F-De \\
    \hyperref[UC15]{UC15} & R-47-F-Ob, R-48-F-Ob, R-50-F-Ob \\
    \hyperref[UC16]{UC16} & R-49-F-Ob \\
    \hyperref[UC17]{UC17} & R-53-F-Ob \\
    \hyperref[UC17.2]{UC17.2} & R-54-F-Ob \\
    \hyperref[UC17.4]{UC17.4} & R-55-F-Ob, R-56-F-Ob \\
    \hyperref[UC19]{UC19} & R-17-F-De \\
    \hyperref[UC20]{UC20} & R-18-F-De \\
    \hyperref[UC22]{UC22} & R-19-F-De \\
    \hyperref[UC23]{UC23} & R-20-F-De \\
    \hyperref[UC24]{UC24} & R-21-F-De \\
    \hyperref[UC25]{UC25} & R-22-F-Ob \\
    \hyperref[UC26]{UC26} & R-23-F-Ob \\
    \hyperref[UC27]{UC27} & R-24-F-Ob \\
    \hyperref[UC28]{UC28} & R-25-F-Ob \\
    \hyperref[UC29]{UC29} & R-26-F-Ob \\
    \hyperref[UC30]{UC30} & R-27-F-De \\
    \hyperref[UC31]{UC31} & R-27-F-De \\
    \hyperref[UC32]{UC32} & R-28-F-De \\
    \hyperref[UC33]{UC33} & R-30-F-Ob \\
    \hyperref[UC33.1.1]{UC33.1.1} & R-32-F-De \\
    \hyperref[UC33.1.2]{UC33.1.2} & R-33-F-Ob \\
    \hyperref[UC33.1.3]{UC33.1.3} & R-34-F-De \\
    \hyperref[UC33.1.4]{UC33.1.4} & R-35-F-De \\
    \hyperref[UC33.1.5]{UC33.1.5} & R-36-F-De \\
    \hyperref[UC33.2.1]{UC33.2.1} & R-37-F-Ob \\
    \hyperref[UC33.2.2]{UC33.2.2} & R-38-F-Ob \\
    \hyperref[UC33.2.3]{UC33.2.3} & R-39-F-Ob \\
    \hyperref[UC33.3]{UC33.3} & R-40-F-Ob \\
    \hyperref[UC33.4.2]{UC33.4.2} & R-42-F-De \\
    \hyperref[UC33.4]{UC33.4} & R-41-F-Ob \\
    \hyperref[UC33.5.1]{UC33.5.1} & R-43-F-Ob \\
    \hyperref[UC33.5.2]{UC33.5.2} & R-44-F-De \\
    \hyperref[UC33.5.3]{UC33.5.3} & R-45-F-Ob \\
    \hyperref[UC33.6]{UC33.6} & R-46-F-De \\
    \hyperref[UC35]{UC35} & R-31-F-Ob \\
    \hyperref[UC36]{UC36} & R-51-F-Ob \\
    \hyperref[UC36.1]{UC36.1} & R-52-F-De \\
    \hyperref[UC17]{UC17} & R-9-V-Ob \\
\end{longtable}

\subsection{Riepilogo dei Requisiti}

Nella seguente tabella viene riportato il riepilogo del numero di requisiti individuati, suddivisi per tipologia e per classe di priorità.

\renewcommand{\arraystretch}{1.5}
\rowcolors{2}{lightgray!20}{white}
\begin{table}[H]
    \centering
    \begin{tabular}{ l c c c c }
        \rowcolor{AccentDark}
        \color{white}\textbf{Tipologia} & \color{white}\textbf{Obbligatori} & \color{white}\textbf{Desiderabili} & \color{white}\textbf{Opzionali} & \color{white}\textbf{Totale} \\
        
        Funzionali & 38 & 18 & 6 & 62 \\
        Qualità    & 5  & 2  & 0 & 7  \\
        Vincolo    & 8  & 1  & 0 & 9  \\
        \textbf{Totale} & \textbf{51} & \textbf{21} & \textbf{6} & \textbf{78} \\
    \end{tabular}
    \label{tab:riepilogo-requisiti}
\end{table}

\end{document}