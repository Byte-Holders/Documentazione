\documentclass[a4paper, 11pt]{article}

% ====== PACCHETTI NECESSARI ======
\usepackage[utf8]{inputenc}
\usepackage[T1]{fontenc}
\usepackage[italian]{babel}
\usepackage{geometry}
\usepackage{graphicx}
\usepackage[table]{xcolor}
\usepackage{tabularx}
\usepackage{array}
\usepackage{amssymb}
\usepackage{fancyhdr}
\setlength{\headheight}{14pt}
\usepackage{titlesec}
\usepackage{helvet}
\renewcommand{\familydefault}{\sfdefault}
\usepackage{lipsum}
\usepackage{hyperref}
\usepackage{booktabs}
\usepackage{enumitem}
\usepackage{titlecaps}
\usepackage[utf8]{inputenc} % Specifica la codifica del file (necessaria per le accentate)
\usepackage[T1]{fontenc}    % Migliora l'output dei font per le lingue europee

% ====== IMPOSTAZIONI GLOBALI DI STILE ======

% 1. DEFINIZIONE COLORI BLU-VIOLA
\definecolor{AccentColor}{RGB}{80, 90, 180} % Blu-viola principale
\definecolor{AccentLight}{RGB}{80, 90, 180} % Versione più chiara
\definecolor{AccentDark}{RGB}{50, 60, 140} % Versione più scura
\definecolor{LightGray}{RGB}{245, 245, 250}
\definecolor{MediumGray}{RGB}{200, 200, 210}

% 2. IMPOSTAZIONE MARGINI
\geometry{a4paper, left=2.5cm, right=2.5cm, top=3.5cm, bottom=3.5cm}

% 3. STILE DEI TITOLI DI SEZIONE
\titleformat{\section}
  {\normalfont\sffamily\Large\bfseries\color{AccentColor}}
  {\thesection}
  {1em}
  {}
\titleformat{\subsection}
  {\normalfont\sffamily\large\bfseries\color{AccentDark}}
  {\thesubsection}
  {1em}
  {}

% 4. IMPOSTAZIONE HEADER E FOOTER
\pagestyle{fancy}
\fancyhf{} 
\fancyhead[L]{\sffamily\bfseries\color{AccentColor}\@BYTE HOLDERS}
\fancyhead[R]{\sffamily\color{AccentColor}\thepage}
\renewcommand{\headrulewidth}{0.8pt}
\renewcommand{\headrule}{\color{AccentColor}\hrule width\headwidth height\headrulewidth \vskip-\headrulewidth}

% 5. IMPOSTAZIONE LINK
\hypersetup{
	colorlinks=true,
	linkcolor=AccentColor,
	urlcolor=AccentLight,
	citecolor=AccentDark,
}

% 6. PERSONALIZZAZIONE ELENCHI
\setlist[itemize]{itemsep=2pt, topsep=4pt}
\setlist[enumerate]{itemsep=2pt, topsep=4pt}

% ====== COMANDI PERSONALIZZATI ======
\newcommand{\refterm}[1]{\textit{#1}\textsuperscript{G}}
\newcommand{\term}[1]{\subsubsection{\refterm{#1}}}

% ====== STILE TABELLE MIGLIORATO ======
\newcolumntype{Y}{>{\raggedright\arraybackslash}X} % Colonna giustificata a sinistra
\setlength{\arrayrulewidth}{0.4pt} % Linee più sottili
\setlength{\tabcolsep}{10pt} % Spaziatura interna celle
\renewcommand{\arraystretch}{1.4} % Altezza righe

% ====== INIZIO DEL DOCUMENTO ======
\begin{document}
	

\pagestyle{empty}

% ====== PAGINA DI TITOLO ======
\begin{titlepage}
	\begin{center}
		\includegraphics[width=0.55\textwidth]{../Assets/ByteHolders1.png}\vspace{1.5cm}

		{\LARGE \sffamily \color{AccentColor}\bfseries Analisi dei Requisiti}
	\end{center}
	
	\vfill
	\rule{\textwidth}{1pt}\par
	\textit{Versione: 0.1.0}
\end{titlepage}

\newpage

% ====== TABELLA DI VERSIONAMENTO ======
{\normalfont\sffamily\huge\bfseries\color{AccentColor} Registro delle versioni}
\vspace{1cm}

\begin{center}
	\rowcolors{2}{LightGray}{white}
	\begin{tabular}{>{\centering\arraybackslash}m{1.5cm} >{\centering\arraybackslash}m{2cm} >{\raggedright\arraybackslash}m{2.5cm} >{\raggedright\arraybackslash}m{6.5cm}}
		\rowcolor{AccentColor}
		\textcolor{white}{\textbf{Versione}} & 
		\textcolor{white}{\textbf{Data}} & 
		\multicolumn{1}{c}{\textcolor{white}{\textbf{Autore}}} &
		\multicolumn{1}{c}{\textcolor{white}{\textbf{Descrizione delle modifiche}}} \\
		0.1.0 & 19/12/2025 & Alessandro Morabito & Inizio stesura\\ 
	\end{tabular}
\end{center}


% ====== INDICE ======
\pagestyle{fancy}
\newpage
\tableofcontents
\newpage

% ====== INTRODUZIONE ======
\section{Introduzione}
\subsection{Scopo del documento}
Con il presente documento il gruppo Byte Holders stabilisce i requisiti funzionali e non funzionali del software CodeGuardian.

Questo documento è rivolto:
\begin{itemize}
	\item all'Azienda Var Group, destinatari anche del software sviluppato
	\item al gruppo Byte Holders, che farà riferimento a questo documento nel corso del progetto
	\item ai professori Tullio Vardanega e Riccardo Cardin
\end{itemize}

All'interno del documento si proponge una visione generale del software proposto nella Sezione \ref{sec:descrizione generale}, per poi passare in rassegna i casi d'uso individuati nella Sezione \ref{sec:casi d'uso}.

Per la redazione del documento si è fatto riferimento allo standard IEEE 830-1998.

\subsection{Scopo del prodotto}
Il prodotto CodeGuardian permetterà di effettuare analisi della qualità di repository GitHub, con una particolare attenzione in merito ai permessi di visualizzazione delle informazioni e lancio delle stesse analisi.

CodeGuardian si propone come soluzione per team di sviluppo eterogenei che vogliono poter monitorare lo stato di repository GitHub e ottenere informazioni aggregate su insemi di progetti analizzati.

\subsection{Glossario}
Per evitare ambiguità, nel corso del documento si farà riferimento a termini indicati nel \href{https://byte-holders.github.io/Documentazione/RTB/Glossario.pdf}{\refterm{Glossario}} utilizzando la lettera \textit{G} ad apice della formula corrispondente, che viene indicata in corsivo (ad es. \refterm{formula in glossario}). La corrispondenza di termini è a meno di coniugazioni e declinazioni.

% Dal momento che il glossario è un documento interno, possiamo mettere qua dentro tutti i termini che ci servono
\subsection{Definizioni, acronimi e abbreviazioni}
\term{Caso d'uso}
Un Caso d'uso è un insieme di scenari che hanno in comune uno scopo finale per un utente.

\term{Scenario}

\term{Attore}


\subsection{Riferimenti}
\subsubsection{Riferimenti normativi}
\begin{itemize}
	\item 
		Norme Di Progetto\\
		\url{https://byte-holders.github.io/Documentazione/RTB/Norme_Di_Progetto.pdf}
	\item
		830-1998 - IEEE Recommended Practice for Software Requirements Specifications\\
		\url{https://ieeexplore.ieee.org/document/720574}
	\item
		Capitolato\\
		\url{https://www.math.unipd.it/~tullio/IS-1/2025/Progetto/C2.pdf}
\end{itemize}
\subsubsection{Riferimenti informativi}
\begin{itemize}
	\item
		\refterm{Glossario}\\
		\url{https://byte-holders.github.io/Documentazione/RTB/Glossario.pdf}
	\item
		Specifica UML 2.5.1\\
		\url{https://www.omg.org/spec/UML/2.5.1/PDF}
\end{itemize}

\section{Descrizione generale} \label{sec:descrizione generale}
\subsection{Prospettiva del prodotto}
Il gruppo Byte Holders propone il software CodeGuardian, un sistema ad agenti che permette di analizzare la qualità del codice, il livello di sicurezza e di manutenzione per una repository GitHub. L'esito dell'analisi sarà disponibile sotto forma di report agli utenti, ai quali sono proposte eventuali soluzioni alle problematiche individuate.

Il gruppo Byte Holders ha offerto particolare attenzione alla rolistica all'interno dell'applicazione, che si è tradotta nella distinzione di tipologie di utenti in base ai loro permessi.

A tutti gli utenti sarà comune la presenza di una dashboard che comprenderà vari workspace, nonché la capacità di effettuare ricerche avanzate al loro interno.

Il prodotto si propone quindi come soluzione per diverse tipologie di utenti, come quelle individuate in prima sessione di \textit{Design Thinking}, che condividono gli stessi progetti.

\subsection{Funzioni del prodotto}

\subsection{Caratteristiche dell'utente}

\section{Casi d'uso} \label{sec:casi d'uso}
\subsection{Lista degli Attori}
Nella creazione dei casi d'uso sono stati individuati i seguenti attori:
\begin{itemize}
	\item 
		\textbf{Utente non Autenticato}\\
		Un utente non riconosciuto dal sistema
	\item 
		\textbf{Utente Autenticato}\\
		Utente generico riconosciuto dal sistema
	\item 
		\textbf{Utente Permesso OWASP}  (eredita da \textit{Utente autenticato})\\
		Utente Autenticato con il permesso per la visione completa delle informazioni su OWASP
	\item
		\textbf{Utente Permesso Utenti/Ruoli} (eredita da \textit{Utente autenticato})\\
	 	Utente Autenticato con il permesso per la gestione degli utenti e dei ruoli
	\item 
		\textbf{Utente Permesso Test} (eredita da \textit{Utente autenticato})\\
		Utente Autenticato con il permesso per la visione completa delle informazioni sui test
	\item
		\textbf{Utente Permesso Documentazione} (eredita da \textit{Utente autenticato})\\
		Utente Autenticato con il permesso per la visione completa delle informazioni sulla documentazione
	\item
		\textbf{Utente Permesso Scansione} (eredita da \textit{Utente autenticato})\\
		Utente Autenticato con il permesso per il lancio di una scansione
	\item
		\textbf{Utente Permesso Qualità Codice} (eredita da \textit{Utente autenticato})\\
		Utente Autenticato con il permesso per la visione completa delle informazioni sulla qualità del codice
	\item
		\textbf{Utente Permesso Informazioni Tecniche} (eredita da \textit{Utente autenticato})\\
		Utente Autenticato con il permesso per la visione completa delle informazioni tecniche di una repository
\end{itemize}
\subsection{Struttura generale di un caso d'uso}
Si è deciso di descrivere ciascun \refterm{caso d'uso} seguendo la seguente struttura (\underline{sottolineati} i campi sempre popolati):

\begin{tabular}{|p{.25\linewidth} p{.6\linewidth}|}
	\hline
	\textbf{\underline{Codice}} 
	&
	Codice identificativo utilizzato per far riferimento al \refterm{caso d'uso} corrente\\
	
	\textbf{\underline{Titolo}} 
	&
	Titolo del \refterm{caso d'uso} corrente\\
	
	\textbf{\underline{Attori principali}} 
	&
	Attori che agiscono sul sistema dando inizio allo scenario\\
	
	\textbf{Attori secondari} 
	&
	Attori di supporto che agiscono in risposta a stimoli del sistema\\
	
	\textbf{\underline{Precondizioni}}
	&
	Condizioni necessarie per l'esecuzione del \refterm{caso d'uso} corrente\\
	
	\textbf{\underline{Postcondizioni}} 
	&
	Condizioni in cui viene lasciato il sistema al termine dello \refterm{scenario principale}\\
	
	\textbf{\underline{Scenario principale}}
	&
	Descrizione degli eventi che avvengono all'interno dello \refterm{scenario principale}\\
	
	\textbf{Inclusioni} 
	&
	Lista dei riferimenti a \refterm{casi d'uso} terzi \refterm{inclusi} dal \refterm{caso d'uso} corrente e al quale si fa riferimento nella sezione \textit{Scenario principale}\\
	
	\textbf{Scenari alternativi} 
	&
	Descrizione delle situazione che portano a \refterm{scenari alternativi}\\
	
	\textbf{Eredita da} 
	&
	Codice del \refterm{caso d'uso} terzo da cui eredita il \refterm{caso d'uso} corrente (non ammettiamo ereditarietà multipla)\\
	
	\hline
\end{tabular}
\subsection{Lista dei \refterm{casi d'uso}}
\end{document}