\documentclass[a4paper, 11pt]{article}

% ====== PACCHETTI NECESSARI ======
\usepackage[utf8]{inputenc}
\usepackage[T1]{fontenc}
\usepackage[italian]{babel}
\usepackage{geometry}
\usepackage{graphicx}
\usepackage[table]{xcolor}
\usepackage{tabularx}
\usepackage{array}
\usepackage{amssymb}
\usepackage{fancyhdr}
\setlength{\headheight}{14pt}
\usepackage{titlesec}
\usepackage{helvet}
\renewcommand{\familydefault}{\sfdefault}
\usepackage{lipsum}
\usepackage{hyperref}
\usepackage{booktabs}
\usepackage{enumitem}
\usepackage{titlecaps}
\usepackage[utf8]{inputenc} % Specifica la codifica del file (necessaria per le accentate)
\usepackage[T1]{fontenc}    % Migliora l'output dei font per le lingue europee

% ====== IMPOSTAZIONI GLOBALI DI STILE ======

% 1. DEFINIZIONE COLORI BLU-VIOLA
\definecolor{AccentColor}{RGB}{80, 90, 180} % Blu-viola principale
\definecolor{AccentLight}{RGB}{80, 90, 180} % Versione più chiara
\definecolor{AccentDark}{RGB}{50, 60, 140} % Versione più scura
\definecolor{LightGray}{RGB}{245, 245, 250}
\definecolor{MediumGray}{RGB}{200, 200, 210}

% 2. IMPOSTAZIONE MARGINI
\geometry{a4paper, left=2.5cm, right=2.5cm, top=3.5cm, bottom=3.5cm}

% 3. STILE DEI TITOLI DI SEZIONE
\titleformat{\section}
  {\normalfont\sffamily\Large\bfseries\color{AccentColor}}
  {\thesection}
  {1em}
  {}
\titleformat{\subsection}
  {\normalfont\sffamily\large\bfseries\color{AccentDark}}
  {\thesubsection}
  {1em}
  {}

% 4. IMPOSTAZIONE HEADER E FOOTER
\pagestyle{fancy}
\fancyhf{} 
\fancyhead[L]{\sffamily\bfseries\color{AccentColor}\@BYTE HOLDERS}
\fancyhead[R]{\sffamily\color{AccentColor}\thepage}
\renewcommand{\headrulewidth}{0.8pt}
\renewcommand{\headrule}{\color{AccentColor}\hrule width\headwidth height\headrulewidth \vskip-\headrulewidth}

% 5. IMPOSTAZIONE LINK
\hypersetup{
	colorlinks=true,
	linkcolor=AccentColor,
	urlcolor=AccentLight,
	citecolor=AccentDark,
}

% 6. PERSONALIZZAZIONE ELENCHI
\setlist[itemize]{itemsep=2pt, topsep=4pt}
\setlist[enumerate]{itemsep=2pt, topsep=4pt}

% ====== COMANDI PERSONALIZZATI ======
\newcommand{\refterm}[1]{\textit{#1}\textsuperscript{G}}
\newcommand{\term}[1]{\subsubsection{\refterm{#1}}}

% ====== STILE TABELLE MIGLIORATO ======
\newcolumntype{Y}{>{\raggedright\arraybackslash}X} % Colonna giustificata a sinistra
\setlength{\arrayrulewidth}{0.4pt} % Linee più sottili
\setlength{\tabcolsep}{10pt} % Spaziatura interna celle
\renewcommand{\arraystretch}{1.4} % Altezza righe

% ====== INIZIO DEL DOCUMENTO ======
\begin{document}
	

\pagestyle{empty}

% ====== PAGINA DI TITOLO ======
\begin{titlepage}
	\begin{center}
		\includegraphics[width=0.55\textwidth]{../Assets/ByteHolders1.png}\vspace{1.5cm}

		{\LARGE \sffamily \color{AccentColor}\bfseries Analisi dei Requisiti}
	\end{center}
	
	\vfill
	\rule{\textwidth}{1pt}\par
	\textit{Versione: 0.1.0}
\end{titlepage}

\newpage

% ====== TABELLA DI VERSIONAMENTO ======
{\normalfont\sffamily\huge\bfseries\color{AccentColor} Registro delle versioni}
\vspace{1cm}

\begin{center}
	\rowcolors{2}{LightGray}{white}
	\begin{tabular}{>{\centering\arraybackslash}m{1.5cm} >{\centering\arraybackslash}m{2cm} >{\raggedright\arraybackslash}m{2.5cm} >{\raggedright\arraybackslash}m{6.5cm}}
		\rowcolor{AccentColor}
		\textcolor{white}{\textbf{Versione}} & 
		\textcolor{white}{\textbf{Data}} & 
		\multicolumn{1}{c}{\textcolor{white}{\textbf{Autore}}} &
		\multicolumn{1}{c}{\textcolor{white}{\textbf{Descrizione delle modifiche}}} \\
		0.2.0 & 14/01/2025 & Lorenzo Grolla & Stesura UC Autenticazione\\ 
		0.1.0 & 19/12/2025 & Alessandro Morabito & Inizio stesura\\ 
	\end{tabular}
\end{center}


% ====== INDICE ======
\pagestyle{fancy}
\newpage
\tableofcontents
\newpage

% ====== INTRODUZIONE ======
\section{Introduzione}
\subsection{Scopo del documento}
Con il presente documento il gruppo Byte Holders stabilisce i requisiti funzionali e non funzionali del software CodeGuardian.

Questo documento è rivolto:
\begin{itemize}
	\item all'Azienda Var Group, destinatari anche del software sviluppato
	\item al gruppo Byte Holders, che farà riferimento a questo documento nel corso del progetto
	\item ai professori Tullio Vardanega e Riccardo Cardin
\end{itemize}

All'interno del documento si proponge una visione generale del software proposto nella Sezione \ref{sec:descrizione generale}, per poi passare in rassegna i casi d'uso individuati nella Sezione \ref{sec:casi d'uso}.

Per la redazione del documento si è fatto riferimento allo standard IEEE 830-1998.

\subsection{Scopo del prodotto}
Il prodotto CodeGuardian permetterà di effettuare analisi della qualità di repository GitHub, con una particolare attenzione in merito ai permessi di visualizzazione delle informazioni e lancio delle stesse analisi.

CodeGuardian si propone come soluzione per team di sviluppo eterogenei che vogliono poter monitorare lo stato di repository GitHub e ottenere informazioni aggregate su insemi di progetti analizzati.

\subsection{Glossario}
Per evitare ambiguità, nel corso del documento si farà riferimento a termini indicati nel \href{https://byte-holders.github.io/Documentazione/RTB/Glossario.pdf}{\refterm{Glossario}} utilizzando la lettera \textit{G} ad apice della formula corrispondente, che viene indicata in corsivo (ad es. \refterm{formula in glossario}). La corrispondenza di termini è a meno di coniugazioni e declinazioni.

% Dal momento che il glossario è un documento interno, possiamo mettere qua dentro tutti i termini che ci servono
\subsection{Definizioni, acronimi e abbreviazioni}
\term{Caso d'uso}
Un Caso d'uso è un insieme di scenari che hanno in comune uno scopo finale per un utente.

\term{Scenario}

\term{Attore}


\subsection{Riferimenti}
\subsubsection{Riferimenti normativi}
\begin{itemize}
	\item 
		Norme Di Progetto\\
		\url{https://byte-holders.github.io/Documentazione/RTB/Norme_Di_Progetto.pdf}
	\item
		830-1998 - IEEE Recommended Practice for Software Requirements Specifications\\
		\url{https://ieeexplore.ieee.org/document/720574}
	\item
		Capitolato\\
		\url{https://www.math.unipd.it/~tullio/IS-1/2025/Progetto/C2.pdf}
\end{itemize}
\subsubsection{Riferimenti informativi}
\begin{itemize}
	\item
		\refterm{Glossario}\\
		\url{https://byte-holders.github.io/Documentazione/RTB/Glossario.pdf}
	\item
		Specifica UML 2.5.1\\
		\url{https://www.omg.org/spec/UML/2.5.1/PDF}
\end{itemize}

\section{Descrizione generale} \label{sec:descrizione generale}
\subsection{Prospettiva del prodotto}
Il gruppo Byte Holders propone il software CodeGuardian, un sistema ad agenti che permette di analizzare la qualità del codice, il livello di sicurezza e di manutenzione per una repository GitHub. L'esito dell'analisi sarà disponibile sotto forma di report agli utenti, ai quali sono proposte eventuali soluzioni alle problematiche individuate.

Il gruppo Byte Holders ha offerto particolare attenzione alla rolistica all'interno dell'applicazione, che si è tradotta nella distinzione di tipologie di utenti in base ai loro permessi.

A tutti gli utenti sarà comune la presenza di una dashboard che comprenderà vari workspace, nonché la capacità di effettuare ricerche avanzate al loro interno.

Il prodotto si propone quindi come soluzione per diverse tipologie di utenti, come quelle individuate in prima sessione di \textit{Design Thinking}, che condividono gli stessi progetti.

\subsection{Funzioni del prodotto}

\subsection{Caratteristiche dell'utente}

\section{Casi d'uso} \label{sec:casi d'uso}
\subsection{Lista degli Attori}
Nella creazione dei casi d'uso sono stati individuati i seguenti attori:
\begin{itemize}
	\item 
		\textbf{Utente non Autenticato}\\
		Un utente non riconosciuto dal sistema
	\item 
		\textbf{Utente Autenticato}\\
		Utente generico riconosciuto dal sistema
	\item 
		\textbf{Utente Permesso OWASP}  (eredita da \textit{Utente autenticato})\\
		Utente Autenticato con il permesso per la visione completa delle informazioni su OWASP
	\item
		\textbf{Utente Permesso Utenti/Ruoli} (eredita da \textit{Utente autenticato})\\
	 	Utente Autenticato con il permesso per la gestione degli utenti e dei ruoli
	\item 
		\textbf{Utente Permesso Test} (eredita da \textit{Utente autenticato})\\
		Utente Autenticato con il permesso per la visione completa delle informazioni sui test
	\item
		\textbf{Utente Permesso Documentazione} (eredita da \textit{Utente autenticato})\\
		Utente Autenticato con il permesso per la visione completa delle informazioni sulla documentazione
	\item
		\textbf{Utente Permesso Scansione} (eredita da \textit{Utente autenticato})\\
		Utente Autenticato con il permesso per il lancio di una scansione
	\item
		\textbf{Utente Permesso Qualità Codice} (eredita da \textit{Utente autenticato})\\
		Utente Autenticato con il permesso per la visione completa delle informazioni sulla qualità del codice
	\item
		\textbf{Utente Permesso Informazioni Tecniche} (eredita da \textit{Utente autenticato})\\
		Utente Autenticato con il permesso per la visione completa delle informazioni tecniche di una repository
\end{itemize}
\subsection{Struttura generale di un caso d'uso}
Si è deciso di descrivere ciascun \refterm{caso d'uso} seguendo la seguente struttura (\underline{sottolineati} i campi sempre popolati):

\begin{tabular}{|p{.25\linewidth} p{.6\linewidth}|}
	\hline
	\textbf{\underline{Codice}} 
	&
	Codice identificativo utilizzato per far riferimento al \refterm{caso d'uso} corrente\\
	
	\textbf{\underline{Titolo}} 
	&
	Titolo del \refterm{caso d'uso} corrente\\
	
	\textbf{\underline{Attori principali}} 
	&
	Attori che agiscono sul sistema dando inizio allo scenario\\
	
	\textbf{Attori secondari} 
	&
	Attori di supporto che agiscono in risposta a stimoli del sistema\\
	
	\textbf{\underline{Precondizioni}}
	&
	Condizioni necessarie per l'esecuzione del \refterm{caso d'uso} corrente\\
	
	\textbf{\underline{Postcondizioni}} 
	&
	Condizioni in cui viene lasciato il sistema al termine dello \refterm{scenario principale}\\
	
	\textbf{\underline{Scenario principale}}
	&
	Descrizione degli eventi che avvengono all'interno dello \refterm{scenario principale}\\
	
	\textbf{Inclusioni} 
	&
	Lista dei riferimenti a \refterm{casi d'uso} terzi \refterm{inclusi} dal \refterm{caso d'uso} corrente e al quale si fa riferimento nella sezione \textit{Scenario principale}\\
	
	\textbf{Scenari alternativi} 
	&
	Descrizione delle situazione che portano a \refterm{scenari alternativi}\\
	
	\textbf{Eredita da} 
	&
	Codice del \refterm{caso d'uso} terzo da cui eredita il \refterm{caso d'uso} corrente (non ammettiamo ereditarietà multipla)\\
	
	\hline
\end{tabular}

\newpage
\subsection{Lista dei \refterm{casi d'uso}}

% UC1: REGISTRAZIONE UTENTE (Padre)
\subsubsection{UC1 - Registrazione}
\begin{figure}[h]
	\centering
	\includegraphics[width=0.8\textwidth]{../Assets/AdR/UC1ext.png}
	\caption{UC1 - Registrazione}
	\label{fig:UC1ext}
\end{figure}

\begin{itemize}
	\item \textbf{Attori principali:} Utente non Autenticato
	\item \textbf{Precondizioni:} 
	\begin{itemize}
		\item Il sistema è attivo e funzionante
		\item L'utente non possiede ancora un account attivo nel sistema
	\end{itemize}
	\item \textbf{Postcondizioni:} Viene creato un nuovo profilo utente nel sistema CodeGuardian con stato "da confermare".
	\item \textbf{Scenario principale:}
	\begin{enumerate}
		\item L'utente accede alla pagina di registrazione
		\item L'utente inserisce l'username (UC1.1)
		\item L'utente inserisce l'email (UC1.2)
		\item L'utente inserisce la password (UC1.3)
		\item L'utente conferma la registrazione
		\item Il sistema valida i dati e crea l'utente su Amazon Cognito
		\item Il sistema invia un codice OTP all'email fornita
	\end{enumerate}
	\item \textbf{Inclusioni:} 
		\begin{itemize}
			\item UC1.1 - Inserimento username
			\item UC1.2 - Inserimento email
			\item UC1.3 - Inserimento password
		\end{itemize} 
	\item \textbf{Estensioni:} 
		\begin{itemize}
			\item UC1.4 - Registrazione fallita
		\end{itemize}
\end{itemize}

Il caso d'Uso UC1 include ulteriori casi d'uso come rappresentato nella seguente immagine:
\begin{figure}[h]
	\centering
	\includegraphics[width=0.6\textwidth]{../Assets/AdR/UC1.png}
	\caption{Inclusioni di UC1: UC1.1,UC1.2,UC1.3}
	\label{fig:UC1}
\end{figure}

% --- Sottocasi di UC1 ---
\subsubsection{UC1.1 - Inserimento username}
\begin{itemize}
	\item \textbf{Attori principali:} Utente non Autenticato
	\item \textbf{Precondizioni:} 
	\begin{itemize}
		\item Il sistema è attivo e funzionante
		\item L'utente si trova nella pagina di registrazione
	\end{itemize}
	\item \textbf{Scenario principale:} L'utente digita l'username scelto nell'apposito campo.
	\item \textbf{Postcondizioni:} L'username è inserito nel sistema.
\end{itemize}

\subsubsection{UC1.2 - Inserimento email}
\begin{itemize}
	\item \textbf{Attori principali:} Utente non Autenticato
	\item \textbf{Precondizioni:} 
	\begin{itemize}
		\item Il sistema è attivo e funzionante
		\item L'utente si trova nella pagina di registrazione
	\end{itemize}
	\item \textbf{Scenario principale:} L'utente digita il proprio indirizzo email nell'apposito campo.
	\item \textbf{Postcondizioni:} L'email è inserita nel sistema.
\end{itemize}

\subsubsection{UC1.3 - Inserimento password}
\begin{itemize}
	\item \textbf{Attori principali:} Utente non Autenticato
	\item \textbf{Precondizioni:} 
	\begin{itemize}
		\item Il sistema è attivo e funzionante
		\item L'utente si trova nella pagina di registrazione
	\end{itemize}
	\item \textbf{Scenario principale:} L'utente digita la password desiderata nell'apposito campo.
	\item \textbf{Postcondizioni:} La password è inserita nel sistema.
\end{itemize}

\subsubsection{UC1.4 - Registrazione fallita}
\begin{itemize}
	\item \textbf{Attori principali:} Utente non Autenticato
	\item \textbf{Precondizioni:} 
	\begin{itemize}
		\item Il sistema è attivo e funzionante
		\item L'utente ha tentato la conferma della registrazione con dati non validi
	\end{itemize}
	\item \textbf{Scenario principale:}
	\begin{enumerate}
		\item Il sistema rileva un errore nei dati inseriti (username già esistente, email già registrata oppure password non conforme ai requisiti di sicurezza)
		\item Il sistema mostra un messaggio di errore specifico all'utente
		\item Il sistema permette all'utente di correggere i dati mantenendo quelli validi
	\end{enumerate}
	\item \textbf{Postcondizioni:} La registrazione non viene completata; l'utente rimane nella pagina di registrazione.
\end{itemize}


% UC2: CONFERMA REGISTRAZIONE 
\subsubsection{UC2 - Conferma registrazione}
\begin{figure}[h]
	\centering
	\includegraphics[width=0.7\textwidth]{../Assets/AdR/UC2.png}
	\caption{UC2 - Conferma Registrazione}
	\label{fig:UC2}
\end{figure}

\begin{itemize}
	\item \textbf{Attori principali:} Utente non Autenticato
	\item \textbf{Precondizioni:} 
		\begin{itemize}
			\item Il sistema è attivo e funzionante
			\item L'utente ha completato la prima fase di registrazione (UC1)
			\item L'utente ha ricevuto il codice OTP via email
		\end{itemize}
	\item \textbf{Postcondizioni:} L'account viene attivato e l'utente può effettuare il login.
	\item \textbf{Scenario principale:}
	\begin{enumerate}
		\item L'utente accede alla pagina di conferma registrazione
		\item Il sistema richiede il codice di verifica
		\item L'utente inserisce il codice OTP (UC2.1)
		\item L'utente conferma l'invio
		\item Il sistema verifica il codice e attiva l'account
	\end{enumerate}
	\item \textbf{Inclusioni:} 
		\begin{itemize}
			\item UC2.1 - Inserimento codice OTP
		\end{itemize}
	\item \textbf{Estensioni:} 
	\begin{itemize}
		\item UC2.2 - Verifica fallita
	\end{itemize}
\end{itemize}

% --- Sottocasi di UC2 ---
\subsubsection{UC2.1 - Inserimento codice OTP}
\begin{itemize}
	\item \textbf{Attori principali:} Utente non Autenticato
	\item \textbf{Precondizioni:} 
	\begin{itemize}
		\item Il sistema è attivo e funzionante
		\item L'utente si trova nella pagina di conferma registrazione
	\end{itemize}
	\item \textbf{Scenario principale:} L'utente inserisce il codice numerico ricevuto via email nell'apposito campo.
	\item \textbf{Postcondizioni:} Il codice OTP è inserito nel sistema.
\end{itemize}

\subsubsection{UC2.2 - Verifica fallita}
\begin{itemize}
	\item \textbf{Attori principali:} Utente non Autenticato
	\item \textbf{Precondizioni:} 
	\begin{itemize}
		\item Il sistema è attivo e funzionante
		\item L'utente ha inviato un codice OTP errato o scaduto
	\end{itemize}
	\item \textbf{Scenario principale:}
	\begin{enumerate}
		\item Il sistema rileva che il codice non è valido o è scaduto
		\item Il sistema mostra un messaggio di errore "Codice non valido o scaduto"
		\item Il sistema offre l'opzione per richiedere un nuovo codice OTP
	\end{enumerate}
	\item \textbf{Postcondizioni:} L'account rimane nello stato "da confermare".
\end{itemize}


% UC3: LOGIN 
\subsubsection{UC3 - Login}
\begin{figure}[h]
	\centering
	\includegraphics[width=0.7\textwidth]{../Assets/AdR/UC3ext.png}
	\caption{UC3 - Login}
	\label{fig:UC3ext}
\end{figure}
\begin{itemize}
	\item \textbf{Attori principali:} Utente non Autenticato
	\item \textbf{Precondizioni:} 
		\begin{itemize}
			\item Il sistema è attivo e funzionante
			\item L'utente possiede un account attivo nel sistema
		\end{itemize}
	\item \textbf{Postcondizioni:} L'utente è autenticato e accede alla home del sistema
	\item \textbf{Scenario principale:}
	\begin{enumerate}
		\item L'utente accede alla pagina di login
		\item L'utente inserisce l'username (UC3.1)
		\item L'utente inserisce la password (UC3.2)
		\item L'utente conferma l'accesso
		\item Il sistema valida le credenziali tramite Amazon Cognito
		\item Il sistema reindirizza l'utente alla home
	\end{enumerate}
	\item \textbf{Inclusioni:} 
		\begin{itemize}
			\item UC3.1 - Inserimento username o email
			\item UC3.2 - Inserimento password
		\end{itemize}
	\item \textbf{Estensioni:} 
	\begin{itemize}
		\item UC3.3 - Login fallito
	\end{itemize}
	\item \textbf{Scenari alternativi:}
	\begin{itemize}
		\item UC4 - Recupero password (accessibile dalla pagina di login)
	\end{itemize}
\end{itemize}

% --- Sottocasi di UC3 ---
Il caso d'Uso UC3 include ulteriori casi d'uso come rappresentato nella seguente immagine:
\begin{figure}[h]
	\centering
	\includegraphics[width=0.6\textwidth]{../Assets/AdR/UC3.png}
	\caption{Inclusioni di UC3: UC3.1,UC3.2}
	\label{fig:UC1}
\end{figure}


\subsubsection{UC3.1 - Inserimento username o email}
\begin{itemize}
	\item \textbf{Attori principali:} Utente non Autenticato
	\item \textbf{Precondizioni:} 
	\begin{itemize}
		\item Il sistema è attivo e funzionante
		\item L'utente si trova nella pagina di login
	\end{itemize}
	\item \textbf{Scenario principale:} L'utente inserisce il proprio username o email nell'apposito campo.
	\item \textbf{Postcondizioni:} L'username è inserito nel sistema.
\end{itemize}

\subsubsection{UC3.2 - Inserimento password}
\begin{itemize}
	\item \textbf{Attori principali:} Utente non Autenticato
	\item \textbf{Precondizioni:} 
	\begin{itemize}
		\item Il sistema è attivo e funzionante
		\item L'utente si trova nella pagina di login
	\end{itemize}
	\item \textbf{Scenario principale:} L'utente inserisce la propria password nell'apposito campo.
	\item \textbf{Postcondizioni:} La password è inserita nel sistema.
\end{itemize}

\subsubsection{UC3.3 - Login fallito}
\begin{itemize}
	\item \textbf{Attori principali:} Utente non Autenticato
	\item \textbf{Precondizioni:} 
	\begin{itemize}
		\item Il sistema è attivo e funzionante
		\item L'utente ha inviato credenziali non valide
	\end{itemize}
	\item \textbf{Scenario principale:}
	\begin{enumerate}
		\item Il sistema verifica che le credenziali non corrispondono a nessun account attivo
		\item Il sistema mostra il messaggio "Username o password errati"
		\item Il sistema permette di riprovare l'inserimento delle credenziali
	\end{enumerate}
	\item \textbf{Postcondizioni:} L'utente rimane non autenticato.
\end{itemize}

% UC4: RECUPERO PASSWORD 
\subsubsection{UC4 - Recupero password}
\begin{figure}[h]
	\centering
	\includegraphics[width=0.7\textwidth]{../Assets/AdR/UC4ext.png}
	\caption{UC4}
	\label{fig:UC4ext}
\end{figure}

\begin{itemize}
	\item \textbf{Attori principali:} Utente non Autenticato
	\item \textbf{Precondizioni:} 
	\begin{itemize}
		\item Il sistema è attivo e funzionante
		\item L'utente possiede un account registrato
		\item L'utente ha dimenticato la password
	\end{itemize}
	\item \textbf{Postcondizioni:} La password viene reimpostata con successo.
	\item \textbf{Scenario principale:}
	\begin{enumerate}
		\item L'utente accede alla funzionalità di recupero password dalla pagina di login
		\item L'utente inserisce l'email associata al proprio account (UC4.1)
		\item L'utente conferma la richiesta
		\item Il sistema valida l'email e genera un codice OTP
		\item Il sistema invia il codice OTP via email
		\item L'utente inserisce il codice OTP ricevuto (UC4.2)
		\item L'utente inserisce la nuova password (UC4.3)
		\item L'utente conferma il cambio password
		\item Il sistema valida il codice OTP, verifica che la nuova password rispetti i requisiti di sicurezza e aggiorna la password
		\item Il sistema conferma l'aggiornamento e reindirizza alla pagina di login
	\end{enumerate}
	\item \textbf{Inclusioni:} 
	\begin{itemize}
		\item UC4.1 - Inserimento email per recupero
		\item UC4.2 - Inserimento codice OTP
		\item UC4.3 - Inserimento nuova password
	\end{itemize} 
	\item \textbf{Estensioni:}
	\begin{itemize}
		\item UC4.4 - Ripristino fallito
	\end{itemize}
\end{itemize}

% --- Sottocasi di UC4 ---
Il caso d'Uso UC4 include ulteriori casi d'uso come rappresentato nella seguente immagine:
\begin{figure}[h]
	\centering
	\includegraphics[width=0.6\textwidth]{../Assets/AdR/UC4.png}
	\caption{Inclusioni di UC4: UC4.1,UC4.2,UC4.3}
	\label{fig:UC4ext}
\end{figure}

\subsubsection{UC4.1 - Inserimento email per recupero}
\begin{itemize}
	\item \textbf{Attori principali:} Utente non Autenticato
	\item \textbf{Precondizioni:} 
	\begin{itemize}
		\item Il sistema è attivo e funzionante
		\item L'utente si trova nella pagina di recupero password
	\end{itemize}
	\item \textbf{Scenario principale:} L'utente inserisce l'email associata all'account nell'apposito campo e conferma la richiesta.
	\item \textbf{Postcondizioni:} Il sistema invia il codice OTP all'email fornita.
\end{itemize}

\subsubsection{UC4.2 - Inserimento codice OTP}
\begin{itemize}
	\item \textbf{Attori principali:} Utente non Autenticato
	\item \textbf{Precondizioni:} 
	\begin{itemize}
		\item Il sistema è attivo e funzionante
		\item L'utente ha ricevuto il codice OTP via email
		\item L'utente si trova nella pagina di reset password
	\end{itemize}
	\item \textbf{Scenario principale:} L'utente inserisce il codice OTP ricevuto via email nell'apposito campo.
	\item \textbf{Postcondizioni:} Il codice OTP è inserito nel sistema.
\end{itemize}

\subsubsection{UC4.3 - Inserimento nuova password}
\begin{itemize}
	\item \textbf{Attori principali:} Utente non Autenticato
	\item \textbf{Precondizioni:} 
	\begin{itemize}
		\item Il sistema è attivo e funzionante
		\item L'utente ha inserito il codice OTP
		\item L'utente si trova nella pagina di reset password
	\end{itemize}
	\item \textbf{Scenario principale:} L'utente inserisce la nuova password desiderata nell'apposito campo.
	\item \textbf{Postcondizioni:} La nuova password è inserita nel sistema.
\end{itemize}

\subsubsection{UC4.4 - Ripristino fallito}
\begin{itemize}
	\item \textbf{Attori principali:} Utente non Autenticato
	\item \textbf{Precondizioni:} 
	\begin{itemize}
		\item Il sistema è attivo e funzionante
		\item L'utente ha inserito un codice OTP non valido o scaduto, oppure una password non conforme ai requisiti
	\end{itemize}
	\item \textbf{Scenario principale:}
	\begin{enumerate}
		\item Il sistema rileva che il codice OTP è errato o scaduto, oppure che la password non rispetta i requisiti di sicurezza
		\item Il sistema mostra un messaggio di errore specifico
		\item Il sistema impedisce il cambio password e permette di richiedere un nuovo codice o correggere la password
	\end{enumerate}
	\item \textbf{Postcondizioni:} La password rimane invariata.
\end{itemize}

\end{document}