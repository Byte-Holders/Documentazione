\documentclass[a4paper, 11pt]{article}

% ====== PACCHETTI NECESSARI ======
\usepackage[utf8]{inputenc}
\usepackage[T1]{fontenc}
\usepackage[italian]{babel}
\usepackage{geometry}
\usepackage{graphicx}
\usepackage[table]{xcolor}
\usepackage{tabularx}
\usepackage{array}
\usepackage{amssymb}
\usepackage{fancyhdr}
\usepackage{titlesec}
\usepackage{helvet}
\renewcommand{\familydefault}{\sfdefault}
\usepackage{lipsum}
\usepackage{hyperref}
\usepackage{booktabs}
\usepackage{enumitem}
\usepackage[utf8]{inputenc} % Specifica la codifica del file (necessaria per le accentate)
\usepackage[T1]{fontenc}    % Migliora l'output dei font per le lingue europee

% ====== IMPOSTAZIONI GLOBALI DI STILE ======

% 1. DEFINIZIONE COLORI BLU-VIOLA
\definecolor{AccentColor}{RGB}{80, 90, 180} % Blu-viola principale
\definecolor{AccentLight}{RGB}{80, 90, 180} % Versione più chiara
\definecolor{AccentDark}{RGB}{50, 60, 140} % Versione più scura
\definecolor{LightGray}{RGB}{245, 245, 250}
\definecolor{MediumGray}{RGB}{200, 200, 210}

% 2. IMPOSTAZIONE MARGINI
\geometry{a4paper, left=2.5cm, right=2.5cm, top=3.5cm, bottom=3.5cm}

% 3. STILE DEI TITOLI DI SEZIONE
\titleformat{\section}
  {\normalfont\sffamily\Large\bfseries\color{AccentColor}}
  {\thesection}
  {1em}
  {}
\titleformat{\subsection}
  {\normalfont\sffamily\large\bfseries\color{AccentDark}}
  {\thesubsection}
  {1em}
  {}
  \titleformat{\subsubsection}
  {\normalfont\sffamily\large\bfseries\color{AccentDark}} % Cambiare colore dellle subsubsection
  {\thesubsubsection}
  {1em}
  {}

% 4. IMPOSTAZIONE HEADER E FOOTER
\pagestyle{fancy}
\fancyhf{}
\fancyhead[L]{\sffamily\bfseries\color{AccentColor}\@BYTE HOLDERS}
\fancyhead[R]{\sffamily\color{AccentColor}\thepage}
\renewcommand{\headrulewidth}{0.8pt}
\renewcommand{\headrule}{\color{AccentColor}\hrule width\headwidth height\headrulewidth \vskip-\headrulewidth}

% 5. IMPOSTAZIONE LINK
\hypersetup{
    colorlinks=true,
    linkcolor=AccentColor,
    urlcolor=AccentLight,
    citecolor=AccentDark,
}

% 6. PERSONALIZZAZIONE ELENCHI
\setlist[itemize]{itemsep=2pt, topsep=4pt}
\setlist[enumerate]{itemsep=2pt, topsep=4pt}

% ====== COMANDI PERSONALIZZATI ======
\makeatletter
\newcommand{\NomeGruppo}[1]{\def\@NomeGruppo{#1}}
\newcommand{\TitoloVerbale}[1]{\def\@TitoloVerbale{#1}}
\newcommand{\Sommario}[1]{\def\@Sommario{#1}}
\newcommand{\Autore}[1]{\def\@Autore{#1}}
\newcommand{\Verificatore}[1]{\def\@Verificatore{#1}}
\makeatother

% ====== STILE TABELLE MIGLIORATO ======
\newcolumntype{Y}{>{\raggedright\arraybackslash}X} % Colonna giustificata a sinistra
\setlength{\arrayrulewidth}{0.4pt} % Linee più sottili
\setlength{\tabcolsep}{10pt} % Spaziatura interna celle
\renewcommand{\arraystretch}{1.4} % Altezza righe

% ====== INIZIO DEL DOCUMENTO ======
\begin{document}

% ====== INFORMAZIONI PER LA PAGINA DI TITOLO ======
\NomeGruppo{BYTE HOLDERS}
\TitoloVerbale{Norme di Progetto}
\Autore{}
\Verificatore{}

\pagestyle{empty}

% ====== PAGINA DI TITOLO ======
\begin{titlepage}
    \centering
    
    \includegraphics[width=0.55\textwidth]{../Assets/ByteHolders1.png}\par\vspace{1.5cm}
\begin{figure}
        \centering
        \label{fig:placeholder}
    \end{figure}
        
    {\LARGE \sffamily \color{AccentColor}\bfseries Norme di Progetto}\par
    \vspace{0.5cm}
    {\large \color{AccentColor}\sffamily }\par
    
    \vfill
    
    \noindent\color{AccentColor}\rule{\textwidth}{1pt}\par
    \vspace{0.5cm}
    
    \begin{tabularx}{0.9\textwidth}{@{}>{\bfseries\sffamily}l X@{}}
    Autore & \sffamily Nicolò Lattanzio\\
    \arrayrulecolor{MediumGray}\hline \\[-1.5ex]
    Verificatore & \sffamily \\
    \arrayrulecolor{MediumGray}\hline \\[-1.5ex] 
    Approvazione & \sffamily \\ 
    \arrayrulecolor{MediumGray}\hline 
\end{tabularx}
    
    \vfill
\end{titlepage}

% ====== INDICE ======
\pagestyle{fancy}
\newpage
\tableofcontents
\newpage

% ====== TABELLA DI VERSIONAMENTO ======
\section{Registro delle versioni}
\begin{center}
    \rowcolors{2}{LightGray}{white}
    \begin{tabular}{>{\centering\arraybackslash}m{1.5cm} >{\centering\arraybackslash}m{2cm} >{\raggedright\arraybackslash}m{2.5cm} >{\raggedright\arraybackslash}m{6.5cm}}
        \rowcolor{AccentColor}
          \textcolor{white}{\textbf{Versione}} & 
          \textcolor{white}{\textbf{Data}} & 
          \multicolumn{1}{c}{\textcolor{white}{\textbf{Autore}}} & 
          \multicolumn{1}{c}{\textcolor{white}{\textbf{Descrizione delle modifiche}}} \\

        0.2 & 06/12/2025 & Nicolò Lattanzio & Aggiunta sezione Processi Primari, con relative sottosezioni, e prima redazione della sezione Processi di Supporto. \\
        0.1 & 30/11/2025 & Nicolò Lattanzio & Creazione e prima redazione del documento.  \\
        
    \end{tabular}
\end{center}

%\vspace{1cm}

% ====== SEZIONE INFORMAZIONI INTRODUTTIVE ======
\section{Informazioni introduttive}
\subsection{Scopo del documento}
Questo documento definisce le norme di progetto adottate dal gruppo \textbf{\textit{BYTE HOLDERS}} per la gestione e lo sviluppo del progetto. Esso stabilisce le linee guida per la documentazione, la comunicazione, la gestione delle versioni e le responsabilità dei membri del team.
\\
Per la realizzazione del documento, e' stato deciso di prendere come riferimento lo standard ISO/IEC 12207, 
che definisce i processi di ciclo di vita del software e fornisce una struttura per la gestione del progetto.
Le tipologie di processi sono le seguenti:
\\
\begin{itemize}
    \item \textbf{Processi Primari} : Consiste nelle attività principali che portano alla creazione, consegna e mantenimento del prodotto finale.
    \item \textbf{Processi di Supporto}: Processi che forniscono supporto alle attività principali, possono essere utilizzato da qualsiasi processo e ne garantiscono la qualità e la correttezza
    \item \textbf{Processi Organizzativi}: Processi necessari a creare l'ambiente e le risorse necessarie affinché i processi primari possano esistere.
\end{itemize}


\subsection{Scopo del prodotto}
Il prodotto sviluppato dal gruppo \textbf{\textit{BYTE HOLDERS}} è un'applicazione web che mira a fornire una piattaforma intuitiva per la gestione delle repository GitHub, 
fornendo informazioni utili in base alle varie necessità.
Tra queste funzionalità vi sono:
\begin{itemize}
    \item Analisi statica della qualità del codice
    \item Analisi della qualità della documentazione
    \item Analisi OWASP per la verifica delle vulnerabilità di sicurezza tramite l'integrazione di strumenti di terze parti
    \item Integrazione con API di GitHub per fornire statistiche dettagliate sulle repository
    \item Analisi specifica sulle pull request per valutare la qualità del codice proposto
    \item Proposte di remedation per migliorare la qualità del codice e della documentazione
\end{itemize}
Il prodotto segue la proposta del \textbf{capitolato C2 - Code Guardian} fornito dal proponente \textbf{\textit{VarGroup S.r.l.}}
\\
Il gruppo si è posto l'obiettivo di completare l'MVP di questo progetto entro il \textbf{20 Marzo 2026}, con un costo totale di \textbf{13.290 Euro.}
\subsection{Riferimenti}
\begin{itemize}
        \item \href{https://www.math.unipd.it/~tullio/IS-1/2025/Progetto/C2.pdf} {Capitolato C2 - Code Guardian}
        \item \href{https://byte-holders.github.io/Documentazione/} {Documentazione del progetto}
\end{itemize}


\section{Processi Primari}
Questa sezione descrive i processi del ciclo di vita primari, definiti dallo standard ISO/IEC 12207, che sono direttamente coinvolti nella creazione, fornitura e manutenzione del prodotto software. 
Essi costituiscono il cuore operativo del progetto, governando le attività principali dall'acquisizione delle esigenze alla consegna e al supporto del prodotto finale.
\\
Possiamo distinguere due macro processi:
\begin{itemize}
    \item \textbf{Processo di Fornitura:} Si occupa della gestione delle relazioni con il committente, dalla risposta alla richiesta iniziale fino alla consegna del prodotto.
    \item \textbf{Processo di Sviluppo:} Copre tutte le attività ingegneristiche necessarie per trasformare i requisiti in un prodotto software funzionante, includendo analisi, progettazione, implementazione, testing e rilascio.
\end{itemize}

\subsection{Processo di Fornitura}
Questo processo definisce le attività che il gruppo Byte Holders, in qualità di fornitore, implementa per garantire la corretta fornitura del prodotto software a VarGroup S.r.l., committente del progetto. 
Esso disciplina l'intero ciclo, dall'aggiudicazione del capitolato alla consegna finale.
\\
Vediamo ora le attività principali coinvolte in questo processo:
\begin{enumerate}
    \item \textbf{Risposta al Bando:} Analisi del capitolato e preparazione dell'offerta (documento \textit{Candidatura}).
    \item \textbf{Negoziazione Requisiti:} Realizzazione di una controproposta per il proponente, definendo i requisiti funzionali e non funzionali che il gruppo Byte Holders, come fornitore, prevede di riuscire a realizzare.
    \item \textbf{Pianificazione della Fornitura:} Creazione del \textit{Piano di Progetto} che dettaglia come verranno eseguiti i processi di sviluppo e gestione.
    \item \textbf{Esecuzione e Controllo:} Realizzazione del progetto conforme al piano, con reporting periodico allo stakeholder.
    \item \textbf{Revisione e Valutazione:} Condotta di revisioni tecniche e attività di verifica e validazione.
    \item \textbf{Consegna e Completamento:} Consegna del prodotto finale insieme alla relativa documentazione.
\end{enumerate}
\subsubsection{Documenti Principali:} 
Per questo progetto saranno prodotti i seguenti documenti chiave:

\subsection{Processo di Sviluppo}
Il processo di sviluppo definisce il ciclo di vita ingegneristico attraverso il quale i requisiti del committente vengono trasformati nel prodotto software finale. 
Questo processo organizza in fasi strutturate le attività di analisi, progettazione, implementazione e validazione, garantendo un approccio sistematico e tracciabile alla realizzazione del sistema.\\

\begin{itemize}
    \item \textbf{Attività Chiave (Fasi del Ciclo di Vita):}
    \begin{enumerate}
        \item \textbf{Analisi dei Requisiti:} Raccolta, analisi, specifica e validazione dei requisiti con gli stakeholder. Prodotto: \textit{Documento di Analisi dei Requisiti}.
        \item \textbf{Progettazione dell'Architettura:} Definizione dell'architettura di sistema e software ad alto livello.
        \item \textbf{Progettazione di Dettaglio:} Definizione dettagliata dei componenti e delle interfacce.
        \item \textbf{Costruzione (Implementazione e Codifica):} Scrittura del codice secondo gli standard di progetto e utilizzo di un sistema di versioning (Git). Prodotto: Codice sorgente e repository.
        \item \textbf{Integrazione e Testing:}
        \begin{itemize}
            \item Testing delle unità (sviluppatori)
            \item Testing d'integrazione (team di integrazione)
            \item Testing di sistema (team di qualità)
        \end{itemize}
    \end{enumerate}
    \item \textbf{Metodologia:} Adotteremo pratiche ibride Agile, con sprint per lo sviluppo e milestone formali per le revisioni.
\end{itemize}


\subsection{Strumenti di supporto}
Per la gestione del progetto, il gruppo \textbf{\textit{BYTE HOLDERS}} utilizza i seguenti strumenti:
\begin{itemize}
  \item \textbf{GitHub}: per il versionamento del codice e la gestione delle issue.
  \item \textbf{Discord}: per la comunicazione interna tra i membri del team.
  \item \textbf{Latex}: per la redazione della documentazione di progetto.
  \item \textbf{Github Actions}: Per la gestione dei latex e la generazione automatica della documentazione.
  \item \textbf{Slack}: Per la comunicazione con il proponente.
\end{itemize}%




\section{Processi Di Supporto}

\subsection{Documentazione}
La documentazione del progetto viene gestita utilizzando \textbf{Latex} e viene archiviata in un repository dedicato su \textbf{GitHub}. 
\\Ogni documento segue una struttura standardizzata per garantire coerenza e facilità di lettura.

\subsubsection{Struttura dei documenti}

\subsubsection{Versionamento}
Il versionamento di ogni documento viene raccolto in una tabella ad inizio documento, che ne traccia le modifiche e la loro data, gli autori e una breve descrizione.
La tabella deve essere di tipo LIFO(Last in First Out), ovvero le versioni più recenti devono essere in cima alla tabella.
\\
In questo modo è possibile vedere fin da subito lo stato di un documento senza dover leggere tutta la tabella fino alla fine; 
inoltre i documenti con la versione più aggiornata sono gli unici documento validi per la consultazione.

Il versionamento di un documento e' composto da tre cifre separate da un punto: a.b.c, se a=1 allora il documento e' stato verificato e approvato, 
se a=0 allora il documento si trova in una fase di redazione e non e' stato ancora verificato.
La cifra c viene incrementata di 1 ad ogni modifica del documento, la cifra b invece viene incrementata di 1 quando avviene una modifica sostanziale del documento.

Un documento che e' stato verificato e approvato puo' essere modificato anche se nella sua versione 1.0.0.

\subsubsection{Tabella Decisioni e Azioni}


\section{Processi Organizzativi}

\subsection{Riunioni}

\subsection{Strumenti a supporto}

\subsection{Gestione dei processi}












\end{document}