\documentclass[a4paper, 11pt]{article}

% ====== PACCHETTI NECESSARI ======
\usepackage[utf8]{inputenc}
\usepackage[T1]{fontenc}
\usepackage[italian]{babel}
\usepackage{geometry}
\usepackage{graphicx}
\usepackage[table]{xcolor}
\usepackage{tabularx}
\usepackage{array}
\usepackage{amssymb}
\usepackage{fancyhdr}
\setlength{\headheight}{14pt}
\usepackage{titlesec}
\usepackage{helvet}
\renewcommand{\familydefault}{\sfdefault}
\usepackage{lipsum}
\usepackage{hyperref}
\usepackage{booktabs}
\usepackage{enumitem}
\usepackage[utf8]{inputenc} % Specifica la codifica del file (necessaria per le accentate)
\usepackage[T1]{fontenc}    % Migliora l'output dei font per le lingue europee
\usepackage{longtable}

% ====== IMPOSTAZIONI GLOBALI DI STILE ======

% 1. DEFINIZIONE COLORI BLU-VIOLA
\definecolor{AccentColor}{RGB}{80, 90, 180} % Blu-viola principale
\definecolor{AccentLight}{RGB}{80, 90, 180} % Versione più chiara
\definecolor{AccentDark}{RGB}{50, 60, 140} % Versione più scura
\definecolor{LightGray}{RGB}{245, 245, 250}
\definecolor{MediumGray}{RGB}{200, 200, 210}

% 2. IMPOSTAZIONE MARGINI
\geometry{a4paper, left=2.5cm, right=2.5cm, top=3.5cm, bottom=3.5cm}

% 3. STILE DEI TITOLI DI SEZIONE
\titleformat{\section}
  {\normalfont\sffamily\Large\bfseries\color{AccentColor}}
  {\thesection}
  {1em}
  {}
\titleformat{\subsection}
  {\normalfont\sffamily\large\bfseries\color{AccentDark}}
  {\thesubsection}
  {1em}
  {}
\titleformat{\subsubsection}
  {\normalfont\sffamily\large\bfseries\color{AccentDark}}
  {\thesubsubsection}
  {1em}
  {}

% 4. IMPOSTAZIONE HEADER E FOOTER
\pagestyle{fancy}
\fancyhf{}
\fancyhead[L]{\sffamily\bfseries\color{AccentColor}\@BYTE HOLDERS}
\fancyhead[R]{\sffamily\color{AccentColor}\thepage}
\renewcommand{\headrulewidth}{0.8pt}
\renewcommand{\headrule}{\color{AccentColor}\hrule width\headwidth height\headrulewidth \vskip-\headrulewidth}

% 5. IMPOSTAZIONE LINK
\hypersetup{
    colorlinks=true,
    linkcolor=AccentColor,
    urlcolor=AccentLight,
    citecolor=AccentDark,
}

% 6. PERSONALIZZAZIONE ELENCHI
\setlist[itemize]{itemsep=2pt, topsep=4pt}
\setlist[enumerate]{itemsep=2pt, topsep=4pt}

% ====== COMANDI PERSONALIZZATI ======
\makeatletter
\newcommand{\NomeGruppo}[1]{\def\@NomeGruppo{#1}}
\newcommand{\TitoloVerbale}[1]{\def\@TitoloVerbale{#1}}
\newcommand{\Sommario}[1]{\def\@Sommario{#1}}
\newcommand{\Autore}[1]{\def\@Autore{#1}}
\newcommand{\Verificatore}[1]{\def\@Verificatore{#1}}
\makeatother

% ====== STILE TABELLE MIGLIORATO ======
\newcolumntype{Y}{>{\raggedright\arraybackslash}X} % Colonna giustificata a sinistra
\setlength{\arrayrulewidth}{0.4pt} % Linee più sottili
\setlength{\tabcolsep}{10pt} % Spaziatura interna celle
\renewcommand{\arraystretch}{1.4} % Altezza righe

% ====== INIZIO DEL DOCUMENTO ======
\begin{document}


% ====== PAGINA DI TITOLO ======
\begin{titlepage}
    \centering
    
    \includegraphics[width=0.55\textwidth]{../../../Assets/ByteHolders1.png}\par\vspace{1.5cm}
    
    {\LARGE \sffamily \color{AccentColor}\bfseries Verbale interno}\par
    \vspace{0.5cm}
    {\large \color{AccentColor}\sffamily 14 Gennaio 2026}\par
    
    \vfill
    
    \noindent\color{AccentColor}\rule{\textwidth}{1pt}\par
    \vspace{0.5cm}
    
    \begin{tabularx}{0.9\textwidth}{@{}>{\bfseries\sffamily}l X@{}}
    Autore & \sffamily Giulia Romanato\\
    \arrayrulecolor{MediumGray}\hline \\[-1.5ex]
    Verificatore & \sffamily \\
    \arrayrulecolor{MediumGray}\hline \\[-1.5ex] 
    Approvazione & \sffamily \\ 
    \arrayrulecolor{MediumGray}\hline 
\end{tabularx}
    
    \vfill
\end{titlepage}

% ====== INDICE ======
\pagestyle{fancy}
\newpage
\tableofcontents
\newpage

% ====== TABELLA DI VERSIONAMENTO ======
\section{Registro delle versioni}
\begin{center}
    \rowcolors{2}{LightGray}{white}
    \begin{tabular}{>{\centering\arraybackslash}m{2cm} >{\centering\arraybackslash}m{2cm} >{\raggedright\arraybackslash}m{2.5cm} >{\raggedright\arraybackslash}m{6.5cm}}
        \rowcolor{AccentColor}
          \textcolor{white}{\textbf{Versione}} & 
          \textcolor{white}{\textbf{Data}} & 
          \multicolumn{1}{c}{\textcolor{white}{\textbf{Autore}}} & 
          \multicolumn{1}{c}{\textcolor{white}{\textbf{Descrizione delle modifiche}}} \\
        1.0.0 & 25/02/2026 &  & Approvazione del verbale \\
        0.1.0 & 25/02/2026 &  & Verifica del verbale \\
        0.0.1 & 25/02/2026 & Giulia Romanato & Stesura del verbale \\ 
        
        & & & \\
    \end{tabular}
\end{center}

%\vspace{1cm}

% ====== SEZIONE INFORMAZIONI INTRODUTTIVE ======
\section{Informazioni introduttive}

\subsection{Durata e luogo}
\begin{itemize}
    \item \textbf{Inizio:} 12:30
    \item \textbf{Fine:} 14:30
    \item \textbf{Luogo:} Chiamata Discord
\end{itemize}

% ====== TABELLA PRESENZE ======
\subsection{Partecipanti}
\begin{center}
    \rowcolors{2}{LightGray}{white}
    \begin{tabular}{>{\raggedright\arraybackslash}p{6cm} c c}
        \rowcolor{AccentColor}
        \textcolor{white}{\textbf{Nome e Cognome}} & 
        \textcolor{white}{\textbf{Presente}} & 
        \textcolor{white}{\textbf{Assente}} \\
        
        Damiano Berti     & \textcolor{AccentColor}{$\blacksquare$}    & $\square$        \\
        Alessandro Frison     & \textcolor{AccentColor}{$\blacksquare$}    & $\square$        \\
        Lorenzo Grolla     & \textcolor{AccentColor}{$\blacksquare$}    & $\square$        \\
        Nicolò Lattanzio    & \textcolor{AccentColor}{$\blacksquare$}    & $\square$        \\
        Alessandro Morabito   & \textcolor{AccentColor}{$\blacksquare$}    & $\square$        \\
        Giacomo Nalotto   & \textcolor{AccentColor}{$\blacksquare$}    & $\square$        \\
        Giulia Romanato   & \textcolor{AccentColor}{$\blacksquare$}    & $\square$        \\
    \end{tabular}
\end{center}

% ====== SEZIONE PRINCIPALE DEL VERBALE ======
\section{Contenuto della riunione}

\subsection{Ordine del giorno}
\begin{enumerate}
    \item Retrospettiva dello Sprint
    \item Revisone casi d'uso identificati e organizzazzione della stesura del documento di Analisi Requisiti
    \item Decisione su come sviluppare il PoC
\end{enumerate}

\subsection{Riassunto della discussione}
\subsubsection{Retrospettiva}
Durante la retrospettiva del terzo Sprint, svoltosi dal 30/12/2025 al 12/01/2026, sono emerse le seguenti riflessioni:
\begin{itemize}
    \item Nella stesura dei documenti sono emerse difficoltà nell'individuare i contenuti da inserire; è stata evidenziata la necessità di definire preventivamente un indice degli argomenti.
    \item Comunicazione non sempre efficace tra i membri incaricati della redazione dei diversi documenti.
    \item Difficoltà nello studio delle tecnologie dovute alla mancanza di account; di conseguenza non è stato possibile realizzare il PoC.
    \item Ulteriori difficoltà legate all'approfondimento e alla comprensione delle tecnologie adottate.
    \item Ritardo nell'avvio della stesura ufficiale dell'Analisi dei Requisiti a causa di dubbi relativi agli attori e della mancanza di riscontro da parte dell'azienda e del professore.
\end{itemize}

Al termine della retrospettiva è stato individuato come responsabile del successivo Sprint (dal 13/01/2026 al 03/02/2026) Alessandro Frison. Sono stati inoltre definiti i seguenti obiettivi dello Sprint:
\begin{itemize}
    \item Sviluppo del PoC.
    \item Stesura dell'Analisi dei Requisiti.
    \item Completamento del Piano di Qualifica (parte relativa all'Analisi dei Requisiti).
    \item Revisione delle Norme di Progetto: nomenclatura dell'Analisi dei Requisiti e misure di qualità.
    \item Consegna RTB.
\end{itemize}

\subsubsection{Discussione PoC e AdR}
Si è inizialmente discusso delle modalità di stesura del documento di Analisi dei Requisiti.\\
A seguito dell'incontro con l'azienda si è deciso di abbandonare l'idea di creare ruoli in maniera dinamica e di concentrarsi su tre ruoli (Project Manager, Teach Lead e Developer) ognuno con particolari permessi.
A seguito della modifica degli attori e delle nuove conoscenze acquisite sulle tecnologie e sugli strumenti utilizzabili (in particolare rispetto ai tool e ai relativi output), sono stati revisionati i casi d'uso precedentemente identificati e la loro associazione agli attori.\\
\\
Successivamente la discussione si è concentrata sulla realizzazione del PoC. Basandosi sulle bozze realizzate da alcuni membri e sugli strumenti individuati, si è deciso di sviluppare un'interfaccia semplice che permetta di avviare la scansione di una repository GitHub. Per ottenere il report sull'analisi della repository verranno utilizzati più agenti Bedrock.\\
\\
Infine, si è deciso di organizzare il lavoro suddividendo il gruppo in due sottogruppi:
\begin{itemize}
    \item uno dedicato alla stesura del documento di Analisi dei Requisiti;
    \item uno dedicato allo sviluppo del PoC.
\end{itemize}

\subsubsection{Analisi dei Requisiti}
Sottogruppo formato da: Giulia Romanato, Lorenzo Grolla, Giacomo Nalotto\\
Per la stesura del documento di Analisi dei Requisiti si è deciso di suddividere i casi d'uso tra i membri del sottogruppo e di definire regole comuni di redazione, al fine di ottenere un documento coerente e uniforme.\\
Le regole individuate sono le seguenti:
\begin{itemize}
    \item I casi d'uso saranno identificati con la sigla UCx, dove x è un numero. In base al livello di profondità verranno utilizzati i seguenti comandi LaTeX:
    \begin{itemize}
        \item 1: subsubsection
        \item 1.1: paragraph
        \item 1.1.1: subparagraph 
    \end{itemize}
    \item La struttura dei vari casi d'uso sarà la seguente:
    \begin{itemize}
        \item Attori: attori associati al caso d'uso in analisi
        \item Attori secondari: nel caso di interazione con servizi esterni
        \item Precondizioni: condizioni necessarie affinché il caso d'uso possa verificarsi
        \item Postcondizioni: condizioni che si verificano al termine dell'esecuzione del caso d'uso
        \item Scenario principale: descrizione del flusso principale; si predilige l'uso della forma attiva (es. “Il sistema mostra…” invece di “L'utente visualizza…”)
        \item Scenario alternativo: per ciascuno si indica un nome significativo, la condizione di attivazione ed eventualmente un breve flusso descrittivo; si distinguono:
        \begin{itemize}
            \item estensioni: scenari che portano a postcondizioni differenti
            \item stato del sistema differente da quallo atteso, esempio in caso si errore invece di visualizzare una certa informazione vine visualizzato un messaggio di errore
        \end{itemize}
        \item Inclusioni: inserimento solo del riferimento al caso d'uso incluso, usando label{UCx} e hyperref[UCx]{Titolo link}
        \item Estensioni: inserimento solo del riferimento al caso d'uso di estensione
        \item Eredita da/UC che ereditano: in caso di generalizzazione 
        \begin{itemize}
            \item nel caso padre: si mette UC che ereditano a seguito di cui si indicano i casi d'uso che ereditano;
            \item nei casi figli: si mette eredita da seguito dal riferimento al caso d'uso padre.
        \end{itemize} 
    \end{itemize}
    \item Per quanto riguarda i diagrammi:
    \begin{itemize}
        \item se sono presenti solo relazioni di include, verrà creato un unico diagramma;
        \item se sono presenti anche extend:
        \begin{itemize}
            \item se vi sono al massimo uno o due extend e un include, possono essere inseriti nello stesso diagramma;
            \item altrimenti si creeranno diagrammi separati per extend e include, il diagramma degli include viene inserito a seguito della descrizione dello scenario principale
        \end{itemize}
    \end{itemize}
\end{itemize}


\subsubsection{Poc} 
Sottogruppo formato da: Alessandro Frison, Damiano Berti, Nicolò Lattanzio, Alessandro Morabito
Si è deciso di strutturare il PoC come una catena composta dai seguenti passaggi:
\begin{enumerate}
    \item Analisi statica del codice per la sicurezza tramite Semgrep (con riferimento a OWASP).
    \item Analisi della test coverage tramite Jest per i progetti TypeScript che lo utilizzano; l'esito dei test dipende dall'output fornito da Jest.
    \item I risultati dei due test precedenti vengono riassunti da un agente che li rende leggibili, fornendo in input il file JSON generato.
    \item Un ulteriore agente analizza il README (senza prompt engineering specifico) e produce una valutazione descrittiva.
    \item Tramite chiamata alla REST API di GitHub (utilizzando la libreria Octokit) vengono ottenuti i linguaggi utilizzati e le rispettive percentuali, che vengono aggiunte al report finale.
    \item Il report viene salvato in un database.
    \item Il report viene letto dal database per la restituzione al frontend.
    \item Il frontend renderizza il report in formato Markdown (da migliorare).
    \item Dal frontend è possibile aggiungere una repository al database; questa viene automaticamente clonata.
\end{enumerate} 
Nel PoC è stato deciso di non implentare filtri per la selezione delle scansioni da eseguire, non effettuare un'anlisi delle best pactice e non implementata l'analisi delle librerie.

% ====== SEZIONE DECISIONI E AZIONI ======
\section{Decisioni e azioni} \label{DecisioniAzioni}
\begin{center}
    \rowcolors{2}{LightGray}{white}
    
    \begin{longtable}{>{\centering\arraybackslash}m{3cm} >{\raggedright\arraybackslash}p{7cm} >{\centering\arraybackslash}p{2.5cm}}
    
    \rowcolor{AccentColor}
    \textcolor{white}{\textbf{Codice}} & 
    \multicolumn{1}{c}{\textcolor{white}{\textbf{Descrizione}}} & 
    \textcolor{white}{\textbf{Assegnatario}} \\
    \endfirsthead
    
    \rowcolor{AccentColor}
    \textcolor{white}{\textbf{Codice}} & 
    \multicolumn{1}{c}{\textcolor{white}{\textbf{Descrizione}}} & 
    \textcolor{white}{\textbf{Assegnatario}} \\
    \endhead
        
        DEC-RTB-041 & Decisione delle regole per la stesura del documento di Analisi dei Requisiti & Giulia Romanato, Lorenzo Grolla, Giacomo Nalotto \\
        DEC-RTB-042 & Decisione della struttura del PoC & Alessandro Frison, Damiano Berti, Nicolò Lattanzio, Alessandro Morabito \\
        \midrule
        AZ-RTB-035 & AdR: stesura casi d'uso sulla gestione account e autenticazione (UC1-UC4) & Lorenzo Grolla\\
        AZ-RTB-036 & AdR: stesura casi d'uso sui workspace (UC5-UC13) & Giacomo Nalotto\\
        AZ-RTB-037 & AdR: stesura casi d'uso sulla gestione delle repository e dei tag (UC19-UC32) & Giulia Romanato\\
        AZ-RTB-038 & AdR: stesura casi d'uso di visualizzazzione del dettaglio della singola repository (UC33) & Giacomo Nalotto\\
        AZ-RTB-039 & AdR: stesura casi d'uso sulla visone aggregata (UC35-UC36) & Lorenzo Grolla\\
        AZ-RTB-040 & AdR: stesura casi d'uso sulla scansione (UC37-UC38) & Giulia Romanato\\
        AZ-RTB-041 & PoC: ottenimento innfomazioni su repository & Alessandro Morabito\\
        AZ-RTB-042 & PoC: git clone & Alessandro Morabito\\
        AZ-RTB-043 & PoC: creazione agenti aws & Alessandro Morabito, Nicolò Lattanzio, Damiano Berti, Alessandro Frison\\
        AZ-RTB-044 & PoC: feedback scansione & Alessandro Morabito\\
        AZ-RTB-045 & PoC: container MongoDB & Alessandro Morabito\\
        AZ-RTB-046 & PoC: agente sulla documantazione & Damiano Berti\\
        AZ-RTB-047 & PoC: selezione repository & Nicolò Lattanzio\\
        AZ-RTB-048 & PoC: setup Sempgrep & Nicolò Lattanzio\\
        AZ-RTB-049 & PoC: test coverage & Alessandro Frison\\
        AZ-RTB-050 & PoC: storico scansioni & Nicolò Lattanzio\\
        AZ-RTB-051 & PoC: lancio scansione & Nicolò Lattanzio, Alessandro Morabito\\
        AZ-RTB-052 & PoC: elenco delle dipendenze & Nicolò Lattanzio\\
        AZ-RTB-053 & PoC: remediation & Nicolò Lattanzio\\
    \end{longtable}
\end{center}

\end{document}