\documentclass[a4paper, 11pt]{article}

% ====== PACCHETTI NECESSARI ======
\usepackage[utf8]{inputenc}
\usepackage[T1]{fontenc}
\usepackage[italian]{babel}
\usepackage{geometry}
\usepackage{graphicx}
\usepackage[table]{xcolor}
\usepackage{tabularx}
\usepackage{array}
\usepackage{amssymb}
\usepackage{fancyhdr}
\setlength{\headheight}{14pt}
\usepackage{titlesec}
\usepackage{helvet}
\renewcommand{\familydefault}{\sfdefault}
\usepackage{lipsum}
\usepackage{hyperref}
\usepackage{booktabs}
\usepackage{enumitem}
\usepackage[utf8]{inputenc} % Specifica la codifica del file (necessaria per le accentate)
\usepackage[T1]{fontenc}    % Migliora l'output dei font per le lingue europee
\usepackage{graphicx} % Necessario per \rotatebox
\usepackage{multirow}
\usepackage{float}

% Comando rapido per ruotare il testo di 45 gradi
\newcommand{\rot}[1]{%
  \multicolumn{1}{c}{%
    \makebox[0pt][l]{%
      \rotatebox[origin=bl]{45}{#1}%
    }%
  }%
}


% ====== IMPOSTAZIONI GLOBALI DI STILE ======

% 1. DEFINIZIONE COLORI BLU-VIOLA
\definecolor{AccentColor}{RGB}{80, 90, 180} % Blu-viola principale
\definecolor{AccentLight}{RGB}{80, 90, 180} % Versione più chiara
\definecolor{AccentDark}{RGB}{50, 60, 140} % Versione più scura
\definecolor{LightGray}{RGB}{245, 245, 250}
\definecolor{MediumGray}{RGB}{200, 200, 210}

% 2. IMPOSTAZIONE MARGINI
\geometry{a4paper, left=2.5cm, right=2.5cm, top=3.5cm, bottom=3.5cm}

% 3. STILE DEI TITOLI DI SEZIONE
\titleformat{\section}
  {\normalfont\sffamily\Large\bfseries\color{AccentColor}}
  {\thesection}
  {1em}
  {}
\titleformat{\subsection}
  {\normalfont\sffamily\large\bfseries\color{AccentDark}}
  {\thesubsection}
  {1em}
  {}
\titleformat{\subsubsection}
  {\normalfont\sffamily\large\bfseries\color{AccentDark}}
  {\thesubsubsection}
  {1em}
  {}

% --- CONFIGURAZIONE LIVELLI AGGIUNTIVI (4 e 5) ---
\setcounter{secnumdepth}{5} 
\setcounter{tocdepth}{5}

\titleformat{\paragraph}
  {\normalfont\sffamily\normalsize\bfseries\color{AccentDark}}
  {\theparagraph}
  {1em}
  {}
\titlespacing*{\paragraph}
  {0pt}
  {3.25ex plus 1ex minus .2ex}
  {1.5ex plus .2ex}

% Alias comando
\newcommand{\subsubsubsection}[1]{\paragraph{#1}}

\titleformat{\subparagraph}
  {\normalfont\sffamily\normalsize\bfseries\color{AccentDark}} % Stile
  {\thesubparagraph}
  {1em}
  {}
\titlespacing*{\subparagraph}
  {0pt}
  {3.25ex plus 1ex minus .2ex}
  {1.5ex plus .2ex}

% Alias comando
\newcommand{\subsubsubsubsection}[1]{\subparagraph{#1}}



% 4. IMPOSTAZIONE HEADER E FOOTER
\pagestyle{fancy}
\fancyhf{} 
\fancyhead[L]{\sffamily\bfseries\color{AccentColor}\@BYTE HOLDERS}
\fancyhead[R]{\sffamily\color{AccentColor}\thepage}
\renewcommand{\headrulewidth}{0.8pt}
\renewcommand{\headrule}{\color{AccentColor}\hrule width\headwidth height\headrulewidth \vskip-\headrulewidth}

% 5. IMPOSTAZIONE LINK
\hypersetup{
    colorlinks=true,
    linkcolor=AccentColor,
    urlcolor=AccentLight,
    citecolor=AccentDark,
}

% 6. PERSONALIZZAZIONE ELENCHI
\setlist[itemize]{itemsep=2pt, topsep=4pt}
\setlist[enumerate]{itemsep=2pt, topsep=4pt}

% ====== COMANDI PERSONALIZZATI ======
\makeatletter
\newcommand{\NomeGruppo}[1]{\def\@NomeGruppo{#1}}
\newcommand{\TitoloVerbale}[1]{\def\@TitoloVerbale{#1}}
\newcommand{\Sommario}[1]{\def\@Sommario{#1}}
\newcommand{\Autore}[1]{\def\@Autore{#1}}
\newcommand{\Verificatore}[1]{\def\@Verificatore{#1}}
\makeatother

% ====== STILE TABELLE MIGLIORATO ======
\newcolumntype{Y}{>{\raggedright\arraybackslash}X} % Colonna giustificata a sinistra
\setlength{\arrayrulewidth}{0.4pt} % Linee più sottili
\setlength{\tabcolsep}{10pt} % Spaziatura interna celle
\renewcommand{\arraystretch}{1.4} % Altezza righe

% ====== INIZIO DEL DOCUMENTO ======
\begin{document}

% ====== INFORMAZIONI PER LA PAGINA DI TITOLO ======
\NomeGruppo{BYTE HOLDERS}
\TitoloVerbale{Verbale Riunione Interna}
\Sommario{Questo verbale documenta la riunione interna avvenuta il 16/10/2025 per la discussione dei capitolati e la definizione degli strumenti di lavoro del team.}
\Autore{}
\Verificatore{}

\pagestyle{empty}

% ====== PAGINA DI TITOLO ======
\begin{titlepage}
    \centering
    
    \includegraphics[width=0.55\textwidth]{../Assets/ByteHolders1.png}\par\vspace{1.5cm}
    
    {\LARGE \sffamily \color{AccentColor}\bfseries Piano di Progetto}\par
    
    \vfill
    
    \noindent\color{AccentColor}\rule{\textwidth}{1pt}\par
    \vspace{0.5cm}
    
    \begin{tabularx}{0.9\textwidth}{@{}>{\bfseries\sffamily}l X@{}}
    Autori & \sffamily Giacomo Nalotto, Damiano Berti\\
    \arrayrulecolor{MediumGray}\hline \\[-1.5ex]
    Verificatori & \sffamily XXX\\
    \arrayrulecolor{MediumGray}\hline \\[-1.5ex] 
    Approvazione & \sffamily YYY\\ 
    \arrayrulecolor{MediumGray}\hline 
\end{tabularx}
    
    \vfill
\end{titlepage}

\newpage

% ====== TABELLA DI VERSIONAMENTO ======
{\normalfont\sffamily\huge\bfseries\color{AccentColor} Registro delle versioni}
\vspace{1cm}

\begin{center}
    \rowcolors{2}{LightGray}{white}
    \begin{tabular}{>{\centering\arraybackslash}m{1.6cm} >{\centering\arraybackslash}m{2cm} >{\raggedright\arraybackslash}m{1.7cm} >{\raggedright\arraybackslash}m{1.7cm} >{\raggedright\arraybackslash}m{5cm}}
        \rowcolor{AccentColor}
          \textcolor{white}{\textbf{Versione}} & 
          \textcolor{white}{\textbf{Data}} & 
          \multicolumn{1}{c}{\textcolor{white}{\textbf{Autore}}} & 
          \multicolumn{1}{c}{\textcolor{white}{\textbf{verificatore}}} &
          {\textcolor{white}{\textbf{Descrizione delle modifiche}}} \\

        0.2.1 & 21/12/2025 & Damiano Berti &  & creazione tabelle tracciamento sprint \\ 
        \hline
        0.2.0 & 20/12/2025 & Damiano Berti &  & Inserita introduzione generale, sezione "analisi e gestione dei rischi" e prima lista di rischi\\ 
        \hline
        0.1.0 & 01/12/2025 & Damiano Berti & & Inizio stesura \\ 
        
    \end{tabular}
\end{center}


% ====== INDICE ======
\pagestyle{fancy}
\newpage
\tableofcontents
\newpage

\section{Introduzione}

\subsection{informazioni generali}
Il presente documento costituisce il \textbf{Piano di Progetto} ufficiale per lo sviluppo del software Code Guardian. Esso definisce la pianificazione temporale, l'allocazione delle risorse e la metodologia di gestione adottata per garantire il rispetto degli standard qualitativi e dei vincoli contrattuali. \\[0.5cm]
Lo scopo primario è fornire una guida strutturata per il monitoraggio dell'avanzamento lavori e la mitigazione dei rischi nell'intero ciclo di vita del progetto, il presente piano sarà quindi soggetto ad aggiornamenti progressivi.

\subsection{Glossario}
Data la natura tecnica del progetto è naturale l'utilizzo di termini settoriali e talvolta anche definiti dal gruppo, per questo è stato appositamente creato un glossario presente su un documento separato, ogni parola che rimanda al glossario sarà contrassegnata come segue:
\begin{center}
  termine\textsuperscript{G}
\end{center}

\subsection{Riferimenti}
\subsubsection{Riferimenti normativi}
\begin{itemize}
  \item \href{https://www.math.unipd.it/~tullio/IS-1/2025/Progetto/C2p.pdf}{Capitolato d'appalto C2: Code Guardian - Var Group}
  \item norme di progetto
\end{itemize}
\subsubsection{Riferimenti informativi}
\begin{itemize}
  \item \href{https://www.math.unipd.it/~tullio/IS-1/2025/Dispense/T02.pdf}{T2: i processi di ciclo di vita del software}
  \item \href{https://www.math.unipd.it/~tullio/IS-1/2025/Dispense/T04.pdf}{T4: Gestione di progetto}
  \item glossario
\end{itemize}

\newpage

\section{Analisi e gestione dei rischi}
\subsection{Introduzione}
L'analisi e la gestione dei rischi rappresentano il processo strategico volto a identificare, valutare e mitigare i potenziali eventi avversi che potrebbero compromettere gli obiettivi del gruppo. \\[0.4cm]
Identificare ed analizzare i rischi in anticipo è essenziale per guidare lo sviluppo del progetto con maggiore controllo, così da poter anticipare le criticità anziché subirle e poter meglio pianificare. Inoltre permette stimare più accuratamente il tempo necessario per le attività così da poter meglio calibrare il backlog degli sprint. \\[0.4cm]
Il processo di gestione dei rischi si articola in quattro fasi essenziali:
\begin{itemize}
  \item \textbf{Identificazione}: consiste nel rintracciare e scrivere un elenco di tutti i possibili problemi o imprevisti che potrebbero ostacolare il progetto
  \item \textbf{Analisi}: serve a capire quanto ogni rischio sia , valutando quante probabilità ha di accadere e quanti danni potrebbe causare.
  \item \textbf{Pianificazione}: Rappresenta l’insieme delle azioni di mitigazione pianificate per ridurre la probabilità che un rischio si manifesti o per limitarne i possibili effetti dannosi.
  \item \textbf{Constrollo}È l’attività di monitoraggio continuo che verifica nel tempo l'efficacia delle misure di prevenzione adottate. Il controllo serve a garantire che i rischi restino entro livelli accettabili, a rilevare l'insorgenza di nuove minacce e ad aggiornare dinamicamente le startegie di mitigazione in base all'evoluzione del progetto.
\end{itemize}
\vspace{0.4cm}
Il gruppo Byte Holders indivudua la seguente lista di possibili rischi, suddividendoli per tipo con annesso codice identificativo:
\begin{itemize}
  \item \textbf{T} rischi teconologici
  \item \textbf{I} rischi individuali
  \item \textbf{O} rischi organizzativi
\end{itemize}
\vspace{0.4cm}
Ogni rischio è quindi contrassegnato con un codice composto da un prefisso indicante il tipo ed una cifra univoca.

\newpage

\subsection{Rischi tecnologici}
\subsubsection{T1 - Inesperienza con le nuove tecnologie}
\begin{center}
  \rowcolors{2}{LightGray}{white}
  \begin{tabular}{|m{5cm}|m{8cm}|}
      \hline
      \multicolumn{2}{|>{\centering\arraybackslash}m{14cm}|}{\cellcolor{AccentColor}\textcolor{white}{\textbf{T1 - Inesperienza con le nuove tecnologie}}} \\ 
      \hline
      Descrizione &  \\ 
      \hline
      Mitigazione &  \\ 
      \hline
      Probabilità di occorrenza &  \\
      \hline
      Pericolosità &  \\
      \hline
  \end{tabular}
\end{center}

\subsubsection{T2 - Mancanza della documentazione necessaria}
\begin{center}
  \rowcolors{2}{LightGray}{white}
  \begin{tabular}{|m{5cm}|m{8cm}|}
      \hline
      \multicolumn{2}{|>{\centering\arraybackslash}m{14cm}|}{\cellcolor{AccentColor}\textcolor{white}{\textbf{T2 - Guasti tecnici hardware o software}}} \\ 
      \hline
      Descrizione &  \\ 
      \hline
      Mitigazione &  \\ 
      \hline
      Probabilità di occorrenza &  \\
      \hline
      Pericolosità &  \\
      \hline
  \end{tabular}
\end{center}

\subsubsection{T2 - Guasti tecnici hardware o software}
\begin{center}
  \rowcolors{2}{LightGray}{white}
  \begin{tabular}{|m{5cm}|m{8cm}|}
      \hline
      \multicolumn{2}{|>{\centering\arraybackslash}m{14cm}|}{\cellcolor{AccentColor}\textcolor{white}{\textbf{T2 - Guasti tecnici hardware o software}}} \\ 
      \hline
      Descrizione &  \\ 
      \hline
      Mitigazione &  \\ 
      \hline
      Probabilità di occorrenza &  \\
      \hline
      Pericolosità &  \\
      \hline
  \end{tabular}
\end{center}

\subsection{Rischi individuali}
\subsubsection{I1 - Impegno personale imprevisto}
\begin{center}
  \rowcolors{2}{LightGray}{white}
  \begin{tabular}{|m{5cm}|m{8cm}|}
      \hline
      \multicolumn{2}{|>{\centering\arraybackslash}m{14cm}|}{\cellcolor{AccentColor}\textcolor{white}{\textbf{I1 - Impegno personale imprevisto}}} \\ 
      \hline
      Descrizione &  \\ 
      \hline
      Mitigazione &  \\ 
      \hline
      Probabilità di occorrenza &  \\
      \hline
      Pericolosità &  \\
      \hline
  \end{tabular}
\end{center}

\subsubsection{I2 - Mancanza di tempo dovuto ad altre attività universitarie}
\begin{center}
  \rowcolors{2}{LightGray}{white}
  \begin{tabular}{|m{5cm}|m{8cm}|}
      \hline
      \multicolumn{2}{|>{\centering\arraybackslash}m{14cm}|}{\cellcolor{AccentColor}\textcolor{white}{\textbf{I2 - Mancanza di tempo dovuto ad atre attività universitarie}}} \\ 
      \hline
      Descrizione &  \\ 
      \hline
      Mitigazione &  \\ 
      \hline
      Probabilità di occorrenza &  \\
      \hline
      Pericolosità &  \\
      \hline
  \end{tabular}
\end{center}


\subsection{Rischi organizzativi}
\subsubsection{O1 - Sottostima del tempo necessario per una task}
\begin{center}
  \rowcolors{2}{LightGray}{white}
  \begin{tabular}{|m{5cm}|m{8cm}|}
      \hline
      \multicolumn{2}{|>{\centering\arraybackslash}m{14cm}|}{\cellcolor{AccentColor}\textcolor{white}{\textbf{O1 - Sottostima del tempo necessario per una task}}} \\ 
      \hline
      Descrizione &  \\ 
      \hline
      Mitigazione &  \\ 
      \hline
      Probabilità di occorrenza &  \\
      \hline
      Pericolosità &  \\
      \hline
  \end{tabular}
\end{center}

\subsubsection{O2 - Sovrastima del tempo necessario per una task}
\begin{center}
  \rowcolors{2}{LightGray}{white}
  \begin{tabular}{|m{5cm}|m{8cm}|}
      \hline
      \multicolumn{2}{|>{\centering\arraybackslash}m{14cm}|}{\cellcolor{AccentColor}\textcolor{white}{\textbf{O2 - Sovrastima del tempo necessario per una task}}} \\ 
      \hline
      Descrizione &  \\ 
      \hline
      Mitigazione &  \\ 
      \hline
      Probabilità di occorrenza &  \\
      \hline
      Pericolosità &  \\
      \hline
  \end{tabular}
\end{center}

\subsubsection{O3 - Problemi comunicativi interni al gruppo o con la proponente}
\begin{center}
  \rowcolors{2}{LightGray}{white}
  \begin{tabular}{|m{5cm}|m{8cm}|}
      \hline
      \multicolumn{2}{|>{\centering\arraybackslash}m{14cm}|}{\cellcolor{AccentColor}\textcolor{white}{\textbf{O3 - Problemi comunicativi interni al gruppo o con la proponente}}} \\ 
      \hline
      Descrizione &  \\ 
      \hline
      Mitigazione &  \\ 
      \hline
      Probabilità di occorrenza &  \\
      \hline
      Pericolosità &  \\
      \hline
  \end{tabular}
\end{center}

\subsubsection{O4 - Gestione errata dei tempi e del budget}
\begin{center}
  \rowcolors{2}{LightGray}{white}
  \begin{tabular}{|m{5cm}|m{8cm}|}
      \hline
      \multicolumn{2}{|>{\centering\arraybackslash}m{14cm}|}{\cellcolor{AccentColor}\textcolor{white}{\textbf{O4 - Gestione errata dei tempi e del budget}}} \\ 
      \hline
      Descrizione &  \\ 
      \hline
      Mitigazione &  \\ 
      \hline
      Probabilità di occorrenza &  \\
      \hline
      Pericolosità &  \\
      \hline
  \end{tabular}
\end{center}



\newpage




\section{Pianificazione nel lungo termine}
\section{Pianificazione nel breve termine}
\subsection{Introduzione}
\subsection{Requirements and Technology Baseline (RTB)}
\subsubsection{Sprint1}
\subsubsubsection{informazioni generali e attività da svolgere}
\subsubsubsection{rischi attesi}
\subsubsubsection{Preventivo}
\begin{table}[H]
    \centering
    \renewcommand{\arraystretch}{1.5}
    
    \begin{tabular}{|m{3.5cm}|*{6}{>{\centering\arraybackslash}m{1.2cm}|}}
        
        \multicolumn{1}{c}{} & 
        \multicolumn{1}{c}{\rule{0pt}{2cm}\makebox[0pt][l]{\rotatebox[origin=bl]{45}{\textcolor{AccentColor}{\textbf{Responsabile}}}}} & 
        
        \rot{\textcolor{AccentColor}{\textbf{Amministratore}}} & 
        \rot{\textcolor{AccentColor}{\textbf{Analista}}} & 
        \rot{\textcolor{AccentColor}{\textbf{Progettista}}} & 
        \rot{\textcolor{AccentColor}{\textbf{Programmatore}}} & 
        \rot{\textcolor{AccentColor}{\textbf{Verificatore}}} \\ \hline 
        
        Giacomo Nalotto     & - & 2 & - & - & - & - \\ \hline
        Damiano Berti       & - & 3 & 2 & - & - & 1 \\ \hline
        Alessandro Morabito & - & - & - & - & - & - \\ \hline
        Alessandro Frison   & - & 1 & - & - & - & - \\ \hline
        Giulia Romanato     & - & - & 2 & - & - & 1 \\ \hline
        Nicolò Lattanzio    & 2 & 2 & - & - & - & - \\ \hline
        Lorenzo Grolla      & - & 2 & - & - & - & 1 \\ \hline
    \end{tabular}
\end{table}
\subsubsubsection{Consuntivo}
\begin{table}[H]
    \centering
    \renewcommand{\arraystretch}{1.5}
    
    \begin{tabular}{|m{3.5cm}|*{6}{>{\centering\arraybackslash}m{1.2cm}|}}
        
        \multicolumn{1}{c}{} & 
        \multicolumn{1}{c}{\rule{0pt}{2.5cm}\makebox[0pt][l]{\rotatebox[origin=bl]{45}{\textcolor{AccentColor}{\textbf{Responsabile}}}}} & 
        
        \rot{\textcolor{AccentColor}{\textbf{Amministratore}}} & 
        \rot{\textcolor{AccentColor}{\textbf{Analista}}} & 
        \rot{\textcolor{AccentColor}{\textbf{Progettista}}} & 
        \rot{\textcolor{AccentColor}{\textbf{Programmatore}}} & 
        \rot{\textcolor{AccentColor}{\textbf{Verificatore}}} \\ \hline 
        
        Giacomo Nalotto     & - & 2 & - & - & - & - \\ \hline
        Damiano Berti       & - & 3 & 2 & - & - & 1 \\ \hline
        Alessandro Morabito & - & - & - & - & - & - \\ \hline
        Alessandro Frison   & - & 2 \textcolor{red}{(+1)} & - & - & - & - \\ \hline
        Giulia Romanato     & - & - & 2 & - & - & 1 \\ \hline
        Nicolò Lattanzio    & 3 \textcolor{red}{(+1)} & 2 & - & - & - & - \\ \hline
        Lorenzo Grolla      & - & 2 & - & - & - & 1 \\ \hline
    \end{tabular}
\end{table}
\subsubsubsection{Aggiornamento delle risorse rimanenti}
\begin{table}[H]
    \centering
    \renewcommand{\arraystretch}{1.5}
    
    \begin{tabular}{|c|c|c|c|c|c|}
        \hline
        \textcolor{AccentColor}{\textbf{Ruolo}} &
        \textcolor{AccentColor}{\textbf{Costo}} &
        \textcolor{AccentColor}{\textbf{Ore}} &
        \textcolor{AccentColor}{\textbf{Costo}} &
        \textcolor{AccentColor}{\textbf{Ore rimanenti}} &
        \textcolor{AccentColor}{\textbf{Budget Rimanente}} \\ \hline
        
        Responsabile & 30€/h & 3 & 90€ &  63 \textcolor{red}{(-3)} &  1.890€ \textcolor{red}{(-90€)}\\ \hline
        
        Amministratore & 20€/h & 11 & 220€ & 39 \textcolor{red}{(-11)} &  780€ \textcolor{red}{(-220€)} \\ \hline
        
        Analista & 25€/h & 4 & 100€ &  96 \textcolor{red}{(-4)} &  2.400€ \textcolor{red}{(-100€)}\\ \hline
        
        Progettista & 25€/h & 0 & 0€ & 160 &  4.000€ \\ \hline
        
        Programmatore & 15€/h & 0 & 0€ &  127 &  1.905€ \\ \hline
        
        Verificatore & 15€/h & 3 & 45€ &  124 \textcolor{red}{(-3)}&  1.860€ \textcolor{red}{(-45€)} \\ \hline
        
        \textbf{Totale} & - & 21 & \textbf{455€} & \textbf{609} \textcolor{red}{(-21)} & \textbf{12.835€} \textcolor{red}{(-455€)}\\ \hline
        
    \end{tabular}
\end{table}
\subsubsubsection{Rischi incontrati}
\subsubsubsection{Retrospettiva}

\end{document}