\documentclass[a4paper, 11pt]{article}

% ====== PACCHETTI NECESSARI ======
\usepackage[utf8]{inputenc}
\usepackage[T1]{fontenc}
\usepackage[italian]{babel}
\usepackage{geometry}
\usepackage{graphicx}
\usepackage[table]{xcolor}
\usepackage{tabularx}
\usepackage{array}
\usepackage{amssymb}
\usepackage{fancyhdr}
\usepackage{titlesec}
\usepackage{helvet}
\renewcommand{\familydefault}{\sfdefault}
\usepackage{lipsum}
\usepackage{hyperref}
\usepackage{booktabs}
\usepackage{enumitem}
\usepackage[utf8]{inputenc} % Specifica la codifica del file (necessaria per le accentate)
\usepackage[T1]{fontenc}    % Migliora l'output dei font per le lingue europee

% ====== IMPOSTAZIONI GLOBALI DI STILE ======

% 1. DEFINIZIONE COLORI BLU-VIOLA
\definecolor{AccentColor}{RGB}{80, 90, 180} % Blu-viola principale
\definecolor{AccentLight}{RGB}{80, 90, 180} % Versione più chiara
\definecolor{AccentDark}{RGB}{50, 60, 140} % Versione più scura
\definecolor{LightGray}{RGB}{245, 245, 250}
\definecolor{MediumGray}{RGB}{200, 200, 210}

% 2. IMPOSTAZIONE MARGINI
\geometry{a4paper, left=2.5cm, right=2.5cm, top=3.5cm, bottom=3.5cm}

% 3. STILE DEI TITOLI DI SEZIONE
\titleformat{\section}
  {\normalfont\sffamily\Large\bfseries\color{AccentColor}}
  {\thesection}
  {1em}
  {}
\titleformat{\subsection}
  {\normalfont\sffamily\large\bfseries\color{AccentDark}}
  {\thesubsection}
  {1em}
  {}

% 4. IMPOSTAZIONE HEADER E FOOTER
\pagestyle{fancy}
\fancyhf{}
\fancyhead[L]{\sffamily\bfseries\color{AccentColor}\@BYTE HOLDERS}
\fancyhead[R]{\sffamily\color{AccentColor}\thepage}
\renewcommand{\headrulewidth}{0.8pt}
\renewcommand{\headrule}{\color{AccentColor}\hrule width\headwidth height\headrulewidth \vskip-\headrulewidth}

% 5. IMPOSTAZIONE LINK
\hypersetup{
    colorlinks=true,
    linkcolor=AccentColor,
    urlcolor=AccentLight,
    citecolor=AccentDark,
}

% 6. PERSONALIZZAZIONE ELENCHI
\setlist[itemize]{itemsep=2pt, topsep=4pt}
\setlist[enumerate]{itemsep=2pt, topsep=4pt}

% ====== COMANDI PERSONALIZZATI ======
\makeatletter
\newcommand{\NomeGruppo}[1]{\def\@NomeGruppo{#1}}
\newcommand{\TitoloVerbale}[1]{\def\@TitoloVerbale{#1}}
\newcommand{\Sommario}[1]{\def\@Sommario{#1}}
\newcommand{\Autore}[1]{\def\@Autore{#1}}
\newcommand{\Verificatore}[1]{\def\@Verificatore{#1}}
\makeatother

% ====== STILE TABELLE MIGLIORATO ======
\newcolumntype{Y}{>{\raggedright\arraybackslash}X} % Colonna giustificata a sinistra
\setlength{\arrayrulewidth}{0.4pt} % Linee più sottili
\setlength{\tabcolsep}{10pt} % Spaziatura interna celle
\renewcommand{\arraystretch}{1.4} % Altezza righe

% ====== INIZIO DEL DOCUMENTO ======
\begin{document}

% ====== INFORMAZIONI PER LA PAGINA DI TITOLO ======
\NomeGruppo{BYTE HOLDERS}
\TitoloVerbale{Verbale Riunione Interna}
\Sommario{Questo verbale documenta la riunione interna avvenuta il 16/10/2025 per la discussione dei capitolati e la definizione degli strumenti di lavoro del team.}
\Autore{}
\Verificatore{}

\pagestyle{empty}

% ====== PAGINA DI TITOLO ======
\begin{titlepage}
    \centering
    
    \includegraphics[width=0.55\textwidth]{../../../Assets/ByteHolders1.png}\par\vspace{1.5cm}
    
    {\LARGE \sffamily \color{AccentColor}\bfseries Verbale interno}\par
    \vspace{0.5cm}
    {\large \color{AccentColor}\sffamily 16 Ottobre 2025}\par
    
    \vfill
    
    \noindent\color{AccentColor}\rule{\textwidth}{1pt}\par
    \vspace{0.5cm}
    
    \begin{tabularx}{0.9\textwidth}{@{}>{\bfseries\sffamily}l X@{}}
    Autore & \sffamily Alessandro Frison\\
    \arrayrulecolor{MediumGray}\hline \\[-1.5ex]
    Verificatore & \sffamily Nicolò Lattanzio\\
    \arrayrulecolor{MediumGray}\hline \\[-1.5ex] 
    Approvazione & \sffamily Giacomo Nalotto\\ 
    \arrayrulecolor{MediumGray}\hline 
\end{tabularx}
    
    \vfill
\end{titlepage}

% ====== INDICE ======
\pagestyle{fancy}
\newpage
\tableofcontents
\newpage

% ====== TABELLA DI VERSIONAMENTO ======
\section{Registro delle versioni}
\begin{center}
    \rowcolors{2}{LightGray}{white}
    \begin{tabular}{>{\centering\arraybackslash}m{1.5cm} >{\centering\arraybackslash}m{2cm} >{\raggedright\arraybackslash}m{2.5cm} >{\raggedright\arraybackslash}m{6.5cm}}
        \rowcolor{AccentColor}
          \textcolor{white}{\textbf{Versione}} & 
          \textcolor{white}{\textbf{Data}} & 
          \multicolumn{1}{c}{\textcolor{white}{\textbf{Autore}}} & 
          \multicolumn{1}{c}{\textcolor{white}{\textbf{Descrizione delle modifiche}}} \\

        2.1& 19/10/2025& Giacomo Nalotto & Revisione e approvazione del verbale.\\
        2.0& 19/10/2025& Nicolò Lattanzio & Verifica del verbale e correzione di alcuni piccoli dettagli.\\
        1.1& 17/10/2025& Alessandro Frison & Compilazione verbale.\\
        1.0 & 16/10/2025 & Nicolò Lattanzio & Creazione del verbale della prima riunione. \\ 
        
        & & & \\
    \end{tabular}
\end{center}

%\vspace{1cm}

% ====== SEZIONE INFORMAZIONI INTRODUTTIVE ======
\section{Informazioni introduttive}

\subsection{Durata e luogo}
\begin{itemize}
    \item \textbf{Inizio:} 17:15
    \item \textbf{Fine:} 19:00
    \item \textbf{Luogo:} Chiamata Discord
\end{itemize}

% ====== TABELLA PRESENZE ======
\subsection{Partecipanti}
\begin{center}
    \rowcolors{2}{LightGray}{white}
    \begin{tabular}{>{\raggedright\arraybackslash}p{6cm} c c}
        \rowcolor{AccentColor}
        \textcolor{white}{\textbf{Nome e Cognome}} & 
        \textcolor{white}{\textbf{Presente}} & 
        \textcolor{white}{\textbf{Assente}} \\
        
        Damiano Berti     & \textcolor{AccentColor}{$\blacksquare$}    & $\square$        \\
        Alessandro Frison     & \textcolor{AccentColor}{$\blacksquare$}    & $\square$        \\
        Lorenzo Grolla     & \textcolor{AccentColor}{$\blacksquare$}    & $\square$        \\
        Nicolò Lattanzio    & \textcolor{AccentColor}{$\blacksquare$}    & $\square$        \\
        Alessandro Morabito   & \textcolor{AccentColor}{$\blacksquare$}    & $\square$        \\
        Giacomo Nalotto   & \textcolor{AccentColor}{$\blacksquare$}    & $\square$        \\
        Giulia Romanato   & \textcolor{AccentColor}{$\blacksquare$}    & $\square$        \\
    \end{tabular}
\end{center}

% ====== SEZIONE PRINCIPALE DEL VERBALE ======
\section{Contenuto della riunione}

\subsection{Ordine del giorno}
\begin{enumerate}
    \item Decisione dei principali canali di comunicazione.
    \item Discussione riguardante i capitolati presentati.
    \item Individuazione di un nome e un logo per il team.
    \item Creazione dell'indirizzo mail di gruppo.
    \item Creazione della repository su GitHub per la documentazione.
\end{enumerate}

\subsection{Riassunto della discussione}
Durante la riunione, che ha anche favorito la conoscenza reciproca e l'allineamento del gruppo, il team si è dedicato alla revisione dei capitolati tecnici, discutendo in merito all'interesse e alla rilevanza degli argomenti proposti e confrontandosi sulle tecnologie già in uso dal gruppo. Durante questa operazione sono state identificate alcune domande chiave che vengono elencate di seguito.\\
Successivamente, si è proceduto alla definizione dell'identità: il gruppo ha votato all'unanimità il nome \textbf{Byte Holders} e, tra le proposte presentate, ha selezionato il logo in calce al documento a maggioranza assoluta. Per quanto riguarda gli strumenti di lavoro, è stato deciso di adottare \textbf{Gmail} come dominio email e di utilizzare una organizzazione \textbf{GitHub} per l'archiviazione e la gestione della documentazione di progetto, in particolare sfruttando la funzione \textbf{GitHub Pages}.\\
In conclusione, si è stabilito di riesaminare in dettaglio ogni capitolato al fine di identificare le domande chiave necessarie per la selezione del progetto migliore a cui candidarsi e di aggregarle in una successiva riunione.

\subsection{Domande per VarGroup}
\begin{enumerate}
    \item Come l’azienda intende accompagnarci nello sviluppo del progetto, e con quali modalità
    \item Chiarimenti sull’implementazione dell’analisi di sicurezza OWASP
    \item Le repository da analizzare dovranno avere un formato standard?
    \item Alcuni chiarimenti in merito all'uso di Swagger API
    \item Chiarimenti sulle modalità di utilizzo degli agenti
\end{enumerate}

\subsection{Domande per Miriade}
\begin{enumerate}
    \item Su quali fattori viene fatta l’analisi delle situazioni di rischio? L’utente descrive una situazione all’IA ed essa risponde? 
    \item Problema privacy: l’applicazione deve avere accesso a messaggi, microfono e geolocalizzazione?
    \item Come si intende accompagnare il gruppo nello sviluppo del progetto? Con quali modalità?
\end{enumerate}

% ====== SEZIONE DECISIONI E AZIONI ======
\section{Decisioni e azioni}
\begin{center}
    \rowcolors{2}{LightGray}{white}
    \begin{tabular}{>{\centering\arraybackslash}m{2cm} >{\raggedright\arraybackslash}p{8cm} >{\centering\arraybackslash}p{2.5cm}}
        \rowcolor{AccentColor}
        \textcolor{white}{\textbf{ID Decisione}} & 
        \multicolumn{1}{c}{\textcolor{white}{\textbf{Descrizione}}} & 
        \textcolor{white}{\textbf{Assegnatario}} \\
        
        DEC-001 & Scelta di Discord come canale di comunicazione principale & Tutti \\
        DEC-002 & Creazione account Gmail per il gruppo & Alessandro Frison \\
        DEC-003 & Creazione organizzazione GitHub per la documentazione &  Alessandro Frison \\
        DEC-004 & Rilettura attenta dei capitolati con lo scopo di individuare eventuali dubbi. & Tutti \\
        & & \\
    \end{tabular}



\end{center}

\end{document}