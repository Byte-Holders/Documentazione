\documentclass[a4paper, 11pt]{article}

% ====== PACCHETTI NECESSARI ======
\usepackage[utf8]{inputenc}
\usepackage[T1]{fontenc}
\usepackage[italian]{babel}
\usepackage{geometry}
\usepackage{graphicx}
\usepackage[table]{xcolor}
\usepackage{tabularx}
\usepackage{array}
\usepackage{amssymb}
\usepackage{fancyhdr}
\usepackage{titlesec}
\usepackage{helvet}
\renewcommand{\familydefault}{\sfdefault}
\usepackage{lipsum}
\usepackage{hyperref}
\usepackage{booktabs}
\usepackage{enumitem}
\usepackage[utf8]{inputenc} % Specifica la codifica del file (necessaria per le accentate)
\usepackage[T1]{fontenc}    % Migliora l'output dei font per le lingue europee

% ====== IMPOSTAZIONI GLOBALI DI STILE ======

% 1. DEFINIZIONE COLORI BLU-VIOLA
\definecolor{AccentColor}{RGB}{80, 90, 180} % Blu-viola principale
\definecolor{AccentLight}{RGB}{80, 90, 180} % Versione più chiara
\definecolor{AccentDark}{RGB}{50, 60, 140} % Versione più scura
\definecolor{LightGray}{RGB}{245, 245, 250}
\definecolor{MediumGray}{RGB}{200, 200, 210}

% 2. IMPOSTAZIONE MARGINI
\geometry{a4paper, left=2.5cm, right=2.5cm, top=3.5cm, bottom=3.5cm}

% 3. STILE DEI TITOLI DI SEZIONE
\titleformat{\section}
  {\normalfont\sffamily\Large\bfseries\color{AccentColor}}
  {\thesection}
  {1em}
  {}
\titleformat{\subsection}
  {\normalfont\sffamily\large\bfseries\color{AccentDark}}
  {\thesubsection}
  {1em}
  {}
  \titleformat{\subsubsection}
  {\normalfont\sffamily\large\bfseries\color{AccentDark}} % Cambiare colore dellle subsubsection
  {\thesubsubsection}
  {1em}
  {}

% 4. IMPOSTAZIONE HEADER E FOOTER
\pagestyle{fancy}
\fancyhf{}
\fancyhead[L]{\sffamily\bfseries\color{AccentColor}\@BYTE HOLDERS}
\fancyhead[R]{\sffamily\color{AccentColor}\thepage}
\renewcommand{\headrulewidth}{0.8pt}
\renewcommand{\headrule}{\color{AccentColor}\hrule width\headwidth height\headrulewidth \vskip-\headrulewidth}

% 5. IMPOSTAZIONE LINK
\hypersetup{
    colorlinks=true,
    linkcolor=AccentColor,
    urlcolor=AccentLight,
    citecolor=AccentDark,
}

% 6. PERSONALIZZAZIONE ELENCHI
\setlist[itemize]{itemsep=2pt, topsep=4pt}
\setlist[enumerate]{itemsep=2pt, topsep=4pt}

% ====== COMANDI PERSONALIZZATI ======
\makeatletter
\newcommand{\NomeGruppo}[1]{\def\@NomeGruppo{#1}}
\newcommand{\TitoloVerbale}[1]{\def\@TitoloVerbale{#1}}
\newcommand{\Sommario}[1]{\def\@Sommario{#1}}
\newcommand{\Autore}[1]{\def\@Autore{#1}}
\newcommand{\Verificatore}[1]{\def\@Verificatore{#1}}
\makeatother

% ====== STILE TABELLE MIGLIORATO ======
\newcolumntype{Y}{>{\raggedright\arraybackslash}X} % Colonna giustificata a sinistra
\setlength{\arrayrulewidth}{0.4pt} % Linee più sottili
\setlength{\tabcolsep}{10pt} % Spaziatura interna celle
\renewcommand{\arraystretch}{1.4} % Altezza righe

% ====== INIZIO DEL DOCUMENTO ======
\begin{document}

% ====== INFORMAZIONI PER LA PAGINA DI TITOLO ======
\NomeGruppo{BYTE HOLDERS}
\TitoloVerbale{Norme di Progetto}
\Autore{}
\Verificatore{}

\pagestyle{empty}

% ====== PAGINA DI TITOLO ======
\begin{titlepage}
    \centering
    
    \includegraphics[width=0.55\textwidth]{../Assets/ByteHolders1.png}\par\vspace{1.5cm}
\begin{figure}
        \centering
        \label{fig:placeholder}
    \end{figure}
        
    {\LARGE \sffamily \color{AccentColor}\bfseries Norme di Progetto}\par
    \vspace{0.5cm}
    {\large \color{AccentColor}\sffamily }\par
    
    \vfill
    
    \noindent\color{AccentColor}\rule{\textwidth}{1pt}\par
    \vspace{0.5cm}
    
    \begin{tabularx}{0.9\textwidth}{@{}>{\bfseries\sffamily}l X@{}}
    Autore & \sffamily Nicolò Lattanzio\\
    \arrayrulecolor{MediumGray}\hline \\[-1.5ex]
    Verificatore & \sffamily \\
    \arrayrulecolor{MediumGray}\hline \\[-1.5ex] 
    Approvazione & \sffamily \\ 
    \arrayrulecolor{MediumGray}\hline 
\end{tabularx}
    
    \vfill
\end{titlepage}

% ====== INDICE ======
\pagestyle{fancy}
\newpage
\tableofcontents
\newpage

% ====== TABELLA DI VERSIONAMENTO ======
\section{Registro delle versioni}
\begin{center}
    \rowcolors{2}{LightGray}{white}
    \begin{tabular}{>{\centering\arraybackslash}m{1.5cm} >{\centering\arraybackslash}m{2cm} >{\raggedright\arraybackslash}m{2.5cm} >{\raggedright\arraybackslash}m{6.5cm}}
        \rowcolor{AccentColor}
          \textcolor{white}{\textbf{Versione}} & 
          \textcolor{white}{\textbf{Data}} & 
          \multicolumn{1}{c}{\textcolor{white}{\textbf{Autore}}} & 
          \multicolumn{1}{c}{\textcolor{white}{\textbf{Descrizione delle modifiche}}} \\

        0.3 & 07/12/2025 & Nicolò Lattanzio & Aggiunta sottosezione 4.1.1 per la Struttura dei documenti, 4.1.2 per il loro versionamento e 4.1.3 per la tabella Decisioni e Azioni.\\
        0.2 & 06/12/2025 & Nicolò Lattanzio & Aggiunta sezione Processi Primari, con relative sottosezioni, e prima redazione della sezione Processi di Supporto. \\
        0.1 & 30/11/2025 & Nicolò Lattanzio & Creazione e prima redazione del documento.  \\
        
    \end{tabular}
\end{center}

%\vspace{1cm}

% ====== SEZIONE INFORMAZIONI INTRODUTTIVE ======
\section{Informazioni introduttive}
\subsection{Scopo del documento}
Questo documento definisce le norme di progetto adottate dal gruppo \textbf{\textit{BYTE HOLDERS}} per la gestione e lo sviluppo del progetto. Esso stabilisce le linee guida per la documentazione, la comunicazione, la gestione delle versioni e le responsabilità dei membri del team.
\\
Per la realizzazione del documento, e' stato deciso di prendere come riferimento lo standard ISO/IEC 12207, 
che definisce i processi di ciclo di vita del software e fornisce una struttura per la gestione del progetto.
Le tipologie di processi sono le seguenti:
\\
\begin{itemize}
    \item \textbf{Processi Primari} : Consiste nelle attività principali che portano alla creazione, consegna e mantenimento del prodotto finale.
    \item \textbf{Processi di Supporto}: Processi che forniscono supporto alle attività principali, possono essere utilizzato da qualsiasi processo e ne garantiscono la qualità e la correttezza
    \item \textbf{Processi Organizzativi}: Processi necessari a creare l'ambiente e le risorse necessarie affinché i processi primari possano esistere.
\end{itemize}


\subsection{Scopo del prodotto}
Il prodotto sviluppato dal gruppo \textbf{\textit{BYTE HOLDERS}} è un'applicazione web che mira a fornire una piattaforma intuitiva per la gestione delle repository GitHub, 
fornendo informazioni utili in base alle varie necessità.
Tra queste funzionalità vi sono:
\begin{itemize}
    \item Analisi statica della qualità del codice
    \item Analisi della qualità della documentazione
    \item Analisi OWASP per la verifica delle vulnerabilità di sicurezza tramite l'integrazione di strumenti di terze parti
    \item Integrazione con API di GitHub per fornire statistiche dettagliate sulle repository
    \item Analisi specifica sulle pull request per valutare la qualità del codice proposto
    \item Proposte di remedation per migliorare la qualità del codice e della documentazione
\end{itemize}
Il prodotto segue la proposta del \textbf{capitolato C2 - Code Guardian} fornito dal proponente \textbf{\textit{VarGroup S.r.l.}}
\\
Il gruppo si è posto l'obiettivo di completare l'MVP di questo progetto entro il \textbf{20 Marzo 2026}, con un costo totale di \textbf{13.290 Euro.}
\subsection{Riferimenti}
\begin{itemize}
        \item \href{https://www.math.unipd.it/~tullio/IS-1/2025/Progetto/C2.pdf} {Capitolato C2 - Code Guardian}
        \item \href{https://byte-holders.github.io/Documentazione/} {Documentazione del progetto}
\end{itemize}


\section{Processi Primari}
Questa sezione descrive i processi del ciclo di vita primari, definiti dallo standard ISO/IEC 12207, che sono direttamente coinvolti nella creazione, fornitura e manutenzione del prodotto software. 
Essi costituiscono il cuore operativo del progetto, governando le attività principali dall'acquisizione delle esigenze alla consegna e al supporto del prodotto finale.
\\
Possiamo distinguere due macro processi:
\begin{itemize}
    \item \textbf{Processo di Fornitura:} Si occupa della gestione delle relazioni con il committente, dalla risposta alla richiesta iniziale fino alla consegna del prodotto.
    \item \textbf{Processo di Sviluppo:} Copre tutte le attività ingegneristiche necessarie per trasformare i requisiti in un prodotto software funzionante, includendo analisi, progettazione, implementazione, testing e rilascio.
\end{itemize}

\subsection{Processo di Fornitura}
Questo processo definisce le attività che il gruppo Byte Holders, in qualità di fornitore, implementa per garantire la corretta fornitura del prodotto software a VarGroup S.r.l., committente del progetto. 
Esso disciplina l'intero ciclo, dall'aggiudicazione del capitolato alla consegna finale.
\\
Vediamo ora le attività principali coinvolte in questo processo:
\begin{enumerate}
    \item \textbf{Risposta al Bando:} Analisi del capitolato e preparazione dell'offerta (documento \textit{Candidatura}).
    \item \textbf{Negoziazione Requisiti:} Realizzazione di una controproposta per il proponente, definendo i requisiti funzionali e non funzionali che il gruppo Byte Holders, come fornitore, prevede di riuscire a realizzare.
    \item \textbf{Pianificazione della Fornitura:} Creazione del \textit{Piano di Progetto} che dettaglia come verranno eseguiti i processi di sviluppo e gestione.
    \item \textbf{Esecuzione e Controllo:} Realizzazione del progetto conforme al piano, con reporting periodico allo stakeholder.
    \item \textbf{Revisione e Valutazione:} Condotta di revisioni tecniche e attività di verifica e validazione.
    \item \textbf{Consegna e Completamento:} Consegna del prodotto finale insieme alla relativa documentazione.
\end{enumerate}
\subsubsection{Documenti Principali:} 
Per questo progetto saranno prodotti i seguenti documenti chiave:

\subsection{Processo di Sviluppo}
Il processo di sviluppo definisce il ciclo di vita ingegneristico attraverso il quale i requisiti del committente vengono trasformati nel prodotto software finale. 
Questo processo organizza in fasi strutturate le attività di analisi, progettazione, implementazione e validazione, garantendo un approccio sistematico e tracciabile alla realizzazione del sistema.\\

\begin{itemize}
    \item \textbf{Attività Chiave (Fasi del Ciclo di Vita):}
    \begin{enumerate}
        \item \textbf{Analisi dei Requisiti:} Raccolta, analisi, specifica e validazione dei requisiti con gli stakeholder. Prodotto: \textit{Documento di Analisi dei Requisiti}.
        \item \textbf{Progettazione dell'Architettura:} Definizione dell'architettura di sistema e software ad alto livello.
        \item \textbf{Progettazione di Dettaglio:} Definizione dettagliata dei componenti e delle interfacce.
        \item \textbf{Costruzione (Implementazione e Codifica):} Scrittura del codice secondo gli standard di progetto e utilizzo di un sistema di versioning (Git). Prodotto: Codice sorgente e repository.
        \item \textbf{Integrazione e Testing:}
        \begin{itemize}
            \item Testing delle unità (sviluppatori)
            \item Testing d'integrazione (team di integrazione)
            \item Testing di sistema (team di qualità)
        \end{itemize}
    \end{enumerate}
    \item \textbf{Metodologia:} Adotteremo pratiche ibride Agile, con sprint per lo sviluppo e milestone formali per le revisioni.
\end{itemize}


\subsection{Strumenti di supporto}
Per i processi primari il gruppo \textbf{\textit{BYTE HOLDERS}} utilizza i seguenti strumenti:
\begin{itemize}
  \item \textbf{GitHub}: per il versionamento del codice e la gestione delle issue.
  \item \textbf{Discord}: per la comunicazione interna tra i membri del team.
  \item \textbf{Slack}: Per la comunicazione con il proponente, in modo da concordare eventuali riunioni e per richiedere supporto tecnico se necessario.
  \item \textbf{Teams}: Per le riunioni virtuali con il proponente, dando la possibilità di condividere lo schermo e mostrare il lavoro svolto.
\end{itemize}

\section{Processi Di Supporto}
Questa sezione descrive i processi di supporto secondo lo standard ISO/IEC 12207, che forniscono l'infrastruttura tecnica e metodologica necessaria per garantire qualità e coerenza nell'esecuzione dei processi primari. Tali processi, di natura trasversale, abilitano e migliorano le attività di sviluppo, gestione e verifica attraverso strumenti, procedure e controlli dedicati.
\subsection{Documentazione}
La documentazione del progetto viene gestita utilizzando \textbf{Latex} e viene archiviata in un repository dedicato su \textbf{GitHub}. 
\\Ogni documento segue una struttura standardizzata per garantire coerenza e facilità di lettura.

\subsubsection{Struttura dei documenti}

Tutti i documenti formali prodotti dal gruppo seguono una struttura standardizzata per garantire coerenza, professionalità e facilità di consultazione. Il template \textbf{Latex} utilizzato definisce chiaramente gli elementi che compongono ciascun documento:

\begin{itemize}[itemsep=6pt]
    \item \textbf{Frontespizio:}
    \begin{itemize}[itemsep=4pt]
        \item Logo del gruppo Byte Holders e titolo del progetto
        \item Nome del documento (es. "Norme di Progetto", "Verbale Interno")
        \item Informazioni di identificazione: data, autore, verificatore e approvatore
    \end{itemize}
    
    \item \textbf{Registro delle Versioni:}
    \begin{itemize}[itemsep=4pt]
        \item Tabella posizionata dopo l'indice
        \item Tracciamento cronologico inverso (LIFO) di tutte le modifiche
        \item Colonne per versione, data, autore e descrizione delle modifiche
    \end{itemize}
    
    \item \textbf{Indice:}
    \begin{itemize}[itemsep=4pt]
        \item Elenca tutte le sezioni, sottosezioni e relative pagine
        \item Fornisce una mappa navigabile del documento
    \end{itemize}
    
    \item \textbf{Contenuto Principale:}
    \begin{itemize}[itemsep=4pt]
        \item Organizzato gerarchicamente in sezioni, sottosezioni e sotto-sottosezioni
    \end{itemize}

    \item \textbf{Stile Grafico Uniforme:}
    \begin{itemize}[itemsep=4pt]
        \item Palette di colori definita dal gruppo (blu-viola in diverse tonalità)
        \item Font sans-serif (Helvetica) per migliorare la leggibilità
        \item Margini standardizzati
    \end{itemize}
\end{itemize}
Questa struttura si applica a tutti i tipi di documenti prodotti, tra cui: Verbali(interni ed esterni), Norme di Progetto, Analisi dei Requisiti\dots

\subsubsection{Versionamento}
Il versionamento dei documenti è un processo fondamentale per garantire la tracciabilità delle modifiche, la collaborazione ordinata tra i membri del team e la chiara identificazione dello stato corrente di ciascun documento. 
\\
Per questo motivo, il gruppo \textbf{Byte Holders} adotta un sistema di numerazione semantica e un registro strutturato delle modifiche.
\\
\\
Per la gestione del versionamento, abbiamo deciso di usare uno schema a tre livelli, X.Y.Z, dove:
\begin{itemize} 
    \item \textbf{X (Approvazione):} Incrementato per cambiamenti significativi che richiedono l'approvazione formale. E quindi il raggiungimento di un documento stabile e definitivo.
    \item \textbf{Y (Major Update):} Incrementato per l'aggiunta di nuove sezioni o modifiche sostanziali che alterano la struttura complessiva del documento. Il documento non è ancora definitivo.
    \item \textbf{Z (Minor Update):} Incrementato per correzioni minori, come errori di battitura o aggiornamenti di dettaglio.
\end{itemize}
\subsubsection{Tabella Decisioni e Azioni}
Ogni verbale relativo al progetto conterrà una tabella decisioni e azioni, che riassume le decisioni prese durante la riunione e le azioni assegnate ai membri del team.
Anche in questo caso, per garantire la tracciabilità, ogni decisione o azione avrà un proprio codice di identificazione,
composto da un prefisso che ne indica la natura (DEC per decisione, AZ per azione); seguito dalla fase del progetto in cui è stata presa (RTB o PB); e infine un numero progressivo che identifica univocamente la decisione o l'azione all'interno di quella fase.
\\
Tutto ciò ci permette di tenere traccia delle decisioni e delle azioni in modo chiaro e organizzato, facilitando il monitoraggio del progresso del progetto e la responsabilizzazione dei membri del team.
\\
Qui sotto è riportato un esempio di tabella decisioni e azioni:
\begin{center}
    \rowcolors{2}{LightGray}{white}
    \begin{tabular}{>{\centering\arraybackslash}m{2cm} >{\raggedright\arraybackslash}p{8cm} >{\centering\arraybackslash}p{2.5cm}}
        \rowcolor{AccentColor}
        \textcolor{white}{\textbf{Codice}} & 
        \multicolumn{1}{c}{\textcolor{white}{\textbf{Descrizione}}} & 
        \textcolor{white}{\textbf{Assegnatario}} \\
        
        DEC-RTB-001 & Scelta del capitolato & Tutti \\
        DEC-PB-003 & Delineamento dell'analisi dei requisiti & Tutti \\
        \midrule
        AZ-RTB-001 & Brainstorming Analisi dei requisiti & Tutti \\
        AZ-PB-004 & Miglioramento struttura dei documenti & Tutti \\
    \end{tabular}
\end{center}

\section{Processi Organizzativi}

\subsection{Riunioni}

\subsection{Strumenti a supporto}

\subsection{Gestione dei processi}












\end{document}