\documentclass[a4paper, 11pt]{article}

% ====== PACCHETTI NECESSARI ======
\usepackage[utf8]{inputenc}
\usepackage[T1]{fontenc}
\usepackage[italian]{babel}
\usepackage{geometry}
\usepackage{graphicx}
\usepackage[table]{xcolor}
\usepackage{tabularx}
\usepackage{array}
\usepackage{amssymb}
\usepackage{fancyhdr}
\setlength{\headheight}{14pt}
\usepackage{titlesec}
\usepackage{helvet}
\renewcommand{\familydefault}{\sfdefault}
\usepackage{lipsum}
\usepackage{hyperref}
\usepackage{booktabs}
\usepackage{enumitem}
\usepackage{titlecaps}
\usepackage{longtable}
\usepackage[utf8]{inputenc} % Specifica la codifica del file (necessaria per le accentate)
\usepackage[T1]{fontenc}    % Migliora l'output dei font per le lingue europee

% ====== IMPOSTAZIONI GLOBALI DI STILE ======

% 1. DEFINIZIONE COLORI BLU-VIOLA
\definecolor{AccentColor}{RGB}{80, 90, 180} % Blu-viola principale
\definecolor{AccentLight}{RGB}{80, 90, 180} % Versione più chiara
\definecolor{AccentDark}{RGB}{50, 60, 140} % Versione più scura
\definecolor{LightGray}{RGB}{245, 245, 250}
\definecolor{MediumGray}{RGB}{200, 200, 210}

% 2. IMPOSTAZIONE MARGINI
\geometry{a4paper, left=2.5cm, right=2.5cm, top=3.5cm, bottom=3.5cm}

% 3. STILE DEI TITOLI DI SEZIONE
\titleformat{\section}
  {\normalfont\sffamily\Large\bfseries\color{AccentColor}}
  {\thesection}
  {1em}
  {}
\titleformat{\subsection}
  {\normalfont\sffamily\large\bfseries\color{AccentDark}}
  {\thesubsection}
  {1em}
  {}

% 4. IMPOSTAZIONE HEADER E FOOTER
\pagestyle{fancy}
\fancyhf{} 
\fancyhead[L]{\sffamily\bfseries\color{AccentColor}\@BYTE HOLDERS}
\fancyhead[R]{\sffamily\color{AccentColor}\thepage}
\renewcommand{\headrulewidth}{0.8pt}
\renewcommand{\headrule}{\color{AccentColor}\hrule width\headwidth height\headrulewidth \vskip-\headrulewidth}

% 5. IMPOSTAZIONE LINK
\hypersetup{
    colorlinks=true,
    linkcolor=AccentColor,
    urlcolor=AccentLight,
    citecolor=AccentDark,
}

% 6. PERSONALIZZAZIONE ELENCHI
\setlist[itemize]{itemsep=2pt, topsep=4pt}
\setlist[enumerate]{itemsep=2pt, topsep=4pt}

% ====== COMANDI PERSONALIZZATI ======
\newcommand{\gloss}[1]{\titlecap{\textit{#1}}\textsuperscript{G}}
\newcommand{\nuovogloss}[1]{\subsection{\gloss{#1}}\label{#1}}

% ====== STILE TABELLE MIGLIORATO ======
\newcolumntype{Y}{>{\raggedright\arraybackslash}X} % Colonna giustificata a sinistra
\setlength{\arrayrulewidth}{0.4pt} % Linee più sottili
\setlength{\tabcolsep}{10pt} % Spaziatura interna celle
\renewcommand{\arraystretch}{1.4} % Altezza righe

% ====== INIZIO DEL DOCUMENTO ======
\begin{document}


\pagestyle{empty}

% ====== PAGINA DI TITOLO ======
\begin{titlepage}
    \begin{center}
    	\includegraphics[width=0.55\textwidth]{../Assets/ByteHolders1.png}\vspace{1.5cm}
    	
    	{\LARGE \sffamily \color{AccentColor}\bfseries Analisi dei Requisiti}
    \end{center}
    
    \vfill
    \rule{\textwidth}{1pt}\par
   	\textit{Versione: 0.1.0}
\end{titlepage}

\newpage

% ====== TABELLA DI VERSIONAMENTO ======
{\normalfont\sffamily\huge\bfseries\color{AccentColor} Registro delle versioni}
\vspace{1cm}

\begin{center}
    \rowcolors{2}{LightGray}{white}
     \begin{tabular}{>{\centering\arraybackslash}m{1.5cm} >{\centering\arraybackslash}m{2cm} >{\raggedright\arraybackslash}m{2.5cm} >{\raggedright\arraybackslash}m{6.5cm}}
        \rowcolor{AccentColor}
          \textcolor{white}{\textbf{Versione}} & 
          \textcolor{white}{\textbf{Data}} & 
          \multicolumn{1}{c}{\textcolor{white}{\textbf{Autore}}} &
          \multicolumn{1}{c}{\textcolor{white}{\textbf{Descrizione delle modifiche}}} \\
        0.1.0 & 19/12/2025 & Alessandro Morabito & Inizio stesura\\ 
    \end{tabular}
\end{center}


% ====== INDICE ======
\pagestyle{fancy}
\newpage
\tableofcontents
\newpage

% ====== INTRODUZIONE ======
\section{Scopo del documento}
Con il presente documento il gruppo \gloss{byte holders} definisce i \gloss{requisiti} dell'applicazione \gloss{Code Guardian} che si impegna a sviluppare per l'\gloss{Azienda}.

\section{Glossario}
Per evitare ambiguità, nel corso del documento si farà riferimento a termini indicati nel \href{https://byte-holders.github.io/Documentazione/RTB/Glossario.pdf}{\gloss{Glossario}}\footnote{\url{https://byte-holders.github.io/Documentazione/RTB/Glossario.pdf}} utilizzando la lettera \textit{G} ad apice della formula corrispondente, che viene indicata in corsivo e con iniziali maiuscole (ad es. \gloss{formula in glossario}). La corrispondenza di termini è a meno di coniugazioni e declinazioni.

\section{Riferimenti}
\subsection{Riferimenti normativi}
\begin{itemize}
	\item 
		\gloss{Norme Di Progetto}\\
		\url{https://byte-holders.github.io/Documentazione/RTB/Glossario.pdf}
	\item
		\gloss{Capitolato}\\
		\url{https://www.math.unipd.it/~tullio/IS-1/2025/Progetto/C2.pdf}
\end{itemize}
\subsection{Riferimenti informativi}
\begin{itemize}
	\item
		\gloss{Glossario}\\
		\url{https://byte-holders.github.io/Documentazione/RTB/Glossario.pdf}
	\item
		Specifica \gloss{UML} 2.5.1\\
		\url{https://www.omg.org/spec/UML/2.5.1/PDF}
\end{itemize}

\section{Termini aggiunti al \gloss{Glossario}}
\nuovogloss{Caso d'uso}
Un \gloss{Caso d'uso} è un insieme di \gloss{scenari} che hanno in comune uno scopo finale per un \gloss{utente}.

\nuovogloss{Scenario}

\nuovogloss{Attore}

\newpage

\section{\gloss{Casi d'uso}}
\subsection{Lista degli \gloss{Attori}}
Nella creazione dei \gloss{casi d'uso} sono stati individuati i seguenti \gloss{attori}:
\begin{itemize}
	\item 
		\textbf{Utente non Autenticato}\\
		Un utente non riconosciuto dal sistema
	\item 
		\textbf{Utente Autenticato}\\
		Utente generico riconosciuto dal sistema
	\item 
		\textbf{Utente Permesso OWASP}  (eredita da \textit{Utente autenticato})\\
		Utente Autenticato con il permesso per la visione completa delle informazioni su \gloss{OWASP}
	\item
		\textbf{Utente Permesso Utenti/Ruoli} (eredita da \textit{Utente autenticato})\\
	 	Utente Autenticato con il permesso per la gestione degli utenti e dei ruoli
	\item 
		\textbf{Utente Permesso Test} (eredita da \textit{Utente autenticato})\\
		Utente Autenticato con il permesso per la visione completa delle informazioni sui test
	\item
		\textbf{Utente Permesso Documentazione} (eredita da \textit{Utente autenticato})\\
		Utente Autenticato con il permesso per la visione completa delle informazioni sulla documentazione
	\item
		\textbf{Utente Permesso Scansione} (eredita da \textit{Utente autenticato})\\
		Utente Autenticato con il permesso per il lancio di una scansione
	\item
		\textbf{Utente Permesso Qualità Codice} (eredita da \textit{Utente autenticato})\\
		Utente Autenticato con il permesso per la visione completa delle informazioni sulla qualità del codice
	\item
		\textbf{Utente Permesso Informazioni Tecniche} (eredita da \textit{Utente autenticato})\\
		Utente Autenticato con il permesso per la visione completa delle informazioni tecniche di una repository
\end{itemize}
\subsection{Struttura generale di un \gloss{caso d'uso}}
Si è deciso di descrivere ciascun \gloss{caso d'uso} seguendo la seguente struttura (\underline{sottolineati} i campi sempre popolati):

\begin{longtable}{|p{.25\linewidth} p{.6\linewidth}|}
	\hline
	\textbf{\underline{Codice}} 
	&
	Codice identificativo utilizzato per far riferimento al \gloss{caso d'uso} corrente\\
	
	\textbf{\underline{Titolo}} 
	&
	Titolo del \gloss{caso d'uso} corrente\\
	
	\textbf{\underline{Attori principali}} 
	&
	Attori che agiscono sul sistema dando inizio allo scenario\\
	
	\textbf{Attori secondari} 
	&
	Attori di supporto che agiscono in risposta a stimoli del sistema\\
	
	\textbf{\underline{Precondizioni}}
	&
	  Condizioni necessarie per l'esecuzione del \gloss{caso d'uso} corrente\\
	
	\textbf{\underline{Postcondizioni}} 
	&
	Condizioni in cui viene lasciato il sistema al termine dello \gloss{scenario principale}\\
	
	\textbf{\underline{Scenario principale}}
	&
	Descrizione degli eventi che avvengono all'interno dello \gloss{scenario principale}\\
	
	\textbf{Inclusioni} 
	&
	Lista dei riferimenti a \gloss{casi d'uso} terzi \gloss{inclusi} dal \gloss{caso d'uso} corrente e al quale si fa riferimento nella sezione \textit{Scenario principale}\\
	
	\textbf{Scenari alternativi} 
	&
	Descrizione delle situazioni che portano a \gloss{scenari alternativi}\\
	
	\textbf{Eredita da} 
	&
	Codice del \gloss{caso d'uso} terzo da cui eredita il \gloss{caso d'uso} corrente (non ammettiamo ereditarietà multipla)\\
	
	\hline
\end{longtable}
\subsection{Lista dei \gloss{casi d'uso}}
\end{document}