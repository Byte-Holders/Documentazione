\documentclass[a4paper, 11pt]{article}

% ====== PACCHETTI NECESSARI ======
\usepackage[utf8]{inputenc}
\usepackage[T1]{fontenc}
\usepackage[italian]{babel}
\usepackage{geometry}
\usepackage{graphicx}
\usepackage[table]{xcolor}
\usepackage{tabularx}
\usepackage{array}
\usepackage{amssymb}
\usepackage{fancyhdr}
\setlength{\headheight}{14pt}
\usepackage{titlesec}
\usepackage{helvet}
\renewcommand{\familydefault}{\sfdefault}
\usepackage{lipsum}
\usepackage{hyperref}
\usepackage{booktabs}
\usepackage{enumitem}
\usepackage[utf8]{inputenc} % Specifica la codifica del file (necessaria per le accentate)
\usepackage[T1]{fontenc}    % Migliora l'output dei font per le lingue europee

% ====== IMPOSTAZIONI GLOBALI DI STILE ======

% 1. DEFINIZIONE COLORI BLU-VIOLA
\definecolor{AccentColor}{RGB}{80, 90, 180} % Blu-viola principale
\definecolor{AccentLight}{RGB}{80, 90, 180} % Versione più chiara
\definecolor{AccentDark}{RGB}{50, 60, 140} % Versione più scura
\definecolor{LightGray}{RGB}{245, 245, 250}
\definecolor{MediumGray}{RGB}{200, 200, 210}

% 2. IMPOSTAZIONE MARGINI
\geometry{a4paper, left=2.5cm, right=2.5cm, top=3.5cm, bottom=3.5cm}

% 3. STILE DEI TITOLI DI SEZIONE
\titleformat{\section}
  {\normalfont\sffamily\Large\bfseries\color{AccentColor}}
  {\thesection}
  {1em}
  {}
\titleformat{\subsection}
  {\normalfont\sffamily\large\bfseries\color{AccentDark}}
  {\thesubsection}
  {1em}
  {}
\titleformat{\subsubsection}
  {\normalfont\sffamily\large\bfseries\color{AccentDark}}
  {\thesubsubsection}
  {1em}
  {}

% 4. IMPOSTAZIONE HEADER E FOOTER
\pagestyle{fancy}
\fancyhf{}
\fancyhead[L]{\sffamily\bfseries\color{AccentColor}\@BYTE HOLDERS}
\fancyhead[R]{\sffamily\color{AccentColor}\thepage}
\renewcommand{\headrulewidth}{0.8pt}
\renewcommand{\headrule}{\color{AccentColor}\hrule width\headwidth height\headrulewidth \vskip-\headrulewidth}

% 5. IMPOSTAZIONE LINK
\hypersetup{
    colorlinks=true,
    linkcolor=AccentColor,
    urlcolor=AccentLight,
    citecolor=AccentDark,
}

% 6. PERSONALIZZAZIONE ELENCHI
\setlist[itemize]{itemsep=2pt, topsep=4pt}
\setlist[enumerate]{itemsep=2pt, topsep=4pt}

% ====== COMANDI PERSONALIZZATI ======
\makeatletter
\newcommand{\NomeGruppo}[1]{\def\@NomeGruppo{#1}}
\newcommand{\TitoloVerbale}[1]{\def\@TitoloVerbale{#1}}
\newcommand{\Sommario}[1]{\def\@Sommario{#1}}
\newcommand{\Autore}[1]{\def\@Autore{#1}}
\newcommand{\Verificatore}[1]{\def\@Verificatore{#1}}
\makeatother

% ====== STILE TABELLE MIGLIORATO ======
\newcolumntype{Y}{>{\raggedright\arraybackslash}X} % Colonna giustificata a sinistra
\setlength{\arrayrulewidth}{0.4pt} % Linee più sottili
\setlength{\tabcolsep}{10pt} % Spaziatura interna celle
\renewcommand{\arraystretch}{1.4} % Altezza righe

% ====== INIZIO DEL DOCUMENTO ======
\begin{document}


% ====== PAGINA DI TITOLO ======
\begin{titlepage}
    \centering
    
    \includegraphics[width=0.55\textwidth]{../../../Assets/ByteHolders1.png}\par\vspace{1.5cm}
    
    {\LARGE \sffamily \color{AccentColor}\bfseries Verbale interno}\par
    \vspace{0.5cm}
    {\large \color{AccentColor}\sffamily 27 novembre 2025}\par
    
    \vfill
    
    \noindent\color{AccentColor}\rule{\textwidth}{1pt}\par
    \vspace{0.5cm}
    
    \begin{tabularx}{0.9\textwidth}{@{}>{\bfseries\sffamily}l X@{}}
    Autore & \sffamily Alessandro Morabito\\
    \arrayrulecolor{MediumGray}\hline \\[-1.5ex]
    Verificatore & \sffamily Damiano Berti\\
    \arrayrulecolor{MediumGray}\hline \\[-1.5ex] 
    Approvazione & \sffamily Nicolò Lattanzio\\ 
    \arrayrulecolor{MediumGray}\hline 
\end{tabularx}
    
    \vfill
\end{titlepage}

% ====== INDICE ======
\pagestyle{fancy}
\newpage
\tableofcontents
\newpage

% ====== TABELLA DI VERSIONAMENTO ======
\section{Registro delle versioni}
\begin{center}
    \rowcolors{2}{LightGray}{white}
    \begin{tabular}{>{\centering\arraybackslash}m{2cm} >{\centering\arraybackslash}m{2cm} >{\raggedright\arraybackslash}m{2.5cm} >{\raggedright\arraybackslash}m{6.5cm}}
        \rowcolor{AccentColor}
          \textcolor{white}{\textbf{Versione}} & 
          \textcolor{white}{\textbf{Data}} & 
          \multicolumn{1}{c}{\textcolor{white}{\textbf{Autore}}} & 
          \multicolumn{1}{c}{\textcolor{white}{\textbf{Descrizione delle modifiche}}} \\
        1.0.0 & 01/12/2025 & Nicolò Lattanzio & Approvazione del verbale \\
        0.1.0 & 01/12/2025 & Damiano Berti & Verifica del verbale \\
        0.0.2 & 01/12/2025 & Alessandro Morabito & Riorganizzazione \ref{RiassuntoDocumentazione}\\
        0.0.1 & 28/11/2025 & Alessandro Morabito & Prima stesura del verbale \\ 
        
        & & & \\
    \end{tabular}
\end{center}

%\vspace{1cm}

% ====== SEZIONE INFORMAZIONI INTRODUTTIVE ======
\section{Informazioni introduttive}

\subsection{Durata e luogo}
\begin{itemize}
    \item \textbf{Inizio:} 15:45
    \item \textbf{Fine:} 18:00
    \item \textbf{Luogo:} Chiamata Discord
\end{itemize}

% ====== TABELLA PRESENZE ======
\subsection{Partecipanti}
\begin{center}
    \rowcolors{2}{LightGray}{white}
    \begin{tabular}{>{\raggedright\arraybackslash}p{6cm} c c}
        \rowcolor{AccentColor}
        \textcolor{white}{\textbf{Nome e Cognome}} & 
        \textcolor{white}{\textbf{Presente}} & 
        \textcolor{white}{\textbf{Assente}} \\
        
        Damiano Berti     & \textcolor{AccentColor}{$\blacksquare$}    & $\square$        \\
        Alessandro Frison     & \textcolor{AccentColor}{$\blacksquare$}    & $\square$        \\
        Lorenzo Grolla     & \textcolor{AccentColor}{$\blacksquare$}    & $\square$        \\
        Nicolò Lattanzio    & \textcolor{AccentColor}{$\blacksquare$}    & $\square$        \\
        Alessandro Morabito   & \textcolor{AccentColor}{$\blacksquare$}    & $\square$        \\
        Giacomo Nalotto   & \textcolor{AccentColor}{$\blacksquare$}    & $\square$        \\
        Giulia Romanato   & \textcolor{AccentColor}{$\blacksquare$}    & $\square$        \\
    \end{tabular}
\end{center}

% ====== SEZIONE PRINCIPALE DEL VERBALE ======
\section{Contenuto della riunione}

\subsection{Ordine del giorno}
\begin{enumerate}
    \item Passare in rassegna i casi d'uso relativi a un attore "Developer" pensati in autonomia
    \item Discutere il significato della figura di "Tech Lead" e "Project Manager"
    \item Suddisivisione compiti per inizio stesura dei documenti sulle Norme di Progetto, il Piano di Progetto e il Piano di Qualifica
\end{enumerate}

\subsection{Riassunto della discussione}
\subsubsection{Analisi dei Casi d'Uso}
Ciascun membro del gruppo ha presentato alcuni casi d'uso per un utente "Developer". Sono stati caricati in un file condiviso e discussi. Si è iniziato a stendere una lista che raccoglie tutte le idee proposte, ma dal momento che è sembrato un metodo poco efficiente in termini di tempo si è deciso che ciascun membro avrebbe steso la lista individualmente. L'unione delle liste è diventata quindi materia dell'incontro seguente.

Attualmente abbiamo individuato i seguenti casi d'uso principali:
\begin{itemize}
	\item Ricerca tra repository
	\item Ricerca all'interno di una repository
	\item Visualizzazione di una repository
	\item Visualizzazione di un report
\end{itemize}

È stato inoltre notato che ci sono molte sovrapposizioni tra i tre attori individuati. Per questo motivo è stato introdotto un ulteriore attore, un "utente", che è loro generalizzazione.

A ciascun membro del gruppo è stato assegnato come compito la produzione di liste di casi d'uso degli attori Tech Lead e Project Manager.

\subsubsection{Terminologia}
È stato introdotto il termine "raccolta" in sostituzione di "progetto", per differenziarlo dal "project" secondo GitHub. 

Attualmente vale la seguente: "Una \textit{raccolta} è un insieme di repository GitHub che contiene tutta la documentazione e il codice riguardante un'applicazione software".

\subsubsection{Documentazione} \label{RiassuntoDocumentazione}
Alla conclusione dell'incontro il gruppo si è spartito i compiti per l'inizio della stesura dei seguenti documenti:
\begin{itemize}
	\item Norme di Progetto
	\item Piano di Progetto
	\item Piano di Qualifica
\end{itemize}
Le assegnazioni sono indicate nella sezione \nameref{DecisioniAzioni}.

% ====== SEZIONE DECISIONI E AZIONI ======
\section{Decisioni e azioni} \label{DecisioniAzioni}
\begin{center}
    \rowcolors{2}{LightGray}{white}
    \begin{tabular}{>{\centering\arraybackslash}m{3cm} >{\raggedright\arraybackslash}p{7cm} >{\centering\arraybackslash}p{2.5cm}}
        \rowcolor{AccentColor}
        \textcolor{white}{\textbf{Codice }} & 
        \multicolumn{1}{c}{\textcolor{white}{\textbf{Descrizione}}} & 
        \textcolor{white}{\textbf{Assegnatario}} \\
        
        DEC-RTB-010 & Discussione casi d'uso per l'attore Developer & Tutti \\
        DEC-RTB-011 & Introduzione del termine "raccolta" & Tutti \\
        \midrule
        AZ-RTB-009 & Elencare casi d'uso per Tech Lead e Project Manager & Tutti\\
        AZ-RTB-010 & Aggiornamento doc. Glossario & Giulia Romanato\\
        AZ-RTB-011 & Inizio stesura doc. Norme di Progetto & Nicolò Lattanzio\\
        AZ-RTB-012 & Inizio stesura doc. Piano di Qualifica & Alessandro Frison, Lorenzo Grolla\\
        AZ-RTB-013 & Inizio stesura doc. Piano di Progetto & Damiano Berti, Giacomo Nalotto
    \end{tabular}
\end{center}

\end{document}