\documentclass[a4paper, 11pt]{article}

% ====== PACCHETTI NECESSARI ======
\usepackage[utf8]{inputenc}
\usepackage[T1]{fontenc}
\usepackage[italian]{babel}
\usepackage{geometry}
\usepackage{graphicx}
\usepackage[table]{xcolor}
\usepackage{tabularx}
\usepackage{array}
\usepackage{amssymb}
\usepackage{fancyhdr}
\usepackage{titlesec}
\usepackage{helvet}
\renewcommand{\familydefault}{\sfdefault}
\usepackage{lipsum}
\usepackage{hyperref}
\usepackage{booktabs}
\usepackage{enumitem}
\usepackage[utf8]{inputenc} % Specifica la codifica del file (necessaria per le accentate)
\usepackage[T1]{fontenc}    % Migliora l'output dei font per le lingue europee

% ====== IMPOSTAZIONI GLOBALI DI STILE ======

% 1. DEFINIZIONE COLORI BLU-VIOLA
\definecolor{AccentColor}{RGB}{80, 90, 180} % Blu-viola principale
\definecolor{AccentLight}{RGB}{80, 90, 180} % Versione più chiara
\definecolor{AccentDark}{RGB}{50, 60, 140} % Versione più scura
\definecolor{LightGray}{RGB}{245, 245, 250}
\definecolor{MediumGray}{RGB}{200, 200, 210}

% 2. IMPOSTAZIONE MARGINI
\geometry{a4paper, left=2.5cm, right=2.5cm, top=3.5cm, bottom=3.5cm}

% 3. STILE DEI TITOLI DI SEZIONE
\titleformat{\section}
  {\normalfont\sffamily\Large\bfseries\color{AccentColor}}
  {\thesection}
  {1em}
  {}
\titleformat{\subsection}
  {\normalfont\sffamily\large\bfseries\color{AccentDark}}
  {\thesubsection}
  {1em}
  {}

% 4. IMPOSTAZIONE HEADER E FOOTER
\pagestyle{fancy}
\fancyhf{}
\fancyhead[L]{\sffamily\bfseries\color{AccentColor}\@BYTE HOLDERS}
\fancyhead[R]{\sffamily\color{AccentColor}\thepage}
\renewcommand{\headrulewidth}{0.8pt}
\renewcommand{\headrule}{\color{AccentColor}\hrule width\headwidth height\headrulewidth \vskip-\headrulewidth}

% 5. IMPOSTAZIONE LINK
\hypersetup{
    colorlinks=true,
    linkcolor=AccentColor,
    urlcolor=AccentLight,
    citecolor=AccentDark,
}

% 6. PERSONALIZZAZIONE ELENCHI
\setlist[itemize]{itemsep=2pt, topsep=4pt}
\setlist[enumerate]{itemsep=2pt, topsep=4pt}

% ====== COMANDI PERSONALIZZATI ======
\makeatletter
\newcommand{\NomeGruppo}[1]{\def\@NomeGruppo{#1}}
\newcommand{\TitoloVerbale}[1]{\def\@TitoloVerbale{#1}}
\newcommand{\Sommario}[1]{\def\@Sommario{#1}}
\newcommand{\Autore}[1]{\def\@Autore{#1}}
\newcommand{\Verificatore}[1]{\def\@Verificatore{#1}}
\makeatother

% ====== STILE TABELLE MIGLIORATO ======
\newcolumntype{Y}{>{\raggedright\arraybackslash}X} % Colonna giustificata a sinistra
\setlength{\arrayrulewidth}{0.4pt} % Linee più sottili
\setlength{\tabcolsep}{10pt} % Spaziatura interna celle
\renewcommand{\arraystretch}{1.4} % Altezza righe

% ====== INIZIO DEL DOCUMENTO ======
\begin{document}

% ====== INFORMAZIONI PER LA PAGINA DI TITOLO ======
\NomeGruppo{BYTE HOLDERS}
\TitoloVerbale{Verbale Riunione Interna}
\Sommario{Questo verbale documenta la riunione interna avvenuta il 13/11/2025 per la discussione dei capitolati e la definizione degli strumenti di lavoro del team.}
\Autore{}
\Verificatore{}

\pagestyle{empty}

% ====== PAGINA DI TITOLO ======
\begin{titlepage}
    \centering
    
    \includegraphics[width=0.55\textwidth]{../../../Assets/ByteHolders1.png}\par\vspace{1.5cm}
    
    {\LARGE \sffamily \color{AccentColor}\bfseries Verbale interno}\par
    \vspace{0.5cm}
    {\large \color{AccentColor}\sffamily 11 Novembre 2025}\par
    
    \vfill
    
    \noindent\color{AccentColor}\rule{\textwidth}{1pt}\par
    \vspace{0.5cm}
    
    \begin{tabularx}{0.9\textwidth}{@{}>{\bfseries\sffamily}l X@{}}
    Autore & \sffamily Lorenzo Grolla\\
    \arrayrulecolor{MediumGray}\hline \\[-1.5ex]
    Verificatore & \sffamily Alessandro Frison\\
    \arrayrulecolor{MediumGray}\hline \\[-1.5ex] 
    Approvatore & \sffamily Giacomo Nalotto\\ 
    \arrayrulecolor{MediumGray}\hline 
\end{tabularx}
    
    \vfill
\end{titlepage}

% ====== INDICE ======
\pagestyle{fancy}
\newpage
\tableofcontents
\newpage

% ====== TABELLA DI VERSIONAMENTO ======
\section{Registro delle versioni}
\begin{center}
    \rowcolors{2}{LightGray}{white}
    \begin{tabular}{>{\centering\arraybackslash}m{1.5cm} >{\centering\arraybackslash}m{2cm} >{\raggedright\arraybackslash}m{2.5cm} >{\raggedright\arraybackslash}m{6.5cm}}
        \rowcolor{AccentColor}
          \textcolor{white}{\textbf{Versione}} & 
          \textcolor{white}{\textbf{Data}} & 
          \multicolumn{1}{c}{\textcolor{white}{\textbf{Autore}}} & 
          \multicolumn{1}{c}{\textcolor{white}{\textbf{Descrizione delle modifiche}}} \\
        
        1.0.0 & 27/11/2025 & Giacomo Nalotto & Revisione e approvazione del verbale. \\
        0.1.0 & 21/11/2025 & Alessandro Frison & Miglioramento del contenuto del riassunto della discussione. \\
        0.0.1 & 14/11/2025 & Lorenzo Grolla & Prima stesura del verbale. \\
        
    \end{tabular}
\end{center}

%\vspace{1cm}

% ====== SEZIONE INFORMAZIONI INTRODUTTIVE ======
\section{Informazioni introduttive}
\subsection{Durata e luogo}
\begin{itemize}
    \item \textbf{Inizio:} 15:30
    \item \textbf{Fine:} 18:00
    \item \textbf{Luogo:} Chiamata Discord
\end{itemize}

% ====== TABELLA PRESENZE ======
\subsection{Partecipanti}
\begin{center}
    \rowcolors{2}{LightGray}{white}
    \begin{tabular}{>{\raggedright\arraybackslash}p{6cm} c c}
        \rowcolor{AccentColor}
        \textcolor{white}{\textbf{Nome e Cognome}} & 
        \textcolor{white}{\textbf{Presente}} & 
        \textcolor{white}{\textbf{Assente}} \\
        
        Damiano Berti     & \textcolor{AccentColor}{$\blacksquare$}    & $\square$        \\
        Alessandro Frison     & \textcolor{AccentColor}{$\blacksquare$}    & $\square$        \\
        Lorenzo Grolla     & \textcolor{AccentColor}{$\blacksquare$}    & $\square$        \\
        Nicolò Lattanzio    & \textcolor{AccentColor}{$\blacksquare$}    & $\square$        \\
        Alessandro Morabito   & \textcolor{AccentColor}{$\blacksquare$}    & $\square$        \\
        Giacomo Nalotto   & \textcolor{AccentColor}{$\blacksquare$}    & $\square$        \\
        Giulia Romanato   & \textcolor{AccentColor}{$\blacksquare$}    & $\square$        \\
    \end{tabular}
\end{center}

% ====== SEZIONE PRINCIPALE DEL VERBALE ======
\section{Contenuto della riunione}

\subsection{Ordine del giorno}
\begin{enumerate}
 \item Decisione data milestone RTB
 \item Suddivisione ruoli
 \item Valutazione formato documenti
 \item Brainstorming analisi dei requisiti
 \item Decisione Workflow
 \item Gestione  documenti: norme di progetto e glossario
\end{enumerate}

\subsection{Riassunto della discussione}
Durante la valutazione iniziale sono state individuate le possibili finestre temporali per la consegna dell’RTB, fissando indicativamente il periodo tra l’\textbf{8 e il 17 gennaio}, tenendo conto della sessione invernale degli esami.
\\
È stata discussa la suddivisione dei ruoli all’interno del gruppo e la relativa rotazione; tuttavia, si è deciso di \textbf{mantenerli invariati} almeno fino al primo incontro con l’azienda proponente.
\\
Il gruppo ha inoltre valutato il formato da adottare per la redazione della documentazione, confrontando l’utilizzo di \LaTeX{} con l’alternativa Typescript. Al termine della discussione, è stato stabilito di proseguire con l’utilizzo di \textbf{\LaTeX{}}, abbandonando la piattaforma Overleaf in favore di un’\textbf{estensione per Visual Studio Code}. 
Contestualmente, è stato pianificato un aggiornamento della \textbf{repository GitHub}, al fine di consentire il caricamento e la modifica condivisa dei verbali in formato \texttt{.tex}, superando le limitazioni di modifica simultanea presenti in Overleaf.
\\
In vista dell’imminente incontro con l’azienda proponente, la cui data è ancora in fase di definizione, il gruppo ha svolto un’attività di \textbf{brainstorming} volta a individuare le principali funzionalità dell’applicazione e a delinearne le possibili modalità di presentazione.
\\
Sono state inoltre formulate e raccolte diverse \textbf{domande da sottoporre a Vargroup}, riguardanti i requisiti dell’applicazione, possibili funzionalità da implementare e aspetti organizzativi necessari per una corretta interazione con l’azienda.
\\
È stato definito il workflow del progetto, stabilendo l'adozione del modello \textbf{GitFlow} per la gestione del versionamento. Si è deciso inoltre di utilizzare \textbf{branch dedicati per ciascun documento}, così da facilitare il lavoro collaborativo e ridurre i conflitti di integrazione.
\\
Per finire, sono stati individuati ulteriori argomenti da integrare alle \textbf{Norme di Progetto} e sono stati definiti i termini da includere e chiarire nel \textbf{Glossario}.

\section{Decisioni e azioni}
\begin{center}
    \rowcolors{2}{LightGray}{white}
    \begin{tabular}{>{\centering\arraybackslash}m{2cm} >{\raggedright\arraybackslash}p{8cm} >{\centering\arraybackslash}p{2.5cm}}
        \rowcolor{AccentColor}
        \textcolor{white}{\textbf{Codice}} & 
        \multicolumn{1}{c}{\textcolor{white}{\textbf{Descrizione}}} & 
        \textcolor{white}{\textbf{Assegnatario}} \\
        
        DEC-RTB-001 & Scelta suddivisione ruoli e definizione rotazione incontri& Tutti \\
        DEC-RTB-002 & Delineamento della dichiarazione degli impegni & Tutti \\
        DEC-RTB-003 & Definizione del Way of Working & Nicolò Lattanzio \\
        DEC-RTB-004 & Definizione del workflow di progetto tramite l'adozione di GitFlow e utilizzo di branch dedicati per ciascun documento & Tutti \\
        
        \midrule
        
        AZ-RTB-001 & Brainstorming Analisi dei Requisiti & Tutti \\
        AZ-RTB-002 & Miglioramento struttura repository GitHub & Damiano Berti \\

    \end{tabular}
\end{center}

\end{document}