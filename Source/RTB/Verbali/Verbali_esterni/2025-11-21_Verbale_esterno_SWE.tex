\documentclass[a4paper, 11pt]{article}

% ====== PACCHETTI NECESSARI ======
\usepackage[utf8]{inputenc}
\usepackage[T1]{fontenc}
\usepackage[italian]{babel}
\usepackage{geometry}
\usepackage{graphicx}
\usepackage[table]{xcolor}
\usepackage{tabularx}
\usepackage{array}
\usepackage{amssymb}
\usepackage{fancyhdr}
\usepackage{titlesec}
\usepackage{helvet}
\renewcommand{\familydefault}{\sfdefault}
\usepackage{lipsum}
\usepackage{hyperref}
\usepackage{booktabs}
\usepackage{enumitem}
\usepackage[utf8]{inputenc} % Specifica la codifica del file (necessaria per le accentate)
\usepackage[T1]{fontenc}    % Migliora l'output dei font per le lingue europee

% ====== IMPOSTAZIONI GLOBALI DI STILE ======

% 1. DEFINIZIONE COLORI BLU-VIOLA
\definecolor{AccentColor}{RGB}{80, 90, 180} % Blu-viola principale
\definecolor{AccentLight}{RGB}{80, 90, 180} % Versione più chiara
\definecolor{AccentDark}{RGB}{50, 60, 140} % Versione più scura
\definecolor{LightGray}{RGB}{245, 245, 250}
\definecolor{MediumGray}{RGB}{200, 200, 210}

% 2. IMPOSTAZIONE MARGINI
\geometry{a4paper, left=2.5cm, right=2.5cm, top=3.5cm, bottom=3.5cm}

% 3. STILE DEI TITOLI DI SEZIONE
\titleformat{\section}
  {\normalfont\sffamily\Large\bfseries\color{AccentColor}}
  {\thesection}
  {1em}
  {}
\titleformat{\subsection}
  {\normalfont\sffamily\large\bfseries\color{AccentDark}}
  {\thesubsection}
  {1em}
  {}

% 4. IMPOSTAZIONE HEADER E FOOTER
\pagestyle{fancy}
\fancyhf{}
\fancyhead[L]{\sffamily\bfseries\color{AccentColor}\@BYTE HOLDERS}
\fancyhead[R]{\sffamily\color{AccentColor}\thepage}
\renewcommand{\headrulewidth}{0.8pt}
\renewcommand{\headrule}{\color{AccentColor}\hrule width\headwidth height\headrulewidth \vskip-\headrulewidth}

% 5. IMPOSTAZIONE LINK
\hypersetup{
    colorlinks=true,
    linkcolor=AccentColor,
    urlcolor=AccentLight,
    citecolor=AccentDark,
}

% 6. PERSONALIZZAZIONE ELENCHI
\setlist[itemize]{itemsep=2pt, topsep=4pt}
\setlist[enumerate]{itemsep=2pt, topsep=4pt}

% ====== COMANDI PERSONALIZZATI ======
\makeatletter
\newcommand{\NomeGruppo}[1]{\def\@NomeGruppo{#1}}
\newcommand{\TitoloVerbale}[1]{\def\@TitoloVerbale{#1}}
\newcommand{\Sommario}[1]{\def\@Sommario{#1}}
\newcommand{\Autore}[1]{\def\@Autore{#1}}
\newcommand{\Verificatore}[1]{\def\@Verificatore{#1}}
\makeatother

% ====== STILE TABELLE MIGLIORATO ======
\newcolumntype{Y}{>{\raggedright\arraybackslash}X} % Colonna giustificata a sinistra
\setlength{\arrayrulewidth}{0.4pt} % Linee più sottili
\setlength{\tabcolsep}{10pt} % Spaziatura interna celle
\renewcommand{\arraystretch}{1.4} % Altezza righe

% ====== INIZIO DEL DOCUMENTO ======
\begin{document}

% ====== INFORMAZIONI PER LA PAGINA DI TITOLO ======
\NomeGruppo{BYTE HOLDERS}
\TitoloVerbale{Verbale Riunione Esterna}
\Sommario{Questo verbale documenta la riunione esterna avvenuta il 21/11/2025 per la discussione dei capitolati e la definizione degli strumenti di lavoro del team.}
\Autore{}
\Verificatore{}

\pagestyle{empty}

% ====== PAGINA DI TITOLO ======
\begin{titlepage}
    \centering
    
    \includegraphics[width=0.55\textwidth]{../../../Assets/ByteHolders1.png}\par\vspace{1.5cm}
    
    {\LARGE \sffamily \bfseries \color{AccentColor}Verbale esterno}\par
    \vspace{0.5cm}
    {\large \sffamily \color{AccentColor}21 Novembre 2025}\par
    
    \vfill
    
    \noindent\color{AccentColor}\rule{\textwidth}{1pt}\par
    \vspace{0.5cm}


    \begin{tabularx}{0.9\textwidth}{@{}>{\bfseries\sffamily}l X@{}}
    Autore & \sffamily Alessandro Frison\\
    \arrayrulecolor{MediumGray}\hline \\[-1.5ex]
    Verificatore & \sffamily Giacomo Nalotto\\
    \arrayrulecolor{MediumGray}\hline \\[-1.5ex] 
    Approvatore & \sffamily Nicolò Lattanzio\\ 
    \arrayrulecolor{MediumGray}\hline 
\end{tabularx}
    
    \vfill
\end{titlepage}

% ====== INDICE ======
\pagestyle{fancy}
\newpage
\tableofcontents
\newpage

% ====== TABELLA DI VERSIONAMENTO ======
\section{Registro delle versioni}
\begin{center}
    \rowcolors{2}{LightGray}{white}
    \begin{tabular}{>{\centering\arraybackslash}m{1.5cm} >{\centering\arraybackslash}m{2cm} >{\raggedright\arraybackslash}m{2.5cm} >{\raggedright\arraybackslash}m{6.5cm}}
        \rowcolor{AccentColor}
          \textcolor{white}{\textbf{Versione}} & 
          \textcolor{white}{\textbf{Data}} & 
          \multicolumn{1}{c}{\textcolor{white}{\textbf{Autore}}} & 
          \multicolumn{1}{c}{\textcolor{white}{\textbf{Descrizione delle modifiche}}} \\

        1.0.0 & 28/11/2025 & Nicolò Lattanzio & Approvazione del verbale. \\
        0.1.0 & 27/11/2025 & Giacomo Nalotto & Verifica del verbale e controllo della tabella delle decisioni ed azioni. \\
        0.0.1 & 25/11/2025 & Alessandro Frison & Creazione e compilazione del verbale della riunione con Var Group. \\
    \end{tabular}
\end{center}

%\vspace{1cm}

% ====== SEZIONE INFORMAZIONI INTRODUTTIVE ======
\section{Informazioni introduttive}

\subsection{Durata e luogo}
\begin{itemize}
    \item \textbf{Inizio:} 15:30
    \item \textbf{Fine:} 18:00
    \item \textbf{Luogo:} Var Group, sede Padova 
\end{itemize}

% ====== TABELLA PRESENZE ======
\subsection{Partecipanti}
\begin{center}
    \rowcolors{2}{LightGray}{white}
    \begin{tabular}{>{\raggedright\arraybackslash}p{6cm} c c}
        \rowcolor{AccentColor}
        \textcolor{white}{\textbf{Nome e Cognome}} & 
        \textcolor{white}{\textbf{Presente}} & 
        \textcolor{white}{\textbf{Assente}} \\
        
        Damiano Berti     & \textcolor{AccentColor}{$\blacksquare$}    & $\square$        \\
        Alessandro Frison     & \textcolor{AccentColor}{$\blacksquare$}    & $\square$        \\
        Lorenzo Grolla     & \textcolor{AccentColor}{$\blacksquare$}    & $\square$        \\
        Nicolò Lattanzio    & \textcolor{AccentColor}{$\blacksquare$}    & $\square$        \\
        Alessandro Morabito   & \textcolor{AccentColor}{$\blacksquare$}    & $\square$        \\
        Giacomo Nalotto   & \textcolor{AccentColor}{$\blacksquare$}    & $\square$        \\
        Giulia Romanato   & \textcolor{AccentColor}{$\blacksquare$}    & $\square$        \\
    \end{tabular}
\end{center}

% ====== SEZIONE PRINCIPALE DEL VERBALE ======
\section{Contenuto della riunione}

\subsection{Ordine del giorno}
\begin{enumerate}
    \item Conoscenza con l'azienda
    \item Definizione della modalità di contatto
    \item Definizione degli attori
    \item Design Thinking
\end{enumerate}


\clearpage
\subsection{Definizione delle modalità di contatto}
Abbiamo discusso le modalità di contatto con l'azienda Var Group, concordando che il canale principale per la comunicazione asincrona sarà tramite \textbf{Slack}, con \textbf{incontri bisettimanali} per aggiornamenti e chiarimenti. 


\subsection{Definizione degli attori}
In modo individuale ci siamo chiesti chi fossero gli attori principali del progetto e li abbiamo appuntati su dei blocchi note. Successivamente, in gruppo, abbiamo discusso le nostre idee e siamo giunti a una lista di attori candidati per il progetto:
\begin{enumerate}
    \item \textbf{Tech Lead} (supervisore tecnico del progetto)
    \item \textbf{Project Manager} (responsabile dei progetti a lui assegnati)
    \item \textbf{Developer} (sviluppatore software)
    \item \textbf{Security Analyst} (analista della sicurezza informatica)
    \item \textbf{End User} (utente finale)
\end{enumerate}
Dopo un'attenta analisi, anche grazie al supporto di Var Group, abbiamo deciso che gli attori principali del progetto saranno:
\begin{enumerate}
    \item \textbf{Tech Lead}
    \item \textbf{Project Manager}
    \item \textbf{Developer}
\end{enumerate}

\subsection{Definizione dei ruoli degli attori}
Successivamente, abbiamo discusso le differenze tra ciascun attore, arrivando alle seguenti conclusioni:
\begin{itemize}
    \item \textbf{Tech Lead}
        \begin{itemize}
            \item supervisione tecnica di tutti i progetti (con filtri per una maggiore leggibilità)
            \item visione delle criticità OWASP
            \item analisi delle repository
            \item visualizzazione dei problermi di documentazione (con suggerimenti per il miglioramento)
            \item verifica lo stato delle repo in base ai risultati dei test sulla repository
        \end{itemize}
    \item \textbf{Project Manager}
        \begin{itemize}
            \item visione d'insieme dei progetti assegnati (con filtri per una maggiore leggibilità)
            \item assegnazione issue / task ai developer
            \item visualizzazione delle criticità OWASP nei progetti assegnati
            \item monitoraggio dello stato di avanzamento dei suoi progetti
            \item possibilità di lanciare analisi sulle repository
            \item gestione della struttura della documentazione
            \item ha l'accesso completo alle repository dei progetti posti sotto la sua supervisione
        \end{itemize}
    \item \textbf{Developer}
        \begin{itemize}
            \item accesso alle repository assegnategli dal Project Manager
            \item vede le issue a suo carico (assegnategli dal Project Manager)
            \item vede i risultati dei test sui suoi push
            \item vede lo stato della documentazione delle sue repository
            \item vede la situazione delle repo a cui deve lavorare
        \end{itemize} 
\end{itemize}

\subsection{Definizione della tipologia di Agenti}
Abbiamo discusso le possibili tipologie di agenti da implementare per il progetto, indicando i pro e i contro di ciascuna, arrivando alle seguenti conclusioni:
\begin{itemize}
    \item \textbf{Agenti democratici}
        \begin{itemize}
            \item Pro
                \begin{itemize}
                    \item Analisi più rapida
                    \item Maggiore affidabilità nei risultati
                \end{itemize}
            \item Contro
                \begin{itemize}
                    \item Maggiore complessità di implementazione
                    \item Maggiore utilizzo di risorse computazionali
                    \item Possono dare risultati diversi in base all'agente che prende in carico la task
                \end{itemize}
        \end{itemize}
    \item \textbf{Agenti autocratici}
        \begin{itemize}
            \item Pro
            \begin{itemize}
                \item Semplicità di implementazione
                \item Minore utilizzo di risorse computazionali
                \item Risultati coerenti in quanto sempre lo stesso agente prende in carico la task
                \item Si adatta meglio alle esigenze del progetto
            \end{itemize}
            \item Contro
            \begin{itemize}
                \item Maggiore tempo di analisi
                \item Minore affidabilità nei risultati
            \end{itemize}
        \end{itemize}
\end{itemize}
Dopo aver valutato le opzioni, il gruppo ha deciso di optare per un'architettura autocratica.

\subsection{Design Thinking}
Infine dopo la prima fase di brainstorming individuale, abbiamo discusso le idee emerse e siamo giunti al diagramma di flusso riportato in Figura \ref{fig:diagramma}.
\\
In particolare, il sistema prevede l'analisi delle repository tramite \textbf{agenti software} che eseguono \textbf{test automatici} per valutare la \textbf{qualità del codice}, la \textbf{sicurezza} e la \textbf{documentazione}. 
\\
I risultati vengono poi aggregati e presentati agli utenti in base al loro ruolo, permettendo una \textbf{gestione efficiente e mirata} dei progetti software.
\begin{figure}[h!]
    \centering
    \includegraphics[width=0.75\textwidth]{../../../Assets/FlowDiagram.png}
    \caption{Diagramma di flusso delle funzionalità principali del progetto}
    \label{fig:diagramma}
\end{figure}

% ====== SEZIONE DECISIONI E AZIONI ======
\section{Decisioni e azioni}
\begin{center}
    \rowcolors{2}{LightGray}{white}
    \begin{tabular}{>{\centering\arraybackslash}m{2cm} >{\raggedright\arraybackslash}p{8cm} >{\centering\arraybackslash}p{2.5cm}}
        \rowcolor{AccentColor}
        \textcolor{white}{\textbf{codice}} & 
        \multicolumn{1}{c}{\textcolor{white}{\textbf{Descrizione}}} & 
        \textcolor{white}{\textbf{Assegnatario}} \\
        
        DEC-RTB-005 & Decisione degli attori & Tutti \\
        DEC-RTB-006 & Definizione a grandi linee dei ruoli degli attori &  Tutti \\
        DEC-RTB-007 & Decisione delle modalità di contatto con l'azienda & Tutti \\
        DEC-RTB-007 & Scelta del tipo di Agenti & Tutti \\
        DEC-RTB-008  & Decisione sulla modalità di svolgimento degli incontri bisettimanali & Tutti \\
        
        \midrule
        
        AZ-RTB-003  & Creazione di un account Slack e iscrizione al canale creato da Var Group & Tutti \\
        AZ-RTB-004  & Definizione più dettagliata dei ruoli degli attori & Tutti \\
        AZ-RTB-005  & Sviluppo del diagramma di Figura \ref{fig:diagramma} & Tutti \\
        AZ-RTB-006  & Inizio Analisi dei Requisiti & Tutti \\
    \end{tabular}

    \vspace{2cm}
    \rule{6cm}{0.4pt}\\
    \textit{Firma del proponente}

\end{center}

\end{document}